% gpl-lgpl.tex                                                  -*- LaTeX -*-
%      Tutorial Text for the Detailed Study and Analysis of GPL and LGPL course
%
% Copyright (C) 2003, 2004, 2005, 2006 Free Software Foundation, Inc.
% Copyright (C) 2014                   Bradley M. Kuhn

% License: CC-By-SA-4.0

% The copyright holders hereby grant the freedom to copy, modify, convey,
% Adapt, and/or redistribute this work under the terms of the Creative
% Commons Attribution Share Alike 4.0 International License.

% This text is distributed in the hope that it will be useful, but
% WITHOUT ANY WARRANTY; without even the implied warranty of
% MERCHANTABILITY or FITNESS FOR A PARTICULAR PURPOSE.

% You should have received a copy of the license with this document in
% a file called 'CC-By-SA-4.0.txt'.  If not, please visit
% https://creativecommons.org/licenses/by-sa/4.0/legalcode to receive
% the license text.

\newcommand{\defn}[1]{\emph{#1}}

\part{Detailed Analysis of the GNU GPL and Related Licenses}

\begin{center}

{\parindent 0in
\tutorialpartsplit{``Detailed Analysis of the GNU GPL and Related Licenses''}{This part} is: \\
\begin{tabbing}
Copyright \= \copyright{} 2003, 2004, 2005, 2006 \= \hspace{.2in} Free Software Foundation, Inc. \\
Copyright \= \copyright{} 2014 \= \hspace{.2in} Bradley M. Kuhn \\
Copyright \= \copyright{} 2014 \= \hspace{.2in} Anthony K. Sebro, Jr. \\
\end{tabbing}

Authors of \tutorialpartsplit{``Detailed Analysis of the GNU GPL and Related Licenses''}{this part} are: \\

Free Software Foundation, Inc. \\
Bradley M. Kuhn \\
David ``Novalis'' Turner \\
Daniel B. Ravicher \\
Tony Sebro \\
John Sullivan

\vspace{.3in}

The copyright holders of \tutorialpartsplit{``Detailed Analysis of the GNU GPL and Related Licenses''}{this part} hereby grant the freedom to copy, modify,
convey, Adapt, and/or redistribute this work under the terms of the Creative
Commons Attribution Share Alike 4.0 International License.  A copy of that
license is available at
\verb=https://creativecommons.org/licenses/by-sa/4.0/legalcode=.  }

\end{center}

\bigskip

\bigskip

\tutorialpartsplit{This tutorial}{This part of the tutorial} gives a
comprehensive explanation of the most popular Free Software copyright
license, the GNU General Public License (``GNU GPL'', or sometimes just
``GPL'') -- both version 2 (``GPLv2'') and version 3 (``GPLv3'') -- and
teaches lawyers, software developers, managers and business people how to use
the GPL (and GPL'd software) successfully both as a community-building
``Constitution'' for a software project, or to incorporate copylefted
software into a new Free Software business and in existing, successful
enterprises.

To successfully benefit of from this part of the tutorial, readers should
have a general familiarity with software development processes.  A vague
understanding of how copyright law applies to software is also helpful.  The
tutorial is of most interest to lawyers, software developers and managers who
run or advise software businesses that modify and/or redistribute software
under terms of the GNU GPL (or who wish to do so in the future), and those
who wish to make use of existing GPL'd software in their enterprise.

Upon completion of this part of the tutorial, successful students can expect
to have learned the following:

\begin{itemize}

  \item The freedom-defending purpose of each term of the GNU GPLv2 and GPLv3.

  \item The differences between GPLv2 and GPLv3.

  \item The redistribution options under the GPLv2 and GPLv3.

  \item The obligations when modifying GPLv2'd or GPLv3'd software.

  \item How to build a plan for proper and successful compliance with the GPL.

  \item The business advantages that the GPL provides.

  \item The most common business models used in conjunction with the GPL.

  \item How existing GPL'd software can be used in existing enterprises.

  \item The basics of the Lesser GPLv2.1 and Lesser GPLv3, and how they
    differs from the GPLv2 and GPLv3, respectively.

  \item The basics to begin understanding the complexities regarding
    derivative and combined works of software.
\end{itemize}

%%%%%%%%%%%%%%%%%%%%%%%%%%%%%%%%%%%%%%%%%%%%%%%%%%%%%%%%%%%%%%%%%%%%%%%%%%%%%%%
% END OF ABSTRACTS SECTION
%%%%%%%%%%%%%%%%%%%%%%%%%%%%%%%%%%%%%%%%%%%%%%%%%%%%%%%%%%%%%%%%%%%%%%%%%%%%%%%
% START OF DAY ONE COURSE
%%%%%%%%%%%%%%%%%%%%%%%%%%%%%%%%%%%%%%%%%%%%%%%%%%%%%%%%%%%%%%%%%%%%%%%%%%%%%%%

\chapter{What Is Software Freedom?}

Study of the GNU General Public License (herein, abbreviated as \defn{GNU
  GPL} or just \defn{GPL}) must begin by first considering the broader world
of software freedom. The GPL was not created from a void, rather, it was
created to embody and defend a set of principles that were set forth at the
founding of the GNU project and the Free Software Foundation (FSF) -- the
organization that upholds, defends and promotes the philosophy of software
freedom. A prerequisite for understanding both of the popular versions of GPL
(GPLv2 and GPLv3) and their terms and conditions is a basic understanding of
the principles behind it.  The GPL family of licenses are unlike almost all
other software licenses in that they are designed to defend and uphold these
principles.

\section{The Free Software Definition}
\label{Free Software Definition}

The Free Software Definition is set forth in full on FSF's website at
\verb0http://fsf.org/0 \verb0philosophy/free-sw.html0. This section presents
an abbreviated version that will focus on the parts that are most pertinent
to the GPL\@.

A particular program grants software freedom to a particular user if that
user is granted the following freedoms:

\begin{itemize}


\item The freedom to run the program, for any purpose.

\item The freedom to study how the program works, and modify it

\item The freedom to redistribute copies.

\item The freedom to distribute copies of  modified versions to others.

\end{itemize}

The focus on ``a particular user'' is particularly pertinent here.  It is not
uncommon for the same version of a specific program to grant these freedoms
to some subset of its user base, while others have none or only some of these
freedoms.  Section~\ref{Proprietary Relicensing} talks in detail about how
this can unfortunately happen even if a program is released under the GPL\@.

Many people refer to software that gives these freedoms as ``Open Source.''
Besides having a different political focus than those who call it Free
Software,\footnote{The political differences between the Free Software
  Movement and the Open Source Movement are documented on FSF's Web site at
  {\tt http://www.fsf.org/licensing/essays/free-software-for-freedom.html}.}
those who call the software ``Open Source'' are often focused on a side
issue.  Specifically, user access to the source code of a program is a
prerequisite to make use of the freedom to modify.  However, the important
issue is what freedoms are granted in the license of that source code.

Software freedom is only complete when no restrictions are imposed on how
these freedoms are exercised.  Specifically, users and programmers can
exercise these freedoms noncommercially or commercially.  Licenses that grant
these freedoms for noncommercial activities but prohibit them for commercial
activities are considered non-free.  Even the Open Source Initiative
(\defn{OSI}) (the arbiter of what is considered ``Open Source'') also rules
such licenses not in fitting with their ``Open Source Definition''.

In general, software for which most or all of these freedoms are
restricted in any way is called ``non-Free Software.''  Typically, the
term ``proprietary software'' is used more or less interchangeably with
``non-Free Software.''  Personally, I tend to use the term ``non-Free
Software'' to refer to noncommercial software that restricts freedom
(such as ``shareware'') and ``proprietary software'' to refer to
commercial software that restricts freedom (such as nearly all of
Microsoft's and Oracle's offerings).

Keep in mind that the none of the terms ``software freedom'', ``open source''
and ``free software'' are not known to be trademarked by any organization in
any jurisdiction.  As such, it's quite common that these terms are abused and
misused by parties who wish to bank on the popularity of software freedom.
When one considers using, modifying or redistributing a software package that
purports to be Open Source or Free Software, one \textbf{must} verify that
the license grants software freedom

Furthermore, throughout this text, we generally prefer the term ``software
freedom'', as this is the least ambiguous term available to describe software
that meets the Free Software Definition.  For example, it is well known and
often discussed that the adjective ``free'' has two unrelated meanings in
English: ``free as in freedom'' and ``free as in price''.  Meanwhile, the
term ``open source'' is even more confusing, because it refers only to the
``freedom to study'', which is merely a subset of one of the four freedoms.

The remainder of this section considers each of each component of software
freedom in detail.

\subsection{The Freedom to Run}

The first tenant of software freedom is the user's fully unfettered right to
run the program.  The software's license must permit any conceivable use of
the software.  Perhaps, for example, the user has discovered an innovative
use for a particular program, one that the programmer never could have
predicted.  Such a use must not be restricted.

It was once rare that this freedom was restricted by even proprietary
software; but such is quite common today. Most End User Licensing Agreements
(EULAs) that cover most proprietary software typically restrict some types of
uses.  Such restrictions of any kind are an unacceptable restriction on
software freedom.

\subsection{The Freedom to Change and Modify}

Perhaps the most useful right of software freedom is the users' right to
change, modify and adapt the software to suit their needs.  Access to the
source code and related build and installation scripts are an essential part
of this freedom.  Without the source code, and the ability to build and
install the binary applications from that source, users cannot effectively
exercise this freedom.

Programmers take direct benefit from this freedom.  However, this freedom
remains important to users who are not programmers.  While it may seem
counterintuitive at first, non-programmer users often exercise this freedom
indirectly in both commercial and noncommercial settings.  For example, users
often seek noncommercial help with the software on email lists and in users
groups.  To make use of such help they must either have the freedom to
recruit programmers who might altruistically assist them to modify their
software, or to at least follow rote instructions to make basic modifications
themselves.

More commonly, users also exercise this freedom commercially.  Each user, or
group of users, may hire anyone they wish in a competitive free market to
modify and change the software.  This means that companies have a right to
hire anyone they wish to modify their Free Software.  Additionally, such
companies may contract with other companies to commission software
modification.

\subsection{The Freedom to Copy and Share}

Users share Free Software in a variety of ways. Software freedom advocates
work to eliminate a fundamental ethical dilemma of the software age: choosing
between obeying a software license and friendship (by giving away a copy of a
program to your friend who likes the software you are using). Licenses that
respect software freedom, therefore, permit altruistic sharing of software
among friends.

The commercial environment also benefits of this freedom.  Commercial sharing
includes selling copies of Free Software: Free Software can be sold at any
price to anyone.  Those who redistribute Free Software commercially also have
the freedom to selectively distribute (i.e., you can pick your customers) and
to set prices at any level that redistributor sees fit.

Of course, most people get copies of Free Software very cheaply (and
sometimes without charge).  The competitive free market of Free Software
tends to keep prices low and reasonable.  However, if someone is willing to
pay billions of dollars for one copy of the GNU Compiler Collection, such a
sale is completely permitted.

Another common instance of commercial sharing is service-oriented
distribution.  For example, some distribution vendors provide immediate
security and upgrade distribution via a special network service.  Such
distribution is not necessarily contradictory with software freedom.

(Section~\ref{Business Models} of this tutorial talks in detail about some
common Free Software business models that take advantage of the freedom to
share commercially.)

\subsection{The Freedom to Share Improvements}

The freedom to modify and improve is somewhat empty without the freedom to
share those improvements.  The Software freedom community is built on the
pillar of altruistic sharing of improved Free Software. Inevitably, a
Free Software project sprouts a mailing list where improvements are shared
freely among members of the development community.  Such noncommercial
sharing is the primary reason that Free Software thrives.

Commercial sharing of modified Free Software is equally important.
For commercial support to exist in a competitive free market, all
developers -- from single-person contractors to large software
companies -- must have the freedom to market their services as
improvers of Free Software.  All forms of such service marketing must
be equally available to all.

For example, selling support services for Free Software is fully
permitted. Companies and individuals can offer themselves as ``the place
to call'' when software fails or does not function properly.  For such a
service to be meaningful, the entity offering that service needs the
right to modify and improve the software for the customer to correct any
problems that are beyond mere user error.

Software freedom licenses also permit any entity to distribute modified
versions of Free Software.  Most Free Software programs have a ``standard
version'' that is made available from the primary developers of the software.
However, all who have the software have the ``freedom to fork'' -- that is,
make available nontrivial modified versions of the software on a permanent or
semi-permanent basis.  Such freedom is central to vibrant developer and user
interaction.

Companies and individuals have the right to make true value-added versions
of Free Software.  They may use freedom to share improvements to
distribute distinct versions of Free Software with different functionality
and features.  Furthermore, this freedom can be exercised to serve a
disenfranchised subset of the user community.  If the developers of the
standard version refuse to serve the needs of some of the software's
users, other entities have the right to create a long- or short-lived fork
to serve that sub-community.

\section{How Does Software Become Free?}

The previous section set forth key freedoms and rights that are referred to
as ``software freedom''.  This section discusses the licensing mechanisms
used to enable software freedom.  These licensing mechanism were ultimately
created as a community-oriented ``answer'' to the existing proprietary
software licensing mechanisms.  Thus, first, consider carefully why
proprietary software exists in the first place.

Proprietary software exists at all only because it is governed by copyright
law.\footnote{This statement is admittedly an oversimplification. Patents and
  trade secrets can cover software and make it effectively non-Free, and one
  can contract away their rights and freedoms regarding software, or source
  code can be practically obscured in binary-only distribution without
  reliance on any legal system.  However, the primary control mechanism for
  software is copyright, and therefore this section focuses on how copyright
  restrictions make software proprietary.} Copyright law, with respect to
software, typically governs copying, modifying, and redistributing that
software (For details of this in the USA, see
\href{http://www.copyright.gov/title17/92chap1.html#106}{\S~106} and
\href{http://www.copyright.gov/title17/92chap1.html#117}{\S~117} of
\href{http://www.law.cornell.edu/uscode/text/17}{Title 17} of the
\textit{United States Code}).\footnote{Copyright law in general also governs
  ``public performance'' of copyrighted works. There is no generally agreed
  definition for public performance of software and both GPLv2 and GPLv3 do
  not govern public performance.} By law (in the USA and in most other
jurisdictions), the copyright holder (most typically, the author) of the work controls
how others may copy, modify and/or distribute the work. For proprietary
software, these controls are used to prohibit these activities. In addition,
proprietary software distributors further impede modification in a practical
sense by distributing only binary code and keeping the source code of the
software secret.

Copyright is not a natural state, it is a legal construction. In the USA, the
Constitution permits, but does not require, the creation of copyright law as
federal legislation.  Software, since it is ``an original works of authorship
fixed in any tangible medium of expression ...  from which they can be
perceived, reproduced, or otherwise communicated, either directly or with the
aid of a machine or device'' (as stated in
\href{http://www.law.cornell.edu/uscode/text/17/102}{17 USC \S~102}), is thus
covered by the statues, and is copyrighted by default.

However, software, in its natural state without copyright, is Free
Software. In an imaginary world with no copyright, the rules would be
different. In this world, when you received a copy of a program's source
code, there would be no default legal system to restrict you from sharing it
with others, making modifications, or redistributing those modified
versions.\footnote{Note that this is again an oversimplification; the
  complexities with this argument are discussed in
  Section~\ref{software-and-non-copyright}.}

Software in the real world is copyrighted by default and is automatically
covered by that legal system.  However, it is possible to move software out
of the domain of the copyright system.  A copyright holder can often
\defn{disclaim} their copyright.  If copyright is disclaimed, the software is
not governed by copyright law.   Software not governed by copyright is in the
``public domain.''

\subsection{Public Domain Software}

Theoretically, an author can create public domain software by disclaiming all
copyright interest on the work. In the USA and other countries that have
signed the Berne convention on copyright, software is copyrighted
automatically by the author when she ``fixes the software into a tangible
medium.''  In the software world, this usually means typing the source code
of the software into a file.

Imagine if authors could truly disclaim those default control of copyright
law.  If so, the software is in the public domain -- no longer covered by
copyright.  Since copyright law is the construction allowing for most
restrictions on software (i.e., prohibition of copying, modification, and
redistribution), removing the software from the copyright system usually
yields software freedom for its users.

Carefully note that software in the public domain is \emph{not} licensed
in any way. It is nonsensical to say software is ``licensed for the
public domain,'' or any phrase that implies the copyright holder gave
expressed permission to take actions governed by copyright law.

By contrast, the copyright holders instead renounced copyright controls on
the work.  The law gave the copyright holder exclusive controls over the
work, and they chose to waive those controls.  Software in the public domain
is absent copyright and absent a license. The software freedoms discussed in
Section~\ref{Free Software Definition} are all granted because there is no
legal system in play to take them away.

Admittedly, a discussion of public domain software is an oversimplified
example.  First, disclaimer of copyright is actually difficult in practice.
Because copyright controls are usually automatically granted and because, in
some jurisdictions, some copyright controls cannot be waived (See
Section~\ref{non-usa-copyright} for further discussion), many copyright
holders sometimes incorrectly believe a work has been placed in the public
domain.  Second, due to aggressive lobbying by the entertainment industry,
the ``exclusive Right'' of copyright, that was supposed to only exist for
``Limited Times'' according to the USA Constitution, appears to be infinite:
simply purchased on the installment plan rather than in whole.  Thus, we must
assume no works of software will fall into the public domain merely due to
the passage of time.

The best example of software known to be in the public domain is software
that is published exclusively produced by the USA government.  Under
\href{http://www.law.cornell.edu/uscode/text/17/105}{17 USC 101 \S~105}, all
works published by the USA Government are not copyrightable.

\subsection{Why Copyright Free Software?}

If simply disclaiming copyright on software yields Free Software, then it
stands to reason that putting software into the public domain is the
easiest and most straightforward way to produce Free Software. Indeed,
some major Free Software projects have chosen this method for making their
software Free. However, most of the Free Software in existence \emph{is}
copyrighted. In most cases (particularly in those of FSF and the GNU
Project), this was done due to very careful planning.

Software released into the public domain does grant freedom to those users
who receive the standard versions on which the original author disclaimed
copyright. However, since the work is not copyrighted, any nontrivial
modification made to the work is fully copyrightable.

Free Software released into the public domain initially is Free, and
perhaps some who modify the software choose to place their work into the
public domain as well. However, over time, some entities will choose to
proprietarize their modified versions. The public domain body of software
feeds the proprietary software. The public commons disappears, because
fewer and fewer entities have an incentive to contribute back to the
commons. They know that any of their competitors can proprietarize their
enhancements. Over time, almost no interesting work is left in the public
domain, because nearly all new work is done by proprietarization.

A legal mechanism is needed to redress this problem. FSF was in fact
originally created primarily as a legal entity to defend software freedom,
and that work of defending software freedom is a substantial part of
its work today. Specifically because of this ``embrace, proprietarize and
extend'' cycle, FSF made a conscious choice to copyright its Free Software,
and then license it under ``copyleft'' terms. Many, including the
developers of the kernel named Linux, have chosen to follow this paradigm.

\label{copyleft-definition}

Copyleft is a legal strategy and mechanism to defend, uphold and propagate software
freedom. The basic technique of copyleft is as follows: copyright the
software, license it under terms that give all the software freedoms, but
use the copyright law controls to ensure that all who receive a copy of
the software have equal rights and freedom. In essence, copyleft grants
freedom, but forbids others to forbid that freedom to anyone else along
the distribution and modification chains.

Copyleft is a general concept. Much like ideas for what a computer might
do must be \emph{implemented} by a program that actually does the job, so
too must copyleft be implemented in some concrete legal structure.
``Share and share alike'' is a phrase that is used often enough to explain the
concept behind copyleft, but to actually make it work in the real world, a
true implementation in legal text must exist. The GPL is the primary
implementation of copyleft in copyright licensing language.

\subsection{Software and Non-Copyright Legal Regimes}
\label{software-and-non-copyright}

The use, modification and distribution of software, like many endeavors,
simultaneously interacts with multiple different legal regimes.  As was noted
early via footnotes, copyright is merely the \textit{most common way} to
restrict users' rights to copy, share, modify and/or redistribute software.
However, proprietary software licenses typically use every mechanism
available to subjugate users.  For example:

\begin{itemize}

\item Unfortunately, despite much effort by many in the software freedom
  community to end patents that read on software (i.e., patents on
  computational ideas), they still ultimately exist.  As such, a software
  program might otherwise be seemly unrestricted, but a patent might read on
  the software and ruin everything for its users.\footnote{See
  \S\S~\ref{gpl-implied-patent-grant},~\ref{GPLv2s7},~\ref{GPLv3s11} for more
  discussion on how the patent system interacts with copyleft, and read
  Richard M.~Stallman's essay,
  \href{http://www.wired.com/opinion/2012/11/richard-stallman-software-patents/}{\textit{Let’s
      Limit the Effect of Software Patents, Since We Can’t Eliminate Them}}
  for more information on the problems these patents present to society.}

\item Digital Restrictions Management (usually called \defn{DRM}) is often
  used to impose technological restrictions on users' ability to exercise
  software freedom that they might otherwise be granted\footnote{See
    \S~\ref{GPLv3s3} for more information on how GPL deals with this issue.}.
  The simplest (and perhaps oldest) form of DRM, of course, is separating
  software source code (read by humans), from their compiled binaries (read
  only by computers).  Furthermore,
  \href{http://www.law.cornell.edu/uscode/text/17/1201}{17 USC 1201} often
  prohibits users legally from circumventing some of these DRM systems.

\item Most EULAs also include a contractual agreement that bind users further
  by forcing them to agree to a contractual, prohibitive software license
  before ever even using the software.

\end{itemize}

Thus, most proprietary software restricts users via multiple interlocking
legal and technological means.  Any license that truly respect the software
freedom of all users must not only grant appropriate copyright permissions,
but also \textit{prevent} restrictions from other legal and technological
means like those listed above.

\subsection{Non-USA Copyright Regimes}
\label{non-usa-copyright}

Generally speaking, copyright law operates similarly enough in countries that
have signed the Berne Convention on Copyright, and software freedom licenses
have generally taken advantage of this international standardization of
copyright law.  However, copyright law does differ from country to country,
and commonly, software freedom licenses like GPL must be considered under the
copyright law in the jurisdiction where any licensing dispute occurs.

Those who are most familiar with the USA's system of copyright often are
surprised to learn that there are certain copyright controls that cannot be
waived nor disclaimed.  Specifically, many copyright regimes outside the USA
recognize a concept of moral rights of authors.  Typically, moral rights are
fully compatible with respecting software freedom, as they are usually
centered around controls that software freedom licenses generally respect,
such as the right of an authors to require proper attribution for their work.

\section{A Community of Equality}

The previous section described the principles of software freedom, a brief
introduction to mechanisms that typically block these freedoms, and the
simplest ways that copyright holders might grant those freedoms to their
users for their copyrighted works of software.  The previous section also
introduced the idea of \textit{copyleft}: a licensing mechanism to use
copyright to not only grant software freedom to users, but also to uphold
those rights against those who might seek to curtail them.

Copyleft, as defined in \S~\ref{copyleft-definition}, is a general term this
mechanism.  The remainder of this text will discuss details of various
real-world implementations of copyleft -- most notably, the GPL\@.

This discussion begins first with some general explanation of what the GPL is
able to do in software development communities.  After that brief discussion
in this section, deeper discussion of how GPL accomplishes this in practice
follows in the next chapter.

Simply put, though, the GPL ultimately creates an community of equality for
both business and noncommercial users.

\subsection{The Noncommercial Community}

A GPL'd code base becomes a center of a vibrant development and user
community.  Traditionally, volunteers, operating noncommercially out of
keen interest or ``scratch an itch'' motivations, produce initial versions
of a GPL'd system.  Because of the efficient distribution channels of the
Internet, any useful GPL'd system is adopted quickly by noncommercial
users.

Fundamentally, the early release and quick distribution of the software
gives birth to a thriving noncommercial community.  Users and developers
begin sharing bug reports and bug fixes across a shared intellectual
commons.  Users can trust the developers, because they know that if the
developers fail to address their needs or abandon the project, the GPL
ensures that someone else has the right to pick up development.
Developers know that the users cannot redistribute their software without
passing along the rights granted by GPL, so they are assured that every
one of their users is treated equally.

Because of the symmetry and fairness inherent in GPL'd distribution,
nearly every GPL'd package in existence has a vibrant noncommercial user
and developer base.

\subsection{The Commercial Community}

By the same token, nearly all established GPL'd software systems have a
vibrant commercial community.  Nearly every GPL'd system that has gained
wide adoption from noncommercial users and developers eventually begins
to fuel a commercial system around that software.

For example, consider the Samba file server system that allows Unix-like
systems (including GNU/Linux) to serve files to Microsoft Windows systems.
Two graduate students originally developed Samba in their spare time and
it was deployed noncommercially in academic environments\footnote{See
  \href{http://turtle.ee.ncku.edu.tw/docs/samba/history}{Andrew Tridgell's
    ``A bit of history and a bit of fun''}}.  However, very
soon for-profit companies discovered that the software could work for them
as well, and their system administrators began to use it in place of
Microsoft Windows NT file-servers.  This served to lower the cost of
running such servers by orders of magnitude. There was suddenly room in
Windows file-server budgets to hire contractors to improve Samba.  Some of
the first people hired to do such work were those same two graduate
students who originally developed the software.

The noncommercial users, however, were not concerned when these two
fellows began collecting paychecks off of their GPL'd work.  They knew
that because of the nature of the GPL that improvements that were
distributed in the commercial environment could easily be folded back into
the standard version.  Companies are not permitted to proprietarize
Samba, so the noncommercial users, and even other commercial users are
safe in the knowledge that the software freedom ensured by GPL will remain
protected.

Commercial developers also work in concert with noncommercial
developers.  Those two now-long-since graduated students continue to
contribute to Samba altruistically, but also get paid work doing it.
Priorities change when a client is in the mix, but all the code is
contributed back to the standard version.  Meanwhile, many other
individuals have gotten involved noncommercially as developers,
because they want to ``cut their teeth on Free Software,'' or because
the problems interest them.  When they get good at it, perhaps they
will move on to another project, or perhaps they will become
commercial developers of the software themselves.

No party is a threat to another in the GPL software scenario because
everyone is on equal ground.  The GPL protects rights of the commercial
and noncommercial contributors and users equally. The GPL creates trust,
because it is a level playing field for all.

\subsection{Law Analogy}

In his introduction to Stallman's \emph{Free Software, Free Society},
Lawrence Lessig draws an interesting analogy between the law and Free
Software. He argues that the laws of a free society must be protected
much like the GPL protects software.  So that I might do true justice to
Lessig's argument, I quote it verbatim:

\begin{quotation}

A ``free society'' is regulated by law. But there are limits that any free
society places on this regulation through law: No society that kept its
laws secret could ever be called free.  No government that hid its
regulations from the regulated could ever stand in our tradition. Law
controls.  But it does so justly only when visibly.  And law is visible
only when its terms are knowable and controllable by those it regulates,
or by the agents of those it regulates (lawyers, legislatures).

This condition on law extends beyond the work of a legislature.  Think
about the practice of law in American courts.  Lawyers are hired by their
clients to advance their clients' interests.  Sometimes that interest is
advanced through litigation. In the course of this litigation, lawyers
write briefs. These briefs in turn affect opinions written by judges.
These opinions decide who wins a particular case, or whether a certain law
can stand consistently with a constitution.

All the material in this process is free in the sense that Stallman means.
Legal briefs are open and free for others to use.  The arguments are
transparent (which is different from saying they are good), and the
reasoning can be taken without the permission of the original lawyers.
The opinions they produce can be quoted in later briefs.  They can be
copied and integrated into another brief or opinion.  The ``source code''
for American law is by design, and by principle, open and free for anyone
to take. And take lawyers do---for it is a measure of a great brief that
it achieves its creativity through the reuse of what happened before.  The
source is free; creativity and an economy is built upon it.

This economy of free code (and here I mean free legal code) doesn't starve
lawyers.  Law firms have enough incentive to produce great briefs even
though the stuff they build can be taken and copied by anyone else.  The
lawyer is a craftsman; his or her product is public.  Yet the crafting is
not charity. Lawyers get paid; the public doesn't demand such work
without price.  Instead this economy flourishes, with later work added to
the earlier.

We could imagine a legal practice that was different --- briefs and
arguments that were kept secret; rulings that announced a result but not
the reasoning. Laws that were kept by the police but published to no one
else. Regulation that operated without explaining its rule.

We could imagine this society, but we could not imagine calling it
``free.''  Whether or not the incentives in such a society would be better
or more efficiently allocated, such a society could not be known as free.
The ideals of freedom, of life within a free society, demand more than
efficient application.  Instead, openness and transparency are the
constraints within which a legal system gets built, not options to be
added if convenient to the leaders.  Life governed by software code should
be no less.

Code writing is not litigation.  It is better, richer, more
productive.  But the law is an obvious instance of how creativity and
incentives do not depend upon perfect control over the products
created.  Like jazz, or novels, or architecture, the law gets built
upon the work that went before. This adding and changing is what
creativity always is.  And a free society is one that assures that its
most important resources remain free in just this sense.\footnote{This
quotation is Copyright \copyright{} 2002, Lawrence Lessig. It is
licensed under the terms of
\href{http://creativecommons.org/licenses/by/1.0/}{the ``Attribution
License'' version 1.0} or any later version as published by Creative
Commons.}
\end{quotation}

In essence, lawyers are paid to service the shared commons of legal
infrastructure.  Few citizens defend themselves in court or write their
own briefs (even though they are legally permitted to do so) because
everyone would prefer to have an expert do that job.

The Free Software economy is a market ripe for experts.  It
functions similarly to other well established professional fields like the
law. The GPL, in turn, serves as the legal scaffolding that permits the
creation of this vibrant commercial and noncommercial Free Software
economy.

%%%%%%%%%%%%%%%%%%%%%%%%%%%%%%%%%%%%%%%%%%%%%%%%%%%%%%%%%%%%%%%%%%%%%%%%%%%%%%%
\chapter{A Tale of Two Copyleft Licenses}
\label{tale-of-two-copylefts}

While determining the proper methodology and criteria to yield an accurate
count remains difficult, the GPL is generally considered one of the most
widely used Free Software licenses.  For most of its history --- for 16 years
from June 1991 to June 2007 --- there was really only one version of the GPL,
version 2.

However, the GPL had both earlier versions before version 2, and, more well
known, a revision to version 3. 

\section{Historical Motivations for the General Public License}

The earliest license to grant software freedom was likely the Berkeley
Software Distribution (``BSD'') license.  This license is typical of what are
often called lax, highly permissive licenses.  Not unlike software in the
public domain, these non-copyleft licenses (usually) grant software freedom
to users, but they do not go to any effort to uphold that software freedom
for users.  The so-called ``downstream'' (those who receive the software and
then build new things based on that software) can restrict the software and
distribute further.

The GNU's Not Unix (``GNU'') project, which Richard M.~Stallman (``RMS'')
founded in 1984 to make a complete Unix-compatible operating system
implementation that assured software freedom for all.  However, RMS saw that
using a license that gave but did not assure software freedom would be
counter to the goals of the GNU project.  RMS invented ``copyleft'' as an
answer to that problem, and began using various copyleft licenses for the
early GNU project programs\footnote{RMS writes more fully about this topic in
  his essay entitled simply
  \href{http://www.gnu.org/gnu/thegnuproject.html}{\textit{The GNU Project}}.
    For those who want to hear the story in his own voice,
    \href{http://audio-video.gnu.org/audio/}{speech recordings} of his talk,
    \textit{The Free Software Movement and the GNU/Linux Operating System}
    are also widely available}.

\section{Proto-GPLs And Their Impact}

The earliest copyleft licenses were specific to various GNU programs.  For
example, \href{http://www.free-soft.org/gpl_history/emacs_gpl.html}{The Emacs
  General Public License} was likely the first copyleft license ever
published.  Interesting to note that even this earliest copyleft license
contains a version of the well-known GPL copyleft clause:

\begin{quotation}
You may modify your copy or copies of GNU Emacs \ldots provided that you also
\ldots cause the whole of any work that you distribute or publish, that in
whole or in part contains or is a derivative of GNU Emacs or any part
thereof, to be licensed at no charge to all third parties on terms identical
to those contained in this License Agreement.
\end{quotation}

This simply stated clause is the fundamental innovation of copyleft.
Specifically, copyleft \textit{uses} the copyright holders' controls on
permission to modify the work to add a conditional requirement.  Namely,
downstream users may only have permission to modify  the work if they pass
along the same permissions on the modified version that came originally to
them.

These original program-specific proto-GPLs give an interesting window into
the central ideas and development of copyleft.  In particular, reviewing them
shows how the text of the GPL we know has evolved to address more of the
issues discussed earlier in \S~\ref{software-and-non-copyright}.

\section{The GNU General Public License, Version 1}
\label{GPLv1}

In January 1989, the FSF announced that the GPL had been converted into a
``subroutine'' that could be reused not just for all FSF-copyrighted
programs, but also by anyone else.  As the FSF claimed in its announcement of
the GPLv1\footnote{The announcement of GPLv1 was published in the
  \href{http://www.gnu.org/bulletins/bull6.html\#SEC8}{GNU'S Bulletin, vol 1,
    number 6 dated January 1989}.  (Thanks very much to Andy Tai for his
  \href{http://www.free-soft.org/gpl_history/}{consolidation of research on
    the history of the pre-v1 GPL's}.)}:
\begin{quotation}
To make it easier to copyleft programs, we have been improving on the
legalbol architecture of the General Public License to produce a new version
that serves as a general-purpose subroutine: it can apply to any program
without modification, no matter who is publishing it.
\end{quotation}

This, like many inventive ideas, seems somewhat obvious in retrospect.  But,
the FSF had some bright people and access to good lawyers when it started.
It took almost five years from the first copyleft licenses to get to a
generalized, reusable GPLv1.  In the context and mindset of the 1980s, this
is not surprising.  The idea of reusable licensing infrastructure was not
only uncommon, it was virtually nonexistent!  Even the early BSD licenses
were simply copied and rewritten slightly for each new use\footnote{It
  remains an interesting accident of history that the early BSD problematic
  ``advertising clause'' (discussion of which is somewhat beyond the scope of
  this tutorial) lives on into current day, simply because while the
  University of California at Berkeley gave unilateral permission to remove
  the clause from \textit{its} copyrighted works, others who adapted the BSD
  license with their own names in place of UC-Berkeley's never have.}.  The
GPLv1's innovation of reuable licensing infrastructure, an obvious fact
today, was indeed a novel invention for its day\footnote{We're all just
  grateful that the FSF also opposes business method patents, since the FSF's
  patent on a ``method for reusable licensing infrastructure'' would have
  not expired until 2006!}.

\section{The GNU General Public License, Version 2}

The GPLv2 was released two and a half years after GPLv1, and over the
following sixteen years, it became the standard for copyleft licensing until
the release of GPLv3 in 2007 (discussed in more detail in the next section).

While this tutorial does not discuss the terms of GPLv1 in detail, it is
worth noting below the three key changes that GPLv2 brought:

\begin{itemize}

\item Software patents and their danger are explicitly mentioned, inspiring
  (in part) the addition of GPLv2~\S\S5--7.  (These sections are discussed in
  detail in \S~\ref{GPLv2s5}, \S~\ref{GPLv2s6} and \S~\ref{GPLv2s7} of this
  tutorial.)

\item GPLv2~\S2's copyleft terms are expanded to more explicitly discuss the
  issue of combined works.  (GPLv2~\S2 is discussed in detail in
  \S~\ref{GPLv2s2} in this tutorial).

\item GPLv2~\S3 includes more detailed requirements, including the phrase
 ``the scripts used to control compilation and installation of the
  executable'', which is a central component of current GPLv2 enforcement
  .  (GPLv2~\S3 is discussed in detail in
  \S~\ref{GPLv2s3} in this tutorial).
\end{itemize}

The next chapter discusses GPLv2 in full detail, and readers who wish to dive
into the section-by-section discussion of the GPL should jump ahead now to
that chapter.  However, the most interesting fact to note here is how GPLv2
was published with little fanfare and limited commentary.  This contrasts
greatly with the creation of GPLv3.

\section{The GNU General Public License, Version 3}

RMS began drafting GPLv2.2 in mid-2002, and FSF ran a few discussion groups
during that era about new text of that license.  However, rampant violations
of the GPL required more immediate attention of FSF's licensing staff, and as
such, much of the early 2000's was spent doing GPL enforcement
work\footnote{More on GPL enforcement is discussed in \tutorialpartsplit{a
    companion tutorial, \texit{A Practical Guide to GPL
      Compliance}}{Part~\ref{gpl-compliance-guide} of this tutorial}.}.  In
2006, FSF began in earnest drafting work for GPLv3.

The GPLv3 process began in earnest in January 2006.  It became clear that
many provisions of the GPL could benefit from modification to fit new
circumstances and to reflect what the entire community learned from
experience with version 2.  Given the scale of revision it seems proper to
approach the work through public discussion in a transparent and accessible
manner.

The GPLv3 process continued through June 2007, culminating in publication of
GPLv3 and LGPLv3 on 29 June 2007, AGPLv3 on 19 November 2007, and the GCC
Runtime Library Exception on 27 January 2009.

All told, four discussion drafts of GPLv3, two discussion drafts of LGPLv3
and two discussion drafts of AGPLv3 were published and discussed.
Ultimately, FSF remained the final arbiter and publisher of the licenses, and
RMS himself their primary author, but input was sought from many parties, and
these licenses do admittedly look and read more like legislation as a result.
Nevertheless, all of the ``v3'' group are substantially better and improved
licenses.

GPLv3 and its terms are discussed in detail in Chapter\~ref{GPLv3}.

\section{The Innovation of Optional ``Or Any Later'' Version}

An interesting fact of all GPL licenses is that the are ultimate multiple
choices for use of the license.  The FSF is the primary steward of GPL (as
discussed later in \S~\ref{GPLv2s9} and \S~\ref{GPLv2s14}).  However, those
who wish to license works under GPL are not required to automatically accept
changes made by the FSF for their own copyrighted works.

Each licensor may chose three different methods of licensing, as follows:

\begin{itemize}

\item explicitly name a single version of GPL for their work (usually
  indicated in shorthand by saying the license is ``GPLv$X$-only''), or

\item name no version of the GPL, thus they allow their downstream recipients
  to select any version of the GPL they chose (usually indicated in shorthand
  by saying the license is simply ``GPL''), or

\item name a specific version of GPL and give downstream recipients the
  option to chose that version ``or any later version as published by the
  FSF'' (usually indicated by saying the license is
  ``GPLv$X$-or-later'')\footnote{The shorthand of ``GPL$X+$'' is also popular
    for this situation.  The authors of this tutorial prefer ``-or-later''
    syntax, because it (a) mirrors the words ``or'' and ``later from the
    licensing statement, (b) the $X+$ doesn't make it abundantly clear that
    $X$ is clearly included as a license option and (c) the $+$ symbol has
    other uses in computing (such as with regular expressions) that mean
    something different.}
\end{itemize}

\label{license-compatibility-first-mentioned}

Oddly, this flexibility has received (in the opinion of the authors, undue)
criticism, primarily because of the complex and oft-debated notion of
``license compatibility'' (which is explained in detail in
\S~\ref{license-compatibility}).  Copyleft licenses are generally
incompatible with each other, because the details of how they implement
copyleft differs.  Specifically, copyleft works only because of its
requirement that downstream licensors use the \textit{same} license for
combined and modified works.  As such, software licensed under the terms of
``GPLv2-only'' cannot be combined with works licensed ``GPLv3-or-later''.
This is admittedly a frustrating outcome.

Other copyleft licenses that appeared after GPL, such
as the Creative Commons ``Share Alike'' licenses, the Eclipse Public License
and the Mozilla Public License \textbf{require} all copyright holders chosing
to use any version of those licenses to automatically accept and relicense
their copyrighted works under new versions.  Of course ,Creative Commons, the
Eclipse Foundation, and the Mozilla Foundation (like the FSF) have generally
served as excellent stewards of their licenses.  Copyright holders using
those licenses seems to find it acceptable that to fully delegate all future
licensing decisions for their copyrights to these organizations without a
second thought.

However, note that FSF gives herein the control of copyright holders to
decide whether or not to implicitly trust the FSF in its work of drafting
future GPL versions.  The FSF, for its part, does encourage copyright holders
to chose by default ``GPLv$X$-or-later'' (where $X$ is the most recent
version of the GPL published by the FSF).  However, the FSF \textbf{does not
  mandate} that a choice to use any GPL requires a copyright holder ceding
its authority for future licensing decisions to the FSF.  In fact, the FSF
considered this possibility for GPLv3 and chose not to do so, instead opting
for the third-party steward designation clause discussed in
Section~\ref{GPlv3S14}.

\section{Complexities of Two Simultaneously Popular Copylefts}

Obviously most GPL advocates would prefer widespread migration to GPLv3, and
many newly formed projects who seek a copyleft license tend to choose a
GPLv3-based license.  However, many existing copylefted projects continue
with GPLv2-only or GPLv2-or-later as their default license.

While GPLv3 introduces many improvements --- many of which were designed to
increase adoption by for-profit companies --- GPLv2 remains a widely used and
extremely popular license.  The GPLv2 is, no doubt, a good and useful
license.

However, unlike GPLv1, which (as pointed out in \S~\ref{GPLv1}), which is
completely out of use by the mid-1990s.  However, unlike GPLv1 before it,
GPLv2 remains a integral part of the copyleft licensing infrastructure for
some time to come.  As such, those who seek to have expertise in current
topics of copyleft licensing need to study both the GPLv2 and GPLv3 family of
licenses.

Furthermore, GPLv3 can is more easily understood by first studying GPLv2.
This is not only because of their chronological order, but also because much
of the discussion material available for GPLv3 tends to talk about GPLv3 in
contrast to GPLv2.  As such, a strong understanding of GPLv2 helps in
understanding most of the third-party material found regarding GPLv3.  Thus,
the following chapter begins a deep discussion of GPLv2.

%%%%%%%%%%%%%%%%%%%%%%%%%%%%%%%%%%%%%%%%%%%%%%%%%%%%%%%%%%%%%%%%%%%%%%%%%%%%%%%
\chapter{GPLv2: Running Software and Verbatim Copying}
\label{run-and-verbatim}


This chapter begins the deep discussion of the details of the terms of
GPLv2\@. In this chapter, we consider the first two sections: GPLv2 \S\S
0--2. These are the straightforward sections of the GPL that define the
simplest rights that the user receives.

\section{GPLv2~\S0: Freedom to Run}
\label{GPLv2s0}

GPLv2~\S0, the opening section of GPLv2, sets forth that the copyright law governs
the work.  It specifically points out that it is the ``copyright
holder'' who decides if a work is licensed under its terms and explains
how the copyright holder might indicate this fact.

A bit more subtly, GPLv2~\S0 makes an inference that copyright law is the only
system that can restrict the software.  Specifically, it states:
\begin{quote}
Activities other than copying, distribution and modification are not
covered by this License; they are outside its scope.
\end{quote}
In essence, the license governs \emph{only} those activities, and all other
activities are unrestricted, provided that no other agreements trump GPLv2
(which they cannot; see Sections~\ref{GPLv2s6} and~\ref{GPLv2s7}).  This is
very important, because the Free Software community heavily supports
users' rights to ``fair use'' and ``unregulated use'' of copyrighted
material.  GPLv2 asserts through this clause that it supports users' rights
to fair and unregulated uses.

Fair use (called ``fair dealing'' in some jurisdictions) of copyrighted
material is an established legal doctrine that permits certain activities
regardless of whether copyright law would other restrict those activities.
Discussion of the various types of fair use activity are beyond the scope of
this tutorial.  However, one important example of fair use is the right to
quote portions of the text in larger work so as to criticize or suggest
changes.  This fair use rights is commonly used on mailing lists when
discussing potential improvements or changes to Free Software.

Fair use is a doctrine established by the courts or by statute.  By
contrast, unregulated uses are those that are not covered by the statue
nor determined by a court to be covered, but are common and enjoyed by
many users.  An example of unregulated use is reading a printout of the
program's source code like an instruction book for the purpose of learning
how to be a better programmer.  The right to read something that you have
access is and should remain unregulated and unrestricted.

\medskip

Thus, the GPLv2 protects users fair and unregulated use rights precisely by
not attempting to cover them.  Furthermore, the GPLv2 ensures the freedom
to run specifically by stating the following:
\begin{quote}
''The act of running the Program is not restricted.''
\end{quote}
Thus, users are explicitly given the freedom to run by GPLv2~\S0.

\medskip

The bulk of GPLv2~\S0 not yet discussed gives definitions for other terms used
throughout.  The only one worth discussing in detail is ``work based on
the Program''.  The reason this definition is particularly interesting is
not for the definition itself, which is rather straightforward, but
because it clears up a common misconception about the GPL\@.

The GPL is often mistakenly criticized because it fails to give a
definition of ``derivative work''.  In fact, it would be incorrect and
problematic if the GPL attempted to define this.  A copyright license, in
fact, has no control over what may or may not be a derivative work.  This
matter is left up to copyright law and the courts --- not the licenses that utilize it.

It is certainly true that copyright law as a whole does not propose clear
and straightforward guidelines for what is and is not a derivative
software work under copyright law.  However, no copyright license --- not
even the GNU GPL --- can be blamed for this.  Legislators and court
opinions must give us guidance to decide the border cases.

\section{GPLv2~\S1: Verbatim Copying}
\label{GPLv2s1}

GPLv2~\S1 covers the matter of redistributing the source code of a program
exactly as it was received. This section is quite straightforward.
However, there are a few details worth noting here.

The phrase ``in any medium'' is important.  This, for example, gives the
freedom to publish a book that is the printed copy of the program's source
code.  It also allows for changes in the medium of distribution.  Some
vendors may ship Free Software on a CD, but others may place it right on
the hard drive of a pre-installed computer.  Any such redistribution media
is allowed.

Preservation of copyright notice and license notifications are mentioned
specifically in GPLv2~\S1.  These are in some ways the most important part of
the redistribution, which is why they are mentioned by name.  GPL
always strives to make it abundantly clear to anyone who receives the
software what its license is.  The goal is to make sure users know their
rights and freedoms under GPL, and to leave no reason that users might be
surprised the software is GPL'd. Thus
throughout the GPL, there are specific references to the importance of
notifying others down the distribution chain that they have rights under
GPL.

Also mentioned by name is the warranty disclaimer. Most people today do
not believe that software comes with any warranty.  Notwithstanding the
\href{http://mlis.state.md.us/2000rs/billfile/hb0019.htm}{Maryland's} and \href{http://leg1.state.va.us/cgi-bin/legp504.exe?001+ful+SB372ER}{Virginia's} UCITA bills, there are few or no implied warranties with software.
However, just to be on the safe side, GPL clearly disclaims them, and the
GPL requires redistributors to keep the disclaimer very visible. (See
Sections~\ref{GPLv2s11} and~\ref{GPLv2s12} of this tutorial for more on GPL's
warranty disclaimers.)

Note finally that GPLv2~\S1 creates groundwork for the important defense of
commercial freedom.  GPLv2~\S1 clearly states that in the case of verbatim
copies, one may make money.  Redistributors are fully permitted to charge
for the redistribution of copies of Free Software. In addition, they may
provide the warranty protection that the GPL disclaims as an additional
service for a fee. (See Section~\ref{Business Models} for more discussion
on making a profit from Free Software redistribution.)

%%%%%%%%%%%%%%%%%%%%%%%%%%%%%%%%%%%%%%%%%%%%%%%%%%%%%%%%%%%%%%%%%%%%%%%%%%%%%%%

\chapter{Derivative Works: Statute and Case Law}

We digress for this chapter from our discussion of GPL's exact text to
consider the matter of derivative works --- a concept that we must
understand fully before considering GPLv2~\S\S2--3\@. GPL, and Free
Software licensing in general, relies critically on the concept of
``derivative work'' since software that is ``independent,'' (i.e., not
``derivative'') of Free Software need not abide by the terms of the
applicable Free Software license. As much is required by \S~106 of the
Copyright Act, 17 U.S.C. \S~106 (2002), and admitted by Free Software
licenses, such as the GPL, which (as we have seen) states in GPLv2~\S0 that ``a
`work based on the Program' means either the Program or any derivative
work under copyright law.'' It is being a derivative work of Free Software
that triggers the necessity to comply with the terms of the Free Software
license under which the original work is distributed. Therefore, one is
left to ask, just what is a ``derivative work''? The answer to that
question differs depending on which court is being asked.

The analysis in this chapter sets forth the differing definitions of
derivative work by the circuit courts. The broadest and most
established definition of derivative work for software is the
abstraction, filtration, and comparison test (``the AFC test'') as
created and developed by the Second Circuit. Some circuits, including
the Ninth Circuit and the First Circuit, have either adopted narrower
versions of the AFC test or have expressly rejected the AFC test in
favor of a narrower standard. Further, several other circuits have yet
to adopt any definition of derivative work for software.

As an introductory matter, it is important to note that literal copying of
a significant portion of source code is not always sufficient to establish
that a second work is a derivative work of an original
program. Conversely, a second work can be a derivative work of an original
program even though absolutely no copying of the literal source code of
the original program has been made. This is the case because copyright
protection does not always extend to all portions of a program's code,
while, at the same time, it can extend beyond the literal code of a
program to its non-literal aspects, such as its architecture, structure,
sequence, organization, operational modules, and computer-user interface.

\section{The Copyright Act}

The copyright act is of little, if any, help in determining the definition
of a derivative work of software. However, the applicable provisions do
provide some, albeit quite cursory, guidance. Section 101 of the Copyright
Act sets forth the following definitions:

\begin{quotation}
A ``computer program'' is a set of statements or instructions to be used
directly or indirectly in a computer in order to bring about a certain
result.

A ``derivative work'' is a work based upon one or more preexisting works,
such as a translation, musical arrangement, dramatization,
fictionalization, motion picture version, sound recording, art
reproduction, abridgment, condensation, or any other form in which a work
may be recast, transformed, or adapted. A work consisting of editorial
revisions, annotations, elaborations, or other modifications which, as a
whole, represent an original work of authorship, is a ``derivative work.''
\end{quotation}

These are the only provisions in the Copyright Act relevant to the
determination of what constitutes a derivative work of a computer
program. Another provision of the Copyright Act that is also relevant to
the definition of derivative work is \S~102(b), which reads as follows:

\begin{quotation}
In no case does copyright protection for an original work of authorship
extend to any idea, procedure, process, system, method of operation,
concept, principle, or discovery, regardless of the form in which it is
described, explained, illustrated, or embodied in such work.
\end{quotation}

Therefore, before a court can ask whether one program is a derivative work
of another program, it must be careful not to extend copyright protection
to any ideas, procedures, processes, systems, methods of operation,
concepts, principles, or discoveries contained in the original program. It
is the implementation of this requirement to ``strip out'' unprotectable
elements that serves as the most frequent issue over which courts
disagree.

\section{Abstraction, Filtration, Comparison Test}

As mentioned above, the AFC test for determining whether a computer
program is a derivative work of an earlier program was created by the
Second Circuit and has since been adopted in the Fifth, Tenth, and
Eleventh Circuits. Computer Associates Intl., Inc. v. Altai, Inc., 982
F.2d 693 (2nd Cir. 1992); Engineering Dynamics, Inc. v. Structural
Software, Inc., 26 F.3d 1335 (5th Cir. 1994); Kepner-Tregoe,
Inc. v. Leadership Software, Inc., 12 F.3d 527 (5th Cir. 1994); Gates
Rubber Co. v. Bando Chem. Indust., Ltd., 9 F.3d 823 (10th Cir. 1993);
Mitel, Inc. v. Iqtel, Inc., 124 F.3d 1366 (10th Cir. 1997); Bateman
v. Mnemonics, Inc., 79 F.3d 1532 (11th Cir. 1996); and, Mitek Holdings,
Inc. v. Arce Engineering Co., Inc., 89 F.3d 1548 (11th Cir. 1996).

Under the AFC test, a court first abstracts from the original program its
constituent structural parts. Then, the court filters from those
structural parts all unprotectable portions, including incorporated ideas,
expression that is necessarily incidental to those ideas, and elements
that are taken from the public domain. Finally, the court compares any and
all remaining kernels of creative expression to the structure of the
second program to determine whether the software programs at issue are
substantially similar so as to warrant a finding that one is the
derivative work of the other.

Often, the courts that apply the AFC test will perform a quick initial
comparison between the entirety of the two programs at issue in order to
help determine whether one is a derivative work of the other. Such a
holistic comparison, although not a substitute for the full application of
the AFC test, sometimes reveals a pattern of copying that is not otherwise
obvious from the application of the AFC test when, as discussed below,
only certain components of the original program are compared to the second
program. If such a pattern is revealed by the quick initial comparison,
the court is more likely to conclude that the second work is indeed a
derivative of the original.

\subsection{Abstraction}

The first step courts perform under the AFC test is separation of the
work's ideas from its expression. In a process akin to reverse
engineering, the courts dissect the original program to isolate each level
of abstraction contained within it. Courts have stated that the
abstractions step is particularly well suited for computer programs
because it breaks down software in a way that mirrors the way it is
typically created. However, the courts have also indicated that this step
of the AFC test requires substantial guidance from experts, because it is
extremely fact and situation specific.

By way of example, one set of abstraction levels is, in descending order
of generality, as follows: the main purpose, system architecture, abstract
data types, algorithms and data structures, source code, and object
code. As this set of abstraction levels shows, during the abstraction step
of the AFC test, the literal elements of the computer program, namely the
source and object code, are defined as particular levels of
abstraction. Further, the source and object code elements of a program are
not the only elements capable of forming the basis for a finding that a
second work is a derivative of the program. In some cases, in order to
avoid a lengthy factual inquiry by the court, the owner of the copyright in
the original work will submit its own list of what it believes to be the
protected elements of the original program. In those situations, the court
will forgo performing its own abstraction, and proceed to the second step of
the AFC test.

\subsection{Filtration}

The most difficult and controversial part of the AFC test is the second
step, which entails the filtration of protectable expression contained in
the original program from any unprotectable elements nestled therein. In
determining which elements of a program are unprotectable, courts employ a
myriad of rules and procedures to sift from a program all the portions
that are not eligible for copyright protection.

First, as set forth in \S~102(b) of the Copyright Act, any and all ideas
embodied in the program are to be denied copyright protection. However,
implementing this rule is not as easy as it first appears. The courts
readily recognize the intrinsic difficulty in distinguishing between ideas
and expression and that, given the varying nature of computer programs,
doing so will be done on an ad hoc basis. The first step of the AFC test,
the abstraction, exists precisely to assist in this endeavor by helping
the court separate out all the individual elements of the program so that
they can be independently analyzed for their expressive nature.

A second rule applied by the courts in performing the filtration step of
the AFC test is the doctrine of merger, which denies copyright protection
to expression necessarily incidental to the idea being expressed. The
reasoning behind this doctrine is that when there is only one way to
express an idea, the idea and the expression merge, meaning that the
expression cannot receive copyright protection due to the bar on copyright
protection extending to ideas. In applying this doctrine, a court will ask
whether the program's use of particular code or structure is necessary for
the efficient implementation of a certain function or process. If so, then
that particular code or structure is not protected by copyright and, as a
result, it is filtered away from the remaining protectable expression.

A third rule applied by the courts in performing the filtration step of
the AFC test is the doctrine of scenes a faire, which denies copyright
protection to elements of a computer program that are dictated by external
factors. Such external factors can include:

\begin{itemize}

  \item The mechanical
specifications of the computer on which a particular program is intended
to operate

  \item Compatibility requirements of other programs with which a
program is designed to operate in conjunction

  \item Computer manufacturers'
design standards

  \item Demands of the industry being serviced, and

widely accepted programming practices within the computer industry

\end{itemize}

Any code or structure of a program that was shaped predominantly in
response to these factors is filtered out and not protected by
copyright. Lastly, elements of a computer program are also to be filtered
out if they were taken from the public domain or fail to have sufficient
originality to merit copyright protection.

Portions of the source or object code of a computer program are rarely
filtered out as unprotectable elements. However, some distinct parts of
source and object code have been found unprotectable. For example,
constant s, the invariable integers comprising part of formulas used to
perform calculations in a program, are unprotectable. Further, although
common errors found in two programs can provide strong evidence of
copying, they are not afforded any copyright protection over and above the
protection given to the expression containing them.

\subsection{Comparison}

The third and final step of the AFC test entails a comparison of the
original program's remaining protectable expression to a second
program. The issue will be whether any of the protected expression is
copied in the second program and, if so, what relative importance the
copied portion has with respect to the original program overall. The
ultimate inquiry is whether there is ``substantial'' similarity between
the protected elements of the original program and the potentially
derivative work. The courts admit that this process is primarily
qualitative rather than quantitative and is performed on a case-by-case
basis. In essence, the comparison is an ad hoc determination of whether
the protectable elements of the original program that are contained in the
second work are significant or important parts of the original program. If
so, then the second work is a derivative work of the first. If, however,
the amount of protectable elements copied in the second work are so small
as to be de minimis, then the second work is not a derivative work of the
original.

\section{Analytic Dissection Test}

The Ninth Circuit has adopted the analytic dissection test to determine
whether one program is a derivative work of another. Apple Computer,
Inc. v. Microsoft Corp., 35 F.3d 1435 (9th Cir. 1994). The analytic
dissection test first considers whether there are substantial similarities
in both the ideas and expressions of the two works at issue. Once the
similar features are identified, analytic dissection is used to determine
whether any of those similar features are protected by copyright. This
step is the same as the filtration step in the AFC test. After identifying
the copyrightable similar features of the works, the court then decides
whether those features are entitled to ``broad'' or ``thin''
protection. ``Thin'' protection is given to non-copyrightable facts or
ideas that are combined in a way that affords copyright protection only
from their alignment and presentation, while ``broad'' protection is given
to copyrightable expression itself. Depending on the degree of protection
afforded, the court then sets the appropriate standard for a subjective
comparison of the works to determine whether, as a whole, they are
sufficiently similar to support a finding that one is a derivative work of
the other. ``Thin'' protection requires the second work be virtually
identical in order to be held a derivative work of an original, while
``broad'' protection requires only a ``substantial similarity.''

\section{No Protection for ``Methods of Operation''}

The First Circuit has taken the position that the AFC test is inapplicable 
when the works in question relate to unprotectable elements set forth in 
\S~102(b).  Their approach results in a much narrower definition
of derivative work for software in comparison to other circuits. Specifically, 
the
First Circuit holds that ``method of operation,'' as used in \S~102(b) of
the Copyright Act, refers to the means by which users operate
computers. Lotus Development Corp. v. Borland Int’l., Inc., 49 F.3d 807
(1st Cir. 1995).  In Lotus, the court held that a menu command
hierarchy for a computer program was uncopyrightable because it did not
merely explain and present the program’s functional capabilities to the
user, but also served as a method by which the program was operated and
controlled. As a result, under the First Circuit’s test, literal copying
of a menu command hierarchy, or any other ``method of operation,'' cannot
form the basis for a determination that one work is a derivative of
another.  As a result, courts in the First Circuit that apply the AFC test
do so only after applying a broad interpretation of \S~102(b) to filter out
unprotected elements. E.g., Real View, LLC v. 20-20 Technologies, Inc., 
683 F. Supp.2d 147, 154 (D. Mass. 2010).


\section{No Test Yet Adopted}

Several circuits, most notably the Fourth and Seventh, have yet to
declare their definition of derivative work and whether or not the
AFC, Analytic Dissection, or some other test best fits their
interpretation of copyright law. Therefore, uncertainty exists with
respect to determining the extent to which a software program is a
derivative work of another in those circuits. However, one may presume
that they would give deference to the AFC test since it is by far the
majority rule amongst those circuits that have a standard for defining
a software derivative work.

\section{Cases Applying Software Derivative Work Analysis}

In the preeminent case regarding the definition of a derivative work for
software, Computer Associates v. Altai, the plaintiff alleged that its
program, Adapter, which was used to handle the differences in operating
system calls and services, was infringed by the defendant's competitive
program, Oscar. About 30\% of Oscar was literally the same code as
that in Adapter. After the suit began, the defendant rewrote those
portions of Oscar that contained Adapter code in order to produce a new
version of Oscar that was functionally competitive with Adapter, without
have any literal copies of its code. Feeling slighted still, the
plaintiff alleged that even the second version of Oscar, despite having no
literally copied code, also infringed its copyrights. In addressing that
question, the Second Circuit promulgated the AFC test.

In abstracting the various levels of the program, the court noted a
similarity between the two programs' parameter lists and macros. However,
following the filtration step of the AFC test, only a handful of the lists
and macros were protectable under copyright law because they were either
in the public domain or required by functional demands on the
program. With respect to the handful of parameter lists and macros that
did qualify for copyright protection, after performing the comparison step
of the AFC test, it was reasonable for the district court to conclude that
they did not warrant a finding of infringement given their relatively minor
contribution to the program as a whole. Likewise, the similarity between
the organizational charts of the two programs was not substantial enough
to support a finding of infringement because they were too simple and
obvious to contain any original expression.

In the case of Oracle America v. Google, 872 F. Supp.2d 974 (N.D. Cal. 2012),
the Northern District of California District Court examined the question of 
whether the application program interfaces (APIs) associated with the Java
programming language are entitled to copyright protection.  While the 
court expressly declined to rule whether all APIs are free to use without 
license (872 F. Supp.2d 974 at 1002), the court held that the command 
structure and taxonomy of the APIs were not protectable under copyright law.
Specifically, the court characterized the command structure and taxonomy as
both a ``method of operation'' (using an approach not dissimilar to the 
First Circuit's analysis in Lotus) and a ``functional requirement for 
compatability'' (using Sega v. Accolade, 977 F.2d 1510 (9th Cir. 1992) and
Sony Computer Ent. v. Connectix, 203 F.3d 596 (9th Cir. 2000) as analogies),
and thus unprotectable subject matter under \S~102(b). 

Perhaps not surprisingly, there have been few other cases involving a highly
detailed software derivative work analysis. Most often, cases involve
clearer basis for decision, including frequent bad faith on the part of
the defendant or overaggressiveness on the part of the plaintiff.  

%%%%%%%%%%%%%%%%%%%%%%%%%%%%%%%%%%%%%%%%%%%%%%%%%%%%%%%%%%%%%%%%%%%%%%%%%%%%%%%

\chapter{Modified Source and Binary Distribution}
\label{source-and-binary}

In this chapter, we discuss the two core sections that define the rights
and obligations for those who modify, improve, and/or redistribute GPL'd
software. These sections, GPLv2~\S\S2--3, define the central core rights and
requirements of GPLv2\@.

\section{GPLv2~\S2: Share and Share Alike}

For many, this is where the ``magic'' happens that defends software
freedom upon redistribution.  GPLv2~\S2 is the only place in GPLv2
that governs the modification controls of copyright law.  If users
modifies a GPLv2'd program, they must follow the terms of GPLv2~\S2 in making
those changes.  Thus, this sections ensures that the body of GPL'd software, as it
continues and develops, remains Free as in freedom.

To achieve that goal, GPLv2~\S2 first sets forth that the rights of
redistribution of modified versions are the same as those for verbatim
copying, as presented in GPLv2~\S1.  Therefore, the details of charging money,
keeping copyright notices intact, and other GPLv2~\S1 provisions are in tact
here as well.  However, there are three additional requirements.

The first (GPLv2~\S2(a)) requires that modified files carry ``prominent
notices'' explaining what changes were made and the date of such
changes. This section does not prescribe some specific way of
marking changes nor does it control the process of how changes are made.
Primarily, GPLv2~\S2(a) seeks to ensure that those receiving modified
versions know the history of changes to the software.  For some users,
it is important to know that they are using the standard version of
program, because while there are many advantages to using a fork,
there are a few disadvantages.  Users should be informed about the
historical context of the software version they use, so that they can
make proper support choices.  Finally, GPLv2~\S2(a) serves an academic
purpose --- ensuring that future developers can use a diachronic
approach to understand the software.

\medskip

The second requirement (GPLv2~\S2(b)) contains the four short lines that embody
the legal details of ``share and share alike''.  These 46 words are
considered by some to be the most worthy of careful scrutiny because
GPLv2~\S2(b), and they
can be a source of great confusion when not properly understood.

In considering GPLv2~\S2(b), first note the qualifier: it \textit{only} applies to
derivative works that ``you distribute or publish''.  Despite years of
education efforts on this matter, many still believe that modifiers
of GPL'd software \textit{must} to publish or otherwise
share their changes.  On the contrary, GPLv2~\S2(b) {\bf does not apply if} the
changes are never distributed.  Indeed, the freedom to make private,
personal, unshared changes to software for personal use only should be
protected and defended.\footnote{Most Free Software enthusiasts believe there is an {\bf
    moral} obligation to redistribute changes that are generally useful,
  and they often encourage companies and individuals to do so.  However, there
  is a clear distinction between what one {\bf ought} to do and what one
  {\bf must} do.}

Next, we again encounter the same matter that appears in GPLv2~\S0, in the
following text:
\begin{quote}
``...that in whole or part contains or is derived from the Program or any part thereof.''
\end{quote}
Again, the GPL relies here on what the copyright law says is a derivative
work.  If, under copyright law, the modified version ``contains or is
derived from'' the GPL'd software, then the requirements of GPLv2~\S2(b)
apply.  The GPL invokes its control as a copyright license over the
modification of the work in combination with its control over distribution
of the work.

The final clause of GPLv2~\S2(b) describes what the licensee must do if she is
distributing or publishing a work that is deemed a derivative work under
copyright law --- namely, the following:
\begin{quote}
[The work must] be licensed as a whole at no charge to all third parties
under the terms of this License.
\end{quote}
That is probably the most tightly-packed phrase in all of the GPL\@.
Consider each subpart carefully.

The work ``as a whole'' is what is to be licensed. This is an important
point that GPLv2~\S2 spends an entire paragraph explaining; thus this phrase is
worthy of a lengthy discussion here.  As a programmer modifies a software
program, she generates new copyrighted material --- fixing expressions of
ideas into the tangible medium of electronic file storage.  That
programmer is indeed the copyright holder of those new changes.  However,
those changes are part and parcel to the original work distributed to
the programmer under GPL\@. Thus, the license of the original work
affects the license of the new whole derivative work.

% {\cal I}
\newcommand{\gplusi}{$\mathcal{G\!\!+\!\!I}$}
\newcommand{\worki}{$\mathcal{I}$}
\newcommand{\workg}{$\mathcal{G}$}

\label{separate-and-independent}

It is certainly possible to take an existing independent work (called
\worki{}) and combine it with a GPL'd program (called \workg{}).  The
license of \worki{}, when it is distributed as a separate and independent
work, remains the prerogative of the copyright holder of \worki{}.
However, when \worki{} is combined with \workg{}, it produces a new work
that is the combination of the two (called \gplusi{}). The copyright of
this combined work, \gplusi{}, is held by the original copyright
holder of each of the two works.

In this case, GPLv2~\S2 lays out the terms by which \gplusi{} may be
distributed and copied.  By default, under copyright law, the copyright
holder of \worki{} would not have been permitted to distribute \gplusi{};
copyright law forbids it without the expressed permission of the copyright
holder of \workg{}. (Imagine, for a moment, if \workg{} were a proprietary
product --- would its copyright holders  give you permission to create and distribute
\gplusi{} without paying them a hefty sum?)  The license of \workg{}, the
GPL, states the  options for the copyright holder of \worki{}
who may want to create and distribute \gplusi{}.  GPL's pregranted
permission to create and distribute derivative works, provided the terms
of GPL are upheld, goes far above and beyond the permissions that one
would get with a typical work not covered by a copyleft license.  (Thus, to
say that this restriction is any way unreasonable is simply ludicrous.)

\medskip

The next phrase of note in GPLv2~\S2(b) is ``licensed \ldots at no charge.''
This phrase  confuses many.  The sloppy reader points out this as ``a
contradiction in GPL'' because (in their confused view) that clause of GPLv2~\S2 says that redistributors cannot
charge for modified versions of GPL'd software, but GPLv2~\S1 says that
they can.  Avoid this confusion: the ``at no charge'' \textbf{does not} prohibit redistributors from
charging when performing the acts governed by copyright
law,\footnote{Recall that you could by default charge for any acts not
governed by copyright law, because the license controls are confined
by copyright.} but rather that they cannot charge a fee for the
\emph{license itself}.  In other words, redistributors of (modified
and unmodified) GPL'd works may charge any amount they choose for
performing the modifications on contract or the act of transferring
the copy to the customer, but they may not charge a separate licensing
fee for the software.

GPLv2~\S2(b) further states that the software must ``be licensed \ldots to all
third parties.''  This too yields some confusion, and feeds the
misconception mentioned earlier --- that all modified versions must made
available to the public at large.  However, the text here does not say
that.  Instead, it says that the licensing under terms of the GPL must
extend to anyone who might, through the distribution chain, receive a copy
of the software.  Distribution to all third parties is not mandated here,
but GPLv2~\S2(b) does require redistributors to license the derivative works in
a way that extends to all third parties who may ultimately receive a
copy of the software.

In summary, GPLv2\ 2(b) says what terms under which the third parties must
receive this no-charge license.  Namely, they receive it ``under the terms
of this License'', the GPLv2.  When an entity \emph{chooses} to redistribute
a derivative work of GPL'd software, the license of that whole 
work must be GPL and only GPL\@.  In this manner, GPLv2~\S2(b) dovetails nicely
with GPLv2~\S6 (as discussed in Section~\ref{GPLv2s6} of this tutorial).

\medskip

The final paragraph of GPLv2~\S2 is worth special mention.  It is possible and
quite common to aggregate various software programs together on one
distribution medium.  Computer manufacturers do this when they ship a
pre-installed hard drive, and GNU/Linux distribution vendors do this to
give a one-stop CD or URL for a complete operating system with necessary
applications.  The GPL very clearly permits such ``mere aggregation'' with
programs under any license.  Despite what you hear from its critics, the
GPL is nothing like a virus, not only because the GPL is good for you and
a virus is bad for you, but also because simple contact with a GPL'd
code-base does not impact the license of other programs.  A programmer must
expended actual effort  to cause a work to fall under the terms
of the GPL.  Redistributors are always welcome to simply ship GPL'd
software alongside proprietary software or other unrelated Free Software,
as long as the terms of GPL are adhered to for those packages that are
truly GPL'd.

\section{GPLv2~\S3: Producing Binaries}
\label{GPL-Section-3}

Software is a strange beast when compared to other copyrightable works.
It is currently impossible to make a film or a book that can be truly
obscured.  Ultimately, the full text of a novel, even one written by
William Faulkner, must presented to the reader as words in some
human-readable language so that they can enjoy the work.  A film, even one
directed by David Lynch, must be perceptible by human eyes and ears to
have any value.

Software is not so.  While the source code --- the human-readable
representation of software is of keen interest to programmers -- users and
programmers alike cannot make the proper use of software in that
human-readable form.  Binary code --- the ones and zeros that the computer
can understand --- must be predicable and attainable for the software to
be fully useful.  Without the binaries, be they in object or executable
form, the software serves only the didactic purposes of computer science.

Under copyright law, binary representations of the software are simply
derivative works of the source code.  Applying a systematic process (i.e.,
``compilation''\footnote{``Compilation'' in this context refers to the
  automated computing process of converting source code into binaries.  It
  has absolutely nothing to do with the term ``compilation'' in copyright statues.}) to a work of source code yields binary code. The binary
code is now a new work of expression fixed in the tangible medium of
electronic file storage.

Therefore, for GPL'd software to be useful, the GPL, since it governs the
rules for creation of derivative works, must grant permission for the
generation of binaries.  Furthermore, notwithstanding the relative
popularity of source-based GNU/Linux distributions like Gentoo, users find
it extremely convenient to receive distribution of binary software.  Such
distribution is the redistribution of derivative works of the software's
source code.  GPLv2~\S3 addresses the matter of creation and distribution of
binary versions.

Under GPLv2~\S3, binary versions may be created and distributed under the
terms of GPLv2~\S1--2, so all the material previously discussed applies
here.  However, GPLv2~\S3 must go a bit further.  Access to the software's
source code is an incontestable prerequisite for the exercise of the
fundamental freedoms to modify and improve the software.  Making even
the most trivial changes to a software program at the binary level is
effectively impossible.  GPLv2~\S3 must ensure that the binaries are never
distributed without the source code, so that these freedoms are passed
through the distribution chain.

GPLv2~\S3 permits distribution of binaries, and then offers three options for
distribution of source code along with binaries. The most common and the
least complicated is the option given under GPLv2~\S3(a).

GPLv2~\S3(a) offers the option to directly accompany the source code alongside
the distribution of the binaries.  This is by far the most convenient
option for most distributors, because it means that the source-code
provision obligations are fully completed at the time of binary
distribution (more on that later).

Under GPLv2~\S3(a), the source code provided must be the ``corresponding source
code.''  Here ``corresponding'' primarily means that the source code
provided must be that code used to produce the binaries being distributed.
That source code must also be ``complete''.   GPLv2~\S3's penultimate paragraph
explains in detail what is meant by ``complete''.  In essence, it is all
the material that a programmer of average skill would need to actually use
the source code to produce the binaries she has received.  Complete source
is required so that, if the licensee chooses, she should be able to
exercise her freedoms to modify and redistribute changes.  Without the
complete source, it would not be possible to make changes that were
actually directly derived from the version received.

Furthermore, GPLv2~\S3 is defending against a tactic that has in fact been
seen in GPL enforcement.  Under GPL, if you pay a high price for
a copy of GPL'd binaries (which comes with corresponding source, of
course), you have the freedom to redistribute that work at any fee you
choose, or not at all.  Sometimes, companies attempt a GPL-violating
cozenage whereby they produce very specialized binaries (perhaps for
an obscure architecture).  They then give source code that does
correspond, but withhold the ``incantations'' and build plans they
used to make that source compile into the specialized binaries.
Therefore, GPLv2~\S3 requires that the source code include ``meta-material'' like
scripts, interface definitions, and other material that is used to
``control compilation and installation'' of the binaries.  In this
manner, those further down the distribution chain are assured that
they have the unabated freedom to build their own derivative works
from the sources provided.

Software distribution comes in many
forms.  Embedded manufacturers, for example, have the freedom to put
GPL'd software into mobile devices with very tight memory and space
constraints.  In such cases, putting the source right alongside the
binaries on the machine itself might not be an option.  While it is
recommended that this be the default way that people comply with GPL, the
GPL does provide options when such distribution is infeasible.

GPLv2~\S3, therefore, allows source code to be provided on any physical
``medium customarily used for software interchange.''  By design, this
phrase covers a broad spectrum --- the phrase seeks to pre-adapt to
changes in  technology.  When GPLv22 was first published in June
1991, distribution on magnetic tape was still common, and CD was
relatively new.  By 2002, CD is the default.  By 2007, DVD's were the
default.  Now, it's common to give software on USB drives and SD card.  This
language in the license must adapt with changing technology.

Meanwhile, the binding created by the word ``customarily'' is key.  Many
incorrectly believe that distributing binary on CD and source on the
Internet is acceptable.  In the corporate world in industrialized countries, it is indeed customary to
simply download a CDs' worth of data quickly.  However, even today in the USA, many computer users are not connected to the Internet, and most people connected
to the Internet still have limited download speeds.  Downloading
CDs full of data is not customary for them in the least.  In some cities
in Africa, computers are becoming more common, but Internet connectivity
is still available only at a few centralized locations.  Thus, the
``customs'' here are normalized for a worldwide userbase.  Simply
providing source on the Internet --- while it is a kind, friendly and
useful thing to do --- is not usually sufficient.

Note, however, a major exception to this rule, given by the last paragraph
of GPLv2~\S3. \emph{If} distribution of the binary files is made only on the
Internet (i.e., ``from a designated place''), \emph{then} simply providing
the source code right alongside the binaries in the same place is
sufficient to comply with GPLv2~\S3.

\medskip

As is shown above, Under GPLv2~\S3(a), embedded manufacturers can put the
binaries on the device and ship the source code along on a CD\@.  However,
sometimes this turns out to be too costly.  Including a CD with every
device could prove too costly, and may practically (although not legally)
prohibit using GPL'd software. For this situation and others like it, GPlv2\S~3(b) is available.

GPLv2~\S3(b) allows a distributor of binaries to instead provide a written
offer for source code alongside those binaries.  This is useful in two
specific ways.  First, it may turn out that most users do not request the
source, and thus the cost of producing the CDs is saved --- a financial
and environmental windfall.  In addition, along with a GPLv2~\S3(b) compliant
offer for source, a binary distributor might choose to \emph{also} give a
URL for source code.  Many who would otherwise need a CD with source might
turn out to have those coveted high bandwidth connections, and are able to
download the source instead --- again yielding environmental and financial
windfalls.

However, note that regardless of how many users prefer to get the
source online, GPLv2~\S3(b) does place lasting long-term obligations on the
binary distributor.  The binary distributor must be prepared to honor
that offer for source for three years and ship it out (just as they
would have had to do under GPLv2~\S3(a)) at a moment's notice when they
receive such a request.  There is real organizational cost here:
support engineers must be trained how to route source requests, and
source CD images for every release version for the last three years
must be kept on hand to burn such CDs quickly. The requests might not
even come from actual customers; the offer for source must be valid
for ``any third party''.

That phrase is another place where some get confused --- thinking again
that full public distribution of source is required.  The offer for source
must be valid for ``any third party'' because of the freedoms of
redistribution granted by GPLv2~\S\S1--2.  A company may ship a binary image
and an offer for source to only one customer.  However, under GPL, that
customer has the right to redistribute that software to the world if she
likes.  When she does, that customer has an obligation to make sure that
those who receive the software from her can exercise their freedoms under
GPL --- including the freedom to modify, rebuild, and redistribute the
source code.

GPLv2~\S3(c) is created to save her some trouble, because by itself GPLv2~\S3(b)
would unfairly favor large companies.  GPLv2~\S3(b) allows the
separation of the binary software from the key tool that people can use
to exercise their freedom. The GPL permits this separation because it is
good for redistributors, and those users who turn out not to need the
source.  However, to ensure equal rights for all software users, anyone
along the distribution chain must have the right to get the source and
exercise those freedoms that require it.

Meanwhile, GPLv2~\S3(b)'s compromise primarily benefits companies who
distribute binary software commercially.  Without GPLv2~\S3(c), that benefit
would be at the detriment of the companies' customers; the burden of
source code provision would be unfairly shifted to the companies'
customers.  A customer, who had received binaries with a GPLv2~\S3(b)-compliant
offer, would be required under GPLv2 (sans GPLv2~\S3(c)) to acquire the source,
merely to give a copy of the software to a friend who needed it.  GPLv2~\S3(c)
reshifts this burden to entity who benefits from GPLv2~\S3(b).

GPLv2~\S3(c) allows those who undertake \emph{noncommercial} distribution to
simply pass along a GPLv2~\S3(b)-compliant source code offer.  The customer who
wishes to give a copy to her friend can now do so without provisioning the
source, as long as she gives that offer to her friend.  By contrast, if
she wanted to go into business for herself selling CDs of that software,
she would have to acquire the source and either comply via GPLv2~\S3(a), or
write her own GPLv2~\S3(b)-compliant source offer.

This process is precisely the reason why a GPLv2~\S3(b) source offer must be
valid for all third parties.  At the time the offer is made, there is no
way of knowing who might end up noncommercially receiving a copy of the
software.  Companies who choose to comply via GPLv2~\S3(b) must thus be
prepared to honor all incoming source code requests.  For this and the
many other additional necessary complications under GPLv2~\S\S3(b--c), it is
only rarely a better option than complying via GPLv2~\S3(a).

%%%%%%%%%%%%%%%%%%%%%%%%%%%%%%%%%%%%%%%%%%%%%%%%%%%%%%%%%%%%%%%%%%%%%%%%%%%%%%%
\chapter{GPL's Implied Patent Grant}
\label{gpl-implied-patent-grant}

We digress again briefly from our section-by-section consideration of GPLv2
to consider the interaction between the terms of GPL and patent law. The
GPLv2, despite being silent with respect to patents, actually confers on its
licensees more rights to a licensor's patents than those licenses that
purport to address the issue. This is the case because patent law, under
the doctrine of implied license, gives to each distributee of a patented
article a license from the distributor to practice any patent claims owned
or held by the distributor that cover the distributed article. The
implied license also extends to any patent claims owned or held by the
distributor that cover ``reasonably contemplated uses'' of the patented
article. To quote the Federal Circuit Court of Appeals, the highest court
for patent cases other than the Supreme Court:

\begin{quotation}
Generally, when a seller sells a product without restriction, it in
effect promises the purchaser that in exchange for the price paid, it will
not interfere with the purchaser's full enjoyment of the product
purchased. The buyer has an implied license under any patents of the
seller that dominate the product or any uses of the product to which the
parties might reasonably contemplate the product will be put.
\end{quotation}
Hewlett-Packard Co. v. Repeat-O-Type Stencil Mfg. Corp., Inc., 123 F.3d
1445, 1451 (Fed. Cir. 1997).

Of course, Free Software is licensed, not sold, and there are indeed
restrictions placed on the licensee, but those differences are not likely
to prevent the application of the implied license doctrine to Free
Software, because software licensed under the GPL grants the licensee the
right to make, use, and sell the software, each of which are exclusive
rights of a patent holder. Therefore, although the GPLv2 does not expressly
grant the licensee the right to do those things under any patents the
licensor may have that cover the software or its reasonably contemplated
uses, by licensing the software under the GPLv2, the distributor impliedly
licenses those patents to the GPLv2 licensee with respect to the GPLv2'd
software.

An interesting issue regarding this implied patent license of GPLv2'd
software is what would be considered ``uses of the [software] to which
the parties might reasonably contemplate the product will be put.'' A
clever advocate may argue that the implied license granted by GPLv2 is
larger in scope than the express license in other Free Software
licenses with express patent grants, in that the patent license
clause of many of those other Free  Software licenses are specifically 
limited to the patent claims covered by the code as licensed by the patentee.

In contrast, a GPLv2 licensee, under the doctrine of implied patent license, 
is free to practice any patent claims held by the licensor that cover 
``reasonably contemplated uses'' of the GPL'd code, which may very well 
include creation and distribution of derivative works since the GPL's terms, 
under which the patented code is distributed, expressly permits such activity.


Further supporting this result is the Federal Circuit's pronouncement that
the recipient of a patented article has, not only an implied license to
make, use, and sell the article, but also an implied patent license to
repair the article to enable it to function properly, Bottom Line Mgmt.,
Inc. v. Pan Man, Inc., 228 F.3d 1352 (Fed. Cir. 2000). Additionally, the
Federal Circuit extended that rule to include any future recipients of the
patented article, not just the direct recipient from the distributor.
This theory comports well with the idea of Free Software, whereby software
is distributed amongst many entities within the community for the purpose
of constant evolution and improvement. In this way, the law of implied
patent license used by the GPLv2 ensures that the community mutually
benefits from the licensing of patents to any single community member.



Note that simply because GPLv2'd software has an implied patent license does
not mean that any patents held by a distributor of GPLv2'd code become
worthless. To the contrary, the patents are still valid and enforceable
against either:

\begin{enumerate}
 \renewcommand{\theenumi}{\alph{enumi}}
 \renewcommand{\labelenumi}{\textup{(\theenumi)}}

\item any software other than that licensed under the GPLv2 by the patent
  holder, and

\item any party that does not comply with the GPLv2
with respect to the licensed software.
\end{enumerate}

\newcommand{\compB}{$\mathcal{B}$}
\newcommand{\compA}{$\mathcal{A}$}

For example, if Company \compA{} has a patent on advanced Web browsing, but
also licenses a Web browsing software program under the GPLv2, then it
cannot assert the patent against any party based on that party's use of 
Company \compA{}'s GPL'ed Web browsing software program, or on that party's
creation and use of derivative works of that GPL'ed program.  However, if a
party uses that program without
complying with the GPLv2, then Company \compA{} can assert both copyright
infringement claims against the non-GPLv2-compliant party and
infringement of the patent, because the implied patent license only
extends to use of the software in accordance with the GPLv2. Further, if
Company \compB{} distributes a competitive advanced Web browsing program 
that is not a derivative work of Company \compA{}'s GPL'ed Web browsing software
program, Company \compA{} is free to assert its patent against any user or
distributor of that product. It is irrelevant whether Company \compB's
program is also distributed under the GPLv2, as Company \compB{} can not grant
implied licenses to Company \compA's patent.

This result also reassures companies that they need not fear losing their
proprietary value in patents to competitors through the GPLv2 implied patent
license, as only those competitors who adopt and comply with the GPLv2's
terms can benefit from the implied patent license. To continue the
example above, Company \compB{} does not receive a free ride on Company
\compA's patent, as Company \compB{} has not licensed-in and then
redistributed Company A's advanced Web browser under the GPLv2. If Company
\compB{} does do that, however, Company \compA{} still has not lost
competitive advantage against Company \compB{}, as Company \compB{} must then,
when it re-distributes Company \compA's program, grant an implied license
to any of its patents that cover the program. Further, if Company \compB{}
relicenses an improved version of Company A's program, it must do so under
the GPLv2, meaning that any patents it holds that cover the improved version
are impliedly licensed to any licensee. As such, the only way Company
\compB{} can benefit from Company \compA's implied patent license, is if it,
itself, distributes Company \compA's software program and grants an
implied patent license to any of its patents that cover that program.

%%%%%%%%%%%%%%%%%%%%%%%%%%%%%%%%%%%%%%%%%%%%%%%%%%%%%%%%%%%%%%%%%%%%%%%%%%%%%%%
\chapter{Defending Freedom on Many Fronts}

Chapters~\ref{run-and-verbatim} and~\ref{source-and-binary} presented the
core freedom-defending provisions of GPLv2\@, which are in GPLv2~\S\S0--3.
GPLv2\S\S~4--7 of the GPLv2 are designed to ensure that GPLv2~\S\S0--3 are
not infringed, are enforceable, are kept to the confines of copyright law but
also  not trumped by other copyright agreements or components of other
entirely separate legal systems.  In short, while GPLv2~\S\S0--3 are the parts
of the license that defend the freedoms of users and programmers,
GPLv2~\S\S4--7 are the parts of the license that keep the playing field clear
so that \S\S~0--3 can do their jobs.

\section{GPLv2~\S4: Termination on Violation}
\label{GPLv2s4}

GPLv2~\S4 is GPLv2's termination clause.  Upon first examination, it seems
strange that a license with the goal of defending users' and programmers'
freedoms for perpetuity in an irrevocable way would have such a clause.
However, upon further examination, the difference between irrevocability
and this termination clause becomes clear.

The GPL is irrevocable in the sense that once a copyright holder grants
rights for someone to copy, modify and redistribute the software under terms
of the GPL, they cannot later revoke that grant.  Since the GPL has no
provision allowing the copyright holder to take such a prerogative, the
license is granted as long as the copyright remains in effect.\footnote{In
  the USA, due to unfortunate legislation, the length of copyright is nearly
  perpetual, even though the Constitution forbids perpetual copyright.} The
copyright holders have the right to relicense the same work under different
licenses (see Section~\ref{Proprietary Relicensing} of this tutorial), or to
stop distributing the GPLv2'd version (assuming GPLv2~\S3(b) was never used),
but they may not revoke the rights under GPLv2 already granted.

In fact, when an entity looses their right to copy, modify and distribute
GPL'd software, it is because of their \emph{own actions}, not that of the
copyright holder.  The copyright holder does not decided when GPLv2~\S4
termination occurs (if ever); rather, the actions of the licensee determine
that.

Under copyright law, the GPL has granted various rights and freedoms to
the licensee to perform specific types of copying, modification, and
redistribution.  By default, all other types of copying, modification, and
redistribution are prohibited.  GPLv2~\S4 says that if you undertake any of
those other types (e.g., redistributing binary-only in violation of GPLv2~\S3),
then all rights under the license --- even those otherwise permitted for
those who have not violated --- terminate automatically.

GPLv2~\S4 makes GPLv2 enforceable.  If licensees fail to adhere to the
license, then they are stuck without any permission under to engage in
activities covered by copyright law.  They must completely cease and desist
from all copying, modification and distribution of the GPL'd software.

At that point, violating licensees must gain the forgiveness of the copyright
holders to have their rights restored.  Alternatively, the violators could
negotiate another agreement, separate from GPL, with the copyright
holder.  Both are common practice, although
\tutorialpartsplit{as discussed in \textit{A Practical Guide to GPL
    Compliance}, there are }{Chapter~\ref{compliance-understanding-whos-enforcing}
  explains further } key differences between these two very different uses of GPL.

\section{GPLv2~\S5: Acceptance, Copyright Style}
\label{GPLv2s5}

GPLv2~\S5 brings us to perhaps the most fundamental misconception and common
confusion about GPLv2\@. Because of the prevalence of proprietary software,
most users, programmers, and lawyers alike tend to be more familiar with
EULAs. EULAs are believed by their authors to be contracts, requiring
formal agreement between the licensee and the software distributor to be
valid. This has led to mechanisms like ``shrink-wrap'' and ``click-wrap''
as mechanisms to perform acceptance ceremonies with EULAs.

The GPL does not need contract law to ``transfer rights.''  Usually, no rights
are transfered between parties.  By contrast, the GPL is primarily a permission
slip to undertake activities that would otherwise have been prohibited
by copyright law.  As such, GPL needs no acceptance ceremony; the
licensee is not even required to accept the license.

However, without the GPL, the activities of copying, modifying and
distributing the software would have otherwise been prohibited.  So, the
GPL says that you only accepted the license by undertaking activities that
you would have otherwise been prohibited without your license under GPL\@.
This is a certainly subtle point, and requires a mindset quite different
from the contractual approach taken by EULA authors.

An interesting side benefit to GPLv2~\S5 is that the bulk of users of Free
Software are not required to accept the license.  Undertaking fair and
unregulated use of the work, for example, does not bind you to the GPL,
since you are not engaging in activity that is otherwise controlled by
copyright law.  Only when you engage in those activities that might have an
impact on the freedom of others does license acceptance occur, and the
terms begin to bind you to fair and equitable sharing of the software.  In
other words, the GPL only kicks in when it needs to for the sake of
freedom.

While GPL is by default a copyright license, it is certainly still possible
to consider GPL as a contract as well.  For example, some distributors chose
to ``wrap'' their software in an acceptance ceremony to GPL, and nothing in
GPL prohibits that use.  Furthermore, the ruling in \textit{Jacobsen
  v. Katzer, 535 F.3d 1373, 1380 (Fed.Cir.2008)} indicates that \textbf{both}
copyright and contractual remedies may be sought by a copyright holder
seeking to enforce a license designed to uphold software freedom.

\section{Using GPL Both as a Contract and Copyright License}

\section{GPLv2~\S6: GPL, My One and Only}
\label{GPLv2s6}

A point that was glossed over in Section~\ref{GPLv2s4}'s discussion of GPLv2~\S4
was the irrevocable nature of the GPL\@. The GPLv2 is indeed irrevocable,
and it is made so formally by GPLv2~\S6.

The first sentence in GPLv2~\S6 ensures that as software propagates down the
distribution chain, that each licensor can pass along the license to each
new licensee.  Under GPLv2~\S6, the act of distributing automatically grants a
license from the original licensor to the next recipient.  This creates a
chain of grants that ensure that everyone in the distribution has rights
under the GPLv2\@.  In a mathematical sense, this bounds the bottom ---
making sure that future licensees get no fewer rights than the licensee before.

The second sentence of GPLv2~\S6 does the opposite; it bounds from the top.  It
prohibits any licensor along the distribution chain from placing
additional restrictions on the user.  In other words, no additional
requirements may trump the rights and freedoms given by GPLv2\@.

The final sentence of GPLv2~\S6 makes it abundantly clear that no individual
entity in the distribution chain is responsible for the compliance of any
other.  This is particularly important for noncommercial users who have
passed along a source offer under GPLv2~\S3(c), as they cannot be assured that
the issuer of the offer will honor their GPLv2~\S3 obligations.

In short, GPLv2~\S6 says that your license for the software is your one and
only copyright license allowing you to copy, modify and distribute the
software.

\section{GPLv2~\S7: ``Give Software Liberty or Give It Death!''}
\label{GPLv2s7}

In essence, GPLv2~\S7 is a verbosely worded way of saying for non-copyright
systems what GPLv2~\S6 says for copyright.  If there exists any reason that a
distributor knows of that would prohibit later licensees from exercising
their full rights under GPL, then distribution is prohibited.

Originally, this was designed as the title of this section suggests --- as
a last ditch effort to make sure that freedom was upheld.  However, in
modern times, it has come to give much more.  Now that the body of GPL'd
software is so large, patent holders who would want to be distributors of
GPL'd software have a tough choice.  They must choose between avoiding
distribution of GPL'd software that exercises the teachings of their
patents, or grant a royalty-free, irrevocable, non-exclusive license to
those patents.  Many companies have chosen the latter.

Thus, GPLv2~\S7 rarely gives software death by stopping its distribution.
Instead, it is inspiring patent holders to share their patents in the same
freedom-defending way that they share their copyrighted works.

\section{GPLv2~\S8: Excluding Problematic Jurisdictions}
\label{GPLv2s8}

GPLv2~\S8 is rarely used by copyright holders.  Its intention is that if a
particular country, say Unfreedonia, grants particular patents or allows
copyrighted interfaces (no country to our knowledge even permits those
yet), that the GPLv2'd software can continue in free and unabated
distribution in the countries where such controls do not exist.

As far as is currently known, GPLv2~\S8 has never been formally used by any
copyright holders.  Some have used GPLv2~\S8 to explain various odd special
topics of distribution, but generally speaking, this section is not
particularly useful and was actually removed in GPLv3.

% FIXME: integrate this into this section.

To our knowledge, no one has invoked this section to add an explicit
geographical distribution limitation since GPLv2 was released in 1991. We
have concluded that this provision is not needed and is not expected to be
needed in the future, and that it therefore should be removed.


%%%%%%%%%%%%%%%%%%%%%%%%%%%%%%%%%%%%%%%%%%%%%%%%%%%%%%%%%%%%%%%%%%%%%%%%%%%%%%%
\chapter{Odds, Ends, and Absolutely No Warranty}

GPLv2~\S\S0--7 constitute the freedom-defending terms of the GPLv2.  The remainder
of the GPLv2 handles administrivia and issues concerning warranties and
liability.

\section{GPLv2~\S9: FSF as Stewards of GPL}
\label{GPLv2s9}

FSF reserves the exclusive right to publish future versions of the GPL\@;
GPLv2~\S9 expresses this.  While the stewardship of the copyrights on the body
of GPL'd software around the world is shared among thousands of
individuals and organizations, the license itself needs a single steward.
Forking of the code is often regrettable but basically innocuous.  Forking
of licensing is disastrous.

(Chapter~\ref{tale-of-two-copylefts} discusses more about the various
versions of GPL.)

\section{GPLv2~\S10: Relicensing Permitted}
\label{GPLv2s10}

GPLv2~\S10 reminds the licensee of what is already implied by the nature of
copyright law.  Namely, the copyright holder of a particular software
program has the prerogative to grant alternative agreements under separate
copyright licenses.

\section{GPLv2~\S11: No Warranty}
\label{GPLv2s11}

Most warranty disclaimer language shout at you.  The
\href{http://www.law.cornell.edu/ucc/2/2-316}{Uniform Commercial
  Code~\S2-316} requires that disclaimers of warranty be ``conspicuous''.
There is apparently general acceptance that \textsc{all caps} is the
preferred way to make something conspicuous, and that has over decades worked
its way into the voodoo tradition of warranty disclaimer writing.

Some have argued the GPL is unenforceable in some jurisdictions because
its disclaimer of warranties is impermissibly broad.  However, GPLv2~\S11
contains a jurisdictional savings provision, which states that it is to be
interpreted only as broadly as allowed by applicable law.  Such a
provision ensures that both it, and the entire GPL, is enforceable in any
jurisdiction, regardless of any particular law regarding the
permissibility of certain warranty disclaimers.

Finally, one important point to remember when reading GPLv2~\S11 is that GPLv2~\S1
permits the sale of warranty as an additional service, which GPLv2~\S11 affirms.

\section{GPLv2~\S12: Limitation of Liability}
\label{GPLv2s12}

There are many types of warranties, and in some jurisdictions some of them
cannot be disclaimed.  Therefore, usually agreements will have both a
warranty disclaimer and a limitation of liability, as we have in GPLv2~\S12.
GPLv2~\S11 thus gets rid of all implied warranties that can legally be
disavowed. GPLv2~\S12, in turn, limits the liability of the actor for any
warranties that cannot legally be disclaimed in a particular jurisdiction.

Again, some have argued the GPL is unenforceable in some jurisdictions
because its limitation of liability is impermissibly broad. However, \S
12, just like its sister, GPLv2~\S11, contains a jurisdictional savings
provision, which states that it is to be interpreted only as broadly as
allowed by applicable law.  As stated above, such a provision ensures that
both GPLv2~\S12, and the entire GPL, is enforceable in any jurisdiction,
regardless of any particular law regarding the permissibility of limiting
liability.

So end the terms and conditions of the GNU General Public License.

%%%%%%%%%%%%%%%%%%%%%%%%%%%%%%%%%%%%%%%%%%%%%%%%%%%%%%%%%%%%%%%%%%%%%%%%%%%%%%%
\chapter{GPLv3}
\label{GPLv3}

This chapter discussed the text of GPLv3.  Much of this material herein
includes text that was adapted (with permission) from text that FSF
originally published as part of the so-called ``rationale documents'' for the
various discussion drafts of GPLv3.

The FSF ran a somewhat public process to develop GPLv3, and it was the first
attempt of its kind to develop a Free Software license this way.  Ultimately,
RMS was the primary author of GPLv3, but he listened to feedback from all
sorts of individuals and even for-profit companies.  Nevertheless, in
attempting to understand GPLv3 after the fact, the materials available from
the GPLv3 process have a somewhat ``drinking from the firehose'' effect.
This chapter seeks to explain GPLv3 to newcomers, who perhaps are familiar
with GPLv2.  Those who wish a to head to the firehose and take a diachronic
approach to GPLv3 study by reading the step-by-step public drafting process
GPLv3 (which occurred from Monday 16 January 2006 through Monday 19 November
2007) should visit \url{http://gplv3.fsf.org/}.

\section{Understanding GPLv3 As An Upgraded GPLv2}

Ultimately, GPLv2 and GPLv3 co-exist as active licenses in regular use.  As
discussed in Chapter\~ref{tale-of-two-copylefts}, GPLv1 was never in regular
use alongside GPLv2.  However, given GPLv2's widespread popularity and
existing longevity by the time GPLv3 was published, it is not surprising that
some licensors have continued to prefer GPLv2-only or GPLv2-or-later as their
preferred license.  GPLv3 has gained major adoption by many projects, old and
new, but many projects have not upgraded due to (in some cases) mere laziness
and (in other cases) policy preference for some of GPLv2's terms.

Given this ``two GPLs'' world is the one we all live in, it makes sense to
consider GPLv3 in terms of how it differs from GPLv2.  Also, most of the best
GPL experts in the world must deal regularly with both licenses, and
admittedly have decades of experience of GPLv2 while the most experience with
GPLv3 that's possible is by default less than a decade.

These two factors usually cause even new students of GPL to start with GPLv2
and move on to GPLv3, and this tutorial follows that pattern.

Overall, the changes made in GPLv3 admittedly \textit{increased} the
complexity of the license.  The FSF stated at the start of the GPLv3 process
that they would have liked to oblige those who have asked for a simpler and
shorter GPL\@.  Ultimately, the FSF gave priority to making GPLv3 do the job
that needs to be done to build a better copyleft.  Obsession for concision
should never trump software freedom.

\section{GPLv3~\S0: Giving In On ``Defined Terms''}

One of lawyers' most common complaints about GPLv2 is that defined terms in
the document appear throughout.  Most licenses define terms up-front.
However, GPL was always designed both as a document that should be easily
understood both by lawyers and by software developers: it is a document
designed to give freedom to software developers and users, and therefore it
should be comprehensible to that constituency.

Interestingly enough, one coauthor of this tutorial who is both a lawyer and
a developer pointed out that in law school, she understood defined terms more
quickly than other law students precisely because of her programming
background.  For developers, having \verb0#define0 (in the C programming
language) or other types of constants and/or macros that automatically expand
in the place where they are used is second nature.  As such, adding a defined
terms section was not terribly problematic for developers, and thus GPLv3
adds one.  Most of these defined terms are somewhat straightforward and bring
forward better worded definitions from GPLv2.  Herein, this tutorial
discusses a few of the new ones.

% FIXME: it's now five, ``Modify''

GPLv3~\S0 includes definitions of four new terms not found in any form in
GPLv2: ``covered work'', ``propagate'', ``convey'', and ``Appropriate Legal
Notices''.

% FIXME: Transition, GPLv2 ref needed.

Although the definition of ``work based on the Program'' made use of a legal
term of art, ``derivative work,'' peculiar to US copyright law, we did not
believe that this presented difficulties as significant as those associated
with the use of the term ``distribution.''  After all, differently-labeled
concepts corresponding to the derivative work are recognized in all copyright
law systems.  That these counterpart concepts might differ to some degree in
scope and breadth from the US derivative work was simply a consequence of
varying national treatment of the right of altering a copyrighted work.

%FIXME: should we keep this? maybe a footnote?

Ironically, the criticism we have received regarding the use of
US-specific legal terminology in the ``work based on the Program''
definition has come not primarily from readers outside the US, but
from those within it, and particularly from members of the technology
licensing bar.  They have argued that the definition of ``work based
on the Program'' effectively misstates what a derivative work is under
US law, and they have contended that it attempts, by indirect means,
to extend the scope of copyleft in ways they consider undesirable.
They have also asserted that it confounds the concepts of derivative
and collective works, two terms of art that they assume, questionably,
to be neatly disjoint under US law.

% FIXME: As above

We do not agree with these views, and we were long puzzled by the
energy with which they were expressed, given the existence of many
other, more difficult legal issues implicated by the GPL.
Nevertheless, we realized that here, too, we can eliminate usage of
local copyright terminology to good effect.  Discussion of GPLv3 will
be improved by the avoidance of parochial debates over the
construction of terms in one imperfectly-drafted copyright statute.
Interpretation of the license in all countries will be made easier by
replacement of those terms with neutral terminology rooted in
description of behavior.

%FIXME: GPLv3, reword a bit.

Draft 2 therefore takes the task of internationalizing the license
further by removing references to derivative works and by providing a
more globally useful definition of a work ``based on'' another work.
We return to the basic principles of users' freedom and the common
elements of copyright law.  Copyright holders of works of software
have the exclusive right to form new works by modification of the
original, a right that may be expressed in various ways in different
legal systems.  The GPL operates to grant this right to successive
generations of users, particularly through the copyleft conditions set
forth in section 5 of GPLv3, which applies to the conveying of works
based on the Program.  In section 0 we simply define a work based on
another work to mean ``any modified version for which permission is
necessary under applicable copyright law,'' without further qualifying
the nature of that permission, though we make clear that modification
includes the addition of material.\footnote{We have also removed the
paragraph in section 5 that makes reference to ``derivative or
collective works based on the Program.''}

%FIXME: transition

While ``covered by this license'' is a phrase found in GPLv2, defining it
more complete in a single as ``covered work'' enables some of the wording in
GPLv3 to be simpler and clearer than its GPLv2 counterparts.

% FIXME: does propagate  definition still work the same way in final draft?

The term ``propagate'' serves two purposes.  First, ``propagate'' provides a
simple and convenient means for distinguishing between the kinds of uses of a
work that the GPL imposes conditions on and the kinds of uses that the GPL
does not (for the most part) impose conditions on.

Second, ``propagate'' furthers our goal of making the license as global as
possible in its wording and effect.  When a work is licensed under the GPL,
the copyright law of some particular country will govern certain legal issues
arising under the license.  A term like ``distribute'' or its equivalent in
languages other than English, is used in several national copyright statutes.

Practical experience with GPLv2 revealed the awkwardness of using the
term ``distribution'' in a license intended for global use.  
The scope of ``distribution'' in the copyright context can differ from
country to country.  The GPL does not seek to necessarily use the specific
meaning of ``distribution'' that exists under United States copyright law or
any other country's copyright law.

%FIXME: rewrite, FSF third person,e tc.

Even within a single country and language, the term distribution may be
ambiguous; as a legal term of art, distribution varies significantly in
meaning among those countries that recognize it.  For example, we have been
told that in at least one country distribution may not include network
transfers of software but may include interdepartmental transfers of physical
copies within an organization.  In many countries the term ``making available
to the public'' or ``communicating to the public'' is the closest counterpart
to the generalized notion of distribution that exists under US law.

Therefore, the GPL defines the term ``propagate'' by reference to activities
that require permission under ``applicable copyright law'', but excludes
execution and private modification from the definition.  GPLv3's definition
also gives examples of activities that may be included within ``propagation''
but it also makes clear that, under the copyright laws of a given country,
``propagation'' may include other activities as well.

% FIXME: probably merge this in

Propagation is defined by behavior, and not by categories drawn from some
particular national copyright statute.  We believe that such factually-based
terminology has the added advantage of being easily understood and applied by
individual developers and users.

% FIXME: transition here to convey definition, maybe with \subsection {},
%        also maybe with: Similar is true with the term ``convey''.

we have further internationalized the license by removing references to
distribution and replacing them with a new factually-based term,
``conveying.'' Conveying is defined to include activities that constitute
propagation of copies to others.  With these changes, GPLv3 addresses
transfers of copies of software in behavioral rather than statutory terms.
At the same time, we have acknowledged the use of ``making available to the
public'' in jurisdictions outside the US by adding it as a specific example
in the definition of ``propagate.'' We decided to leave the precise
definition of an organizational licensee, and the line drawn between
licensees and other parties, for determination under local law.

% FIXME: paragraph number change , and more on Convey once definition comes.

The third paragraph of section 2 represents another effort to compensate for
variation in national copyright law.  We distinguish between propagation that
enables parties other than the licensee to make or receive copies, and other
forms of propagation.  As noted above, the meaning of ``distribution'' under
copyright law varies from country to country, including with respect to
whether making copies available to other parties (such as related public or
corporate entities) is ``distribution.'' ``Propagation,'' however, is a term
not tied to any statutory language.  Propagation that does not enable other
parties to make or receive copies --- for example, making private copies or
privately viewing the program --- is permitted unconditionally.  Propagation
that does enable other parties to make or receive copies is permitted as
``distribution,'' subject to the conditions set forth in sections 4--6.

% FIXME: Appropriate Legal Notices

\section{GPLv3~\S1: Understanding CCS}

% FIXME: Talk briefly about importance of CCS and reference compliance guide

% FIXME: verify this still matches final GPLv3 text.
% FIXME:  link to GPLv2 tutorial sections if possible and where appropriate.

GPLv3\~S1 retains GPLv2's definition of ``source code'' and adds an explicit
definition of ``object code'' as ``any non-source version of a work''.
Object code is not restricted to a narrow technical meaning and is to be
understood broadly as including any form of the work other than the preferred
form for making modifications to it.  Object code therefore includes any kind
of transformed version of source code, such as bytecode or minified
Javascript.  The definition of object code also ensures that licensees cannot
escape their obligations under the GPL by resorting to shrouded source or
obfuscated programming.

% FIXME: CCS Coresponding Source updated to newer definition in later drafts

Keeping with the desire to ``round up'' definitions that were spread
throughout the text of GPLv2, the definition of CCS\footnote{Note that the
  preferred term by those who work with both GPLv2 and GPLv3 is ``Complete
  Corresponding Source'', abbreviated to ``CCS''.  Admittedly, the word
  ``complete'' no longer appears in GPLv3 (which uses the word ``all''
  instead).  However, both GPLv2 and the early drafts of GPLv3 itself used
  the word complete, and early GPLv3 drafts even included the phrase
  ``Complete Corresponding Source''.  Meanwhile, use of the acronym ``CCS''
  (sometimes, ``C\&CS'') was so widespread among GPL enforcers that its use
  continues even though GPLv3-focused experts tend to say just the defined
  term of ``Corresponding Source''.}, or, as GPLv3 officially calls it,
``Corresponding Source'', is given in GPLv3~\S1\P4.  This definition is as
broad as necessary to protect users' exercise of their rights under the
GPL\@.  We follow the definition with particular examples to remove any doubt
that they are to be considered Complete Corresponding Source Code.  We wish to
make completely clear that a licensee cannot avoid complying with the
requirements of the GPL by dynamically linking an add-on component to the
original version of a program.

Though the definition of Complete Corresponding Source Code in the
second paragraph of section 1 is expansive, it is not sufficient to
protect users' freedoms in many circumstances.  For example, a GPL'd
program, or a modified version of such a program, might need to be
signed with a key or authorized with a code in order for it to run on
a particular machine and function properly. Similarly, a program that
produces digitally-restricted files might require a decryption code in
order to read the output.  

% FIXME: Standard Interface

% FIXME: System Libraries: it's in a different place and changed in later drafts

The final paragraph of section 1 revises the exception to the source code
distribution requirement in GPLv2 that we have sometimes called the system
library exception. This exception has been read to prohibit certain
distribution arrangements that we consider reasonable and have not sought to
prevent, such as distribution of gcc linked with a non-free C library that is
included as part of a larger non-free system. This is not to say that such
non-free libraries are legitimate; rather, preventing free software from
linking with these libraries would hurt free software more than it would hurt
proprietary software.

As revised, the exception has two parts. Part (a) rewords the GPLv2
exception for clarity but also removes the words ``unless that
component itself accompanies the executable.''  By itself, (a) would
be too permissive, allowing distributors to evade their
responsibilities under the GPL.  We have therefore added part (b) to
specify when a system library that is an adjunct of a major essential
operating system component, compiler, or interpreter does not trigger
the requirement to distribute source code.  The more low-level the
functionality provided by the library, the more likely it is to be
qualified for this exception.

\section{GPLv3~\S2: Basic Permissions}

% FIXME: phrase ``unmodified Program'' appears due to User Products exception

We have included the first sentence of section 2 to further internationalize
the GPL. Under the copyright laws of some countries, it may be necessary for
a copyright license to include an explicit provision setting forth the
duration of the rights being granted. In other countries, including the
United States, such a provision is unnecessary but permissible.

The first paragraph of section 2 also acknowledges that licensees under the
GPL enjoy rights of copyright fair use, or the equivalent under applicable
law. These rights are compatible with, and not in conflict with, the freedoms
that the GPL seeks to protect, and the GPL cannot and should not restrict
them.

% FIXME: propagate and convey

Section 2 distinguishes between activities of a licensee that are permitted
without limitation and activities that trigger additional requirements. The
second paragraph of section 2 guarantees the basic freedoms of privately
modifying and running the program. However, the right to privately modify and
run the program is terminated if the licensee brings a patent infringement
lawsuit against anyone for activities relating to a work based on the
program.


\section{GPLv3~\S3: What Hath DMCA Wrought}
\label{GPLv3s3}

% FIXME: reference the section in DMCA about this, maybe already there in
%        GPLv2 section?

% FIXME: Wrong paragraph now.

The second paragraph of section 3 declares that no GPL'd program is part of
an effective technological protection measure, regardless of what the program
does. Ill-advised legislation in the United States and other countries has
prohibited circumvention of such technological measures. If a covered work is
distributed as part of a system for generating or accessing certain data, the
effect of this paragraph is to prevent someone from claiming that some other
GPL'd program that accesses the same data is an illegal circumvention.

\section{GPLv3~\S4: Verbatim Copying}

% FIXME: there appear to be minor changes here in later drafts, fix that.

Section 4 has been revised from its corresponding section in GPLv2 in light
of the new section 7 on license compatibility. A distributor of verbatim
copies of the program's source code must obey any existing additional terms
that apply to parts of the program. In addition, the distributor is required
to keep intact all license notices, including notices of such additional
terms.

\section{GPLv3~\S5: Modified Source}

% FIXME: 5(a) is slightly different in final version

Section 5 contains a number of changes relative to the corresponding section
in GPLv2. Subsection 5a slightly relaxes the requirements regarding notice of
changes to the program. In particular, the modified files themselves need no
longer be marked. This reduces administrative burdens for developers of
modified versions of GPL'd software.

Under subsection 5a, as in the corresponding provision of GPLv2, the notices
must state ``the date of any change,'' which we interpret to mean the date of
one or more of the licensee's changes.  The best practice would be to include
the date of the latest change.  However, in order to avoid requiring revision
of programs distributed under ``GPL version 2 or later,'' we have retained
the existing wording.

% FIXME:  It's now (b) and (c).  Also, ``validity'' of proprietary
%         relicensing?  Give me a break.  I'll fix that.

Subsection 5b is the central copyleft provision of the license.  It now
states that the GPL applies to the whole of the work.  The license must be
unmodified, except as permitted by section 7, which allows GPL'd code to be
combined with parts covered by certain other kinds of free software licensing
terms. Another change in subsection 5b is the removal of the words ``at no
charge,'' which was often misinterpreted by commentators.  The last sentence
of subsection 5b explicitly recognizes the validity of disjunctive
dual-licensing.

%  FIXME: 5d.  Related to Appropriatey Legal notices


% follows 5d now, call it the ``final paragraph''

The paragraph following subsection 5c has been revised for clarity, but the
underlying meaning is unchanged. When independent non-derivative sections are
distributed for use in a combination that is a covered work, the whole of the
combination must be licensed under the GPL, regardless of the form in which
such combination occurs, including combination by dynamic linking. The final
sentence of the paragraph adapts this requirement to the new compatibility
provisions of section 7.

\section{GPLv3~\S6: Non-Source and Corresponding Source}

Section 6 of GPLv3, which clarifies and revises GPLv2 section 3, requires
distributors of GPL'd object code to provide access to the corresponding
source code, in one of four specified ways. As noted above, ``object code''
in GPLv3 is defined broadly to mean any non-source version of a work.

% FIXME:  probably mostly still right, needs some updates, though.

Subsections 6a and 6b now apply specifically to distribution of object code
in a physical product. Physical products include embedded systems, as well as
physical software distribution media such as CDs. As in GPLv2, the
distribution of object code may either be accompanied by the machine-readable
source code, or it may be accompanied by a written offer to provide the
machine-readable source code to any third party. GPLv3 clarifies that the
medium for software interchange on which the machine-readable source code is
provided must be a durable physical medium. Subsection 6b does not prevent a
distributor from offering to provide source code to a third party by some
other means, such as transmission over a network, so long as the option of
obtaining source code on a physical medium is presented.

% FIXME:  probably mostly still right, needs some updates, though.

Subsection 6b revises the requirements for the written offer to provide
source code. As before, the offer must remain valid for at least three
years. In addition, even after three years, a distributor of a product
containing GPL'd object code must offer to provide source code for as long as
the distributor also continues to offer spare parts or customer support for
the product model. We believe that this is a reasonable and appropriate
requirement; a distributor should be prepared to provide source code if he or
she is prepared to provide support for other aspects of a physical product.

% FIXME: 10x language is gone.

Subsection 6b also increases the maximum permitted price for providing a copy
of the source code. GPLv2 stated that the price could be no more than the
cost of physically performing source distribution; GPLv3 allows the price to
be up to ten times the distributor's cost. It may not be practical to expect
some organizations to provide such copies at cost. Moreover, permitting such
organizations to charge ten times the cost is not particularly harmful, since
some recipient of the code can be expected to make the code freely available
on a public network server. We also recognize that there is nothing wrong
with profiting from providing copies of source code, provided that the price
of a copy is not so unreasonably high as to make it effectively unavailable.

% FIXME:  probably mostly still right, needs some updates, though.

Subsection 6c gives narrower permission than the corresponding subsection in
GPLv2.  The option of including a copy of an offer received in accordance
with subsection 6b is available only for private distribution of object code;
moreover, such private distribution is restricted to ``occasional
non-commercial distribution.''  This subsection makes clear that a
distributor cannot comply with the GPL merely by making object code available
on a publicly-accessible network server accompanied by a copy of the written
offer to provide source code received from an upstream distributor.

% FIXME:  probably mostly still right, needs some updates, though.

New subsection 6d, which revises the final paragraph of GPLv2 section 3,
addresses distribution of object code by offering access to copy the code
from a designated place, such as by enabling electronic access to a network
server.  Subsection 6d clarifies that the distributor must offer equivalent
access to copy the source code ``in the same way through the same place.''
This wording permits a distributor to offer a third party access to both
object code and source code on a single network portal or web page, even
though the access may include links to different physical servers.  For
example, a downstream distributor may provide a link to an upstream
distributor's server and arrange with the operator of that server to keep the
source code available for copying for as long as the downstream distributor
enables access to the object code.  This codifies what has been our
interpretation of GPLv2.

%FIXME: 6e, peer-to-peer


%  FIXME: Not final paragraph anymore. 

The final paragraph of section 6 takes account of the fact that the Complete
Corresponding Source Code may include added parts that carry non-GPL terms,
as permitted by section 7.

% FIXME: update lock-down section to work with more recent drafts

Though the definition of Complete Corresponding Source Code in the second
paragraph of section 1 is expansive, it is not sufficient to protect users'
freedoms in many circumstances. For example, a GPL'd program, or a modified
version of such a program, might need to be signed with a key or authorized
with a code in order for it to run on a particular machine and function
properly. Similarly, a program that produces digitally-restricted files might
require a decryption code in order to read the output.

The third paragraph of section 1 addresses this problem by making clear that
Complete Corresponding Source Code includes any such encryption,
authorization, and decryption codes. By requiring the inclusion of this
information whenever the GPL requires distribution of Complete Corresponding
Source Code, we thwart efforts to obstruct the goals of the GPL, and we
ensure that users will remain in control over their own machines. We
recognize an exception where use of the program normally implies that the
user already has the codes. For example, in secure systems a computer owner
might possess any keys needed to run a program, while the distributor of the
program might not have the keys.

% FIXME: installation information

%FIXME: publicly documented format

\section{Understanding License Compatibility}
\label{license-compatibility}

% FIXME: more about license compatibility here.

A challenge that faced the Free Software community heavily through out the
early 2000s was the proliferation of incompatible Free Software licenses.  Of
course, we cannot make the GPL compatible with all such licenses. GPLv3
contains provisions that are designed to reduce license incompatibility by
making it easier for developers to combine code carrying non-GPL terms with
GPL'd code.

% FIXME: connecting text

\subsection{Additional Permissions}

% FIXME: rework and fix formatting.

The GPL is a statement of permissions, some of which have conditions.
Additional terms, terms that supplement those of the GPL, may come to be
placed on, or removed from, GPL-covered code in certain common ways.  We
consider those added terms ``additional permissions'' if they grant
exceptions from the conditions of the GPL, and ``additional requirements'' if
they add conditions to the basic permissions of the GPL. The treatment of
additional permissions and additional requirements under GPLv3 is necessarily
asymmetrical, because they do not raise the same ethical and interpretive
issues; in particular, additional requirements, if allowed without careful
limitation, could transform a GPL'd program into a non-free one.  With these
principles in the background, section 7 answers the following questions: (1)
How do the presence of additional terms on all or part of a GPL'd program
affect users' rights? (2) When and how may a licensee add terms to code being
distributed under the GPL? (3) When may a licensee remove additional terms?

% FIXME: FSF third person, etc.

Additional permissions present the easier case.  We have licensed some of our
own software under GPLv2 with permissive exceptions that allow combination
with non-free code, and that allow removal of those permissions by downstream
recipients; similarly, LGPLv2.1 is in essence a permissive variant of GPLv2,
and it permits relicensing under the GPL.  We have generalized these
practices in section 7.  A licensee may remove any additional permission from
a covered work, whether it was placed by the original author or by an
upstream distributor.  A licensee may also add any kind of additional
permission to any part of a work for which the licensee has, or can give,
appropriate copyright permission. For example, if the licensee has written
that part, the licensee is the copyright holder for that part and can
therefore give additional permissions that are applicable to it.
Alternatively, the part may have been written by someone else and licensed,
with the additional permissions, to that licensee.  Any additional
permissions on that part are, in turn, removable by downstream recipients.
As subsection 7a explains, the effect of an additional permission depends on
whether the permission applies to the whole work or a part.

% FIXME: rework this a bit

We have drafted version 3 of the GNU LGPL, which we have released with Draft
2 of GPLv3, as a simple list of additional permissions supplementing the
terms of GPLv3.  Section 7 has thus provided the basis for recasting a
formally complex license as an elegant set of added terms, without changing
any of the fundamental features of the existing LGPL.  We offer this draft of
LGPLv3 as as a model for developers wishing to license their works under the
GPL with permissive exceptions.  The removability of additional permissions
under section 7 does not alter any existing behavior of the LGPL; the LGPL
has always allowed relicensing under the ordinary GPL.

\subsection{Additional Requirements and License Compatibility}

% FIXME: minor rewrites needed

We broadened the title of section 7 because license compatibility, as it is
conventionally understood, is only one of several facets of the placement of
additional terms on GPL'd code.  The license compatibility issue arises for
three reasons.  First, the GPL is a strong copyleft license, requiring
modified versions to be distributed under the GPL.  Second, the GPL states
that no further restrictions may be placed on the rights of recipients.
Third, all other free software licenses in common use contain certain
requirements, many of which are not conditions made by the GPL.  Thus, when
GPL'd code is modified by combination with code covered by another formal
license that specifies other requirements, and that modified code is then
distributed to others, the freedom of recipients may be burdened by
additional requirements in violation of the GPL.  It can be seen that
additional permissions in other licenses do not raise any problems of license
compatibility.

% FIXME: minor rewrites needed

Section 7 relaxes the prohibition on further restrictions slightly by
enumerating, in subsection 7b, a limited list of categories of additional
requirements that may be placed on code without violating GPLv3.  The list
includes the items that were listed in Draft 1, though rewritten for clarity.
It also includes a new catchall category for terms that might not obviously
fall within one of the other categories but which are precisely equivalent to
GPLv3 conditions, or which deny permission for activities clearly not
permitted by GPLv3.  We have carefully considered but rejected proposals to
expand this list further.  We have also rejected suggestions, made by some
discussion committee members, that the Affero clause requirement (7d in Draft
1 and 7b4 in Draft 2) be removed, though we have revised it in response to
certain comments.  We are unwavering in our view that the Affero requirement
is a legitimate one, and we are committed to achieving compatibility of the
Affero GPL with GPLv3.

% FIXME: minor rewrites needed

A GPL licensee may place an additional requirement on code for which the
licensee has or can give appropriate copyright permission, but only if that
requirement falls within the list given in subsection 7b.  Placement of any
other kind of additional requirement continues to be a violation of the
license.  Additional requirements that are in the 7b list may not be removed,
but if a user receives GPL'd code that purports to include an additional
requirement not in the 7b list, the user may remove that requirement.  Here
we were particularly concerned to address the problem of program authors who
purport to license their works in a misleading and possibly
self-contradictory fashion, using the GPL together with unacceptable added
restrictions that would make those works non-free software.

\section{GPLv3~\S7: Explicit Compatibility}


% FIXME:  probably mostly still right, needs some updates, though.

In GPLv3 we take a new approach to the issue of combining GPL'd code with
code governed by the terms of other free software licenses. Our view, though
it was not explicitly stated in GPLv2 itself, was that GPLv2 allowed such
combinations only if the non-GPL licensing terms permitted distribution under
the GPL and imposed no restrictions on the code that were not also imposed by
the GPL. In practice, we supplemented this policy with a structure of
exceptions for certain kinds of combinations.

% FIXME:  probably mostly still right, needs some updates, though.

Section 7 of GPLv3 implements a more explicit policy on license
compatibility. It formalizes the circumstances under which a licensee may
release a covered work that includes an added part carrying non-GPL terms. We
distinguish between terms that provide additional permissions, and terms that
place additional requirements on the code, relative to the permissions and
requirements established by applying the GPL to the code.

% FIXME:  probably mostly still right, needs some updates, though.

Section 7 first explicitly allows added parts covered by terms with
additional permissions to be combined with GPL'd code. This codifies our
existing practice of regarding such licensing terms as compatible with the
GPL. A downstream user of a combined GPL'd work who modifies such an added
part may remove the additional permissions, in which case the broader
permissions no longer apply to the modified version, and only the terms of
the GPL apply to it.

% FIXME:  probably mostly still right, needs some updates, though.

In its treatment of terms that impose additional requirements, section 7
extends the range of licensing terms with which the GPL is compatible. An
added part carrying additional requirements may be combined with GPL'd code,
but only if those requirements belong to an set enumerated in section 7. We
must, of course, place some limit on the kinds of additional requirements
that we will accept, to ensure that enhanced license compatibility does not
defeat the broader freedoms advanced by the GPL. Unlike terms that grant
additional permissions, terms that impose additional requirements cannot be
removed by a downstream user of the combined GPL'd work, because no such user
would have the right to do so.

% FIXME:  probably mostly still right, needs some updates, though.

Under subsections 7a and 7b, the requirements may include preservation of
copyright notices, information about the origins of the code or alterations
of the code, and different warranty disclaimers. Under subsection 7c, the
requirements may include limitations on the use of names of contributors and
on the use of trademarks for publicity purposes. In general, we permit these
requirements in added terms because many free software licenses include them
and we consider them to be unobjectionable. Because we support trademark fair
use, the limitations on the use of trademarks may seek to enforce only what
is required by trademark law, and may not prohibit what would constitute fair
use.

% FIXME: 7d-f

\section{GPLv3~\S7(e): Peer-to-Peer Sharing Networks}

% FIXME: rewrite a bit, maybe drop reference to bitorrent?

Certain decentralized forms of peer-to-peer file sharing present a challenge
to the unidirectional view of distribution that is implicit in GPLv2 and
Draft 1 of GPLv3.  It is neither straightforward nor reasonable to identify
an upstream/downstream link in BitTorrent distribution; such distribution is
multidirectional, cooperative and anonymous.  In systems like BitTorrent,
participants act both as transmitters and recipients of blocks of a
particular file, but they see themselves as users and receivers, and not as
distributors in any conventional sense.  At any given moment of time, most
peers will not have the complete file.

% FIXME: rewrite a bit.

The GPL permits distribution of a work in object code form over a network,
provided that the distributor offers equivalent access to copy the
Corresponding Source Code ``in the same way through the same place.''  This
wording might be interpreted to permit BitTorrent distribution of binaries if
they are packaged together with the source code, but this impractical, for at
least two reasons. First, even if the source code is packaged with the
binary, it will only be available to a non-seeding peer at the end of the
distribution process, but the peer will already have been providing parts of
the binary to others in the network, functioning rather like a router or a
cache proxy.  Second, in practice BitTorrent and similar peer-to-peer forms
of transmission have been less suitable means for distributing source code.
In large distributions, packaging source code with the binary may result in a
substantial increase in file size and transmission time.  Source code
packages themselves tend not to be transmitted through BitTorrent owing to
reduced demand. There generally will be too few participants downloading the
same source package at the same time to enable effective seeding and
distribution.

% FIXME: rewrite a bit.

We have made two changes that recognize and facilitate distribution of
covered works in object code form using BitTorrent or similar peer-to-peer
methods.  First, under new subsection 6e, if a licensee conveys such a work
using peer-to-peer transmission, that licensee is in compliance with section
6 so long as the licensee knows, and informs other peers where, the object
code and its Corresponding Source are publicly available at no charge under
subsection 6d.  The Corresponding Source therefore need not be provided
through the peer-to-peer system that was used for providing the binary.
Second, we have revised section 9 to make clear that ancillary propagation of
a covered work that occurs as part of the process of peer-to-peer file
transmission does not require acceptance, just as mere receipt and execution
of the Program does not require acceptance.  Such ancillary propagation is
permitted without limitation or further obligation.

% FIXME:  removing additional restrictions

% FIXME:  probably mostly still right, needs some updates, though.

Section 7 requires a downstream user of a covered work to preserve the
non-GPL terms covering the added parts just as they must preserve the GPL, as
long as any substantial portion of those parts is present in the user's
version.

% FIXME: minor rewrites needed

Section 7 points out that GPLv3 itself makes no assertion that an additional
requirement is enforceable by the copyright holder.  However, section 7 makes
clear that enforcement of such requirements is expected to be by the
termination procedure given in section 8 of GPLv3.

% FIXME: better context, etc.

Some have questioned whether section 7 is needed, and some have suggested
that it creates complexity that did not previously exist.  We point out to
those readers that there is already GPLv2-licensed code that carries
additional terms.  One of the objectives of section 7 is to rationalize
existing practices of program authors and modifiers by setting clear
guidelines regarding the removal and addition of such terms.  With its
carefully limited list of allowed additional requirements, section 7
accomplishes additional objectives, permitting the expansion of the base of
code available for GPL developers, while also encouraging useful
experimentation with requirements we do not include in the GPL itself.

\section{GPLv3~\S8: A Lighter Termination}

% FIXME:  probably mostly still right, needs some updates, though.

GPLv2 provided for automatic termination of the rights of a person who
copied, modified, sublicensed, or distributed a work in violation of the
license.  Automatic termination can be too harsh for those who have committed
an inadvertent violation, particularly in cases involving distribution of
large collections of software having numerous copyright holders.  A violator
who resumes compliance with GPLv2 would need to obtain forgiveness from all
copyright holders, but even to contact them all might be impossible.

% FIXME: needs to be updated to describe more complex termination

Section 8 of GPLv3 replaces automatic termination with a non-automatic
termination process.  Any copyright holder for the licensed work may opt to
terminate the rights of a violator of the license, provided that the
copyright holder has first given notice of the violation within 60 days of
its most recent occurrence. A violator who has been given notice may make
efforts to enter into compliance and may request that the copyright holder
agree not exercise the right of termination; the copyright holder may choose
to grant or refuse this request.

% FIXME: needs to be updated to describe more complex termination

If a licensee who is in violation of GPLv3 acts to correct the violation and
enter into compliance, and the licensee receives no notice of the past
violation within 60 days, then the licensee need not worry about termination
of rights under the license.

\section{GPLv3~\S9: Acceptance}

% FIXME

\section{GPLv3~\S10: Explicit Downstream License}

% FIXME

\section{GPLv3~\S11: Explicit Patent Licensing}
\label{GPLv3s11}

% FIXME: just brought in words here, needs rewriting.

is rooted in the basic principles of the GPL.
Our license has always stated that distributors may not impose further
restrictions on users' exercise of GPL rights.  To make the suggested
distinction between contribution and distribution is to allow a
distributor to demand patent royalties from a direct or indirect
recipient, based on claims embodied in the distributed code. This
undeniably burdens users with an additional legal restriction on their
rights, in violation of the license.

%FIXME: possible useful text, but maybe not.

In the covenant provided in the revised section 11, the set of claims
that a party undertakes not to assert against downstream users are that
party's ``essential patent claims'' in the work conveyed by the party.
``Essential patent claims,'' a new term defined in section 0, are simply
all claims ``that would be infringed by making, using, or selling the
work.''  We have abandoned the phrase ``reasonably contemplated use.''
This change makes the obligations of distributing patent holders more
predictable.

% FIXME:  probably needs a lot of work, these provisions changed over time.

GPLv3 adds a new section on licensing of patents. GPLv2 relies on an implied
patent license. The doctrine of implied license is one that is recognized
under United States patent law but may not be recognized in other
jurisdictions. We have therefore decided to make the patent license grant
explicit in GPLv3. Under section 11, a redistributor of a GPL'd work
automatically grants a nonexclusive, royalty-free and worldwide license for
any patent claims held by the redistributor, if those claims would be
infringed by the work or a reasonably contemplated use of the work.

% FIXME:  probably needs a lot of work, these provisions changed over time.

The patent license is granted both to recipients of the redistributed work
and to any other users who have received any version of the work. Section 11
therefore ensures that downstream users of GPL'd code and works derived from
GPL'd code are protected from the threat of patent infringement allegations
made by upstream distributors, regardless of which country's laws are held to
apply to any particular aspect of the distribution or licensing of the GPL'd
code.

% FIXME:  probably needs a lot of work, these provisions changed over time.

A redistributor of GPL'd code may benefit from a patent license that has been
granted by a third party, where the third party otherwise could bring a
patent infringement lawsuit against the redistributor based on the
distribution or other use of the code. In such a case, downstream users of
the redistributed code generally remain vulnerable to the applicable patent
claims of the third party. This threatens to defeat the purposes of the GPL,
for the third party could prevent any downstream users from exercising the
freedoms that the license seeks to guarantee.

% FIXME:  probably needs a lot of work, these provisions changed over time.

The second paragraph of section 11 addresses this problem by requiring the
redistributor to act to shield downstream users from these patent claims. The
requirement applies only to those redistributors who distribute knowingly
relying on a patent license. Many companies enter into blanket patent
cross-licensing agreements. With respect to some such agreements, it would
not be reasonable to expect a company to know that a particular patent
license covered by the agreement, but not specifically mentioned in it,
protects the company's distribution of GPL'd code.

% FIXME: does this still fit with the final retaliation provision?

This narrowly-targeted patent retaliation provision is the only form of
patent retaliation that GPLv3 imposes by its own force. We believe that it
strikes a proper balance between preserving the freedom of a user to run and
modify a program, and protecting the rights of other users to run, modify,
copy, and distribute code free from threats by patent holders. It is
particularly intended to discourage a GPL licensee from securing a patent
directed to unreleased modifications of GPL'd code and then suing the
original developers or others for making their own equivalent modifications.

Several other free software licenses include significantly broader patent
retaliation provisions. In our view, too little is known about the
consequences of these forms of patent retaliation. As we explain below,
section 7 permits distribution of a GPL'd work that includes added parts
covered by terms other than those of the GPL. Such terms may include certain
kinds of patent retaliation provisions that are broader than those of section
2.

\section{GPLv3~\S12: Familiar as GPLv2 \S~7}

% FIXME:  probably mostly still right, needs some updates, though.

The wording in the first sentence of section 12 has been revised
slightly to clarify that an agreement, such as a litigation settlement
agreement or a patent license agreement, is one of the ways in which
conditions may be ``imposed'' on a GPL licensee that may contradict the
conditions of the GPL, but which do not excuse the licensee from
compliance with those conditions.  This change codifies what has been
our interpretation of GPLv2.  

% FIXME:  probably mostly still right, needs some updates, though.

We have removed the limited severability clause of GPLv2 section 7 as a
matter of tactical judgment, believing that this is the best way to ensure
that all provisions of the GPL will be upheld in court. We have also removed
the final sentence of GPLv2 section 7, which we consider to be unnecessary.

\section{GPLv3~\S13: The Great Affero Compromise}

% FIXME

\section{GPLv3~\S14: So, When's GPLv4?}
\label{GPlv2s14}

% FIXME Say more

No substantive change has been made in section 14. The wording of the section
has been revised slightly to make it clearer.

% FIXME; proxy

\section{GPLv3~\S15--17: Warranty Disclaimers and Liability Limitation}

No substantive changes have been made in sections 15 and 16.

% FIXME: more, plus 17

% FIXME: Section header needed here about choice of law.

% FIXME: reword into tutorial

Some have asked us to address the difficulties of internationalization
by including, or permitting the inclusion of, a choice of law
provision.  We maintain that this is the wrong approach.  Free
software licenses should not contain choice of law clauses, for both
legal and pragmatic reasons.  Choice of law clauses are creatures of
contract, but the substantive rights granted by the GPL are defined
under applicable local copyright law. Contractual free software
licenses can operate only to diminish these rights.  Choice of law
clauses also raise complex questions of interpretation when works of
software are created by combination and extension.  There is also the
real danger that a choice of law clause will specify a jurisdiction
that is hostile to free software principles.

% FIXME: reword into tutorial, \ref to section 7.

Our revised version of section 7 makes explicit our view that the
inclusion of a choice of law clause by a licensee is the imposition of
an additional requirement in violation of the GPL.  Moreover, if a
program author or copyright holder purports to supplement the GPL with
a choice of law clause, section 7 now permits any licensee to remove
that clause.

%%%%%%%%%%%%%%%%%%%%%%%%%%%%%%%%%%%%%%%%%%%%%%%%%%%%%%%%%%%%%%%%%%%%%%%%%%%%%%%
\chapter{The Lesser GPL}

As we have seen in our consideration of the GPL, its text is specifically
designed to cover all possible derivative works under copyright law. Our
goal in designing GPL was to make sure that any derivative work of GPL'd
software was itself released under GPL when distributed. Reaching as far
as copyright law will allow is the most direct way to reach that goal.

However, while the strategic goal is to bring as much Free Software
into the world as possible, particular tactical considerations
regarding software freedom dictate different means. Extending the
copyleft effect as far as copyright law allows is not always the most
prudent course in reaching the goal. In particular situations, even
those of us with the goal of building a world where all published
software is Free Software realize that full copyleft does not best
serve us. The GNU Lesser General Public License (``GNU LGPL'') was
designed as a solution for such situations.

\section{The First LGPL'd Program}

The first example that FSF encountered where such altered tactics were
needed was when work began on the GNU C Library. The GNU C Library would
become (and today, now is) a drop-in replacement for existing C libraries.
On a Unix-like operating system, C is the lingua franca and the C library
is an essential component for all programs. It is extremely difficult to
construct a program that will run with ease on a Unix-like operating
system without making use of services provided by the C library --- even
if the program is written in a language other than C\@. Effectively, all
user application programs that run on any modern Unix-like system must
make use of the C library.

By the time work began on the GNU implementation of the C libraries, there
were already many C libraries in existence from a variety of vendors.
Every proprietary Unix vendor had one, and many third parties produced
smaller versions for special purpose use. However, our goal was to create
a C library that would provide equivalent functionality to these other C
libraries on a Free Software operating system (which in fact happens today
on modern GNU/Linux systems, which all use the GNU C Library).

Unlike existing GNU application software, however, the licensing
implications of releasing the GNU C Library (``glibc'') under GPL were
somewhat different. Applications released under GPL would never
themselves become part of proprietary software. However, if glibc were
released under GPL, it would require that any application distributed for
the GNU/Linux platform be released under GPL\@.

Since all applications on a Unix-like system depend on the C library, it
means that they must link with that library to function on the system. In
other words, all applications running on a Unix-like system must be
combined with the C library to form a new whole derivative work that is
composed of the original application and the C library. Thus, if glibc
were GPL'd, each and every application distributed for use on GNU/Linux
would also need to be GPL'd, since to even function, such applications
would need to be combined into larger derivative works by linking with
glibc.

At first glance, such an outcome seems like a windfall for Free Software
advocates, since it stops all proprietary software development on
GNU/Linux systems. However, the outcome is a bit more subtle. In a world
where many C libraries already exist, many of which could easily be ported
to GNU/Linux, a GPL'd glibc would be unlikely to succeed. Proprietary
vendors would see the excellent opportunity to license their C libraries
to anyone who wished to write proprietary software for GNU/Linux systems.
The de-facto standard for the C library on GNU/Linux would likely be not
glibc, but the most popular proprietary one.

Meanwhile, the actual goal of releasing glibc under GPL --- to ensure no
proprietary applications on GNU/Linux --- would be unattainable in this
scenario. Furthermore, users of those proprietary applications would also
be users of a proprietary C library, not the Free glibc.

The Lesser GPL was initially conceived to handle this scenario. It was
clear that the existence of proprietary applications for GNU/Linux was
inevitable. Since there were so many C libraries already in existence, a
new one under GPL would not stop that tide. However, if the new C library
were released under a license that permitted proprietary applications
to link with it, but made sure that the library itself remained Free,
an ancillary goal could be met. Users of proprietary applications, while
they would not have the freedom to copy, share, modify and redistribute
the application itself, would have the freedom to do so with respect to
the C library.

There was no way the license of glibc could stop or even slow the creation
of proprietary applications on GNU/Linux. However, loosening the
restrictions on the licensing of glibc ensured that nearly all proprietary
applications at least used a Free C library rather than a proprietary one.
This trade-off is central to the reasoning behind the LGPL\@.

Of course, many people who use the LGPL today are not thinking in these
terms. In fact, they are often choosing the LGPL because they are looking
for a ``compromise'' between the GPL and the X11-style liberal licensing.
However, understanding FSF's reasoning behind the creation of the LGPL is
helpful when studying the license.


\section{What's the Same?}

Much of the text of the LGPL is identical to the GPL\@. As we begin our
discussion of the LGPL, we will first eliminate the sections that are
identical, or that have the minor modification changing the word
``Program'' to ``Library.''

First, LGPLv2.1~\S1, the rules for verbatim copying of source, are
equivalent to those in GPLv2~\S1.

Second, LGPLv2.1~\S8 is equivalent GPLv2~\S4\@. In both licenses, this
section handles termination in precisely the same manner.

LGPLv2.1~\S9 is equivalent to GPLv2~\S5\@. Both sections assert that
the license is a copyright license, and handle the acceptance of those
copyright terms.

LGPLv2.1~\S10 is equivalent to GPLv2~\S6. They both protect the
distribution system of Free Software under these licenses, to ensure that
up, down, and throughout the distribution chain, each recipient of the
software receives identical rights under the license and no other
restrictions are imposed.

LGPLv2.1~\S11 is GPLv2~\S7. As discussed, it is used to ensure that
other claims and legal realities, such as patent licenses and court
judgments, do not trump the rights and permissions granted by these
licenses, and requires that distribution be halted if such a trump is
known to exist.

LGPLv2.1~\S12 adds the same features as GPLv2~\S8. These sections are
used to allow original copyright holders to forbid distribution in
countries with draconian laws that would otherwise contradict these
licenses.

LGPLv2.1~\S13 sets up FSF as the steward of the LGPL, just as GPLv2~\S9
does for GPL. Meanwhile, LGPLv2.1~\S14 reminds licensees that copyright
holders can grant exceptions to the terms of LGPL, just as GPLv2~\S10
reminds licensees of the same thing.

Finally, the assertions of no warranty and limitations of liability are
identical; thus LGPLv2.1~\S15 and LGPLv2.1~\S16 are the same as GPLv2~\S11 and \S
12.

As we see, the entire latter half of the license is identical.
The parts which set up the legal boundaries and meta-rules for the license
are the same. It is our intent that the two licenses operate under the
same legal mechanisms and are enforced precisely the same way.

We strike a difference only in the early portions of the license.
Namely, in the LGPL we go into deeper detail of granting various permissions to
create derivative works, so the redistributors can make
some proprietary derivatives. Since we simply do not allow the
license to stretch as far as copyright law does regarding what
derivative works must be relicensed under the same terms, we must go
further to explain which derivative works we will allow to be
proprietary. Thus, we'll see that the front matter of the LGPL is a
bit more wordy and detailed with regards to the permissions granted to
those who modify or redistribute the software.

\section{Additions to the Preamble}

Most of LGPL's Preamble is identical, but the last seven paragraphs
introduce the concepts and reasoning behind creation of the license,
presenting a more generalized and briefer version of the story with which
we began our consideration of LGPL\@.

In short, FSF designed LGPL for those edge cases where the freedom of the
public can better be served by a more lax licensing system. FSF doesn't
encourage use of LGPL automatically for any software that happens to be a
library; rather, FSF suggests that it only be used in specific cases, such
as the following:

\begin{itemize}

\item To encourage the widest possible use of a Free Software library, so
  it becomes a de-facto standard over similar, although not
  interface-identical, proprietary alternatives

\item To encourage use of a Free Software library that already has
  interface-identical proprietary competitors that are more developed

\item To allow a greater number of users to get freedom, by encouraging
  proprietary companies to pick a Free alternative for its otherwise
  proprietary products

\end{itemize}

LGPL's preamble sets forth the limits to which the license seeks to go in
chasing these goals. LGPL is designed to ensure that users who happen to
acquire software linked with such libraries have full freedoms with
respect to that library. They should have the ability to upgrade to a newer
or modified Free version or to make their own modifications, even if they
cannot modify the primary software program that links to that library.

Finally, the preamble introduces two terms used throughout the license to
clarify between the different types of derivative works: ``works that use
the library,'' and ``works based on the library.''  Unlike GPL, LGPL must
draw some lines regarding derivative works. We do this here in this
license because we specifically seek to liberalize the rights afforded to
those who make derivative works. In GPL, we reach as far as copyright law
allows. In LGPL, we want to draw a line that allows some derivative works
copyright law would otherwise prohibit if the copyright holder exercised
his full permitted controls over the work.

\section{An Application: A Work that Uses the Library}

In the effort to allow certain proprietary derivative works and prohibit
others, LGPL distinguishes between two classes of derivative works:
``works based on the library,'' and ``works that use the library.''  The
distinction is drawn on the bright line of binary (or runtime) derivative
works and source code derivatives. We will first consider the definition
of a ``work that uses the library,'' which is set forth in LGPLv2.1~\S5.

We noted in our discussion of GPLv2~\S3 (discussed in
Section~\ref{GPL-Section-3} of this document) that binary programs when
compiled and linked with GPL'd software are derivative works of that GPL'd
software. This includes both linking that happens at compile-time (when
the binary is created) or at runtime (when the binary -- including library
and main program both -- is loaded into memory by the user). In GPL,
binary derivative works are controlled by the terms of the license (in GPLv2~\S3),
and distributors of such binary derivatives must release full
corresponding source\@.

In the case of LGPL, these are precisely the types of derivative works
we wish to permit. This scenario, defined in LGPL as ``a work that uses
the library,'' works as follows:

\newcommand{\workl}{$\mathcal{L}$}
\newcommand{\lplusi}{$\mathcal{L\!\!+\!\!I}$}

\begin{itemize}

\item A new copyright holder creates a separate and independent work,
  \worki{}, that makes interface calls (e.g., function calls) to the
  LGPL'd work, called \workl{}, whose copyright is held by some other
  party. Note that since \worki{} and \workl{} are separate and
  independent works, there is no copyright obligation on this new copyright
  holder with regard to the licensing of \worki{}, at least with regard to
  the source code.

\item The new copyright holder, for her software to be useful, realizes
  that it cannot run without combining \worki{} and \workl{}.
  Specifically, when she creates a running binary program, that running
  binary must be a derivative work, called \lplusi{}, that the user can
  run.

\item Since \lplusi{} is a derivative work of both \worki{} and \workl{},
  the license of \workl{} (the LGPL) can put restrictions on the license
  of \lplusi{}. In fact, this is what LGPL does.

\end{itemize}

We will talk about the specific restrictions LGPLv2.1 places on ``works
that use the library'' in detail in Section~\ref{lgpl-section-6}. For
now, focus on the logic related to how the LGPLv2.1 places requirements on
the license of \lplusi{}. Note, first of all, the similarity between
this explanation and that in Section~\ref{separate-and-independent},
which discussed the combination of otherwise separate and independent
works with GPL'd code. Effectively, what LGPLv2.1 does is say that when a
new work is otherwise separate and independent, but has interface
calls out to an LGPL'd library, then it is considered a ``work that
uses the library.''

In addition, the only reason that LGPLv2.1 has any control over the licensing
of a ``work that uses the library'' is for the same reason that GPL has
some say over separate and independent works. Namely, such controls exist
because the {\em binary combination\/} (\lplusi{}) that must be created to
make the separate work (\worki{}) at all useful is a derivative work of
the LGPLv2.1'd software (\workl{}).

Thus, a two-question test that will help indicate if a particular work is
a ``work that uses the library'' under LGPLv2.1 is as follows:

\begin{enumerate}

\item Is the source code of the new copyrighted work, \worki{}, a
  completely independent work that stands by itself, and includes no
  source code from \workl{}?

\item When the source code is compiled, does it create a derivative work
  by combining with \workl{}, either by static (compile-time) or dynamic
  (runtime) linking, to create a new binary work, \lplusi{}?
\end{enumerate}

If the answers to both questions are ``yes,'' then \worki{} is most likely
a ``work that uses the library.''  If the answer to the first question
``yes,'' but the answer to the second question is ``no,'' then most likely
\worki{} is neither a ``work that uses the library'' nor a ``work based on
the library.''  If the answer to the first question is ``no,'' but the
answer to the second question is ``yes,'' then an investigation into
whether or not \worki{} is in fact a ``work based on the library'' is
warranted.

\section{The Library, and Works Based On It}

In short, a ``work based on the library'' could be defined as any
derivative work of LGPL'd software that cannot otherwise fit the
definition of a ``work that uses the library.''  A ``work based on the
library'' extends the full width and depth of copyright derivative works,
in the same sense that GPL does.

Most typically, one creates a ``work based on the library'' by directly
modifying the source of the library. Such a work could also be created by
tightly integrating new software with the library. The lines are no doubt
fuzzy, just as they are with GPL'd works, since copyright law gives us no
litmus test for derivative works of a software program.

Thus, the test to use when considering whether something is a ``work
based on the library'' is as follows:

\begin{enumerate}

\item Is the new work, when in source form, a derivative work under
  copyright law of the LGPL'd work?

\item Is there no way in which the new work fits the definition of a
  ``work that uses the library''?
\end{enumerate}


If the answer is ``yes'' to both these questions, then you most likely
have a ``work based on the library.''  If the answer is ``no'' to the
first but ``yes'' to the second, you are in a gray area between ``work
based on the library'' and a ``work that uses the library.''

In our years of work with the LGPLv2.1, however, we have never seen a work
of software that was not clearly one or the other; the line is quite
bright. At times, though, we have seen cases where a derivative work
appeared in some ways to be a work that used the library and in other
ways a work based on the library. We overcame this problem by
dividing the work into smaller subunits. It was soon discovered that
what we actually had were three distinct components: the original
LGPL'd work, a specific set of works that used that library, and a
specific set of works that were based on the library. Once such
distinctions are established, the licensing for each component can be
considered independently and the LGPLv2.1 applied to each work as
prescribed.


\section{Subtleties in Defining the Application}

In our discussion of the definition of ``works that use the library,'' we
left out a few more complex details that relate to lower-level programming
details. The fourth paragraph of LGPLv2.1~\S5 covers these complexities,
and it has been a source of great confusion. Part of the confusion comes
because a deep understanding of how compiler programs work is nearly
mandatory to grasp the subtle nature of what LGPLv2.1~\S5, \P 4 seeks to
cover. It helps some to note that this is a border case that we cover in
the license only so that when such a border case is hit, the implications
of using LGPL continue in the expected way.

To understand this subtle point, we must recall the way that a compiler
operates. The compiler first generates object code, which are the binary
representations of various programming modules. Each of those modules is
usually not useful by itself; it becomes useful to a user of a full program
when those modules are {\em linked\/} into a full binary executable.

As we have discussed, the assembly of modules can happen at compile-time
or at runtime. Legally, there is no distinction between the two --- both
create a derivative work by copying and combining portions of one work and
mixing them with another. However, under LGPL, there is a case in the
compilation process where the legal implications are different.
Specifically, while we know that a ``work that uses the library'' is one
whose final binary is a derivative work, but whose source is not, there
are cases where the object code --- that intermediate step between source
and final binary --- is a derivative work created by copying verbatim code
from the LGPL'd software.

For efficiency, when a compiler turns source code into object code, it
sometimes places literal portions of the copyrighted library code into the
object code for an otherwise separate independent work. In the normal
scenario, the derivative would not be created until final assembly and
linking of the executable occurred. However, when the compiler does this
efficiency optimization, at the intermediate object code step, a
derivative work is created.

LGPLv2.1~\S5\P4 is designed to handle this specific case. The intent of
the license is clearly that simply compiling software to ``make use'' of
the library does not in itself cause the compiled work to be a ``work
based on the library.''  However, since the compiler copies verbatim,
copyrighted portions of the library into the object code for the otherwise
separate and independent work, it would actually cause that object file to be a
``work based on the library.''  It is not FSF's intent that a mere
compilation idiosyncrasy would change the requirements on the users of the
LGPLv2.1'd software. This paragraph removes that restriction, allowing the
implications of the license to be the same regardless of the specific
mechanisms the compiler uses underneath to create the ``work that uses the
library.''

As it turns out, we have only once had anyone worry about this specific
idiosyncrasy, because that particular vendor wanted to ship object code
(rather than final binaries) to their customers and was worried about
this edge condition. The intent of clarifying this edge condition is
primarily to quell the worries of software engineers who understand the
level of verbatim code copying that a compiler often does, and to help
them understand that the full implications of LGPLv2.1 are the same regardless
of the details of the compilation progress.

\section{LGPLv2.1~\S6 \& LGPLv2.1~\S5: Combining the Works}
\label{lgpl-section-6}
Now that we have established a good working definition of works that
``use'' and works that ``are based on'' the library, we will consider the
rules for distributing these two different works.

The rules for distributing ``works that use the library'' are covered in
LGPLv2.1~\S6\@. LGPLv2.1~\S6 is much like GPLv2~\S3, as it requires the release
of source when a binary version of the LGPL'd software is released. Of
course, it only requires that source code for the library itself be made
available. The work that ``uses'' the library need not be provided in
source form. However, there are also conditions in LGPLv2.1~\S6 to make sure
that a user who wishes to modify or update the library can do so.

LGPLv2.1~\S6 lists five choices with regard to supplying library source
and granting the freedom to modify that library source to users. We
will first consider the option given by \S~6(b), which describes the
most common way currently used for LGPLv2.1 compliance on a ``work that
uses the library.''

LGPLv2.1~\S6(b) allows the distributor of a ``work that uses the library'' to
simply use a dynamically linked, shared library mechanism to link with the
library. This is by far the easiest and most straightforward option for
distribution. In this case, the executable of the work that uses the
library will contain only the ``stub code'' that is put in place by the
shared library mechanism, and at runtime the executable will combine with
the shared version of the library already resident on the user's computer.
If such a mechanism is used, it must allow the user to upgrade and
replace the library with interface-compatible versions and still be able
to use the ``work that uses the library.''  However, all modern shared
library mechanisms function as such, and thus LGPLv2.1~\S6(b) is the simplest
option, since it does not even require that the distributor of the ``work
2based on the library'' ship copies of the library itself.

LGPLv2.1~\S6(a) is the option to use when, for some reason, a shared library
mechanism cannot be used. It requires that the source for the library be
included, in the typical GPL fashion, but it also has a requirement beyond
that. The user must be able to exercise her freedom to modify the library
to its fullest extent, and that means recombining it with the ``work based
on the library.''  If the full binary is linked without a shared library
mechanism, the user must have available the object code for the ``work
based on the library,'' so that the user can relink the application and
build a new binary.

The remaining options in LGPLv2.1~\S6 are very similar to the other choices
provided by GPLv2~\S3. There are some additional options, but time does
not permit us in this course to go into those additional options. In
almost all cases of distribution under LGPL, either LGPLv2.1~\S6(a) or LGPLv2.1~\S6(b) are
exercised.

\section{Distribution of the Combined Works}

Essentially, ``works based on the library'' must be distributed under the
same conditions as works under full GPL\@. In fact, we note that 
LGPLv2.1~\S2 is nearly identical in its terms and requirements to GPLv2~\S2.
There are again subtle differences and additions, which time does not
permit us to cover in this course.

\section{And the Rest}

The remaining variations between LGPL and GPL cover the following
conditions:

\begin{itemize}

\item Allowing a licensing ``upgrade'' from LGPL to GPL\@ (in LGPLv2.1~\S3)

\item Binary distribution of the library only, covered in LGPLv2.1~\S4,
  which is effectively equivalent to LGPLv2.1~\S3

\item Creating aggregates of libraries that are not derivative works of
  each other, and distributing them as a unit (in LGPLv2.1~\S7)

\end{itemize}


Due to time constraints, we cannot cover these additional terms in detail,
but they are mostly straightforward. The key to understanding LGPLv2.1 is
understanding the difference between a ``work based on the library'' and a
``work that uses the library.''  Once that distinction is clear, the
remainder of LGPLv2.1 is close enough to GPL that the concepts discussed in
our more extensive GPL unit can be directly applied.

%%%%%%%%%%%%%%%%%%%%%%%%%%%%%%%%%%%%%%%%%%%%%%%%%%%%%%%%%%%%%%%%%%%%%%%%%%%%%%%
\chapter{Integrating the GPL into Business Practices}

Since GPL'd software is now extremely prevalent through the industry, it
is useful to have some basic knowledge about using GPL'd software in
business and how to build business models around GPL'd software.

\section{Using GPL'd Software In-House}

As discussed in Sections~\ref{GPLv2s0} and~\ref{GPLv2s5} of this tutorial,
the GPL only governs the activities of copying, modifying and
distributing software programs that are not governed by the license.
Thus, in FSF's view, simply installing the software on a machine and
using it is not controlled or limited in any way by GPL\@. Using Free
Software in general requires substantially fewer agreements and less
license compliance activity than any known proprietary software.

Even if a company engages heavily in copying the software throughout the
enterprise, such copying is not only permitted by GPLv2~\S\S1 and 3, but it is
encouraged!  If the company simply deploys unmodified (or even modified)
Free Software throughout the organization for its employees to use, the
obligations under the license are very minimal. Using Free Software has a
substantially lower cost of ownership --- both in licensing fees and in
licensing checking and handling -- than the proprietary software
equivalents.

\section{Business Models}
\label{Business Models}

Using Free Software in house is certainly helpful, but a thriving
market for Free Software-oriented business models also exists. There is the
traditional model of selling copies of Free Software distributions.
Many companies, including IBM and Red Hat, make substantial revenue
from this model. IBM primarily chooses this model because they have
found that for higher-end hardware, the cost of the profit made from
proprietary software licensing fees is negligible. The real profit is
in the hardware, but it is essential that software be stable, reliable
and dependable, and the users be allowed to have unfettered access to
it. Free Software, and GPL'd software in particular (because IBM can
be assured that proprietary versions of the same software will not
exists to compete on their hardware) is the right choice.

Red Hat has actually found that a ``convenience fee'' for Free Software,
when set at a reasonable price (around \$60 or so), can produce some
profit. Even though Red Hat's system is fully downloadable on their
Web site, people still go to local computer stores and buy copies of their
box set, which is simply a printed version of the manual (available under
a Free license as well) and the Free Software system it documents.

\medskip

However, custom support, service, and software improvement contracts
are the most widely used models for GPL'd software. The GPL is
central to their success, because it ensures that the code base
remains common, and that large and small companies are on equal
footing for access to the technology. Consider, for example, the GNU
Compiler Collection (GCC). Cygnus Solutions, a company started in the
early 1990s, was able to grow steadily simply by providing services
for GCC --- mostly consisting of new ports of GCC to different or new,
embedded targets. Eventually, Cygnus was so successful that
it was purchased by Red Hat where it remains a profitable division.

However, there are very small companies like CodeSourcery, as well as
other medium-sized companies like MontaVista and OpenTV that compete in
this space. Because the code-base is protect by GPL, it creates and
demands industry trust. Companies can cooperate on the software and
improve it for everyone. Meanwhile, companies who rely on GCC for their
work are happy to pay for improvements, and for ports to new target
platforms. Nearly all the changes fold back into the standard
versions, and those forks that exist remain freely available.

\medskip

\label{Proprietary Relicensing}

A final common business model that is perhaps the most controversial is
proprietary relicensing of a GPL'd code base. This is only an option for
software in which a particular entity is the sole copyright holder. As
discussed earlier in this tutorial, a copyright holder is permitted under
copyright law to license a software system under her copyright as many
different ways as she likes to as many different parties as she wishes.

Some companies, such as MySQL AB and TrollTech, use this to their
financial advantage with regard to a GPL'd code base. The standard
version is available from the company under the terms of the GPL\@.
However, parties can purchase separate proprietary software licensing for
a fee.

This business model is problematic because it means that the GPL'd code
base must be developed in a somewhat monolithic way, because volunteer
Free Software developers may be reluctant to assign their copyrights to
the company because it will not promise to always and forever license the
software as Free Software. Indeed, the company will surely use such code
contributions in proprietary versions licensed for fees.

\section{Ongoing Compliance}

GPL compliance is in fact a very simple matter -- much simpler than
typical proprietary software agreements and EULAs. Usually, the most
difficult hurdle is changing from a proprietary software mindset to one
that seeks to foster a community of sharing and mutual support. Certainly
complying with the GPL from a users' perspective gives substantially fewer
headaches than proprietary license compliance.

For those who go into the business of distributing {\em modified\\}
versions of GPL'd software, the burden is a bit higher, but not by
much. The glib answer is that by releasing the whole product as Free
Software, it is always easy to comply with the GPL. However,
admittedly to the dismay of FSF, many modern and complex software
systems are built using both proprietary and GPL'd components that are
not legally derivative works of each other. Sometimes, it is easier simply to
improve existing GPL'd application than to start from scratch. In
exchange for that benefit, the license requires that the modifier give
back to the commons that made the work easier in the first place. It is a
reasonable trade-off and a way to help build a better world while also
making a profit.

Note that FSF does provide services to assist companies who need
assistance in complying with the GPL. You can contact FSF's GPL
Compliance Labs at $<$compliance@fsf.org$>$.

If you are particularly interested in matters of GPL compliance, we
recommend the second course in this series, {\em GPL Compliance Case
  Studies and Legal Ethics in Free Software Licensing\/}, in which we
discuss some real GPL violation cases that FSF has worked to resolve.
Consideration of such cases can help give insight on how to handle GPL
compliance in new situations.


% =====================================================================
% END OF FIRST DAY SEMINAR SECTION
% =====================================================================
