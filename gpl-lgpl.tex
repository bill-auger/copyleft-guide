% gpl-buisness.tex                                                -*- LaTeX -*-
%      Tutorial Text for the GPL for Businesspeople and Developers course
%
% Copyright (C) 2003 Free Software Foundation, Inc.

% Verbatim copying and distribution of this entire document is permitted in
% any medium, provided this notice is preserved.

\documentclass[12pt]{report}
% FILTER_PS:  \usepackage[dvips]{graphicx}

\newcommand\href[2]{#2}

%\newcommand\textcolor[2]{#2}

\input{/usr/share/texmf/tex/plain/dvips/colordvi.tex}

% FILTER_PDF: \usepackage[pdftex]{graphicx}
\usepackage[
pdftex=true,
latex2html=false,
pdftitle={The GNU General Public License for Businesspeople and Developers},
pdfauthor={Bradley M. Kuhn},
pdfsubject={GNU General Public License},
pdfkeywords={computer, science, free, software, freedom, licensing, licenses, GPL, GNU, general, public, license}
]{hyperref}

% I could not get this to work!
%\usepackage[pdftex,usenames]{color}
% So I used this:

\input{/usr/share/texmf/pdftex/plain/misc/pdfcolor.tex}

%\usepackage[ref]{backref}

% FILTER_HTML: 
%\usepackage{html}

%\usepackage[dvips]{graphicx}

%\usepackage[latex2html]{hyperref}
\newcommand\href[2]{\htmladdnormallink{#1}{#2}}
%\renewcommand\cite[1]{\hypercite{#1}}
%\renewcommand\ref[1]{\hyperref{#1}}

\usepackage[usenames]{color}

%\input{/usr/share/texmf/tex/plain/dvips/colordvi.tex}

% one-inch-margins.tex                                            -*- LaTeX -*-
%      code to create one inch margins in LaTeX
%
% COPYRIGHT (C) 1994 Bradley M. Kuhn
%
% Written   :   Bradley M. Kuhn         Loyola College
%   By


%% This code creates one inch margins for a LaTeX document

\oddsidemargin 0in
\evensidemargin 0in
\textwidth 6.5in

\topmargin 0in
\textheight 8.0in


%\setlength\parskip{0.7em}
%\setlength\parindent{0pt}

\newcommand{\defn}[1]{\emph{#1}}

%\pagestyle{empty}

\begin{document}

\begin{titlepage}

{\Large

\begin{center}

\vspace{.5in}

{\sc The GNU General Public License for Businesspeople and Developers } \\

\vspace{1in}

A Tutorial By:

\vspace{.3in}
Bradley M. Kuhn

Executive Director

Free Software Foundation


\end{center}
}

\vfill

{\parindent 0in
Copyright \copyright{} 2003 \hspace{.2in} Free Software Foundation, Inc.

\vspace{.3in}

Verbatim copying and distribution of this entire document is permitted in
any medium, provided this notice is preserved.
}

\end{titlepage}

\pagestyle{plain}
\pagenumbering{roman}

\begin{abstract}

This tutorial gives a section-by-section explanation of the most popular
Free Software copyright license, the GNU General Public License (GNU GPL),
and teaches software developers, managers and businesspeople how to use
the GPL and GPL'ed software successfully in new Free Software business and
in existing, successful enterprises.

Attendees should have a general familiarity with software development
processes.  A vague understanding of how copyright law applies to software
is also helpful.  The tutorial is of most interest to software developers
and managers who run software businesses that modify and/or redistribute
software under terms of the GNU GPL (or who wish to do so in the future),
and those who wish to make use of existing GPL'ed software in their
enterprise.

This tutorial introduces the GNU GPL and its terms to professionals who
are not well versed in the details of copyright law.  Presented by a
software developer and manager, this tutorial informs those who wish to
have a deeper understanding of how the GNU GPL uses copyright law to
protect software freedom and to assist in the formation of Free Software
businesses, and of the organizational motivations behind the GNU GPL.

Upon completion of the tutorial, successful attendees can expect to have
learned the following:

\begin{itemize}

  \item the freedom-defending purpose of each term of the GNU GPL.

  \item the redistribution options under the GPL.

  \item the obligations when modifying GPL'ed software.

  \item how to properly apply the GPL to a new software.

  \item how to build a plan for proper and successful compliance with the GPL.

  \item the business advantages that the GPL provides.

  \item the most common business models used in conjunction with the GPL.

  \item how existing GPL'ed software can be used in existing enterprises.
\end{itemize}

\end{abstract}

\tableofcontents

\pagebreak

\pagenumbering{arabic}

%%%%%%%%%%%%%%%%%%%%%%%%%%%%%%%%%%%%%%%%%%%%%%%%%%%%%%%%%%%%%%%%%%%%%%%%%%%%%%%
\chapter{What Is Free Software?}

Consideration of the GNU General Public License (herein, abbreviated as
\defn{GNU GPL} or just \defn{GPL}) must begin by first considering the broader
world of Free Software.  The GPL was not created from a void, rather,
it was created to embody and defend a set of principles that were set
forth at the founding of the GNU project and the Free Software Foundation
(FSF)---the organization that upholds, defends and promotes the philosophy
of software freedom.  A prerequisite for understanding the GPL and its
terms and conditions is a basic understanding of the principles behind it.
The GPL is unlike almost all other software licenses in that it is
designed to defend and uphold these principles.

\section{The Free Software Definition}
\label{Free Software Definition}

The Free Software Definition is set forth in full on FSF's website at
\href{http://www.fsf.org/philosophy/free-sw.html}{http://www.fsf.org/philosophy/free-sw.html}.
This section presents an abbreviated version that will focus on the parts
that are most pertinent to the terms of the GPL\@.

A particular program is Free Software if it grants a particular user of
that program, the following freedoms:

\begin{itemize}

\item the freedom to run the program for any purpose.

\item the freedom to change and modify the program.

\item the freedom to copy and share the program.

\item the freedom to share improved versions of the program.

\end{itemize}

The focus on ``a particular user'' is very pertinent here.  It is not
uncommon for the same version of a specific program to grant these
freedoms to some subset of its user base, while others have none or only
some of these freedoms.  Section~\ref{relicensing} talks in detail about
how this can happen even if a program is released under the GPL\@.

Some people refer to software that gives these freedoms as ``Open
Source''.  Besides having a different political focus than those who call
it Free Software\footnote{The political differences between the Free
Software Movement and the Open Source Movement are documented on FSF's
website at
\href{http://www.fsf.org/philosophy/free-software-for-freedom.html}
{http://www.gnu.org/philosophy/free-software-for-freedom.html}.},
those who call the software ``Open Source'' are focused on a side issue.
User access to the source code of a program is a prerequisite to make use
of the freedom to modify.  However, the important issue is what freedoms
are granted in the license of that source code.  Microsoft's ``Shared
Source'' program, for example, gives various types of access to source
code, but almost none of the freedoms described in this section.

One key issue that is central to these freedoms is that there are no
restrictions on how these freedoms can be exercised.  Specifically, users
and programmers can exercise these freedoms non-commercially or
commercially.  Licenses that grant these freedoms for non-commercial
activities but prohibit them for commercial activities are considered
non-Free.

In general, software for which most or all of these freedoms are
restricted in any way is called ``non-Free Software''.  Typically, the
term ``proprietary software'' is used more or less interchangeably with
``non-Free Software''.  Personally, I tend to use the term ``non-Free
Software'' to refer to non-commercial software that restricts freedom
(such as ``shareware'') and ``proprietary software'' to refer to
commercial software that restricts freedom (such as nearly all of
Microsoft's and Oracle's offerings).

The remainder of this section considers each of the four freedoms in
detail.

\subsection{The Freedom to Run}

For a program to be Free Software, the freedom to run the program must be
completely unrestricted.  This means that any use for that software that
the user can come up with must be permitted.  Perhaps, for example, the
user has discovered an innovative new use for a particular program, one
that the programmer never could have predicted.  Such a use much not be
restricted.

It was once rare that this freedom was restricted by even proprietary
software; today it is not so rare.  Most End User Licensing Agreements
(EULAs) that cover most proprietary software restrict some types of use.
For example, some versions of Microsoft's FrontPage software prohibit use
of the software to create websites that generate negative publicity for
Microsoft.  Free Software has no such restrictions; everyone is free to
use Free Software for any purpose whatsoever.

\subsection{The Freedom to Change and Modify}

Free Software programs allow users to change, modify and adapt the
software to suit their needs.  Access to the source code and related build
scripts are an essential part of this freedom.  Without the source code
and the ability to build the binary applications from that source, the
freedom cannot be properly exercised.

Programmers can take direct benefit from this freedom, and often do.
However, this freedom is essential to users who are not programmers.
Users must have the right to engage in a non-commercial environment of
finding help with the software (as often happens on email lists and in
users groups).  This means they must have the freedom to recruit
programmers who might altruistically assist them to modify their software.

The commercial exercise of this freedom is also essential.  Each user, or
group of users, must have the right to hire anyone they wish on a
competitive free market to modify and change the software.  This means
that companies have a right to hire anyone they wish to modify their Free
Software.  Additionally, such companies may contract with other companies
to commission software modification.

\subsection{The Freedom to Copy and Share}

Users may share Free Software in a variety of ways.  Free Software
advocates work to eliminate fundamental ethical dilemma of the software
age: choosing between obeying a software license, and friendship (by
giving away a copy of a program your friend who likes the software you are
using).  Free Software licenses, therefore, must permit this sort of
altruistic sharing of software among friends.

The commercial environment must also have the benefits of this freedom.
Commercial sharing typically takes the form of selling copies of Free
Software.  Free Software can be sold at any price to anyone.  Those who
redistribute Free Software commercially have the freedom to selectively
distribute (you can pick your customers) and to set prices at any level
the redistributor sees fit.

It is true that many people get copies of Free Software very cheaply (and
sometimes without charge). The competitive free market of Free Software
tends to keep prices low and reasonable.  However, if someone is willing
to pay a billion dollars for one copy of the GNU Compiler Collection, such
a sale is completely permitted.

Another common instance of commercial sharing is service-oriented
distribution.  For example, a distribution vendor may provide immediate
security and upgrade distribution via a special network service.  Such
distribution is completely permitted for Free Software.

\subsection{The Freedom to Share Improvements}

The freedom to modify and improve is somewhat empty without the freedom to
share those improvements.  The Free Software community is built on the
pillar of altruistic sharing of improved Free Software.  Inevitably, a
Free Software project sprouts a mailing list where improvements are shared
freely among members of the development community.  Such non-commercial
sharing must be permitted for Free Software to thrive.

Commercial sharing of modified Free Software is equally important.  For a
competitive free market for support to exist, all developers --- from
single-person contractors to large software companies --- must have the
freedom to market their services as improvers of Free Software.  All forms
of such service marketing must be equally available to all.

For example, selling support services for Free Software is fully
permitted.  Companies and individuals can offer themselves as ``the place
to call'' when software fails or does not function properly.  For such a
service to be meaningful, the entity offering that service must have the
right to modify and improve the software for the customer to correct any
problems that are beyond mere user error.

Entities must also be permitted to make available modified versions of
Free Software.  Most Free Software programs have a so-called ``canonical
version'' that is made available from the primary developers of the
software.  However, all who have the software have the ``freedom to fork''
--- that is, make available non-trivial modified versions of the software
on a permanent or semi-permanent basis.  Such freedom is central to
vibrant developer and user interaction.

Companies and individuals have the right to make true value-added versions
of Free Software.  They may use freedom to share improvements to
distribute distinct versions of Free Software with different functionality
and features.  Furthermore, this freedom can be exercised to serve a
disenfranchised subset of the user community.  If the developers of the
canonical version refuse to serve the needs of some of the software's
users, other entities have the right to create long- or short-lived fork
that serves that sub-community.

\section{How Does Software Become Free?}

The last section set forth the freedoms and rights are respected by Free
Software.  It presupposed, however, that such software exists.  This
section discusses how Free Software comes into existence.  But first, it
addresses how software can be non-free in the first place.

Software can be made proprietary only because it is governed by copyright
law\footnote{This statement is a bit of an oversimplification.  Patents
  and trade secrets can cover software and make it effectively non-free,
  and one can contract away their rights and freedoms regarding software.
  However, the primary control mechanism for software is copyright.}.
Copyright law, with respect to software governs copying, modifying, and
redistributing that software\footnote{Copyright law in general also
  governs ``public performance'' of copyrighted works.  There is no
  generally agreed definition for public performance of software and
  version 2 of the GPL does not govern public performance.}.  By law, the
copyright holder (aka the author) of the work controls how others my copy,
modify and/or distribute the work.  For proprietary software, these
controls are used to prohibit these activities.  In addition, proprietary
software distributors further impede modification in a practical sense by
distributing only binary code and keeping the source code of the software
secret.

Copyright law is a construction.  In the USA, the Constitution permits,
but does not require, the creation of copyright law as federal
legislation.  Software, since it is tangible expression of an idea, is
thus covered by the statues, and is copyrighted by default.

However, software, in its natural state without copyright, is Free
Software.  In an imaginary world, which has no copyright, the rules would
be different.  In this world, when you received a copy of a program's
source code, there would be no default legal system to restrict you from
sharing it with others, making modifications, or redistributing those
modified versions\footnote{There could still exist legal systems, like our
  modern patent system, which could restrict the software in other ways.}.

Software in the real world is copyrighted by default, and that default
legal system does exist.  However, it is possible to move software out of
the domain of the copyright system.  A copyright holder is always
permitted to \defn{disclaim} their copyright.  If copyright is disclaimed,
the software is not governed by copyright law.  Software not governed by
copyright is in the ``public domain''.

\subsection{Public Domain Software}

An author can create public domain software by disclaiming all copyright
interest on the work.  In the USA and other countries that have signed the
Berne convention on copyright, software is copyrighted automatically by
the author when (s)he ``fixes the software into a tangible medium''.  In
the software world, this usually means typing the source code of the
software into a file.

However, an author can disclaim that default control given to her by the
copyright laws.  Once this is done, the software is in the public domain
--- it is no longer covered by copyright.  Since it is copyright law that
allows for various controls on software (i.e., prohibition of copying,
modification, and redistribution), removing the software from the
copyright system and placing it into the public domain does yield Free
Software.

Carefully note that software in the public domain is \emph{not} licensed
in any way.  It is nonsensical to say software is ``licensed for the
public domain'', or any phrase that implies the copyright holder gave an
expressed permission to take actions governed by copyright law.

By contrast, what the copyright holder has done is renounce her copyright
controls on the work.  The law gave her controls over the work, and she
has chosen to waive those controls.  Software in the public domain is
absent copyright and absent a license.  The software freedoms discussed in
Section~\ref{Free Software Definition} are all granted because there is no
legal system in play to take them away.

\subsection{Why Copyright Free Software?}

If simply disclaiming copyright on software yields Free Software, then it
stands to reason that putting software into the public domain is the
easiest and most straightforward way to produce Free Software.  Indeed,
some major Free Software projects have chosen this method for making their
software Free.  However, most of the Free Software in existence \emph{is}
copyrighted.  In most cases (particularly in that of FSF and the GNU
Project), this was done due to very careful planning.

Software released into the public domain does grant freedom to those users
who receive the canonical versions on which the original author disclaimed
copyright.  However, since the work is not copyrighted, any non-trivial
modification made to the work is fully copyrightable.

Free Software released into the public domain initially is Free, and
perhaps some who modify the software choose to place their work into the
public domain as well.  However, over time, some entities will choose to
proprietarize their modified versions.  The public domain body of software
feeds the proprietary software.  The public commons disappears, because
fewer and fewer entities have an incentive to contribute back to the
commons, since they know that any of their competitors can proprietarize
their enhancements.  Over time, almost no interesting work is left in the
public domain, because nearly all new work is done by proprietarization.

A legal mechanism is needed to redress this problem.  FSF was in fact
originally created primarily as a legal entity to defend software freedom,
and that work of of defending software freedom is a substantial part of
its work today.  Specifically because of this ``embrace, proprietarize and
extend'' cycle, FSF made a conscious choice to copyright its Free Software,
and then license it under ``copyleft'' terms, and many, including the
developers of the kernel named Linux has chosen to follow this paradigm.

Copyleft is a legal strategy to defend, uphold and propagate software
freedom.  The basic technique of copyleft is as follows: copyright the
software, license it under terms that give all the software freedoms, but
use the copyright law controls to ensure that all who receive a copy of
the software have equal rights and freedom.  In essence, copyleft grants
freedom, but forbids others to forbid that freedom from anyone else along
the distribution and modification chains.

Copyleft is a general concept.  Much like ideas for what a computer might
do must be \emph{implemented} by a program that actually does the job, so
too must copyleft be implemented in some concrete legal structure.
``Share and share alike'' is a phrase that is often enough to explain the
concept behind copyleft, but to actually make it work in the real world, a
true implementation in legal text must exist.  The GPL is the primary
implementation of copyleft in copyright licensing language.

\section{An Ecosystem of Equality}

The GPL uses copyright law to defend freedom and equally ensure users'
rights.  This ultimately creates an ecosystem of equality for both
business and non-commercial users.

\subsection{The Non-Commercial Ecosystem}

A GPL'ed code base becomes a center of a vibrant development and user
community.  Traditionally, volunteers, operating non-commercially out of
keen interest or ``scratch an itch'' motivations, produce initial versions
of a GPL'ed system.  Because of the efficient distribution channels of the
Internet, any useful GPL'ed system is adopted quickly by non-commercial
users.

Fundamentally, the early release and quick distribution of the software
gives birth to a thriving non-commercial community.  Users and developers
begin sharing bug reports and bug fixes across a shared intellectual
commons.  Users can trust the developers, because they know that if the
developers fail to address their needs or abandon the project, the GPL
ensures that someone else has the right to pick up development.
Developers know that the users cannot redistribute their software without
passing along the rights granted by GPL, so they are assured that every
one of their users is treated equally.

Because of the symmetry and fairness inherent in GPL'ed distribution,
nearly every GPL'ed package in existence has a vibrant non-commercial user
and developer base.

\subsection{The Commercial Ecosystem}

By the same token, nearly all established GPL'ed software systems have a
vibrant commercial community.  Nearly every GPL'ed system that has gained
wide adoption from non-commercial users and developers eventually begins
to fuel a commercial system around that software.

For example, consider the Samba file server system that allows Unix-like
systems (including GNU/Linux) to serve files to Microsoft Windows systems.
Two graduate students originally developed Samba in their spare time and
it was deployed non-commercially in academic environments.  However, very
soon for-profit companies discovered that the software could work for them
as well, and their system administrators began to use it in place of
Microsoft Windows NT file-servers.  This served to lower the cost of
ownership by orders of magnitude.  There was suddenly room in Windows
file-server budgets to hire contractors to improve Samba.  Some of the first
people hired to do such work were those same two graduate students who
originally developed the software.

The non-commercial users, however, were not concerned when these two
fellows began collecting paychecks off of their GPL'ed work.  They knew
that because of the nature of the GPL that improvements that were
distributed in the commercial environment could easily be folded back into
the canonical version.  Companies are not permitted to proprietarize
Samba, so the non-commercial users, and even other commercial users are
safe in the knowledge that the software freedom ensured by GPL will remain
protected.

Commercial developers also work in concert with non-commercial developers.
Those two now-long-since graduated students continue to contribute to
Samba altruistically, but also get work doing it.  Priorities change when a
client is in the mix, but all the code is contributed back to the
canonical version.  Meanwhile, many other individuals have gotten involved
non-commercially as developers, because they want to ``cut their teeth on
Free Software'' or because the problem interest them.  When they get good
at it, perhaps they will move on to another project or perhaps they will
become commercial developers of the software themselves.

No party is a threat to another in the GPL software scenario because
everyone is on equal ground.  The GPL protects rights of the commercial
and non-commercial contributors and users equally.  The GPL creates trust,
because it is a level playing field for all.

\subsection{Law Analogy}

In his introduction to Stallman's \emph{Free Software, Free Society},
Lawrence Lessig draws an interesting analogy between the law and Free
Software.  He argues that the laws of a Free society must be protected
much like the GPL protects software.  So that I might do true justice to
Lessig's argument, I quote it verbatim:

\begin{quotation}

A ``free society'' is regulated by law. But there are limits that any free
society places on this regulation through law: No society that kept its
laws secret could ever be called free. No government that hid its
regulations from the regulated could ever stand in our tradition. Law
controls.  But it does so justly only when visibly. And law is visible
only when its terms are knowable and controllable by those it regulates,
or by the agents of those it regulates (lawyers, legislatures).

This condition on law extends beyond the work of a legislature.  Think
about the practice of law in American courts.  Lawyers are hired by their
clients to advance their clients' interests. Sometimes that interest is
advanced through litigation. In the course of this litigation, lawyers
write briefs.  These briefs in turn affect opinions written by judges.
These opinions decide who wins a particular case, or whether a certain law
can stand consistently with a constitution.

All the material in this process is free in the sense that Stallman means.
Legal briefs are open and free for others to use.  The arguments are
transparent (which is different from saying they are good) and the
reasoning can be taken without the permission of the original lawyers.
The opinions they produce can be quoted in later briefs.  They can be
copied and integrated into another brief or opinion.  The ``source code''
for American law is by design, and by principle, open and free for anyone
to take. And take lawyers do---for it is a measure of a great brief that
it achieves its creativity through the reuse of what happened before.  The
source is free; creativity and an economy is built upon it.

This economy of free code (and here I mean free legal code) doesn't starve
lawyers.  Law firms have enough incentive to produce great briefs even
though the stuff they build can be taken and copied by anyone else.  The
lawyer is a craftsman; his or her product is public.  Yet the crafting is
not charity.  Lawyers get paid; the public doesn't demand such work
without price.  Instead this economy flourishes, with later work added to
the earlier.

We could imagine a legal practice that was different---briefs and
arguments that were kept secret; rulings that announced a result but not
the reasoning.  Laws that were kept by the police but published to no one
else.  Regulation that operated without explaining its rule.

We could imagine this society, but we could not imagine calling it
``free.''  Whether or not the incentives in such a society would be better
or more efficiently allocated, such a society could not be known as free.
The ideals of freedom, of life within a free society, demand more than
efficient application.  Instead, openness and transparency are the
constraints within which a legal system gets built, not options to be
added if convenient to the leaders.  Life governed by software code should
be no less.

Code writing is not litigation.  It is better, richer, more
productive. But the law is an obvious instance of how creativity and
incentives do not depend upon perfect control over the products created.
Like jazz, or novels, or architecture, the law gets built upon the work
that went before. This adding and changing is what creativity always is.
And a free society is one that assures that its most important resources
remain free in just this sense.\footnote{This quotation is Copyright
  \copyright{} 2002, Lawrence Lessig.  It is licensed under the terms of
  \href{http://creativecommons.org/licenses/by/1.0/}{the ``Attribution
    License'', version 1.0} or any later version as published by Creative
  Commons.}
\end{quotation}

In essence, lawyers are paid to service the shared commons of legal
infrastructure.  Few defend themselves in court or write their own briefs
(even though they legally permitted to do so) because everyone would
prefer to have an expert do that job.

The Free Software economy is a market that is ripe for experts.  It
functions similarly to other well established professional fields like the
law.  The GPL, in turn, serves as the legal scaffolding that permits the
creation of this vibrant commercial and non-commercial Free Software
economy.

%%%%%%%%%%%%%%%%%%%%%%%%%%%%%%%%%%%%%%%%%%%%%%%%%%%%%%%%%%%%%%%%%%%%%%%%%%%%%%%
\chapter{Copying, Modifying and Redistributing}

This chapter begins the deep discussion of the details of the terms of
GPL\@.  In this chapter, we consider the core terms: GPL \S\S 0--3.  These
are the sections of the GPL that fundamentally define the legal details of
how software freedom is respected.

\section{GPL \S 0: Freedom to Run}

\S 0, the opening section of GPL, sets forth that the work is governed by
copyright law.  It specifically points out that it is the ``copyright
holder'' who decides if a work is licensed under its terms, and explains
how the copyright holder might indicate this fact.

A bit more subtly, \S 0 makes an inference that copyright law is the only
system under which it is governed.  Specifically, it states:
\begin{quote}
Activities other than copying, distribution and modification are not
covered by this License; they are outside its scope.
\end{quote}
In essence, the license governs \emph{only} those activities and all other
activities are unrestricted, provided that no other agreements trump GPL
(which they cannot; see Sections~\ref{GPLs6} and~\ref{GPLs7}).  This is
very important, because the Free Software community heavily supports
users' rights to ``fair use'' and ``unregulated use'' of copyrighted
material.  GPL asserts through this clause that it supports users' rights
to fair and unregulated uses.

Fair use of copyrighted material is an established legal doctrine that
permits certain activities.  Discussion of the various types of fair use
activity are beyond the scope of this tutorial.  However, one important
example of fair use is the right to reverse engineering software.

Fair use is a doctrine established by the courts or by statute.  By
contrast, unregulated uses are those that are not covered by the statue
nor determined by a court to be covered, but are common and enjoyed by
many users.  An example of unregulated use is reading a program like a
novel for the purpose of learning how to be a better programmer.

\medskip

Thus, the GPL protects users fair and unregulated use rights precisely by
not attempting to cover them.  Furthermore, the GPL ensures the freedom
to run specifically by stating the following:
\begin{quote}
The act of running the Program is not restricted
\end{quote}
Thus, users are explicitly given the freedom to run by \S 0.

\medskip

The bulk of \S 0 not yet discussed gives definitions for other terms used
throughout.  The only one worth discussing in detail is ``work based on
the Program''.  The reason this definition is particular interesting is
not for the definition itself, which is rather straightforward, but the
because it clears up a common misconception about the GPL\@.

The GPL is often mistakenly criticized because it fails to give a
definition of ``derivative work''.  In fact, it would be incorrect and
problematic if the GPL attempt to define this.  A copyright license, in
fact, has no control over what may or may not be a derivative work.  This
matter is left up to copyright law, not the licenses that utilize it.

It is certainly true that copyright law as a whole does not propose clear
and straightforward guidelines for what is and is not a derivative
software work under copyright law.  However, no copyright license --- not
even the GNU GPL -- can be blamed for this.  Legislators and court
opinions must give us guidance to decide the border cases.

\section{GPL \S 1: Verbatim Copying}

GPL \S 1 covers the matter of redistributing the source code of a program
exactly as it was received.  This section is quite straightforward.
However, there are a few details worth noting here.

The phrase ``in any medium'' is important.  This, for example, gives the
freedom to publish a book that is the printed copy of the program's source
code.  It also allows for changes in the medium of distribution.  Some
vendors may ship Free Software on a CD, but others may place it right on
the hard drive of a pre-installed computer.  Any such redistribution media
is allowed.

Preservation of copyright notice and license notifications are mentioned
specifically in \S 1.  These are in some ways the most important part of
the redistribution, which is why they are mentioned by name.  The GPL
always strives to make it abundantly clear to anyone who receives the
software what its license is.  The goal is to leave no reason that someone
would be surprised that the software she got was licensed under GPL\@.
Thus throughout the GPL, there are specific reference to the importance of
notifying others down the distribution chain that they have rights under
GPL.

Also mentioned by name is the warranty disclaimer.  Most people today do
not believe that software comes with any warranty.  Notwithstanding the
proposed state-level UCITA bills (which have never obtained widespread
adoption), there are little or no implied warranties with software.
However, just to be on the safe side, GPL clearly disclaims them, and the
GPL requires redistributors to keep the disclaimer very visible.  (See
Sections~\ref{GPLs11} and~\ref{GPLs12} of this tutorial for more on GPL's
warranty disclaimers.)

Note finally that \S 1 begins to set forth the important defense of
commercial freedom.  \S 1 clearly states that in the case of verbatim
copies, one may make money.  Redistributors are fully permitted to charge
for the redistribution of copies of Free Software.  In addition, they may
provide the warranty protection that the GPL disclaims as an additional
service for a fee.  (See Section~\ref{Business Models} for more discussion
on making profit from Free Software redistribution.)

\section{GPL \S 2: Share and Share Alike}

Many consider \S 2 the heart and soul of the GPL\@.  For many, this is
where the ``magic'' happens that defends software freedom along the
distribution chain.  I certainly agree that if GPL has a soul, this is
where it is.  However, I argue that the heart is in fact contained in \SS
4--5 (see Section~\ref{GPLs4} and~\ref{GPLs5} of this tutorial).  But, for
the moment, let us consider the soul.

\S 2 gives the only permission in the GPL that governs the modification
controls of copyright law.  If someone modifies a GPL'ed program, she is
bound in the making those changes by \S 2.  The goal here is to ensure
that the body of GPL'ed software, as it continues and develops, remains
Free as in freedom.

To achieve that goal, \S 2 first sets forth that the rights of
redistribution modified versions are the same as those for verbatim
copying, as presented in \S 1.  Therefore, the details of charging,
keeping copyright notices in tact, and other \S 1 provisions are in tact
here as well.  However, there are three additional requirements.

The first (\S 2(a)) requires that modified files carry ``prominent
notices'' explaining what changes were made and the date of such changes.
The goal here is not to put forward some specific way of marking changes,
or controlling the process of how changes get made.  Primarily, \S 2(a)
seeks to ensure that those receiving modified versions what path it took
to them.  For some users, it is important to know that they are using the
canonical version of program, because while there are many advantages to
using a fork, there are a few disadvantages.  Users should be informed the
historical context of the software version they use, so that they can make
proper support choices.  Finally, \S 2(a) serves an academic purpose ---
ensuring that future developers can use a diachronic approach to
understand the software.

\medskip

The second requirement (\S 2(b)) contains the four short lines that embody
the legal details of ``share and share alike''.  These 46 words are
considered by some to be the most worthy of careful scrutiny.  It is worth
the effort to carefully understand what each clause is saying, because \S
2(b) can be a source of great confusion when not properly understood.

In considering \S 2(b), first note the qualifier: it only applies to
derivative works that ``you distribute or publish''.  Despite years of
education efforts by FSF on this matter, many still believe that modifiers
of GPL'ed software are required by the license to publish or otherwise
share their changes.  On the contrary, \S 2(b) {\bf does not apply if} the
changes are never distributed.  Indeed, the freedom to make private,
personal changes to software that are not shared should be protected and
defended\footnote{FSF does maintain that there is an {\bf ethical}
  obligation to redistributor changes that are generally useful, and often
  encourages companies and individuals to do so.  However, there is a
  clear distinction between what one {\bf ought} to do and what one {\bf
    must} do.}.

Next, we again encounter the same matter that appears in \S 0, in the
following text:
\begin{quote}
... that in whole or part contains or is derived from the Program or any
  part thereof,
\end{quote}
Again, the GPL relies here on what the copyright law says is a derivative
work.  If, under copyright law, the modified version ``contains or is
derived from'' the GPL'ed software, then the requirements of \S 2(b)
apply.  The GPL invokes its control as a copyright license over the
modification of the work in combination with its control over distribution
of the work.

The final clause of \S 2(b) describes what the licensee must do if she is
distributing or publishing a work that is deemed a derivative work under
copyright law --- namely, the following:
\begin{quote}
[The work must] be licensed as a whole at no charge to all third parties
under the terms of this License.
\end{quote}
That is probably the most tightly-packed phrase in all of the GPL\@.
Consider each subpart carefully.

The work ``as a whole'' is what is to be licensed.  This is an important
point that \S 2 spends an entire paragraph explaining; thus this phrase is
worthy of a lengthy discussion here.  As a programmer modifies a software
program, she generates new copyrighted material --- fixing ideas in the
tangible medium of electronic file storage.  That programmer is indeed the
copyright holder of those new changes.  However, those changes are part
and parcel to the original worked distributed to the programmer under
GPL\@.  Thus, the license of the original work affects the license of the
new whole derivative work.

% {\cal I}
\newcommand{\gplusi}{$\mathcal{G\!\!+\!\!I}$}
\newcommand{\worki}{$\mathcal{I}$}
\newcommand{\workg}{$\mathcal{G}$}

It is certainly possible to take an existing independent work (called
\worki{}) and combine it with a GPL'ed program (called \workg{}).  The
license of \worki{}, when it is distributed as a separate and independent
work, remains the prerogative of the copyright holder of .  However, when
\worki{} is combined with \workg{}, it produces a new work that is the
combination of the two (called \gplusi{}).  The copyright of this
derivative work, \gplusi{}, is jointly held by the original copyright
holder of each of the two works.

In this case, \S 2 lays out the terms by which \gplusi{} may be
distributed and copied.  By default, under copyright law, the copyright
holder of \worki{} would not have been permitted to distribute \gplusi{};
copyright law forbids it without the expressed permission of the copyright
holder of \workg{}.  (Imagine, for a moment, if \workg{} were a Microsoft
product --- would they give you permission to create and distribute
\gplusi{} without paying them a hefty sum?)  The license of \workg{}, the
GPL, sets forth ahead of time options for the copyright holder of \worki{}
who may want to create and distribute \gplusi{}.  This pre-granted
permission to create and distribute derivative works, provided the terms
of GPL are uphold, goes far above and beyond the permissions that one
would get with a typical work not covered by a copyleft license.  Thus, to
say that this restriction is any way unreasonable is simply ludicrous.

\medskip

The next phrase of note in \S 2(b) is ``licensed ... at no charge''.  This
is a source of great confusion to many.  Not a month goes by that FSF does
not receive an email that claims to point out ``a contradiction in GPL''
because \S 2 says that redistributors cannot charge for modified versions
of GPL'ed software, but \S 1 says that they can.  The ``at no charge''
means not that redistributors cannot charge for performing the acts
governed by copyright law\footnote{Recall that you could by default charge
  for any acts not governed by copyright law, because the license controls
  are confined by copyright.}, but rather that they cannot charge a fee
for the \emph{license itself}.  In other words, redistributors of
(modified and unmodified) GPL'ed works may charge any amount they choose
for performing the modifications on contract or the act of transferring
the copy to the customer, but they may not charge a separate licensing fee
for the software.

\S 2(b) further states that the software must ``be licensed ... to all
third parties''.  This too has led to some confusions, and feeds the
misconception mentioned earlier --- that all modified versions must made
available to the public at large.  However, the text here does not say
that.  Instead, it says that the licensing under terms of the GPL must
extend to anyone who might, through the typical distribution chain,
receive a copy of the software.  Distribution to all third parties is not
mandated here, but \S 2(b) does require redistributors to license the
derivative works in a way that is extends FIXME to all third parties who may
ultimately receive a copy of the software.

In summary, \S 2(b) says what terms under which the third parties must
receive this no-charge license.  Namely, they receive it ``under the terms
of this License'', the GPL.  When an entity \emph{chooses} to redistribute
a derivative work of GPL'ed software, the license of that whole derivative
work must be GPL and only GPL\@.  In this manner, \S 2(b) dovetails nicely
with \S 6 (as discussed in Section\~ref{GPLs6} of this tutorial).

\medskip

The final paragraph of \S 2 is worth special mention.  It is possible and
quite common to aggregate various software programs together on one
distribution medium.  Computer manufacturers do this when they ship a
pre-installed hard drive, and GNU/Linux distribution vendors do this to
give a one-stop CD or URL for a complete operating system with necessary
applications.  The GPL very clearly permits such ``mere aggregation'' with
programs under any license.  Despite what you hear from its critics, the
GPL is nothing like a virus, not only because the GPL is good for you and
a virus is bad for you, but also because simple contact with a GPL'ed
code-base does not impact the license of other programs.  Actual effort
must be expended by a programmer to cause a work to fall under the terms
of the GPL.  Redistributors are always welcome to simply ship GPL'ed
software alongside proprietary software or other unrelated Free Software,
as long as the terms of GPL are adhered to for those packages that are
truly GPL'ed.

\section{GPL, \S 3}

%%%%%%%%%%%%%%%%%%%%%%%%%%%%%%%%%%%%%%%%%%%%%%%%%%%%%%%%%%%%%%%%%%%%%%%%%%%%%%%
\chapter{Defending Freedom On Many Fronts}

\section{GPL, Section 4}
\label{GPLs4}

\section{GPL, Section 5}
\label{GPLs5}

\section{GPL, Section 6}
\label{GPLs6}

\section{GPL, Section 7}
\label{GPLs7}

%%%%%%%%%%%%%%%%%%%%%%%%%%%%%%%%%%%%%%%%%%%%%%%%%%%%%%%%%%%%%%%%%%%%%%%%%%%%%%%
\chapter{Odds, Ends, and Absolutely No Warranty}

\section{GPL, \S 8}
\label{GPLs8}

\section{GPL, \S 9}
\label{GPLs9}

\section{GPL, \S 10}
\label{GPLs10}

\section{GPL, \S 11}
\label{GPLs11}

There was a case where the disclaimer of a contract was negated because it
was not "conspicuous" to the person entering into the contract.  Therefore,
to make such language "conspicuous" people started placing it in bold or caps it.  My question
has always been, does that mean all the other parts of the document aren't
important such that they too need to be "conspicuous."

As for disclaiming warranties, remember that there are many types of
warranties, and in some jurisdictions some of them cannot be disclaimed.
Therefore, usually agreements will have both a warranty disclaimer and a
limitation of liability.  The former gets rid of everything that can be
gotten rid of, while the latter limits the liability of the actor for any
warranties that cannot be disclaimed (such as personal injury, etc.).

\section{GPL, \S 12}
\label{GPLs12}

%%%%%%%%%%%%%%%%%%%%%%%%%%%%%%%%%%%%%%%%%%%%%%%%%%%%%%%%%%%%%%%%%%%%%%%%%%%%%%%
\chapter{Integrating the GPL into Business Practices}

\section{Using Free Software In-House}

\section{Business Models}
\label{Business Models}

\section{Ongoing Compliance}

\appendix

\chapter{The GNU General Public License}

\begin{center}
{\parindent 0in

Version 2, June 1991

Copyright \copyright\ 1989, 1991 Free Software Foundation, Inc.

\bigskip

59 Temple Place - Suite 330, Boston, MA  02111-1307, USA

\bigskip

Everyone is permitted to copy and distribute verbatim copies
of this license document, but changing it is not allowed.
}
\end{center}

\begin{center}
{\bf\large Preamble}
\end{center}


The licenses for most software are designed to take away your freedom to
share and change it.  By contrast, the GNU General Public License is
intended to guarantee your freedom to share and change free software---to
make sure the software is free for all its users.  This General Public
License applies to most of the Free Software Foundation's software and to
any other program whose authors commit to using it.  (Some other Free
Software Foundation software is covered by the GNU Library General Public
License instead.)  You can apply it to your programs, too.

When we speak of free software, we are referring to freedom, not price.
Our General Public Licenses are designed to make sure that you have the
freedom to distribute copies of free software (and charge for this service
if you wish), that you receive source code or can get it if you want it,
that you can change the software or use pieces of it in new free programs;
and that you know you can do these things.

To protect your rights, we need to make restrictions that forbid anyone to
deny you these rights or to ask you to surrender the rights.  These
restrictions translate to certain responsibilities for you if you
distribute copies of the software, or if you modify it.

For example, if you distribute copies of such a program, whether gratis or
for a fee, you must give the recipients all the rights that you have.  You
must make sure that they, too, receive or can get the source code.  And
you must show them these terms so they know their rights.

We protect your rights with two steps: (1) copyright the software, and (2)
offer you this license which gives you legal permission to copy,
distribute and/or modify the software.

Also, for each author's protection and ours, we want to make certain that
everyone understands that there is no warranty for this free software.  If
the software is modified by someone else and passed on, we want its
recipients to know that what they have is not the original, so that any
problems introduced by others will not reflect on the original authors'
reputations.

Finally, any free program is threatened constantly by software patents.
We wish to avoid the danger that redistributors of a free program will
individually obtain patent licenses, in effect making the program
proprietary.  To prevent this, we have made it clear that any patent must
be licensed for everyone's free use or not licensed at all.

The precise terms and conditions for copying, distribution and
modification follow.

\begin{center}
{\Large \sc Terms and Conditions For Copying, Distribution and
  Modification}
\end{center}


%\renewcommand{\theenumi}{\alpha{enumi}}
\begin{enumerate}

\addtocounter{enumi}{-1}

\item

This License applies to any program or other work which contains a notice
placed by the copyright holder saying it may be distributed under the
terms of this General Public License.  The ``Program'', below, refers to
any such program or work, and a ``work based on the Program'' means either
the Program or any derivative work under copyright law: that is to say, a
work containing the Program or a portion of it, either verbatim or with
modifications and/or translated into another language.  (Hereinafter,
translation is included without limitation in the term ``modification''.)
Each licensee is addressed as ``you''.

Activities other than copying, distribution and modification are not
covered by this License; they are outside its scope.  The act of
running the Program is not restricted, and the output from the Program
is covered only if its contents constitute a work based on the
Program (independent of having been made by running the Program).
Whether that is true depends on what the Program does.

\item You may copy and distribute verbatim copies of the Program's source
  code as you receive it, in any medium, provided that you conspicuously
  and appropriately publish on each copy an appropriate copyright notice
  and disclaimer of warranty; keep intact all the notices that refer to
  this License and to the absence of any warranty; and give any other
  recipients of the Program a copy of this License along with the Program.

You may charge a fee for the physical act of transferring a copy, and you
may at your option offer warranty protection in exchange for a fee.

\item

You may modify your copy or copies of the Program or any portion
of it, thus forming a work based on the Program, and copy and
distribute such modifications or work under the terms of Section 1
above, provided that you also meet all of these conditions:

\begin{enumerate}

\item

You must cause the modified files to carry prominent notices stating that
you changed the files and the date of any change.

\item

You must cause any work that you distribute or publish, that in
whole or in part contains or is derived from the Program or any
part thereof, to be licensed as a whole at no charge to all third
parties under the terms of this License.

\item
If the modified program normally reads commands interactively
when run, you must cause it, when started running for such
interactive use in the most ordinary way, to print or display an
announcement including an appropriate copyright notice and a
notice that there is no warranty (or else, saying that you provide
a warranty) and that users may redistribute the program under
these conditions, and telling the user how to view a copy of this
License.  (Exception: if the Program itself is interactive but
does not normally print such an announcement, your work based on
the Program is not required to print an announcement.)

\end{enumerate}


These requirements apply to the modified work as a whole.  If
identifiable sections of that work are not derived from the Program,
and can be reasonably considered independent and separate works in
themselves, then this License, and its terms, do not apply to those
sections when you distribute them as separate works.  But when you
distribute the same sections as part of a whole which is a work based
on the Program, the distribution of the whole must be on the terms of
this License, whose permissions for other licensees extend to the
entire whole, and thus to each and every part regardless of who wrote it.

Thus, it is not the intent of this section to claim rights or contest
your rights to work written entirely by you; rather, the intent is to
exercise the right to control the distribution of derivative or
collective works based on the Program.

In addition, mere aggregation of another work not based on the Program
with the Program (or with a work based on the Program) on a volume of
a storage or distribution medium does not bring the other work under
the scope of this License.

\item
You may copy and distribute the Program (or a work based on it,
under Section 2) in object code or executable form under the terms of
Sections 1 and 2 above provided that you also do one of the following:

\begin{enumerate}

\item

Accompany it with the complete corresponding machine-readable
source code, which must be distributed under the terms of Sections
1 and 2 above on a medium customarily used for software interchange; or,

\item

Accompany it with a written offer, valid for at least three
years, to give any third party, for a charge no more than your
cost of physically performing source distribution, a complete
machine-readable copy of the corresponding source code, to be
distributed under the terms of Sections 1 and 2 above on a medium
customarily used for software interchange; or,

\item

Accompany it with the information you received as to the offer
to distribute corresponding source code.  (This alternative is
allowed only for noncommercial distribution and only if you
received the program in object code or executable form with such
an offer, in accord with Subsection b above.)

\end{enumerate}


The source code for a work means the preferred form of the work for
making modifications to it.  For an executable work, complete source
code means all the source code for all modules it contains, plus any
associated interface definition files, plus the scripts used to
control compilation and installation of the executable.  However, as a
special exception, the source code distributed need not include
anything that is normally distributed (in either source or binary
form) with the major components (compiler, kernel, and so on) of the
operating system on which the executable runs, unless that component
itself accompanies the executable.

If distribution of executable or object code is made by offering
access to copy from a designated place, then offering equivalent
access to copy the source code from the same place counts as
distribution of the source code, even though third parties are not
compelled to copy the source along with the object code.

\item
You may not copy, modify, sublicense, or distribute the Program
except as expressly provided under this License.  Any attempt
otherwise to copy, modify, sublicense or distribute the Program is
void, and will automatically terminate your rights under this License.
However, parties who have received copies, or rights, from you under
this License will not have their licenses terminated so long as such
parties remain in full compliance.

\item
You are not required to accept this License, since you have not
signed it.  However, nothing else grants you permission to modify or
distribute the Program or its derivative works.  These actions are
prohibited by law if you do not accept this License.  Therefore, by
modifying or distributing the Program (or any work based on the
Program), you indicate your acceptance of this License to do so, and
all its terms and conditions for copying, distributing or modifying
the Program or works based on it.

\item
Each time you redistribute the Program (or any work based on the
Program), the recipient automatically receives a license from the
original licensor to copy, distribute or modify the Program subject to
these terms and conditions.  You may not impose any further
restrictions on the recipients' exercise of the rights granted herein.
You are not responsible for enforcing compliance by third parties to
this License.

\item
If, as a consequence of a court judgment or allegation of patent
infringement or for any other reason (not limited to patent issues),
conditions are imposed on you (whether by court order, agreement or
otherwise) that contradict the conditions of this License, they do not
excuse you from the conditions of this License.  If you cannot
distribute so as to satisfy simultaneously your obligations under this
License and any other pertinent obligations, then as a consequence you
may not distribute the Program at all.  For example, if a patent
license would not permit royalty-free redistribution of the Program by
all those who receive copies directly or indirectly through you, then
the only way you could satisfy both it and this License would be to
refrain entirely from distribution of the Program.

If any portion of this section is held invalid or unenforceable under
any particular circumstance, the balance of the section is intended to
apply and the section as a whole is intended to apply in other
circumstances.

It is not the purpose of this section to induce you to infringe any
patents or other property right claims or to contest validity of any
such claims; this section has the sole purpose of protecting the
integrity of the free software distribution system, which is
implemented by public license practices.  Many people have made
generous contributions to the wide range of software distributed
through that system in reliance on consistent application of that
system; it is up to the author/donor to decide if he or she is willing
to distribute software through any other system and a licensee cannot
impose that choice.

This section is intended to make thoroughly clear what is believed to
be a consequence of the rest of this License.

\item
If the distribution and/or use of the Program is restricted in
certain countries either by patents or by copyrighted interfaces, the
original copyright holder who places the Program under this License
may add an explicit geographical distribution limitation excluding
those countries, so that distribution is permitted only in or among
countries not thus excluded.  In such case, this License incorporates
the limitation as if written in the body of this License.

\item
The Free Software Foundation may publish revised and/or new versions
of the General Public License from time to time.  Such new versions will
be similar in spirit to the present version, but may differ in detail to
address new problems or concerns.

Each version is given a distinguishing version number.  If the Program
specifies a version number of this License which applies to it and ``any
later version'', you have the option of following the terms and conditions
either of that version or of any later version published by the Free
Software Foundation.  If the Program does not specify a version number of
this License, you may choose any version ever published by the Free Software
Foundation.

\item
If you wish to incorporate parts of the Program into other free
programs whose distribution conditions are different, write to the author
to ask for permission.  For software which is copyrighted by the Free
Software Foundation, write to the Free Software Foundation; we sometimes
make exceptions for this.  Our decision will be guided by the two goals
of preserving the free status of all derivatives of our free software and
of promoting the sharing and reuse of software generally.

\begin{center}
{\Large\sc
No Warranty
}
\end{center}

\item
{\sc Because the program is licensed free of charge, there is no warranty
for the program, to the extent permitted by applicable law.  Except when
otherwise stated in writing the copyright holders and/or other parties
provide the program ``as is'' without warranty of any kind, either expressed
or implied, including, but not limited to, the implied warranties of
merchantability and fitness for a particular purpose.  The entire risk as
to the quality and performance of the program is with you.  Should the
program prove defective, you assume the cost of all necessary servicing,
repair or correction.}

\item
{\sc In no event unless required by applicable law or agreed to in writing
will any copyright holder, or any other party who may modify and/or
redistribute the program as permitted above, be liable to you for damages,
including any general, special, incidental or consequential damages arising
out of the use or inability to use the program (including but not limited
to loss of data or data being rendered inaccurate or losses sustained by
you or third parties or a failure of the program to operate with any other
programs), even if such holder or other party has been advised of the
possibility of such damages.}

\end{enumerate}


\begin{center}
{\Large\sc End of Terms and Conditions}
\end{center}


\pagebreak[2]

\section*{Appendix: How to Apply These Terms to Your New Programs}

If you develop a new program, and you want it to be of the greatest
possible use to the public, the best way to achieve this is to make it
free software which everyone can redistribute and change under these
terms.

  To do so, attach the following notices to the program.  It is safest to
  attach them to the start of each source file to most effectively convey
  the exclusion of warranty; and each file should have at least the
  ``copyright'' line and a pointer to where the full notice is found.

\begin{quote}
one line to give the program's name and a brief idea of what it does. \\
Copyright (C) yyyy  name of author \\

This program is free software; you can redistribute it and/or modify
it under the terms of the GNU General Public License as published by
the Free Software Foundation; either version 2 of the License, or
(at your option) any later version.

This program is distributed in the hope that it will be useful,
but WITHOUT ANY WARRANTY; without even the implied warranty of
MERCHANTABILITY or FITNESS FOR A PARTICULAR PURPOSE.  See the
GNU General Public License for more details.

You should have received a copy of the GNU General Public License
along with this program; if not, write to the Free Software
Foundation, Inc., 59 Temple Place - Suite 330, Boston, MA  02111-1307, USA.
\end{quote}

Also add information on how to contact you by electronic and paper mail.

If the program is interactive, make it output a short notice like this
when it starts in an interactive mode:

\begin{quote}
Gnomovision version 69, Copyright (C) yyyy  name of author \\
Gnomovision comes with ABSOLUTELY NO WARRANTY; for details type `show w'. \\
This is free software, and you are welcome to redistribute it
under certain conditions; type `show c' for details.
\end{quote}


The hypothetical commands {\tt show w} and {\tt show c} should show the
appropriate parts of the General Public License.  Of course, the commands
you use may be called something other than {\tt show w} and {\tt show c};
they could even be mouse-clicks or menu items---whatever suits your
program.

You should also get your employer (if you work as a programmer) or your
school, if any, to sign a ``copyright disclaimer'' for the program, if
necessary.  Here is a sample; alter the names:

\begin{quote}
Yoyodyne, Inc., hereby disclaims all copyright interest in the program \\
`Gnomovision' (which makes passes at compilers) written by James Hacker. \\

signature of Ty Coon, 1 April 1989 \\
Ty Coon, President of Vice
\end{quote}


This General Public License does not permit incorporating your program
into proprietary programs.  If your program is a subroutine library, you
may consider it more useful to permit linking proprietary applications
with the library.  If this is what you want to do, use the GNU Library
General Public License instead of this License.

\end{document}

% LocalWords:  proprietarize redistributors sublicense yyyy Gnomovision EULAs
% LocalWords:  Yoyodyne FrontPage improvers Berne copyrightable Stallman's GPLs
% LocalWords:  Lessig Lessig's UCITA pre
