% comprehensive-gpl-guide.tex                                    -*- LaTeX -*-
%
% Toplevel file to build the entire book.
\documentclass[10pt, letterpaper, openany, oneside]{book}
% I'm somewhat convinced that this book would be better formatted using
%  the memoir class :
%    http://www.ctan.org/pkg/memoir
%   http://mirror.unl.edu/ctan/macros/latex/contrib/memoir/memman.pdf

% For the moment, I've thrown in fancychap because I don't have time to
% research memoir.


% FIXME: Some overall formatting hacks that would really help:

%   * I have started using  \hyperref[LABEL]{text} extensively, which seems
%     to work great in the PDF and HTML versions, but in the Postscript
%     version, the link lost entirely.  I think we need an additional command
%     to replace \hyperref which takes an optional third argument that will
%     insert additional text only when generating print versions, such as:
%      \newhyperref[GPLv2s3]{the requirements for binary distribution under
%      GPLv2}{(see section~\ref*{GPLv2s3} for more information)}
%
%     This is a careful balance, because it'd be all too easy to over-pepper
%     the printed version with back/forward references, but there are
%     probably times when this is useful.

%   * Similar issue: \href{} is well known not to carry the URLs in the print
%     versions.  Adding a footnote with the URL for the print version is
%     probably right.  (or maybe a References page?)

%   * The text is extremely inconsistent regarding formatting of code and
%     commands.  The following varied different methods have been used:
%         + the \verb%..% inline form
%         + verbatim environment (i.e., \begin{verbatim}
%         + {\tt }
%         + \texttt{}
%         + the lstlisting environment (i.e., \begin{lstlisting}
%     These should be made consistent, using only two forms: one for line and
%     one for a long quoted section.

\usepackage{hyperref}
\usepackage{listings}
\usepackage{enumerate}
\usepackage[Conny]{fncychap}
\usepackage[dvips]{graphicx}
\usepackage[verbose, twoside, dvips,
              paperwidth=8.5in, paperheight=11in,
              left=1in, right=1in, top=1.25in, bottom=.75in,
           ]{geometry}

\newcommand{\tutorialpartsplit}[2]{#2}

%% BEGIN CODE TO FORCE NO PAGE NUMBER ON ToC
\usepackage{tocloft}
\addtocontents{toc}{\cftpagenumbersoff{part}} %% Similarly for subsection, figure... as you wish
\addtocontents{toc}{\cftpagenumbersoff{section}} %% Similarly for subsection, figure... as you wish
\addtocontents{toc}{\cftpagenumbersoff{chapter}}
\addtocontents{toc}{\cftpagenumbersoff{section}} %% Similarly for subsection, figure... as you wish
\addtocontents{toc}{\cftpagenumbersoff{subsection}} %% Similarly for subsection, figure... as you wish
\renewcommand{\cftdot}{} %empty {} for no dots. you can have any symbol inside. For example put {\ensuremath{\ast}} and see what happens.
% END  CODE TO FORCE NO PAGE NUMBER ON ToC


\hypersetup{pdfinfo={Title={Copyleft and the GNU General Public License: A Comprehensive Tutorial and Guide}}}

    \begin{document}

\pagestyle{plain}
\pagenumbering{roman}

\frontmatter

\begin{titlepage}

\begin{center}

{\Huge
{\sc Copyleft and the  \\

GNU General Public License:

\vspace{.25in}

A Comprehensive Tutorial \\

and Guide
}}
\vfill

{\parindent 0in
\begin{tabbing}
Copyright \= \copyright{} 2003--2007, 2014 \hspace{1.mm} \=  \kill
Copyright \> \copyright{} 2014 \>  Bradley M. Kuhn. \\
Copyright \> \copyright{} 2014 \>  Anthony K. Sebro, Jr. \\
Copyright \= \copyright{} 2014 \> Denver Gingerich \\
Copyright \= \copyright{} 2003--2007, 2014 \> \hspace{.2in} Free Software Foundation, Inc. \\
Copyright \> \copyright{} 2008 \>  Software Freedom Law Center. \\
\end{tabbing}

\vspace{.3in}

The copyright holders hereby grant the freedom to copy, modify, convey,
Adapt, and/or redistribute this work under the terms of the Creative Commons
Attribution Share Alike 4.0 International License.  A copy of that license is
available at \url{https://creativecommons.org/licenses/by-sa/4.0/legalcode}.

Each part of this book, except the appendix, is separately under this same
license, but copyrighted by different entities at different times.  Each part
therefore also contains its own copyright and licensing notice.  The notice
above is for the entire work, and includes the full copyright and licensing
details, except for the appendix.

The appendix includes copies of the texts of various licenses published
by the FSF, and they are all licensed under the license, ``Everyone is permitted
to copy and distribute verbatim copies of this license document, but changing
it is not allowed.''.  However, those who seek to make modified versions of
those licenses should note the
\href{https://www.gnu.org/licenses/gpl-faq.html#ModifyGPL}{explanation given in the GPL FAQ}.

\vfill

Patches are welcome to this material.  Sources can be found in the Git
repository at: \url{https://gitorious.org/gpl-compliance-tools/tutorial/}
}
\end{center}

\end{titlepage}

\tableofcontents

\chapter{Preface}

This tutorial is the culmination of nearly a decade of studying and writing
about software freedom licensing and the GPL\@.  Each part of this tutorial
is a course unto itself, educating the reader on a myriad of topics from the
deep details of the GPLv2 and GPLv3, common business models in the copyleft
licensing area (both the friendly and unfriendly kind), best practices for
compliance with the GPL, for engineers, managers, and lawyers, as well as
real-world case studies of GPL enforcement matters.

It is unlikely that all the information herein is necessary to learn all at
once, and therefore this tutorial likely serves best as a reference book.
The material herein has been used as the basis for numerous live tutorials
and discussion groups since 2002, and the materials have been periodically
updated.   They likely stand on their own as excellent reference material.

However, if you are reading these course materials without attending a live
tutorial session, please note that this material is merely a summary of the
highlights of the various CLE and other tutorial courses based on this
material.  Please be aware that during the actual courses, class discussion
and presentation supplements this printed curriculum.  Simply reading this
material is \textbf{not an equivalent} for attending a course.

\mainmatter

% gpl-lgpl.tex                                                  -*- LaTeX -*-
%      Tutorial Text for the Detailed Study and Analysis of GPL and LGPL course
%
% Copyright (C) 2003, 2004, 2005, 2006 Free Software Foundation, Inc.
% Copyright (C) 2014                   Bradley M. Kuhn

% License: CC-By-SA-4.0

% The copyright holders hereby grant the freedom to copy, modify, convey,
% Adapt, and/or redistribute this work under the terms of the Creative
% Commons Attribution Share Alike 4.0 International License.

% This text is distributed in the hope that it will be useful, but
% WITHOUT ANY WARRANTY; without even the implied warranty of
% MERCHANTABILITY or FITNESS FOR A PARTICULAR PURPOSE.

% You should have received a copy of the license with this document in
% a file called 'CC-By-SA-4.0.txt'.  If not, please visit
% https://creativecommons.org/licenses/by-sa/4.0/legalcode to receive
% the license text.

% FIXME-LATER: I should make a macro like the Texinfo @xref stuff for places
%      where I'm saying ``see section X in this tutorial'', so that the extra
%      verbiage isn't there in the HTML versions that I'll eventually do.
%      Maybe something like that already exists?  In the worst case, I could
%      adapt @xref from texinfo.texi for it.

\newcommand{\defn}[1]{\emph{#1}}

\part{Detailed Analysis of the GNU GPL and Related Licenses}

{\parindent 0in
\tutorialpartsplit{``Detailed Analysis of the GNU GPL and Related Licenses''}{This part} is: \\
\begin{tabbing}
Copyright \= \copyright{} 2003--2007 \hspace{.1mm} \=  \kill
Copyright \> \copyright{} 2014 \> Bradley M. Kuhn \\
Copyright \> \copyright{} 2014 \>  Anthony K. Sebro, Jr. \\
Copyright \> \copyright{} 2003--2007 \>  Free Software Foundation, Inc.
\end{tabbing}


\vspace{1in}

\begin{center}
Authors of \tutorialpartsplit{``Detailed Analysis of the GNU GPL and Related Licenses''}{this part} are: \\


Free Software Foundation, Inc. \\
Bradley M. Kuhn \\
David ``Novalis'' Turner \\
Daniel B. Ravicher \\
Tony Sebro \\
John Sullivan

\vspace{.3in}

The copyright holders of \tutorialpartsplit{``Detailed Analysis of the GNU GPL and Related Licenses''}{this part} hereby grant the freedom to copy, modify,
convey, Adapt, and/or redistribute this work under the terms of the Creative
Commons Attribution Share Alike 4.0 International License.  A copy of that
license is available at
\verb=https://creativecommons.org/licenses/by-sa/4.0/legalcode=.
\end{center}
}

\bigskip

\bigskip

\tutorialpartsplit{This tutorial}{This part of the tutorial} gives a
comprehensive explanation of the most popular Free Software copyright
license, the GNU General Public License (``GNU GPL'', or sometimes just
``GPL'') -- both version 2 (``GPLv2'') and version 3 (``GPLv3'') -- and
teaches lawyers, software developers, managers and business people how to use
the GPL (and GPL'd software) successfully both as a community-building
``Constitution'' for a software project, and to incorporate copylefted
software into a new Free Software business and in existing, successful
enterprises.

To successfully benefit of from this part of the tutorial, readers should
have a general familiarity with software development processes.  A basic
understanding of how copyright law applies to software is also helpful.  The
tutorial is of most interest to lawyers, software developers and managers who
run or advise software businesses that modify and/or redistribute software
under the terms of the GNU GPL (or who wish to do so in the future), and those
who wish to make use of existing GPL'd software in their enterprise.

Upon completion of this part of the tutorial, successful readers can expect
to have learned the following:

\begin{itemize}

  \item The freedom-defending purpose of various terms in the GNU GPLv2 and GPLv3.

  \item The differences between GPLv2 and GPLv3.

  \item The redistribution options under the GPLv2 and GPLv3.

  \item The obligations when modifying GPLv2'd or GPLv3'd software.

  \item How to build a plan for proper and successful compliance with the GPL.

  \item The business advantages that the GPL provides.

  \item The most common business models used in conjunction with the GPL.

  \item How existing GPL'd software can be used in existing enterprises.

  \item The basics of LGPLv2.1 and LGPLv3, and how they
    differs from the GPLv2 and GPLv3, respectively.

  \item The basics to begin understanding the complexities regarding
    derivative and combined works of software.
\end{itemize}

%%%%%%%%%%%%%%%%%%%%%%%%%%%%%%%%%%%%%%%%%%%%%%%%%%%%%%%%%%%%%%%%%%%%%%%%%%%%%%%
% END OF ABSTRACTS SECTION
%%%%%%%%%%%%%%%%%%%%%%%%%%%%%%%%%%%%%%%%%%%%%%%%%%%%%%%%%%%%%%%%%%%%%%%%%%%%%%%
% START OF DAY ONE COURSE
%%%%%%%%%%%%%%%%%%%%%%%%%%%%%%%%%%%%%%%%%%%%%%%%%%%%%%%%%%%%%%%%%%%%%%%%%%%%%%%

\chapter{What Is Software Freedom?}

Study of the GNU General Public License (herein, abbreviated as \defn{GNU
  GPL} or just \defn{GPL}) must begin by first considering the broader world
of software freedom. The GPL was not created in a vacuum. Rather, it was
created to embody and defend a set of principles that were set forth at the
founding of the GNU project and the Free Software Foundation (FSF) -- the
preeminent organization that upholds, defends and promotes the philosophy of software
freedom. A prerequisite for understanding both of the popular versions
of the GPL
(GPLv2 and GPLv3) and their terms and conditions is a basic understanding of
the principles behind them.  The GPL family of licenses are unlike nearly all
other software licenses in that they are designed to defend and uphold these
principles.

\section{The Free Software Definition}
\label{Free Software Definition}

The Free Software Definition is set forth in full on FSF's website at
\verb0http://fsf.org/0 \verb0philosophy/free-sw.html0. This section presents
an abbreviated version that will focus on the parts that are most pertinent
to the GPL\@.

A particular program grants software freedom to a particular user if that
user is granted the following freedoms:

\begin{itemize}


\item The freedom to run the program, for any purpose.

\item The freedom to study how the program works, and modify it

\item The freedom to redistribute copies.

\item The freedom to distribute copies of  modified versions to others.

\end{itemize}

The focus on ``a particular user'' is particularly pertinent here.  It is not
uncommon for the same version of a specific program to grant these freedoms
to some subset of its user base, while others have none or only some of these
freedoms.  Section~\ref{Proprietary Relicensing} talks in detail about how
this can unfortunately happen even if a program is released under the GPL\@.

Many people refer to software that gives these freedoms as ``Open Source.''
Besides having a different political focus than those who call it Free
Software,\footnote{The political differences between the Free Software
  Movement and the Open Source Movement are documented on FSF's Web site at
  {\tt http://www.fsf.org/licensing/essays/free-software-for-freedom.html}.}
Those who call the software ``Open Source'' are often focused on a side
issue.  Specifically, user access to the source code of a program is a
prerequisite to make use of the freedom to modify.  However, the important
issue is what freedoms are granted in the license of that source code.

Software freedom is only complete when no restrictions are imposed on how
these freedoms are exercised.  Specifically, users and programmers can
exercise these freedoms noncommercially or commercially.  Licenses that grant
these freedoms for noncommercial activities but prohibit them for commercial
activities are considered non-free.  Even the Open Source Initiative
(\defn{OSI}) (the arbiter of what is considered ``Open Source'') also rules
such licenses not in fitting with its ``Open Source Definition''.

In general, software for which most or all of these freedoms are
restricted in any way is called ``non-Free Software.''  Typically, the
term ``proprietary software'' is used more or less interchangeably with
``non-Free Software.''  Personally, I tend to use the term ``non-Free
Software'' to refer to noncommercial software that restricts freedom
(such as ``shareware'') and ``proprietary software'' to refer to
commercial software that restricts freedom (such as nearly all of
Microsoft's and Oracle's offerings).

Keep in mind that the none of the terms ``software freedom'', ``open source''
and ``free software'' are known to be trademarked or otherwise legally
restricted by any organization in
any jurisdiction.  As such, it's quite common that these terms are abused and
misused by parties who wish to bank on the popularity of software freedom.
When one considers using, modifying or redistributing a software package that
purports to be Open Source or Free Software, one \textbf{must} verify that
the license grants software freedom.

Furthermore, throughout this text, we generally prefer the term ``software
freedom'', as this is the least ambiguous term available to describe software
that meets the Free Software Definition.  For example, it is well known and
often discussed that the adjective ``free'' has two unrelated meanings in
English: ``free as in freedom'' and ``free as in price''.  Meanwhile, the
term ``open source'' is even more confusing, because it appears to refer only to the
``freedom to study'', which is merely a subset of one of the four freedoms.

The remainder of this section considers each of each component of software
freedom in detail.

\subsection{The Freedom to Run}
\label{freedom-to-run}

The first tenet of software freedom is the user's fully unfettered right to
run the program.  The software's license must permit any conceivable use of
the software.  Perhaps, for example, the user has discovered an innovative
use for a particular program, one that the programmer never could have
predicted.  Such a use must not be restricted.

It was once rare that this freedom was restricted by even proprietary
software; but such is quite common today. Most End User License Agreements
(EULAs) that cover most proprietary software typically restrict some types of
uses.  Such restrictions of any kind are an unacceptable restriction on
software freedom.

\subsection{The Freedom to Change and Modify}

Perhaps the most useful right of software freedom is the users' right to
change, modify and adapt the software to suit their needs.  Access to the
source code and related build and installation scripts are an essential part
of this freedom.  Without the source code, and the ability to build and
install the binary applications from that source, users cannot effectively
exercise this freedom.

Programmers directly benefit from this freedom.  However, this freedom
remains important to users who are not programmers.  While it may seem
counterintuitive at first, non-programmer users often exercise this freedom
indirectly in both commercial and noncommercial settings.  For example, users
often seek noncommercial help with the software on email lists and in user
groups.  To make use of such help they must either have the freedom to
recruit programmers who might altruistically assist them to modify their
software, or to at least follow rote instructions to make basic modifications
themselves.

More commonly, users also exercise this freedom commercially.  Each user, or
group of users, may hire anyone they wish in a competitive free market to
modify and change the software.  This means that companies have a right to
hire anyone they wish to modify their Free Software.  Additionally, such
companies may contract with other companies to commission software
modification.

\subsection{The Freedom to Copy and Share}

Users share Free Software in a variety of ways. Software freedom advocates
work to eliminate a fundamental ethical dilemma of the software age: choosing
between obeying a software license and friendship (by giving away a copy of a
program to your friend who likes the software you are using). Licenses that
respect software freedom, therefore, permit altruistic sharing of software
among friends.

The commercial environment also benefits of this freedom.  Commercial sharing
includes selling copies of Free Software: that is, Free Software can
be distribted for any monetary
price to anyone.  Those who redistribute Free Software commercially also have
the freedom to selectively distribute (i.e., you can pick your customers) and
to set prices at any level that redistributor sees fit.

Of course, most people get copies of Free Software very cheaply (and
sometimes without charge).  The competitive free market of Free Software
tends to keep prices low and reasonable.  However, if someone is willing to
pay billions of dollars for one copy of the GNU Compiler Collection, such a
sale is completely permitted.

Another common instance of commercial sharing is service-oriented
distribution.  For example, some distribution vendors provide immediate
security and upgrade distribution via a special network service.  Such
distribution is not necessarily contradictory with software freedom.

(Section~\ref{Business Models} of this tutorial talks in detail about some
common Free Software business models that take advantage of the freedom to
share commercially.)

\subsection{The Freedom to Share Improvements}

The freedom to modify and improve is somewhat empty without the freedom to
share those improvements.  The Software freedom community is built on the
pillar of altruistic sharing of improved Free Software. Historically
it was typical for a
Free Software project to sprout a mailing list where improvements
would be shared
freely among members of the development community\footnote{This is still
commonly the case, though today there are other or additional ways of
sharing Free Software.}.   Such noncommercial
sharing is the primary reason that Free Software thrives.

Commercial sharing of modified Free Software is equally important.
For commercial support to exist in a competitive free market, all
developers -- from single-person contractors to large software
companies -- must have the freedom to market their services as
improvers of Free Software.  All forms of such service marketing must
be equally available to all.

For example, selling support services for Free Software is fully
permitted. Companies and individuals can offer themselves as ``the place
to call'' when software fails or does not function properly.  For such a
service to be meaningful, the entity offering that service needs the
right to modify and improve the software for the customer to correct any
problems that are beyond mere user error.

Software freedom licenses also permit any entity to distribute modified
versions of Free Software.  Most Free Software programs have a ``standard
version'' that is made available from the primary developers of the software.
However, all who have the software have the ``freedom to fork'' -- that is,
make available nontrivial modified versions of the software on a permanent or
semi-permanent basis.  Such freedom is central to vibrant developer and user
interaction.

Companies and individuals have the right to make true value-added versions
of Free Software.  They may use freedom to share improvements to
distribute distinct versions of Free Software with different functionality
and features.  Furthermore, this freedom can be exercised to serve a
disenfranchised subset of the user community.  If the developers of the
standard version refuse to serve the needs of some of the software's
users, other entities have the right to create a long- or short-lived fork
to serve that sub-community.

\section{How Does Software Become Free?}

The previous section set forth key freedoms and rights that are referred to
as ``software freedom''.  This section discusses the licensing mechanisms
used to enable software freedom.  These licensing mechanism were ultimately
created as a community-oriented ``answer'' to the existing proprietary
software licensing mechanisms.  Thus, first, consider carefully why
proprietary software exists in the first place.

Proprietary software exists at all only because it is governed by copyright
law.\footnote{This statement is admittedly an oversimplification. Patents and
  trade secrets can cover software and make it effectively non-Free, and one
  can contract away their rights and freedoms regarding software, or source
  code can be practically obscured in binary-only distribution without
  reliance on any legal system.  However, the primary control mechanism for
  software is copyright, and therefore this section focuses on how copyright
  restrictions make software proprietary.} Copyright law, with respect to
software, typically governs copying, modifying, and redistributing that
software (For details of this in the USA, see
\href{http://www.copyright.gov/title17/92chap1.html#106}{\S~106} and
\href{http://www.copyright.gov/title17/92chap1.html#117}{\S~117} of
\href{http://www.law.cornell.edu/uscode/text/17}{Title 17} of the
\textit{United States Code}).\footnote{Copyright law in general also governs
  ``public performance'' of copyrighted works. There is no generally agreed
  definition for public performance of software and both GPLv2 and GPLv3 do
  not restrict public performance.} By law (in the USA and in most other
jurisdictions), the copyright holder (most typically, the author) of the work controls
how others may copy, modify and/or distribute the work. For proprietary
software, these controls are used to prohibit these activities. In addition,
proprietary software distributors further impede modification in a practical
sense by distributing only binary code and keeping the source code of the
software secret.

Copyright is not a natural state, it is a legal construction. In the USA, the
Constitution permits, but does not require, the creation of copyright law as
federal legislation.  Software, since it is ``an original works of authorship
fixed in any tangible medium of expression ...  from which they can be
perceived, reproduced, or otherwise communicated, either directly or with the
aid of a machine or device'' (as stated in
\href{http://www.law.cornell.edu/uscode/text/17/102}{17 USC \S~102}), is thus
covered by the statute, and is copyrighted by default.

However, software, in its natural state without copyright, is Free
Software. In an imaginary world with no copyright, the rules would be
different. In this world, when you received a copy of a program's source
code, there would be no default legal system to restrict you from sharing it
with others, making modifications, or redistributing those modified
versions.\footnote{Note that this is again an oversimplification; the
  complexities with this argument are discussed in
  Section~\ref{software-and-non-copyright}.}

Software in the real world is copyrighted by default and is automatically
covered by that legal system.  However, it is possible to move software out
of the domain of the copyright system.  A copyright holder can often
\defn{disclaim} their copyright. (For example, under USA copyright law
it is possible for a copyright holder to engage in conduct resulting
in abandonment of copyright.)  If copyright is disclaimed, the software is
effectively no longer restricted by copyright law.   Software not restricted by copyright is in the
``public domain.''

\subsection{Public Domain Software}

In the USA and other countries that
are parties to the Berne Convention on Copyright, software is copyrighted
automatically by the author when she fixes the software in a tangible
medium.  In the software world, this usually means typing the source code
of the software into a file.

Imagine if authors could truly disclaim those default control of copyright
law.  If so, the software is in the public domain --- no longer covered by
copyright.  Since copyright law is the construction allowing for most
restrictions on software (i.e., prohibition of copying, modification, and
redistribution), removing the software from the copyright system usually
yields software freedom for its users.

Carefully note that software truly in the public domain is \emph{not} licensed
in any way.  It is confusing to say software is ``licensed for the
public domain,'' or any phrase that implies the copyright holder gave
express permission to take actions governed by copyright law.

Copyright holders who state that they are releasing their code into
the public domain are effectively renouncing copyright controls on
the work.  The law gave the copyright holders exclusive controls over the
work, and they chose to waive those controls.  Software that is, in
this sense, in the public domain
is conceptualized by the developer as having no copyright and thus no license. The software freedoms discussed in
Section~\ref{Free Software Definition} are all granted because there is no
legal system in play to take them away.

Admittedly, a discussion of public domain software is an oversimplified
example.  
Because copyright controls are usually automatically granted and because, in
some jurisdictions, some copyright controls cannot be waived (see
Section~\ref{non-usa-copyright} for further discussion), many copyright
holders sometimes incorrectly believe a work has been placed in the public
domain.  Second, due to aggressive lobbying by the entertainment industry,
the ``exclusive Right'' of copyright, that was supposed to only exist for
``Limited Times'' according to the USA Constitution, appears to be infinite:
simply purchased on the installment plan rather than in whole.  Thus, we must
assume no works of software will fall into the public domain merely due to
the passage of time.

Nevertheless, under USA law it is likely that the typical
disclaimers of copyright or public domain dedications we see in the
Free Software world would be interpreted by courts as copyright
abandonment, leading to a situation in which the user effectively receives a
maximum grant of copyright freedoms, similar to a maximally-permissive
Free Software license.

The best example of software known to truly be in the public domain is software
that is published by the USA government.  Under
\href{http://www.law.cornell.edu/uscode/text/17/105}{17 USC 101 \S~105}, all
works published by the USA Government are not copyrightable in the USA.

\subsection{Why Copyright Free Software?}

If simply disclaiming copyright on software yields Free Software, then it
stands to reason that putting software into the public domain is the
easiest and most straightforward way to produce Free Software. Indeed,
some major Free Software projects have chosen this method for making their
software Free. However, most of the Free Software in existence \emph{is}
copyrighted. In most cases (particularly in those of FSF and the GNU
Project), this was done due to very careful planning.

Software released into the public domain does grant freedom to those users
who receive the standard versions on which the original author disclaimed
copyright. However, since the work is not copyrighted, any nontrivial
modification made to the work is fully copyrightable.

Free Software released into the public domain initially is Free, and
perhaps some who modify the software choose to place their work into the
public domain as well. However, over time, some entities will choose to
proprietarize their modified versions. The public domain body of software
feeds the proprietary software. The public commons disappears, because
fewer and fewer entities have an incentive to contribute back to the
commons. They know that any of their competitors can proprietarize their
enhancements. Over time, almost no interesting work is left in the public
domain, because nearly all new work is done by proprietarization.

A legal mechanism is needed to redress this problem. FSF was in fact
originally created primarily as a legal entity to defend software freedom,
and that work of defending software freedom is a substantial part of
its work today. Specifically because of this ``embrace, proprietarize and
extend'' cycle, FSF made a conscious choice to copyright its Free Software,
and then license it under ``copyleft'' terms. Many, including the
developers of the kernel named Linux, have chosen to follow this paradigm.

\label{copyleft-definition}

Copyleft is a legal strategy and mechanism to defend, uphold and propagate software
freedom. The basic technique of copyleft is as follows: copyright the
software, license it under terms that give all the software freedoms, but
use the copyright law controls to ensure that all who receive a copy of
the software have equal rights and freedom. In essence, copyleft grants
freedom, but forbids others to forbid that freedom to anyone else along
the distribution and modification chains.

Copyleft is a general concept. Much like ideas for what a computer might
do must be \emph{implemented} by a program that actually does the job, so
too must copyleft be implemented in some concrete legal structure.
``Share and share alike'' is a phrase that is used often enough to explain the
concept behind copyleft, but to actually make it work in the real world, a
true implementation in legal text must exist. The GPL is the primary
implementation of copyleft in copyright licensing language.

\subsection{Software and Non-Copyright Legal Regimes}
\label{software-and-non-copyright}

The use, modification and distribution of software, like many endeavors,
simultaneously interacts with multiple different legal regimes.  As was noted
early via footnotes, copyright is merely the \textit{most common way} to
restrict users' rights to copy, share, modify and/or redistribute software.
However, proprietary software licenses typically use every mechanism
available to subjugate users.  For example:

\begin{itemize}

\item Unfortunately, despite much effort by many in the software freedom
  community to end patents that read on software (i.e., patents on
  computational ideas), they still ultimately exist.  As such, a software
  program might otherwise be seemly unrestricted, but a patent might read on
  the software and ruin everything for its users.\footnote{See
  \S\S~\ref{gpl-implied-patent-grant},~\ref{GPLv2s7},~\ref{GPLv3s11} for more
  discussion on how the patent system interacts with copyleft, and read
  Richard M.~Stallman's essay,
  \href{http://www.wired.com/opinion/2012/11/richard-stallman-software-patents/}{\textit{Let’s
      Limit the Effect of Software Patents, Since We Can’t Eliminate Them}}
  for more information on the problems these patents present to society.}

\item Digital Restrictions Management (usually called \defn{DRM}) is often
  used to impose technological restrictions on users' ability to exercise
  software freedom that they might otherwise be granted\footnote{See
    \S~\ref{GPLv3-drm} for more information on how GPL deals with this issue.}.
  The simplest (and perhaps oldest) form of DRM, of course, is separating
  software source code (read by humans), from their compiled binaries (read
  only by computers).  Furthermore,
  \href{http://www.law.cornell.edu/uscode/text/17/1201}{17 USC~\S1201} often
  prohibits users legally from circumventing some of these DRM systems.

\item Most EULAs also include a contractual agreement that bind users further
  by forcing them to agree to a contractual, prohibitive software license
  before ever even using the software.

\end{itemize}

Thus, most proprietary software restricts users via multiple interlocking
legal and technological means.  Any license that truly respect the software
freedom of all users must not only grant appropriate copyright permissions,
but also \textit{prevent} restrictions from other legal and technological
means like those listed above.

\subsection{Non-USA Copyright Regimes}
\label{non-usa-copyright}

Generally speaking, copyright law operates similarly enough in countries that
have signed the Berne Convention on Copyright, and software freedom licenses
have generally taken advantage of this international standardization of
copyright law.  However, copyright law does differ from country to country,
and commonly, software freedom licenses like GPL must be considered under the
copyright law in the jurisdiction where any licensing dispute occurs.

Those who are most familiar with the USA's system of copyright often are
surprised to learn that there are certain copyright controls that cannot be
waived nor disclaimed.  Specifically, many copyright regimes outside the USA
recognize a concept of moral rights of authors.  Typically, moral rights are
fully compatible with respecting software freedom, as they are usually
centered around controls that software freedom licenses generally respect,
such as the right of an authors to require proper attribution for their work.

\section{A Community of Equality}

The previous section described the principles of software freedom, a brief
introduction to mechanisms that typically block these freedoms, and the
simplest ways that copyright holders might grant those freedoms to their
users for their copyrighted works of software.  The previous section also
introduced the idea of \textit{copyleft}: a licensing mechanism to use
copyright to not only grant software freedom to users, but also to uphold
those rights against those who might seek to curtail them.

Copyleft, as defined in \S~\ref{copyleft-definition}, is a general term this
mechanism.  The remainder of this text will discuss details of various
real-world implementations of copyleft -- most notably, the GPL\@.

This discussion begins first with some general explanation of what the GPL is
able to do in software development communities.  After that brief discussion
in this section, deeper discussion of how GPL accomplishes this in practice
follows in the next chapter.

Simply put, though, the GPL ultimately creates an community of equality for
both business and noncommercial users.

\subsection{The Noncommercial Community}

A GPL'd code base becomes a center of a vibrant development and user
community.  Traditionally, volunteers, operating noncommercially out of
keen interest or ``scratch an itch'' motivations, produce initial versions
of a GPL'd system.  Because of the efficient distribution channels of the
Internet, any useful GPL'd system is adopted quickly by noncommercial
users.

Fundamentally, the early release and quick distribution of the software
gives birth to a thriving noncommercial community.  Users and developers
begin sharing bug reports and bug fixes across a shared intellectual
commons.  Users can trust the developers, because they know that if the
developers fail to address their needs or abandon the project, the GPL
ensures that someone else has the right to pick up development.
Developers know that the users cannot redistribute their software without
passing along the rights granted by GPL, so they are assured that every
one of their users is treated equally.

Because of the symmetry and fairness inherent in GPL'd distribution,
nearly every GPL'd package in existence has a vibrant noncommercial user
and developer base.

\subsection{The Commercial Community}

By the same token, nearly all established GPL'd software systems have a
vibrant commercial community.  Nearly every GPL'd system that has gained
wide adoption from noncommercial users and developers eventually begins
to fuel a commercial system around that software.

For example, consider the Samba file server system that allows Unix-like
systems (including GNU/Linux) to serve files to Microsoft Windows systems.
Two graduate students originally developed Samba in their spare time and
it was deployed noncommercially in academic environments\footnote{See
  \href{http://turtle.ee.ncku.edu.tw/docs/samba/history}{Andrew Tridgell's
    ``A bit of history and a bit of fun''}}.  However, very
soon for-profit companies discovered that the software could work for them
as well, and their system administrators began to use it in place of
Microsoft Windows NT file-servers.  This served to lower the cost of
running such servers by orders of magnitude. There was suddenly room in
Windows file-server budgets to hire contractors to improve Samba.  Some of
the first people hired to do such work were those same two graduate
students who originally developed the software.

The noncommercial users, however, were not concerned when these two
fellows began collecting paychecks off of their GPL'd work.  They knew
that because of the nature of the GPL that improvements that were
distributed in the commercial environment could easily be folded back into
the standard version.  Companies are not permitted to proprietarize
Samba, so the noncommercial users, and even other commercial users are
safe in the knowledge that the software freedom ensured by GPL will remain
protected.

Commercial developers also work in concert with noncommercial
developers.  Those two now-long-since graduated students continue to
contribute to Samba altruistically, but also get paid work doing it.
Priorities change when a client is in the mix, but all the code is
contributed back to the standard version.  Meanwhile, many other
individuals have gotten involved noncommercially as developers,
because they want to ``cut their teeth on Free Software,'' or because
the problems interest them.  When they get good at it, perhaps they
will move on to another project, or perhaps they will become
commercial developers of the software themselves.

No party is a threat to another in the GPL software scenario because
everyone is on equal ground.  The GPL protects rights of the commercial
and noncommercial contributors and users equally. The GPL creates trust,
because it is a level playing field for all.

\subsection{Law Analogy}

In his introduction to Stallman's \emph{Free Software, Free Society},
Lawrence Lessig draws an interesting analogy between the law and Free
Software. He argues that the laws of a free society must be protected
much like the GPL protects software.  So that I might do true justice to
Lessig's argument, I quote it verbatim:

\begin{quotation}

A ``free society'' is regulated by law. But there are limits that any free
society places on this regulation through law: No society that kept its
laws secret could ever be called free.  No government that hid its
regulations from the regulated could ever stand in our tradition. Law
controls.  But it does so justly only when visibly.  And law is visible
only when its terms are knowable and controllable by those it regulates,
or by the agents of those it regulates (lawyers, legislatures).

This condition on law extends beyond the work of a legislature.  Think
about the practice of law in American courts.  Lawyers are hired by their
clients to advance their clients' interests.  Sometimes that interest is
advanced through litigation. In the course of this litigation, lawyers
write briefs. These briefs in turn affect opinions written by judges.
These opinions decide who wins a particular case, or whether a certain law
can stand consistently with a constitution.

All the material in this process is free in the sense that Stallman means.
Legal briefs are open and free for others to use.  The arguments are
transparent (which is different from saying they are good), and the
reasoning can be taken without the permission of the original lawyers.
The opinions they produce can be quoted in later briefs.  They can be
copied and integrated into another brief or opinion.  The ``source code''
for American law is by design, and by principle, open and free for anyone
to take. And take lawyers do---for it is a measure of a great brief that
it achieves its creativity through the reuse of what happened before.  The
source is free; creativity and an economy is built upon it.

This economy of free code (and here I mean free legal code) doesn't starve
lawyers.  Law firms have enough incentive to produce great briefs even
though the stuff they build can be taken and copied by anyone else.  The
lawyer is a craftsman; his or her product is public.  Yet the crafting is
not charity. Lawyers get paid; the public doesn't demand such work
without price.  Instead this economy flourishes, with later work added to
the earlier.

We could imagine a legal practice that was different --- briefs and
arguments that were kept secret; rulings that announced a result but not
the reasoning. Laws that were kept by the police but published to no one
else. Regulation that operated without explaining its rule.

We could imagine this society, but we could not imagine calling it
``free.''  Whether or not the incentives in such a society would be better
or more efficiently allocated, such a society could not be known as free.
The ideals of freedom, of life within a free society, demand more than
efficient application.  Instead, openness and transparency are the
constraints within which a legal system gets built, not options to be
added if convenient to the leaders.  Life governed by software code should
be no less.

Code writing is not litigation.  It is better, richer, more
productive.  But the law is an obvious instance of how creativity and
incentives do not depend upon perfect control over the products
created.  Like jazz, or novels, or architecture, the law gets built
upon the work that went before. This adding and changing is what
creativity always is.  And a free society is one that assures that its
most important resources remain free in just this sense.\footnote{This
quotation is Copyright \copyright{} 2002, Lawrence Lessig. It is
licensed under the terms of
\href{http://creativecommons.org/licenses/by/1.0/}{the ``Attribution
License'' version 1.0} or any later version as published by Creative
Commons.}
\end{quotation}

In essence, lawyers are paid to service the shared commons of legal
infrastructure.  Few citizens defend themselves in court or write their
own briefs (even though they are legally permitted to do so) because
everyone would prefer to have an expert do that job.

The Free Software economy is a market ripe for experts.  It
functions similarly to other well established professional fields like the
law. The GPL, in turn, serves as the legal scaffolding that permits the
creation of this vibrant commercial and noncommercial Free Software
economy.

%%%%%%%%%%%%%%%%%%%%%%%%%%%%%%%%%%%%%%%%%%%%%%%%%%%%%%%%%%%%%%%%%%%%%%%%%%%%%%%
\chapter{A Tale of Two Copyleft Licenses}
\label{tale-of-two-copylefts}

While determining the proper methodology and criteria to yield an accurate
count remains difficult, the GPL is generally considered one of the most
widely used Free Software licenses.  For most of its history --- for 16 years
from June 1991 to June 2007 --- there was really only one version of the GPL,
version 2.

However, the GPL had both earlier versions before version 2, and, more well
known, a revision to version 3. 

\section{Historical Motivations for the General Public License}

The earliest license to grant software freedom was likely the Berkeley
Software Distribution (``BSD'') license.  This license is typical of what are
often called lax, highly permissive licenses.  Not unlike software in the
public domain, these non-copyleft licenses (usually) grant software freedom
to users, but they do not go to any effort to uphold that software freedom
for users.  The so-called ``downstream'' (those who receive the software and
then build new things based on that software) can restrict the software and
distribute further.

The GNU's Not Unix (``GNU'') project, which Richard M.~Stallman (``RMS'')
founded in 1984 to make a complete Unix-compatible operating system
implementation that assured software freedom for all.  However, RMS saw that
using a license that gave but did not assure software freedom would be
counter to the goals of the GNU project.  RMS invented ``copyleft'' as an
answer to that problem, and began using various copyleft licenses for the
early GNU project programs\footnote{RMS writes more fully about this topic in
  his essay entitled simply
  \href{http://www.gnu.org/gnu/thegnuproject.html}{\textit{The GNU Project}}.
    For those who want to hear the story in his own voice,
    \href{http://audio-video.gnu.org/audio/}{speech recordings} of his talk,
    \textit{The Free Software Movement and the GNU/Linux Operating System}
    are also widely available}.

\section{Proto-GPLs And Their Impact}

The earliest copyleft licenses were specific to various GNU programs.  For
example, \href{http://www.free-soft.org/gpl_history/emacs_gpl.html}{The Emacs
  General Public License} was likely the first copyleft license ever
published.  Interesting to note that even this earliest copyleft license
contains a version of the well-known GPL copyleft clause:

\begin{quotation}
You may modify your copy or copies of GNU Emacs \ldots provided that you also
\ldots cause the whole of any work that you distribute or publish, that in
whole or in part contains or is a derivative of GNU Emacs or any part
thereof, to be licensed at no charge to all third parties on terms identical
to those contained in this License Agreement.
\end{quotation}

This simply stated clause is the fundamental innovation of copyleft.
Specifically, copyleft \textit{uses} the copyright holders' controls on
permission to modify the work to add a conditional requirement.  Namely,
downstream users may only have permission to modify  the work if they pass
along the same permissions on the modified version that came originally to
them.

These original program-specific proto-GPLs give an interesting window into
the central ideas and development of copyleft.  In particular, reviewing them
shows how the text of the GPL we know has evolved to address more of the
issues discussed earlier in \S~\ref{software-and-non-copyright}.

\section{The GNU General Public License, Version 1}
\label{GPLv1}

In January 1989, the FSF announced that the GPL had been converted into a
``subroutine'' that could be reused not just for all FSF-copyrighted
programs, but also by anyone else.  As the FSF claimed in its announcement of
the GPLv1\footnote{The announcement of GPLv1 was published in the
  \href{http://www.gnu.org/bulletins/bull6.html\#SEC8}{GNU'S Bulletin, vol 1,
    number 6 dated January 1989}.  (Thanks very much to Andy Tai for his
  \href{http://www.free-soft.org/gpl_history/}{consolidation of research on
    the history of the pre-v1 GPL's}.)}:
\begin{quotation}
To make it easier to copyleft programs, we have been improving on the
legalbol architecture of the General Public License to produce a new version
that serves as a general-purpose subroutine: it can apply to any program
without modification, no matter who is publishing it.
\end{quotation}

This, like many inventive ideas, seems somewhat obvious in retrospect.  But,
the FSF had some bright people and access to good lawyers when it started.
It took almost five years from the first copyleft licenses to get to a
generalized, reusable GPLv1.  In the context and mindset of the 1980s, this
is not surprising.  The idea of reusable licensing infrastructure was not
only uncommon, it was virtually nonexistent!  Even the early BSD licenses
were simply copied and rewritten slightly for each new use\footnote{It
  remains an interesting accident of history that the early BSD problematic
  ``advertising clause'' (discussion of which is somewhat beyond the scope of
  this tutorial) lives on into current day, simply because while the
  University of California at Berkeley gave unilateral permission to remove
  the clause from \textit{its} copyrighted works, others who adapted the BSD
  license with their own names in place of UC-Berkeley's never have.}.  The
GPLv1's innovation of reuable licensing infrastructure, an obvious fact
today, was indeed a novel invention for its day\footnote{We're all just
  grateful that the FSF also opposes business method patents, since the FSF's
  patent on a ``method for reusable licensing infrastructure'' would have
  not expired until 2006!}.

\section{The GNU General Public License, Version 2}

The GPLv2 was released two and a half years after GPLv1, and over the
following sixteen years, it became the standard for copyleft licensing until
the release of GPLv3 in 2007 (discussed in more detail in the next section).

While this tutorial does not discuss the terms of GPLv1 in detail, it is
worth noting below the three key changes that GPLv2 brought:

\begin{itemize}

\item Software patents and their danger are explicitly mentioned, inspiring
  (in part) the addition of GPLv2~\S\S5--7.  (These sections are discussed in
  detail in \S~\ref{GPLv2s5}, \S~\ref{GPLv2s6} and \S~\ref{GPLv2s7} of this
  tutorial.)

\item GPLv2~\S2's copyleft terms are expanded to more explicitly discuss the
  issue of combined works.  (GPLv2~\S2 is discussed in detail in
  \S~\ref{GPLv2s2} in this tutorial).

\item GPLv2~\S3 includes more detailed requirements, including the phrase
 ``the scripts used to control compilation and installation of the
  executable'', which is a central component of current GPLv2 enforcement
  .  (GPLv2~\S3 is discussed in detail in
  \S~\ref{GPLv2s3} in this tutorial).
\end{itemize}

The next chapter discusses GPLv2 in full detail, and readers who wish to dive
into the section-by-section discussion of the GPL should jump ahead now to
that chapter.  However, the most interesting fact to note here is how GPLv2
was published with little fanfare and limited commentary.  This contrasts
greatly with the creation of GPLv3.

\section{The GNU General Public License, Version 3}

RMS began drafting GPLv2.2 in mid-2002, and FSF ran a few discussion groups
during that era about new text of that license.  However, rampant violations
of the GPL required more immediate attention of FSF's licensing staff, and as
such, much of the early 2000's was spent doing GPL enforcement
work\footnote{More on GPL enforcement is discussed in \tutorialpartsplit{a
    companion tutorial, \textit{A Practical Guide to GPL
      Compliance}}{Part~\ref{gpl-compliance-guide} of this tutorial}.}.  In
2006, FSF began in earnest drafting work for GPLv3.

The GPLv3 process began in earnest in January 2006.  It became clear that
many provisions of the GPL could benefit from modification to fit new
circumstances and to reflect what the entire community learned from
experience with version 2.  Given the scale of revision it seems proper to
approach the work through public discussion in a transparent and accessible
manner.

The GPLv3 process continued through June 2007, culminating in publication of
GPLv3 and LGPLv3 on 29 June 2007, AGPLv3 on 19 November 2007, and the GCC
Runtime Library Exception on 27 January 2009.

All told, four discussion drafts of GPLv3, two discussion drafts of LGPLv3
and two discussion drafts of AGPLv3 were published and discussed.
Ultimately, FSF remained the final arbiter and publisher of the licenses, and
RMS himself their primary author, but input was sought from many parties, and
these licenses do admittedly look and read more like legislation as a result.
Nevertheless, all of the ``v3'' group are substantially better and improved
licenses.

GPLv3 and its terms are discussed in detail in Chapter\~ref{GPLv3}.

\section{The Innovation of Optional ``Or Any Later'' Version}

An interesting fact of all GPL licenses is that the are ultimate multiple
choices for use of the license.  The FSF is the primary steward of GPL (as
discussed later in \S~\ref{GPLv2s9} and \S~\ref{GPLv2s14}).  However, those
who wish to license works under GPL are not required to automatically accept
changes made by the FSF for their own copyrighted works.

Each licensor may chose three different methods of licensing, as follows:

\begin{itemize}

\item explicitly name a single version of GPL for their work (usually
  indicated in shorthand by saying the license is ``GPLv$X$-only''), or

\item name no version of the GPL, thus they allow their downstream recipients
  to select any version of the GPL they chose (usually indicated in shorthand
  by saying the license is simply ``GPL''), or

\item name a specific version of GPL and give downstream recipients the
  option to chose that version ``or any later version as published by the
  FSF'' (usually indicated by saying the license is
  ``GPLv$X$-or-later'')\footnote{The shorthand of ``GPL$X+$'' is also popular
    for this situation.  The authors of this tutorial prefer ``-or-later''
    syntax, because it (a) mirrors the words ``or'' and ``later from the
    licensing statement, (b) the $X+$ doesn't make it abundantly clear that
    $X$ is clearly included as a license option and (c) the $+$ symbol has
    other uses in computing (such as with regular expressions) that mean
    something different.}
\end{itemize}

\label{license-compatibility-first-mentioned}

Oddly, this flexibility has received (in the opinion of the authors, undue)
criticism, primarily because of the complex and oft-debated notion of
``license compatibility'' (which is explained in detail in
\S~\ref{license-compatibility}).  Copyleft licenses are generally
incompatible with each other, because the details of how they implement
copyleft differs.  Specifically, copyleft works only because of its
requirement that downstream licensors use the \textit{same} license for
combined and modified works.  As such, software licensed under the terms of
``GPLv2-only'' cannot be combined with works licensed ``GPLv3-or-later''.
This is admittedly a frustrating outcome.

Other copyleft licenses that appeared after GPL, such
as the Creative Commons ``Share Alike'' licenses, the Eclipse Public License
and the Mozilla Public License \textbf{require} all copyright holders chosing
to use any version of those licenses to automatically accept and relicense
their copyrighted works under new versions.  Of course ,Creative Commons, the
Eclipse Foundation, and the Mozilla Foundation (like the FSF) have generally
served as excellent stewards of their licenses.  Copyright holders using
those licenses seems to find it acceptable that to fully delegate all future
licensing decisions for their copyrights to these organizations without a
second thought.

However, note that FSF gives herein the control of copyright holders to
decide whether or not to implicitly trust the FSF in its work of drafting
future GPL versions.  The FSF, for its part, does encourage copyright holders
to chose by default ``GPLv$X$-or-later'' (where $X$ is the most recent
version of the GPL published by the FSF).  However, the FSF \textbf{does not
  mandate} that a choice to use any GPL requires a copyright holder ceding
its authority for future licensing decisions to the FSF.  In fact, the FSF
considered this possibility for GPLv3 and chose not to do so, instead opting
for the third-party steward designation clause discussed in
Section~\ref{GPLv3s14}.

\section{Complexities of Two Simultaneously Popular Copylefts}

Obviously most GPL advocates would prefer widespread migration to GPLv3, and
many newly formed projects who seek a copyleft license tend to choose a
GPLv3-based license.  However, many existing copylefted projects continue
with GPLv2-only or GPLv2-or-later as their default license.

While GPLv3 introduces many improvements --- many of which were designed to
increase adoption by for-profit companies --- GPLv2 remains a widely used and
extremely popular license.  The GPLv2 is, no doubt, a good and useful
license.

However, unlike GPLv1, which (as pointed out in \S~\ref{GPLv1}), which is
completely out of use by the mid-1990s.  However, unlike GPLv1 before it,
GPLv2 remains a integral part of the copyleft licensing infrastructure for
some time to come.  As such, those who seek to have expertise in current
topics of copyleft licensing need to study both the GPLv2 and GPLv3 family of
licenses.

Furthermore, GPLv3 can is more easily understood by first studying GPLv2.
This is not only because of their chronological order, but also because much
of the discussion material available for GPLv3 tends to talk about GPLv3 in
contrast to GPLv2.  As such, a strong understanding of GPLv2 helps in
understanding most of the third-party material found regarding GPLv3.  Thus,
the following chapter begins a deep discussion of GPLv2.

%%%%%%%%%%%%%%%%%%%%%%%%%%%%%%%%%%%%%%%%%%%%%%%%%%%%%%%%%%%%%%%%%%%%%%%%%%%%%%%
\chapter{Running Software and Verbatim Copying}
\label{run-and-verbatim}


This chapter begins the deep discussion of the details of the terms of
GPLv2\@. In this chapter, we consider the first two sections: GPLv2 \S\S
0--2. These are the straightforward sections of the GPL that define the
simplest rights that the user receives.

\section{GPLv2~\S0: Freedom to Run}
\label{GPLv2s0}

GPLv2~\S0, the opening section of GPLv2, sets forth that the copyright law governs
the work.  It specifically points out that it is the ``copyright
holder'' who decides if a work is licensed under its terms and explains
how the copyright holder might indicate this fact.

A bit more subtly, GPLv2~\S0 makes an inference that copyright law is the only
system that can restrict the software.  Specifically, it states:
\begin{quote}
Activities other than copying, distribution and modification are not
covered by this License; they are outside its scope.
\end{quote}
In essence, the license governs \emph{only} those activities, and all other
activities are unrestricted, provided that no other agreements trump GPLv2
(which they cannot; see Sections~\ref{GPLv2s6} and~\ref{GPLv2s7}).  This is
very important, because the Free Software community heavily supports
users' rights to ``fair use'' and ``unregulated use'' of copyrighted
material.  GPLv2 asserts through this clause that it supports users' rights
to fair and unregulated uses.

Fair use (called ``fair dealing'' in some jurisdictions) of copyrighted
material is an established legal doctrine that permits certain activities
regardless of whether copyright law would other restrict those activities.
Discussion of the various types of fair use activity are beyond the scope of
this tutorial.  However, one important example of fair use is the right to
quote portions of the text in larger work so as to criticize or suggest
changes.  This fair use rights is commonly used on mailing lists when
discussing potential improvements or changes to Free Software.

Fair use is a doctrine established by the courts or by statute.  By
contrast, unregulated uses are those that are not covered by the statue
nor determined by a court to be covered, but are common and enjoyed by
many users.  An example of unregulated use is reading a printout of the
program's source code like an instruction book for the purpose of learning
how to be a better programmer.  The right to read something that you have
access is and should remain unregulated and unrestricted.

\medskip

Thus, the GPLv2 protects users fair and unregulated use rights precisely by
not attempting to cover them.  Furthermore, the GPLv2 ensures the freedom
to run specifically by stating the following:
\begin{quote}
''The act of running the Program is not restricted.''
\end{quote}
Thus, users are explicitly given the freedom to run by GPLv2~\S0.

\medskip

The bulk of GPLv2~\S0 not yet discussed gives definitions for other terms used
throughout.  The only one worth discussing in detail is ``work based on
the Program''.  The reason this definition is particularly interesting is
not for the definition itself, which is rather straightforward, but
because it clears up a common misconception about the GPL\@.

The GPL is often mistakenly criticized because it fails to give a
definition of ``derivative work''.  In fact, it would be incorrect and
problematic if the GPL attempted to define this.  A copyright license, in
fact, has no control over what may or may not be a derivative work.  This
matter is left up to copyright law and the courts --- not the licenses that utilize it.

It is certainly true that copyright law as a whole does not propose clear
and straightforward guidelines for what is and is not a derivative
software work under copyright law.  However, no copyright license --- not
even the GNU GPL --- can be blamed for this.  Legislators and court
opinions must give us guidance to decide the border cases.

\section{GPLv2~\S1: Verbatim Copying}
\label{GPLv2s1}

GPLv2~\S1 covers the matter of redistributing the source code of a program
exactly as it was received. This section is quite straightforward.
However, there are a few details worth noting here.

The phrase ``in any medium'' is important.  This, for example, gives the
freedom to publish a book that is the printed copy of the program's source
code.  It also allows for changes in the medium of distribution.  Some
vendors may ship Free Software on a CD, but others may place it right on
the hard drive of a pre-installed computer.  Any such redistribution media
is allowed.

Preservation of copyright notice and license notifications are mentioned
specifically in GPLv2~\S1.  These are in some ways the most important part of
the redistribution, which is why they are mentioned by name.  GPL
always strives to make it abundantly clear to anyone who receives the
software what its license is.  The goal is to make sure users know their
rights and freedoms under GPL, and to leave no reason that users might be
surprised the software is GPL'd. Thus
throughout the GPL, there are specific references to the importance of
notifying others down the distribution chain that they have rights under
GPL.

Also mentioned by name is the warranty disclaimer. Most people today do
not believe that software comes with any warranty.  Notwithstanding the
\href{http://mlis.state.md.us/2000rs/billfile/hb0019.htm}{Maryland's} and \href{http://leg1.state.va.us/cgi-bin/legp504.exe?001+ful+SB372ER}{Virginia's} UCITA bills, there are few or no implied warranties with software.
However, just to be on the safe side, GPL clearly disclaims them, and the
GPL requires redistributors to keep the disclaimer very visible. (See
Sections~\ref{GPLv2s11} and~\ref{GPLv2s12} of this tutorial for more on GPL's
warranty disclaimers.)

Note finally that GPLv2~\S1 creates groundwork for the important defense of
commercial freedom.  GPLv2~\S1 clearly states that in the case of verbatim
copies, one may make money.  Redistributors are fully permitted to charge
for the redistribution of copies of Free Software. In addition, they may
provide the warranty protection that the GPL disclaims as an additional
service for a fee. (See Section~\ref{Business Models} for more discussion
on making a profit from Free Software redistribution.)

%%%%%%%%%%%%%%%%%%%%%%%%%%%%%%%%%%%%%%%%%%%%%%%%%%%%%%%%%%%%%%%%%%%%%%%%%%%%%%%

\chapter{Derivative Works: Statute and Case Law}

We digress for this chapter from our discussion of GPL's exact text to
consider the matter of derivative works --- a concept that we must
understand fully before considering GPLv2~\S\S2--3\@. GPL, and Free
Software licensing in general, relies critically on the concept of
``derivative work'' since software that is ``independent,'' (i.e., not
``derivative'') of Free Software need not abide by the terms of the
applicable Free Software license. As much is required by \S~106 of the
Copyright Act, 17 U.S.C. \S~106 (2002), and admitted by Free Software
licenses, such as the GPL, which (as we have seen) states in GPLv2~\S0 that ``a
`work based on the Program' means either the Program or any derivative
work under copyright law.'' It is being a derivative work of Free Software
that triggers the necessity to comply with the terms of the Free Software
license under which the original work is distributed. Therefore, one is
left to ask, just what is a ``derivative work''? The answer to that
question differs depending on which court is being asked.

The analysis in this chapter sets forth the differing definitions of
derivative work by the circuit courts. The broadest and most
established definition of derivative work for software is the
abstraction, filtration, and comparison test (``the AFC test'') as
created and developed by the Second Circuit. Some circuits, including
the Ninth Circuit and the First Circuit, have either adopted narrower
versions of the AFC test or have expressly rejected the AFC test in
favor of a narrower standard. Further, several other circuits have yet
to adopt any definition of derivative work for software.

As an introductory matter, it is important to note that literal copying of
a significant portion of source code is not always sufficient to establish
that a second work is a derivative work of an original
program. Conversely, a second work can be a derivative work of an original
program even though absolutely no copying of the literal source code of
the original program has been made. This is the case because copyright
protection does not always extend to all portions of a program's code,
while, at the same time, it can extend beyond the literal code of a
program to its non-literal aspects, such as its architecture, structure,
sequence, organization, operational modules, and computer-user interface.

\section{The Copyright Act}

The copyright act is of little, if any, help in determining the definition
of a derivative work of software. However, the applicable provisions do
provide some, albeit quite cursory, guidance. Section 101 of the Copyright
Act sets forth the following definitions:

\begin{quotation}
A ``computer program'' is a set of statements or instructions to be used
directly or indirectly in a computer in order to bring about a certain
result.

A ``derivative work'' is a work based upon one or more preexisting works,
such as a translation, musical arrangement, dramatization,
fictionalization, motion picture version, sound recording, art
reproduction, abridgment, condensation, or any other form in which a work
may be recast, transformed, or adapted. A work consisting of editorial
revisions, annotations, elaborations, or other modifications which, as a
whole, represent an original work of authorship, is a ``derivative work.''
\end{quotation}

These are the only provisions in the Copyright Act relevant to the
determination of what constitutes a derivative work of a computer
program. Another provision of the Copyright Act that is also relevant to
the definition of derivative work is \S~102(b), which reads as follows:

\begin{quotation}
In no case does copyright protection for an original work of authorship
extend to any idea, procedure, process, system, method of operation,
concept, principle, or discovery, regardless of the form in which it is
described, explained, illustrated, or embodied in such work.
\end{quotation}

Therefore, before a court can ask whether one program is a derivative work
of another program, it must be careful not to extend copyright protection
to any ideas, procedures, processes, systems, methods of operation,
concepts, principles, or discoveries contained in the original program. It
is the implementation of this requirement to ``strip out'' unprotectable
elements that serves as the most frequent issue over which courts
disagree.

\section{Abstraction, Filtration, Comparison Test}

As mentioned above, the AFC test for determining whether a computer
program is a derivative work of an earlier program was created by the
Second Circuit and has since been adopted in the Fifth, Tenth, and
Eleventh Circuits. Computer Associates Intl., Inc. v. Altai, Inc., 982
F.2d 693 (2nd Cir. 1992); Engineering Dynamics, Inc. v. Structural
Software, Inc., 26 F.3d 1335 (5th Cir. 1994); Kepner-Tregoe,
Inc. v. Leadership Software, Inc., 12 F.3d 527 (5th Cir. 1994); Gates
Rubber Co. v. Bando Chem. Indust., Ltd., 9 F.3d 823 (10th Cir. 1993);
Mitel, Inc. v. Iqtel, Inc., 124 F.3d 1366 (10th Cir. 1997); Bateman
v. Mnemonics, Inc., 79 F.3d 1532 (11th Cir. 1996); and, Mitek Holdings,
Inc. v. Arce Engineering Co., Inc., 89 F.3d 1548 (11th Cir. 1996).

Under the AFC test, a court first abstracts from the original program its
constituent structural parts. Then, the court filters from those
structural parts all unprotectable portions, including incorporated ideas,
expression that is necessarily incidental to those ideas, and elements
that are taken from the public domain. Finally, the court compares any and
all remaining kernels of creative expression to the structure of the
second program to determine whether the software programs at issue are
substantially similar so as to warrant a finding that one is the
derivative work of the other.

Often, the courts that apply the AFC test will perform a quick initial
comparison between the entirety of the two programs at issue in order to
help determine whether one is a derivative work of the other. Such a
holistic comparison, although not a substitute for the full application of
the AFC test, sometimes reveals a pattern of copying that is not otherwise
obvious from the application of the AFC test when, as discussed below,
only certain components of the original program are compared to the second
program. If such a pattern is revealed by the quick initial comparison,
the court is more likely to conclude that the second work is indeed a
derivative of the original.

\subsection{Abstraction}

The first step courts perform under the AFC test is separation of the
work's ideas from its expression. In a process akin to reverse
engineering, the courts dissect the original program to isolate each level
of abstraction contained within it. Courts have stated that the
abstractions step is particularly well suited for computer programs
because it breaks down software in a way that mirrors the way it is
typically created. However, the courts have also indicated that this step
of the AFC test requires substantial guidance from experts, because it is
extremely fact and situation specific.

By way of example, one set of abstraction levels is, in descending order
of generality, as follows: the main purpose, system architecture, abstract
data types, algorithms and data structures, source code, and object
code. As this set of abstraction levels shows, during the abstraction step
of the AFC test, the literal elements of the computer program, namely the
source and object code, are defined as particular levels of
abstraction. Further, the source and object code elements of a program are
not the only elements capable of forming the basis for a finding that a
second work is a derivative of the program. In some cases, in order to
avoid a lengthy factual inquiry by the court, the owner of the copyright in
the original work will submit its own list of what it believes to be the
protected elements of the original program. In those situations, the court
will forgo performing its own abstraction, and proceed to the second step of
the AFC test.

\subsection{Filtration}

The most difficult and controversial part of the AFC test is the second
step, which entails the filtration of protectable expression contained in
the original program from any unprotectable elements nestled therein. In
determining which elements of a program are unprotectable, courts employ a
myriad of rules and procedures to sift from a program all the portions
that are not eligible for copyright protection.

First, as set forth in \S~102(b) of the Copyright Act, any and all ideas
embodied in the program are to be denied copyright protection. However,
implementing this rule is not as easy as it first appears. The courts
readily recognize the intrinsic difficulty in distinguishing between ideas
and expression and that, given the varying nature of computer programs,
doing so will be done on an ad hoc basis. The first step of the AFC test,
the abstraction, exists precisely to assist in this endeavor by helping
the court separate out all the individual elements of the program so that
they can be independently analyzed for their expressive nature.

A second rule applied by the courts in performing the filtration step of
the AFC test is the doctrine of merger, which denies copyright protection
to expression necessarily incidental to the idea being expressed. The
reasoning behind this doctrine is that when there is only one way to
express an idea, the idea and the expression merge, meaning that the
expression cannot receive copyright protection due to the bar on copyright
protection extending to ideas. In applying this doctrine, a court will ask
whether the program's use of particular code or structure is necessary for
the efficient implementation of a certain function or process. If so, then
that particular code or structure is not protected by copyright and, as a
result, it is filtered away from the remaining protectable expression.

A third rule applied by the courts in performing the filtration step of
the AFC test is the doctrine of scenes a faire, which denies copyright
protection to elements of a computer program that are dictated by external
factors. Such external factors can include:

\begin{itemize}

  \item The mechanical
specifications of the computer on which a particular program is intended
to operate

  \item Compatibility requirements of other programs with which a
program is designed to operate in conjunction

  \item Computer manufacturers'
design standards

  \item Demands of the industry being serviced, and

widely accepted programming practices within the computer industry

\end{itemize}

Any code or structure of a program that was shaped predominantly in
response to these factors is filtered out and not protected by
copyright. Lastly, elements of a computer program are also to be filtered
out if they were taken from the public domain or fail to have sufficient
originality to merit copyright protection.

Portions of the source or object code of a computer program are rarely
filtered out as unprotectable elements. However, some distinct parts of
source and object code have been found unprotectable. For example,
constant s, the invariable integers comprising part of formulas used to
perform calculations in a program, are unprotectable. Further, although
common errors found in two programs can provide strong evidence of
copying, they are not afforded any copyright protection over and above the
protection given to the expression containing them.

\subsection{Comparison}

The third and final step of the AFC test entails a comparison of the
original program's remaining protectable expression to a second
program. The issue will be whether any of the protected expression is
copied in the second program and, if so, what relative importance the
copied portion has with respect to the original program overall. The
ultimate inquiry is whether there is ``substantial'' similarity between
the protected elements of the original program and the potentially
derivative work. The courts admit that this process is primarily
qualitative rather than quantitative and is performed on a case-by-case
basis. In essence, the comparison is an ad hoc determination of whether
the protectable elements of the original program that are contained in the
second work are significant or important parts of the original program. If
so, then the second work is a derivative work of the first. If, however,
the amount of protectable elements copied in the second work are so small
as to be de minimis, then the second work is not a derivative work of the
original.

\section{Analytic Dissection Test}

The Ninth Circuit has adopted the analytic dissection test to determine
whether one program is a derivative work of another. Apple Computer,
Inc. v. Microsoft Corp., 35 F.3d 1435 (9th Cir. 1994). The analytic
dissection test first considers whether there are substantial similarities
in both the ideas and expressions of the two works at issue. Once the
similar features are identified, analytic dissection is used to determine
whether any of those similar features are protected by copyright. This
step is the same as the filtration step in the AFC test. After identifying
the copyrightable similar features of the works, the court then decides
whether those features are entitled to ``broad'' or ``thin''
protection. ``Thin'' protection is given to non-copyrightable facts or
ideas that are combined in a way that affords copyright protection only
from their alignment and presentation, while ``broad'' protection is given
to copyrightable expression itself. Depending on the degree of protection
afforded, the court then sets the appropriate standard for a subjective
comparison of the works to determine whether, as a whole, they are
sufficiently similar to support a finding that one is a derivative work of
the other. ``Thin'' protection requires the second work be virtually
identical in order to be held a derivative work of an original, while
``broad'' protection requires only a ``substantial similarity.''

\section{No Protection for ``Methods of Operation''}

The First Circuit has taken the position that the AFC test is inapplicable 
when the works in question relate to unprotectable elements set forth in 
\S~102(b).  Their approach results in a much narrower definition
of derivative work for software in comparison to other circuits. Specifically, 
the
First Circuit holds that ``method of operation,'' as used in \S~102(b) of
the Copyright Act, refers to the means by which users operate
computers. Lotus Development Corp. v. Borland Int’l., Inc., 49 F.3d 807
(1st Cir. 1995).  In Lotus, the court held that a menu command
hierarchy for a computer program was uncopyrightable because it did not
merely explain and present the program’s functional capabilities to the
user, but also served as a method by which the program was operated and
controlled. As a result, under the First Circuit’s test, literal copying
of a menu command hierarchy, or any other ``method of operation,'' cannot
form the basis for a determination that one work is a derivative of
another.  As a result, courts in the First Circuit that apply the AFC test
do so only after applying a broad interpretation of \S~102(b) to filter out
unprotected elements. E.g., Real View, LLC v. 20-20 Technologies, Inc., 
683 F. Supp.2d 147, 154 (D. Mass. 2010).


\section{No Test Yet Adopted}

Several circuits, most notably the Fourth and Seventh, have yet to
declare their definition of derivative work and whether or not the
AFC, Analytic Dissection, or some other test best fits their
interpretation of copyright law. Therefore, uncertainty exists with
respect to determining the extent to which a software program is a
derivative work of another in those circuits. However, one may presume
that they would give deference to the AFC test since it is by far the
majority rule amongst those circuits that have a standard for defining
a software derivative work.

\section{Cases Applying Software Derivative Work Analysis}

In the preeminent case regarding the definition of a derivative work for
software, Computer Associates v. Altai, the plaintiff alleged that its
program, Adapter, which was used to handle the differences in operating
system calls and services, was infringed by the defendant's competitive
program, Oscar. About 30\% of Oscar was literally the same code as
that in Adapter. After the suit began, the defendant rewrote those
portions of Oscar that contained Adapter code in order to produce a new
version of Oscar that was functionally competitive with Adapter, without
have any literal copies of its code. Feeling slighted still, the
plaintiff alleged that even the second version of Oscar, despite having no
literally copied code, also infringed its copyrights. In addressing that
question, the Second Circuit promulgated the AFC test.

In abstracting the various levels of the program, the court noted a
similarity between the two programs' parameter lists and macros. However,
following the filtration step of the AFC test, only a handful of the lists
and macros were protectable under copyright law because they were either
in the public domain or required by functional demands on the
program. With respect to the handful of parameter lists and macros that
did qualify for copyright protection, after performing the comparison step
of the AFC test, it was reasonable for the district court to conclude that
they did not warrant a finding of infringement given their relatively minor
contribution to the program as a whole. Likewise, the similarity between
the organizational charts of the two programs was not substantial enough
to support a finding of infringement because they were too simple and
obvious to contain any original expression.

In the case of Oracle America v. Google, 872 F. Supp.2d 974 (N.D. Cal. 2012),
the Northern District of California District Court examined the question of 
whether the application program interfaces (APIs) associated with the Java
programming language are entitled to copyright protection.  While the 
court expressly declined to rule whether all APIs are free to use without 
license (872 F. Supp.2d 974 at 1002), the court held that the command 
structure and taxonomy of the APIs were not protectable under copyright law.
Specifically, the court characterized the command structure and taxonomy as
both a ``method of operation'' (using an approach not dissimilar to the 
First Circuit's analysis in Lotus) and a ``functional requirement for 
compatability'' (using Sega v. Accolade, 977 F.2d 1510 (9th Cir. 1992) and
Sony Computer Ent. v. Connectix, 203 F.3d 596 (9th Cir. 2000) as analogies),
and thus unprotectable subject matter under \S~102(b). 

Perhaps not surprisingly, there have been few other cases involving a highly
detailed software derivative work analysis. Most often, cases involve
clearer basis for decision, including frequent bad faith on the part of
the defendant or overaggressiveness on the part of the plaintiff.  

%%%%%%%%%%%%%%%%%%%%%%%%%%%%%%%%%%%%%%%%%%%%%%%%%%%%%%%%%%%%%%%%%%%%%%%%%%%%%%%

\chapter{Modified Source and Binary Distribution}
\label{source-and-binary}

In this chapter, we discuss the two core sections that define the rights
and obligations for those who modify, improve, and/or redistribute GPL'd
software. These sections, GPLv2~\S\S2--3, define the central core rights and
requirements of GPLv2\@.

\section{GPLv2~\S2: Share and Share Alike}
\label{GPLv2s2}

For many, this is where the ``magic'' happens that defends software
freedom upon redistribution.  GPLv2~\S2 is the only place in GPLv2
that governs the modification controls of copyright law.  If users
modifies a GPLv2'd program, they must follow the terms of GPLv2~\S2 in making
those changes.  Thus, this sections ensures that the body of GPL'd software, as it
continues and develops, remains Free as in freedom.

To achieve that goal, GPLv2~\S2 first sets forth that the rights of
redistribution of modified versions are the same as those for verbatim
copying, as presented in GPLv2~\S1.  Therefore, the details of charging money,
keeping copyright notices intact, and other GPLv2~\S1 provisions are in tact
here as well.  However, there are three additional requirements.

The first (GPLv2~\S2(a)) requires that modified files carry ``prominent
notices'' explaining what changes were made and the date of such
changes. This section does not prescribe some specific way of
marking changes nor does it control the process of how changes are made.
Primarily, GPLv2~\S2(a) seeks to ensure that those receiving modified
versions know the history of changes to the software.  For some users,
it is important to know that they are using the standard version of
program, because while there are many advantages to using a fork,
there are a few disadvantages.  Users should be informed about the
historical context of the software version they use, so that they can
make proper support choices.  Finally, GPLv2~\S2(a) serves an academic
purpose --- ensuring that future developers can use a diachronic
approach to understand the software.

\medskip

The second requirement (GPLv2~\S2(b)) contains the four short lines that embody
the legal details of ``share and share alike''.  These 46 words are
considered by some to be the most worthy of careful scrutiny because
GPLv2~\S2(b), and they
can be a source of great confusion when not properly understood.

In considering GPLv2~\S2(b), first note the qualifier: it \textit{only} applies to
derivative works that ``you distribute or publish''.  Despite years of
education efforts on this matter, many still believe that modifiers
of GPL'd software \textit{must} to publish or otherwise
share their changes.  On the contrary, GPLv2~\S2(b) {\bf does not apply if} the
changes are never distributed.  Indeed, the freedom to make private,
personal, unshared changes to software for personal use only should be
protected and defended.\footnote{Most Free Software enthusiasts believe there is an {\bf
    moral} obligation to redistribute changes that are generally useful,
  and they often encourage companies and individuals to do so.  However, there
  is a clear distinction between what one {\bf ought} to do and what one
  {\bf must} do.}

Next, we again encounter the same matter that appears in GPLv2~\S0, in the
following text:
\begin{quote}
``...that in whole or part contains or is derived from the Program or any part thereof.''
\end{quote}
Again, the GPL relies here on what the copyright law says is a derivative
work.  If, under copyright law, the modified version ``contains or is
derived from'' the GPL'd software, then the requirements of GPLv2~\S2(b)
apply.  The GPL invokes its control as a copyright license over the
modification of the work in combination with its control over distribution
of the work.

The final clause of GPLv2~\S2(b) describes what the licensee must do if she is
distributing or publishing a work that is deemed a derivative work under
copyright law --- namely, the following:
\begin{quote}
[The work must] be licensed as a whole at no charge to all third parties
under the terms of this License.
\end{quote}
That is probably the most tightly-packed phrase in all of the GPL\@.
Consider each subpart carefully.

The work ``as a whole'' is what is to be licensed. This is an important
point that GPLv2~\S2 spends an entire paragraph explaining; thus this phrase is
worthy of a lengthy discussion here.  As a programmer modifies a software
program, she generates new copyrighted material --- fixing expressions of
ideas into the tangible medium of electronic file storage.  That
programmer is indeed the copyright holder of those new changes.  However,
those changes are part and parcel to the original work distributed to
the programmer under GPL\@. Thus, the license of the original work
affects the license of the new whole derivative work.

% {\cal I}
\newcommand{\gplusi}{$\mathcal{G\!\!+\!\!I}$}
\newcommand{\worki}{$\mathcal{I}$}
\newcommand{\workg}{$\mathcal{G}$}

\label{separate-and-independent}

It is certainly possible to take an existing independent work (called
\worki{}) and combine it with a GPL'd program (called \workg{}).  The
license of \worki{}, when it is distributed as a separate and independent
work, remains the prerogative of the copyright holder of \worki{}.
However, when \worki{} is combined with \workg{}, it produces a new work
that is the combination of the two (called \gplusi{}). The copyright of
this combined work, \gplusi{}, is held by the original copyright
holder of each of the two works.

In this case, GPLv2~\S2 lays out the terms by which \gplusi{} may be
distributed and copied.  By default, under copyright law, the copyright
holder of \worki{} would not have been permitted to distribute \gplusi{};
copyright law forbids it without the expressed permission of the copyright
holder of \workg{}. (Imagine, for a moment, if \workg{} were a proprietary
product --- would its copyright holders  give you permission to create and distribute
\gplusi{} without paying them a hefty sum?)  The license of \workg{}, the
GPL, states the  options for the copyright holder of \worki{}
who may want to create and distribute \gplusi{}.  GPL's pregranted
permission to create and distribute derivative works, provided the terms
of GPL are upheld, goes far above and beyond the permissions that one
would get with a typical work not covered by a copyleft license.  (Thus, to
say that this restriction is any way unreasonable is simply ludicrous.)

\medskip

\label{GPLv2s2-at-no-charge}
The next phrase of note in GPLv2~\S2(b) is ``licensed \ldots at no charge.''
This phrase  confuses many.  The sloppy reader points out this as ``a
contradiction in GPL'' because (in their confused view) that clause of GPLv2~\S2 says that redistributors cannot
charge for modified versions of GPL'd software, but GPLv2~\S1 says that
they can.  Avoid this confusion: the ``at no charge'' \textbf{does not} prohibit redistributors from
charging when performing the acts governed by copyright
law,\footnote{Recall that you could by default charge for any acts not
governed by copyright law, because the license controls are confined
by copyright.} but rather that they cannot charge a fee for the
\emph{license itself}.  In other words, redistributors of (modified
and unmodified) GPL'd works may charge any amount they choose for
performing the modifications on contract or the act of transferring
the copy to the customer, but they may not charge a separate licensing
fee for the software.

GPLv2~\S2(b) further states that the software must ``be licensed \ldots to all
third parties.''  This too yields some confusion, and feeds the
misconception mentioned earlier --- that all modified versions must made
available to the public at large.  However, the text here does not say
that.  Instead, it says that the licensing under terms of the GPL must
extend to anyone who might, through the distribution chain, receive a copy
of the software.  Distribution to all third parties is not mandated here,
but GPLv2~\S2(b) does require redistributors to license the derivative works in
a way that extends to all third parties who may ultimately receive a
copy of the software.

In summary, GPLv2\ 2(b) says what terms under which the third parties must
receive this no-charge license.  Namely, they receive it ``under the terms
of this License'', the GPLv2.  When an entity \emph{chooses} to redistribute
a derivative work of GPL'd software, the license of that whole 
work must be GPL and only GPL\@.  In this manner, GPLv2~\S2(b) dovetails nicely
with GPLv2~\S6 (as discussed in Section~\ref{GPLv2s6} of this tutorial).

\medskip

The final paragraph of GPLv2~\S2 is worth special mention.  It is possible and
quite common to aggregate various software programs together on one
distribution medium.  Computer manufacturers do this when they ship a
pre-installed hard drive, and GNU/Linux distribution vendors do this to
give a one-stop CD or URL for a complete operating system with necessary
applications.  The GPL very clearly permits such ``mere aggregation'' with
programs under any license.  Despite what you hear from its critics, the
GPL is nothing like a virus, not only because the GPL is good for you and
a virus is bad for you, but also because simple contact with a GPL'd
code-base does not impact the license of other programs.  A programmer must
expended actual effort  to cause a work to fall under the terms
of the GPL.  Redistributors are always welcome to simply ship GPL'd
software alongside proprietary software or other unrelated Free Software,
as long as the terms of GPL are adhered to for those packages that are
truly GPL'd.

\section{GPLv2~\S3: Producing Binaries}
\label{GPLv2s3}

Software is a strange beast when compared to other copyrightable works.
It is currently impossible to make a film or a book that can be truly
obscured.  Ultimately, the full text of a novel, even one written by
William Faulkner, must presented to the reader as words in some
human-readable language so that they can enjoy the work.  A film, even one
directed by David Lynch, must be perceptible by human eyes and ears to
have any value.

Software is not so.  While the source code --- the human-readable
representation of software is of keen interest to programmers -- users and
programmers alike cannot make the proper use of software in that
human-readable form.  Binary code --- the ones and zeros that the computer
can understand --- must be predicable and attainable for the software to
be fully useful.  Without the binaries, be they in object or executable
form, the software serves only the didactic purposes of computer science.

Under copyright law, binary representations of the software are simply
derivative works of the source code.  Applying a systematic process (i.e.,
``compilation''\footnote{``Compilation'' in this context refers to the
  automated computing process of converting source code into binaries.  It
  has absolutely nothing to do with the term ``compilation'' in copyright statues.}) to a work of source code yields binary code. The binary
code is now a new work of expression fixed in the tangible medium of
electronic file storage.

Therefore, for GPL'd software to be useful, the GPL, since it governs the
rules for creation of derivative works, must grant permission for the
generation of binaries.  Furthermore, notwithstanding the relative
popularity of source-based GNU/Linux distributions like Gentoo, users find
it extremely convenient to receive distribution of binary software.  Such
distribution is the redistribution of derivative works of the software's
source code.  GPLv2~\S3 addresses the matter of creation and distribution of
binary versions.

Under GPLv2~\S3, binary versions may be created and distributed under the
terms of GPLv2~\S1--2, so all the material previously discussed applies
here.  However, GPLv2~\S3 must go a bit further.  Access to the software's
source code is an incontestable prerequisite for the exercise of the
fundamental freedoms to modify and improve the software.  Making even
the most trivial changes to a software program at the binary level is
effectively impossible.  GPLv2~\S3 must ensure that the binaries are never
distributed without the source code, so that these freedoms are passed
through the distribution chain.

GPLv2~\S3 permits distribution of binaries, and then offers three options for
distribution of source code along with binaries. The most common and the
least complicated is the option given under GPLv2~\S3(a).

GPLv2~\S3(a) offers the option to directly accompany the source code alongside
the distribution of the binaries.  This is by far the most convenient
option for most distributors, because it means that the source-code
provision obligations are fully completed at the time of binary
distribution (more on that later).

Under GPLv2~\S3(a), the source code provided must be the ``corresponding source
code.''  Here ``corresponding'' primarily means that the source code
provided must be that code used to produce the binaries being distributed.
That source code must also be ``complete''.   GPLv2~\S3's penultimate paragraph
explains in detail what is meant by ``complete''.  In essence, it is all
the material that a programmer of average skill would need to actually use
the source code to produce the binaries she has received.  Complete source
is required so that, if the licensee chooses, she should be able to
exercise her freedoms to modify and redistribute changes.  Without the
complete source, it would not be possible to make changes that were
actually directly derived from the version received.

Furthermore, GPLv2~\S3 is defending against a tactic that has in fact been
seen in GPL enforcement.  Under GPL, if you pay a high price for
a copy of GPL'd binaries (which comes with corresponding source, of
course), you have the freedom to redistribute that work at any fee you
choose, or not at all.  Sometimes, companies attempt a GPL-violating
cozenage whereby they produce very specialized binaries (perhaps for
an obscure architecture).  They then give source code that does
correspond, but withhold the ``incantations'' and build plans they
used to make that source compile into the specialized binaries.
Therefore, GPLv2~\S3 requires that the source code include ``meta-material'' like
scripts, interface definitions, and other material that is used to
``control compilation and installation'' of the binaries.  In this
manner, those further down the distribution chain are assured that
they have the unabated freedom to build their own derivative works
from the sources provided.

Software distribution comes in many
forms.  Embedded manufacturers, for example, have the freedom to put
GPL'd software into mobile devices with very tight memory and space
constraints.  In such cases, putting the source right alongside the
binaries on the machine itself might not be an option.  While it is
recommended that this be the default way that people comply with GPL, the
GPL does provide options when such distribution is infeasible.

\label{GPLv2s3-medium-customarily}
GPLv2~\S3, therefore, allows source code to be provided on any physical
``medium customarily used for software interchange.''  By design, this
phrase covers a broad spectrum --- the phrase seeks to pre-adapt to
changes in  technology.  When GPLv22 was first published in June
1991, distribution on magnetic tape was still common, and CD was
relatively new.  By 2002, CD is the default.  By 2007, DVD's were the
default.  Now, it's common to give software on USB drives and SD card.  This
language in the license must adapt with changing technology.

Meanwhile, the binding created by the word ``customarily'' is key.  Many
incorrectly believe that distributing binary on CD and source on the
Internet is acceptable.  In the corporate world in industrialized countries, it is indeed customary to
simply download a CDs' worth of data quickly.  However, even today in the USA, many computer users are not connected to the Internet, and most people connected
to the Internet still have limited download speeds.  Downloading
CDs full of data is not customary for them in the least.  In some cities
in Africa, computers are becoming more common, but Internet connectivity
is still available only at a few centralized locations.  Thus, the
``customs'' here are normalized for a worldwide userbase.  Simply
providing source on the Internet --- while it is a kind, friendly and
useful thing to do --- is not usually sufficient.

Note, however, a major exception to this rule, given by the last paragraph
of GPLv2~\S3. \emph{If} distribution of the binary files is made only on the
Internet (i.e., ``from a designated place''), \emph{then} simply providing
the source code right alongside the binaries in the same place is
sufficient to comply with GPLv2~\S3.

\medskip

As is shown above, Under GPLv2~\S3(a), embedded manufacturers can put the
binaries on the device and ship the source code along on a CD\@.  However,
sometimes this turns out to be too costly.  Including a CD with every
device could prove too costly, and may practically (although not legally)
prohibit using GPL'd software. For this situation and others like it, GPLv2\S~3(b) is available.

GPLv2~\S3(b) allows a distributor of binaries to instead provide a written
offer for source code alongside those binaries.  This is useful in two
specific ways.  First, it may turn out that most users do not request the
source, and thus the cost of producing the CDs is saved --- a financial
and environmental windfall.  In addition, along with a GPLv2~\S3(b) compliant
offer for source, a binary distributor might choose to \emph{also} give a
URL for source code.  Many who would otherwise need a CD with source might
turn out to have those coveted high bandwidth connections, and are able to
download the source instead --- again yielding environmental and financial
windfalls.

However, note that regardless of how many users prefer to get the
source online, GPLv2~\S3(b) does place lasting long-term obligations on the
binary distributor.  The binary distributor must be prepared to honor
that offer for source for three years and ship it out (just as they
would have had to do under GPLv2~\S3(a)) at a moment's notice when they
receive such a request.  There is real organizational cost here:
support engineers must be trained how to route source requests, and
source CD images for every release version for the last three years
must be kept on hand to burn such CDs quickly. The requests might not
even come from actual customers; the offer for source must be valid
for ``any third party''.

That phrase is another place where some get confused --- thinking again
that full public distribution of source is required.  The offer for source
must be valid for ``any third party'' because of the freedoms of
redistribution granted by GPLv2~\S\S1--2.  A company may ship a binary image
and an offer for source to only one customer.  However, under GPL, that
customer has the right to redistribute that software to the world if she
likes.  When she does, that customer has an obligation to make sure that
those who receive the software from her can exercise their freedoms under
GPL --- including the freedom to modify, rebuild, and redistribute the
source code.

GPLv2~\S3(c) is created to save her some trouble, because by itself GPLv2~\S3(b)
would unfairly favor large companies.  GPLv2~\S3(b) allows the
separation of the binary software from the key tool that people can use
to exercise their freedom. The GPL permits this separation because it is
good for redistributors, and those users who turn out not to need the
source.  However, to ensure equal rights for all software users, anyone
along the distribution chain must have the right to get the source and
exercise those freedoms that require it.

Meanwhile, GPLv2~\S3(b)'s compromise primarily benefits companies who
distribute binary software commercially.  Without GPLv2~\S3(c), that benefit
would be at the detriment of the companies' customers; the burden of
source code provision would be unfairly shifted to the companies'
customers.  A customer, who had received binaries with a GPLv2~\S3(b)-compliant
offer, would be required under GPLv2 (sans GPLv2~\S3(c)) to acquire the source,
merely to give a copy of the software to a friend who needed it.  GPLv2~\S3(c)
reshifts this burden to entity who benefits from GPLv2~\S3(b).

GPLv2~\S3(c) allows those who undertake \emph{noncommercial} distribution to
simply pass along a GPLv2~\S3(b)-compliant source code offer.  The customer who
wishes to give a copy to her friend can now do so without provisioning the
source, as long as she gives that offer to her friend.  By contrast, if
she wanted to go into business for herself selling CDs of that software,
she would have to acquire the source and either comply via GPLv2~\S3(a), or
write her own GPLv2~\S3(b)-compliant source offer.

This process is precisely the reason why a GPLv2~\S3(b) source offer must be
valid for all third parties.  At the time the offer is made, there is no
way of knowing who might end up noncommercially receiving a copy of the
software.  Companies who choose to comply via GPLv2~\S3(b) must thus be
prepared to honor all incoming source code requests.  For this and the
many other additional necessary complications under GPLv2~\S\S3(b--c), it is
only rarely a better option than complying via GPLv2~\S3(a).

%%%%%%%%%%%%%%%%%%%%%%%%%%%%%%%%%%%%%%%%%%%%%%%%%%%%%%%%%%%%%%%%%%%%%%%%%%%%%%%
\chapter{GPL's Implied Patent Grant}
\label{gpl-implied-patent-grant}

We digress again briefly from our section-by-section consideration of GPLv2
to consider the interaction between the terms of GPL and patent law. The
GPLv2, despite being silent with respect to patents, actually confers on its
licensees more rights to a licensor's patents than those licenses that
purport to address the issue. This is the case because patent law, under
the doctrine of implied license, gives to each distributee of a patented
article a license from the distributor to practice any patent claims owned
or held by the distributor that cover the distributed article. The
implied license also extends to any patent claims owned or held by the
distributor that cover ``reasonably contemplated uses'' of the patented
article. To quote the Federal Circuit Court of Appeals, the highest court
for patent cases other than the Supreme Court:

\begin{quotation}
Generally, when a seller sells a product without restriction, it in
effect promises the purchaser that in exchange for the price paid, it will
not interfere with the purchaser's full enjoyment of the product
purchased. The buyer has an implied license under any patents of the
seller that dominate the product or any uses of the product to which the
parties might reasonably contemplate the product will be put.
\end{quotation}
Hewlett-Packard Co. v. Repeat-O-Type Stencil Mfg. Corp., Inc., 123 F.3d
1445, 1451 (Fed. Cir. 1997).

Of course, Free Software is licensed, not sold, and there are indeed
restrictions placed on the licensee, but those differences are not likely
to prevent the application of the implied license doctrine to Free
Software, because software licensed under the GPL grants the licensee the
right to make, use, and sell the software, each of which are exclusive
rights of a patent holder. Therefore, although the GPLv2 does not expressly
grant the licensee the right to do those things under any patents the
licensor may have that cover the software or its reasonably contemplated
uses, by licensing the software under the GPLv2, the distributor impliedly
licenses those patents to the GPLv2 licensee with respect to the GPLv2'd
software.

An interesting issue regarding this implied patent license of GPLv2'd
software is what would be considered ``uses of the [software] to which
the parties might reasonably contemplate the product will be put.'' A
clever advocate may argue that the implied license granted by GPLv2 is
larger in scope than the express license in other Free Software
licenses with express patent grants, in that the patent license
clause of many of those other Free  Software licenses are specifically 
limited to the patent claims covered by the code as licensed by the patentee.

In contrast, a GPLv2 licensee, under the doctrine of implied patent license, 
is free to practice any patent claims held by the licensor that cover 
``reasonably contemplated uses'' of the GPL'd code, which may very well 
include creation and distribution of derivative works since the GPL's terms, 
under which the patented code is distributed, expressly permits such activity.


Further supporting this result is the Federal Circuit's pronouncement that
the recipient of a patented article has, not only an implied license to
make, use, and sell the article, but also an implied patent license to
repair the article to enable it to function properly, Bottom Line Mgmt.,
Inc. v. Pan Man, Inc., 228 F.3d 1352 (Fed. Cir. 2000). Additionally, the
Federal Circuit extended that rule to include any future recipients of the
patented article, not just the direct recipient from the distributor.
This theory comports well with the idea of Free Software, whereby software
is distributed amongst many entities within the community for the purpose
of constant evolution and improvement. In this way, the law of implied
patent license used by the GPLv2 ensures that the community mutually
benefits from the licensing of patents to any single community member.



Note that simply because GPLv2'd software has an implied patent license does
not mean that any patents held by a distributor of GPLv2'd code become
worthless. To the contrary, the patents are still valid and enforceable
against either:

\begin{enumerate}
 \renewcommand{\theenumi}{\alph{enumi}}
 \renewcommand{\labelenumi}{\textup{(\theenumi)}}

\item any software other than that licensed under the GPLv2 by the patent
  holder, and

\item any party that does not comply with the GPLv2
with respect to the licensed software.
\end{enumerate}

\newcommand{\compB}{$\mathcal{B}$}
\newcommand{\compA}{$\mathcal{A}$}

For example, if Company \compA{} has a patent on advanced Web browsing, but
also licenses a Web browsing software program under the GPLv2, then it
cannot assert the patent against any party based on that party's use of 
Company \compA{}'s GPL'ed Web browsing software program, or on that party's
creation and use of derivative works of that GPL'ed program.  However, if a
party uses that program without
complying with the GPLv2, then Company \compA{} can assert both copyright
infringement claims against the non-GPLv2-compliant party and
infringement of the patent, because the implied patent license only
extends to use of the software in accordance with the GPLv2. Further, if
Company \compB{} distributes a competitive advanced Web browsing program 
that is not a derivative work of Company \compA{}'s GPL'ed Web browsing software
program, Company \compA{} is free to assert its patent against any user or
distributor of that product. It is irrelevant whether Company \compB's
program is also distributed under the GPLv2, as Company \compB{} can not grant
implied licenses to Company \compA's patent.

This result also reassures companies that they need not fear losing their
proprietary value in patents to competitors through the GPLv2 implied patent
license, as only those competitors who adopt and comply with the GPLv2's
terms can benefit from the implied patent license. To continue the
example above, Company \compB{} does not receive a free ride on Company
\compA's patent, as Company \compB{} has not licensed-in and then
redistributed Company A's advanced Web browser under the GPLv2. If Company
\compB{} does do that, however, Company \compA{} still has not lost
competitive advantage against Company \compB{}, as Company \compB{} must then,
when it re-distributes Company \compA's program, grant an implied license
to any of its patents that cover the program. Further, if Company \compB{}
relicenses an improved version of Company A's program, it must do so under
the GPLv2, meaning that any patents it holds that cover the improved version
are impliedly licensed to any licensee. As such, the only way Company
\compB{} can benefit from Company \compA's implied patent license, is if it,
itself, distributes Company \compA's software program and grants an
implied patent license to any of its patents that cover that program.

%%%%%%%%%%%%%%%%%%%%%%%%%%%%%%%%%%%%%%%%%%%%%%%%%%%%%%%%%%%%%%%%%%%%%%%%%%%%%%%
\chapter{Defending Freedom on Many Fronts}

Chapters~\ref{run-and-verbatim} and~\ref{source-and-binary} presented the
core freedom-defending provisions of GPLv2\@, which are in GPLv2~\S\S0--3.
GPLv2\S\S~4--7 of the GPLv2 are designed to ensure that GPLv2~\S\S0--3 are
not infringed, are enforceable, are kept to the confines of copyright law but
also  not trumped by other copyright agreements or components of other
entirely separate legal systems.  In short, while GPLv2~\S\S0--3 are the parts
of the license that defend the freedoms of users and programmers,
GPLv2~\S\S4--7 are the parts of the license that keep the playing field clear
so that \S\S~0--3 can do their jobs.

\section{GPLv2~\S4: Termination on Violation}
\label{GPLv2s4}

GPLv2~\S4 is GPLv2's termination clause.  Upon first examination, it seems
strange that a license with the goal of defending users' and programmers'
freedoms for perpetuity in an irrevocable way would have such a clause.
However, upon further examination, the difference between irrevocability
and this termination clause becomes clear.

The GPL is irrevocable in the sense that once a copyright holder grants
rights for someone to copy, modify and redistribute the software under terms
of the GPL, they cannot later revoke that grant.  Since the GPL has no
provision allowing the copyright holder to take such a prerogative, the
license is granted as long as the copyright remains in effect.\footnote{In
  the USA, due to unfortunate legislation, the length of copyright is nearly
  perpetual, even though the Constitution forbids perpetual copyright.} The
copyright holders have the right to relicense the same work under different
licenses (see Section~\ref{Proprietary Relicensing} of this tutorial), or to
stop distributing the GPLv2'd version (assuming GPLv2~\S3(b) was never used),
but they may not revoke the rights under GPLv2 already granted.

In fact, when an entity looses their right to copy, modify and distribute
GPL'd software, it is because of their \emph{own actions}, not that of the
copyright holder.  The copyright holder does not decided when GPLv2~\S4
termination occurs (if ever); rather, the actions of the licensee determine
that.

Under copyright law, the GPL has granted various rights and freedoms to
the licensee to perform specific types of copying, modification, and
redistribution.  By default, all other types of copying, modification, and
redistribution are prohibited.  GPLv2~\S4 says that if you undertake any of
those other types (e.g., redistributing binary-only in violation of GPLv2~\S3),
then all rights under the license --- even those otherwise permitted for
those who have not violated --- terminate automatically.

GPLv2~\S4 makes GPLv2 enforceable.  If licensees fail to adhere to the
license, then they are stuck without any permission under to engage in
activities covered by copyright law.  They must completely cease and desist
from all copying, modification and distribution of the GPL'd software.

At that point, violating licensees must gain the forgiveness of the copyright
holders to have their rights restored.  Alternatively, the violators could
negotiate another agreement, separate from GPL, with the copyright
holder.  Both are common practice, although
\tutorialpartsplit{as discussed in \textit{A Practical Guide to GPL
    Compliance}, there are }{Chapter~\ref{compliance-understanding-whos-enforcing}
  explains further } key differences between these two very different uses of GPL.

\section{GPLv2~\S5: Acceptance, Copyright Style}
\label{GPLv2s5}

GPLv2~\S5 brings us to perhaps the most fundamental misconception and common
confusion about GPLv2\@. Because of the prevalence of proprietary software,
most users, programmers, and lawyers alike tend to be more familiar with
EULAs. EULAs are believed by their authors to be contracts, requiring
formal agreement between the licensee and the software distributor to be
valid. This has led to mechanisms like ``shrink-wrap'' and ``click-wrap''
as mechanisms to perform acceptance ceremonies with EULAs.

The GPL does not need contract law to ``transfer rights.''  Usually, no rights
are transfered between parties.  By contrast, the GPL is primarily a permission
slip to undertake activities that would otherwise have been prohibited
by copyright law.  As such, GPL needs no acceptance ceremony; the
licensee is not even required to accept the license.

However, without the GPL, the activities of copying, modifying and
distributing the software would have otherwise been prohibited.  So, the
GPL says that you only accepted the license by undertaking activities that
you would have otherwise been prohibited without your license under GPL\@.
This is a certainly subtle point, and requires a mindset quite different
from the contractual approach taken by EULA authors.

An interesting side benefit to GPLv2~\S5 is that the bulk of users of Free
Software are not required to accept the license.  Undertaking fair and
unregulated use of the work, for example, does not bind you to the GPL,
since you are not engaging in activity that is otherwise controlled by
copyright law.  Only when you engage in those activities that might have an
impact on the freedom of others does license acceptance occur, and the
terms begin to bind you to fair and equitable sharing of the software.  In
other words, the GPL only kicks in when it needs to for the sake of
freedom.

While GPL is by default a copyright license, it is certainly still possible
to consider GPL as a contract as well.  For example, some distributors chose
to ``wrap'' their software in an acceptance ceremony to GPL, and nothing in
GPL prohibits that use.  Furthermore, the ruling in \textit{Jacobsen
  v. Katzer, 535 F.3d 1373, 1380 (Fed.Cir.2008)} indicates that \textbf{both}
copyright and contractual remedies may be sought by a copyright holder
seeking to enforce a license designed to uphold software freedom.

\section{Using GPL Both as a Contract and Copyright License}

\section{GPLv2~\S6: GPL, My One and Only}
\label{GPLv2s6}

A point that was glossed over in Section~\ref{GPLv2s4}'s discussion of GPLv2~\S4
was the irrevocable nature of the GPL\@. The GPLv2 is indeed irrevocable,
and it is made so formally by GPLv2~\S6.

The first sentence in GPLv2~\S6 ensures that as software propagates down the
distribution chain, that each licensor can pass along the license to each
new licensee.  Under GPLv2~\S6, the act of distributing automatically grants a
license from the original licensor to the next recipient.  This creates a
chain of grants that ensure that everyone in the distribution has rights
under the GPLv2\@.  In a mathematical sense, this bounds the bottom ---
making sure that future licensees get no fewer rights than the licensee before.

The second sentence of GPLv2~\S6 does the opposite; it bounds from the top.  It
prohibits any licensor along the distribution chain from placing
additional restrictions on the user.  In other words, no additional
requirements may trump the rights and freedoms given by GPLv2\@.

The final sentence of GPLv2~\S6 makes it abundantly clear that no individual
entity in the distribution chain is responsible for the compliance of any
other.  This is particularly important for noncommercial users who have
passed along a source offer under GPLv2~\S3(c), as they cannot be assured that
the issuer of the offer will honor their GPLv2~\S3 obligations.

In short, GPLv2~\S6 says that your license for the software is your one and
only copyright license allowing you to copy, modify and distribute the
software.

\section{GPLv2~\S7: ``Give Software Liberty or Give It Death!''}
\label{GPLv2s7}

In essence, GPLv2~\S7 is a verbosely worded way of saying for non-copyright
systems what GPLv2~\S6 says for copyright.  If there exists any reason that a
distributor knows of that would prohibit later licensees from exercising
their full rights under GPL, then distribution is prohibited.

Originally, this was designed as the title of this section suggests --- as
a last ditch effort to make sure that freedom was upheld.  However, in
modern times, it has come to give much more.  Now that the body of GPL'd
software is so large, patent holders who would want to be distributors of
GPL'd software have a tough choice.  They must choose between avoiding
distribution of GPL'd software that exercises the teachings of their
patents, or grant a royalty-free, irrevocable, non-exclusive license to
those patents.  Many companies have chosen the latter.

Thus, GPLv2~\S7 rarely gives software death by stopping its distribution.
Instead, it is inspiring patent holders to share their patents in the same
freedom-defending way that they share their copyrighted works.

\section{GPLv2~\S8: Excluding Problematic Jurisdictions}
\label{GPLv2s8}

GPLv2~\S8 is rarely used by copyright holders.  Its intention is that if a
particular country, say Unfreedonia, grants particular patents or allows
copyrighted interfaces (no country to our knowledge even permits those
yet), that the GPLv2'd software can continue in free and unabated
distribution in the countries where such controls do not exist.

As far as is currently known, GPLv2~\S8 has very rarely been formally used by
copyright holders.  Admittedly, some have used GPLv2~\S8 to explain various
odd special topics of distribution (usually related in some way to
GPLv2~\S7).  However, generally speaking, this section is not proven
particularly useful in the more than two decades of GPLv2 history.

Meanwhile, despite many calls by the FSF (and others) for those licensors who
explicitly use this section to come forward and explain their reasoning, no
one ever did.  Furthermore, research conducted during the GPLv3 drafting
process found exactly one licensor who had invoked this section to add an
explicit geographical distribution limitation, and the reasoning for that one
invocation was not fitting with FSF's intended spirit of GPLv2~\S8.  As such,
GPLv2~\S8 was not included at all in GPLv3.

%%%%%%%%%%%%%%%%%%%%%%%%%%%%%%%%%%%%%%%%%%%%%%%%%%%%%%%%%%%%%%%%%%%%%%%%%%%%%%%
\chapter{Odds, Ends, and Absolutely No Warranty}

GPLv2~\S\S0--7 constitute the freedom-defending terms of the GPLv2.  The remainder
of the GPLv2 handles administrivia and issues concerning warranties and
liability.

\section{GPLv2~\S9: FSF as Stewards of GPL}
\label{GPLv2s9}

FSF reserves the exclusive right to publish future versions of the GPL\@;
GPLv2~\S9 expresses this.  While the stewardship of the copyrights on the body
of GPL'd software around the world is shared among thousands of
individuals and organizations, the license itself needs a single steward.
Forking of the code is often regrettable but basically innocuous.  Forking
of licensing is disastrous.

(Chapter~\ref{tale-of-two-copylefts} discusses more about the various
versions of GPL.)

\section{GPLv2~\S10: Relicensing Permitted}
\label{GPLv2s10}

GPLv2~\S10 reminds the licensee of what is already implied by the nature of
copyright law.  Namely, the copyright holder of a particular software
program has the prerogative to grant alternative agreements under separate
copyright licenses.

\section{GPLv2~\S11: No Warranty}
\label{GPLv2s11}

Most warranty disclaimer language shout at you.  The
\href{http://www.law.cornell.edu/ucc/2/2-316}{Uniform Commercial
  Code~\S2-316} requires that disclaimers of warranty be ``conspicuous''.
There is apparently general acceptance that \textsc{all caps} is the
preferred way to make something conspicuous, and that has over decades worked
its way into the voodoo tradition of warranty disclaimer writing.

That said, there is admittedly some authority under USA law suggesting that
effective warranty disclaimers that conspicuousness can be established by
capitalization and is absent when a disclaimer has the same typeface as the
terms surrounding it (see \textit{Stevenson v.~TRW, Inc.}, 987 F.2d 288, 296
(5th Cir.~1993)).  While GPLv3's drafters doubted that such authority would
apply to copyright licenses like the GPL, the FSF has nevertheless left
warranty and related disclaimers in \textsc{all caps} throughout all versions
of GPL\@\footnote{One of the authors of this tutorial, Bradley M.~Kuhn, has
  often suggested the aesthetically preferable compromise of a
  \textsc{specifically designed ``small caps'' font, such as this one, as an
    alternative to} WRITING IN ALL CAPS IN THE DEFAULT FONT (LIKE THIS),
  since the latter adds more ugliness than conspicuousness.  Kuhn once
  engaged in reversion war with a lawyer who disagreed, but that lawyer never
  answered Kuhn's requests for case law that argues THIS IS INHERENTLY MORE
  CONSPICUOUS \textsc{Than this is}.}.

Some have argued the GPL is unenforceable in some jurisdictions because
its disclaimer of warranties is impermissibly broad.  However, GPLv2~\S11
contains a jurisdictional savings provision, which states that it is to be
interpreted only as broadly as allowed by applicable law.  Such a
provision ensures that both it, and the entire GPL, is enforceable in any
jurisdiction, regardless of any particular law regarding the
permissibility of certain warranty disclaimers.

Finally, one important point to remember when reading GPLv2~\S11 is that GPLv2~\S1
permits the sale of warranty as an additional service, which GPLv2~\S11 affirms.

\section{GPLv2~\S12: Limitation of Liability}
\label{GPLv2s12}

There are many types of warranties, and in some jurisdictions some of them
cannot be disclaimed.  Therefore, usually agreements will have both a
warranty disclaimer and a limitation of liability, as we have in GPLv2~\S12.
GPLv2~\S11 thus gets rid of all implied warranties that can legally be
disavowed. GPLv2~\S12, in turn, limits the liability of the actor for any
warranties that cannot legally be disclaimed in a particular jurisdiction.

Again, some have argued the GPL is unenforceable in some jurisdictions
because its limitation of liability is impermissibly broad. However, \S
12, just like its sister, GPLv2~\S11, contains a jurisdictional savings
provision, which states that it is to be interpreted only as broadly as
allowed by applicable law.  As stated above, such a provision ensures that
both GPLv2~\S12, and the entire GPL, is enforceable in any jurisdiction,
regardless of any particular law regarding the permissibility of limiting
liability.

So end the terms and conditions of the GNU General Public License.

%%%%%%%%%%%%%%%%%%%%%%%%%%%%%%%%%%%%%%%%%%%%%%%%%%%%%%%%%%%%%%%%%%%%%%%%%%%%%%%
\chapter{GPL Version 3}
\label{GPLv3}

This chapter discussed the text of GPLv3.  Much of this material herein
includes text that was adapted (with permission) from text that FSF
originally published as part of the so-called ``rationale documents'' for the
various discussion drafts of GPLv3.

The FSF ran a somewhat public process to develop GPLv3, and it was the first
attempt of its kind to develop a Free Software license this way.  Ultimately,
RMS was the primary author of GPLv3, but he listened to feedback from all
sorts of individuals and even for-profit companies.  Nevertheless, in
attempting to understand GPLv3 after the fact, the materials available from
the GPLv3 process have a somewhat ``drinking from the firehose'' effect.
This chapter seeks to explain GPLv3 to newcomers, who perhaps are familiar
with GPLv2 and who did not participate in the GPLv3 process.

Those who wish to drink from the firehose and take a diachronic approach to
GPLv3 study by reading the step-by-step public drafting process GPLv3 (which
occurred from Monday 16 January 2006 through Monday 19 November 2007) should
visit \url{http://gplv3.fsf.org/}.

\section{Understanding GPLv3 As An Upgraded GPLv2}

Ultimately, GPLv2 and GPLv3 co-exist as active licenses in regular use.  As
discussed in Chapter\~ref{tale-of-two-copylefts}, GPLv1 was never regularly
used alongside GPLv2.  However, given GPLv2's widespread popularity and
existing longevity by the time GPLv3 was published, it is not surprising that
some licensors still prefer GPLv2-only or GPLv2-or-later.  GPLv3 gained major
adoption by many projects, old and new, but many projects have not upgraded
due to (in some cases) mere laziness and (in other cases) policy preference
for some of GPLv2's terms and/or policy opposition to GPLv3's terms.

Given this ``two GPLs world'' is reality, it makes sense to consider GPLv3 in
terms of how it differs from GPLv2.  Also, most of the best GPL experts in
the world must deal regularly with both licenses, and admittedly have decades
of experience of GPLv2 while the most experience with GPLv3 that's possible
is by default less than a decade.  These two factors usually cause even new
students of GPL to start with GPLv2 and move on to GPLv3, and this tutorial
follows that pattern.

Overall, the changes made in GPLv3 admittedly \textit{increased} the
complexity of the license.  The FSF stated at the start of the GPLv3 process
that they would have liked to oblige those who have asked for a simpler and
shorter GPL\@.  Ultimately, the FSF gave priority to making GPLv3 a better
copyleft in the spirit of past GPL's.  Obsession for concision should never
trump software freedom.

The FSF had many different, important goals in seeking to upgrade to GPLv3.
However, one important goal that is often lost in the discussion of policy
minutia is a rather simple but important issue.  Namely, FSF sought to assure
that GPLv3 was more easily internationalized than GPLv2.  In particular, the
FSF sought to ease interpretation of GPL in other countries by replacement of
USA-centric\footnote{See Section~\ref{non-usa-copyright} of this tutorial for
  a brief discussion about non-USA copyright systems.}  copyright phrases and
wording with neutral terminology rooted in description of behavior rather
than specific statue.  As can be seen in the section-by-section discussion of
GPLv3 that follows, nearly every section had changes related to issues of
internationalization.
 
\section{GPLv3~\S0: Giving In On ``Defined Terms''}

One of lawyers' most common complaints about GPLv2 is that defined terms in
the document appear throughout.  Most licenses define terms up-front.
However, GPL was always designed both as a document that should be easily
understood both by lawyers and by software developers: it is a document
designed to give freedom to software developers and users, and therefore it
should be comprehensible to that constituency.

Interestingly enough, one coauthor of this tutorial who is both a lawyer and
a developer pointed out that in law school, she understood defined terms more
quickly than other law students precisely because of her programming
background.  For developers, having \verb0#define0 (in the C programming
language) or other types of constants and/or macros that automatically expand
in the place where they are used is second nature.  As such, adding a defined
terms section was not terribly problematic for developers, and thus GPLv3
adds one.  Most of these defined terms are somewhat straightforward and bring
forward better worded definitions from GPLv2.  Herein, this tutorial
discusses a few of the new ones.

GPLv3~\S0 includes definitions of five new terms not found in any form in
GPLv2: ``modify'' ``covered work'', ``propagate'', ``convey'', and
``Appropriate Legal Notices''. 

\subsection{Modify and the Work Based on the Program}

GPLv2 included a defined term, ``work based on the Program'', but also used
the term ``modify'' and ``based on'' throughout the license.  GPLv2's ``work
based on the Program'' definition made use of a legal term of art,
``derivative work'', which is peculiar to USA copyright law.  However,
ironically, the most criticism of USA-specific legal terminology in GPLv2's
``work based on the Program'' definition historically came not primarily from
readers outside the USA, but from those within it\footnote{The FSF noted in
  that it did not generally agree with these views, and expressed puzzlement
  by the energy with which they were expressed, given the existence of many
  other, more difficult legal issues implicated by the GPL.  Nevertheless,
  the FSF argued that it made sense to eliminate usage of local copyright
  terminology to good effect.}.  Admittedly, even though differently-labeled
concepts corresponding to the derivative work are recognized in all copyright
law systems, these counterpart concepts might differ to some degree in scope
and breadth from the USA derivative work.

The goal and intention of GPLv2 was always to cover all rights governed by
relevant copyright law, in the USA and elsewhere.  GPLv3 therefore takes the
task of internationalizing the license further by removing references to
derivative works and by providing a more globally useful definition.  The new
definition returns to the common elements of copyright law.  Copyright
holders of works of software have the exclusive right to form new works by
modification of the original --- a right that may be expressed in various
ways in different legal systems.  GPLv3 operates to grant this right to
successive generations of users (particularly through the copyleft conditions
set forth in GPLv3~\S5, as described later in this tutorial in its
\S~\ref{GPLv3s5}).  Here in GPLv3~\S0, ``modify'' refers to basic copyright
rights, and then this definition of ``modify'' is used to define ``modified
version of'' and ``work based on,'' as synonyms.

\section{The Covered Work}

GPLv3 uses a common license drafting technique of building upon simpler
definitions to make complex ones.  The Program is a defined term found
throughout GPLv2, and the word ``covered'' and the phrase ``covered by this
license'' are used in tandem with the Program in GPLv2, but not as part of a
definition.  GPLv3 offers a single term ``covered work'', which enables some
of the wording in GPLv3 to be simpler and clearer than its GPLv2
counterparts.

\section{Propagate}

The term ``propagate'' serves two purposes.  First, ``propagate'' provides a
simple and convenient means for distinguishing between the kinds of uses of a
work that GPL imposes conditions on and the kinds of uses that GPL does not
(for the most part) impose conditions on.

Second, ``propagate'' helps globalize GPL in its wording and effect.  When a
work is GPL'd, the copyright law of some particular country will govern
certain legal issues arising under the license.  A term like ``distribute''
(or its equivalent in languages other than English) is used in several
national copyright statutes.  Yet, practical experience with GPLv2 revealed
the awkwardness of using the term ``distribution'' in a license intended for
global use: the scope of ``distribution'' in the copyright context can differ
from country to country.  The GPL never necessarily intended the specific
meaning of ``distribution'' that exists under USA (or any other country's)
copyright law.

Indeed, even within a single country and language, the term distribution may
be ambiguous; as a legal term of art, distribution varies significantly in
meaning among those countries that recognize it.  For example, comments
during GPLv3's drafting process indicated that in at least one country,
distribution may not include network transfers of software but may include
interdepartmental transfers of physical copies within an organization.
Meanwhile, the copyright laws of many countries, as well as certain
international copyright treaties, recognize ``making available to the
public'' or ``communication to the public'' as one of the exclusive rights of
copyright holders.

Therefore, the GPL defines the term ``propagate'' by reference to activities
that require permission under ``applicable copyright law'', but excludes
execution and private modification from the definition.  GPLv3's definition
also gives examples of activities that may be included within ``propagation''
but it also makes clear that, under the copyright laws of a given country,
``propagation'' may include other activities as well.

Thus, propagation is defined by behavior, and not by categories drawn from
some particular national copyright statute.  This helps not only with
internationalization, but also factually-based terminology aids in
developers' and users' understanding of GPL\@.

\section{Convey}

Further to this point, a subset of propagate --- ``convey'' --- is defined.
Conveying includes activities that constitute propagation of copies to
others.  As with the definition of propagate, GPLv3 thus addresses transfers
of copies of software in behavioral rather than statutory terms.  

\section{Appropriate Legal Notices}

GPLv2 used the term ``appropriate copyright notice and disclaimer of
warranty'' in two places, which is a rather bulk term.  Also, experience with
GPLv2 and other licenses that grant software freedom showed throughout the
1990s that the scope of types of notices that need preservation upon
conveyance were more broad that merely the copyright notices.  The
Appropriate Legal Notice definition consolidates the material that GPLv2
traditionally required preserved into one definition.

\section{Other Defined Terms}

Note finally that not all defined terms in GPLv3 appear in GPLv3~\S0.
Specifically, those defined terms that are confined in use to a single
section are defined in the section in which they are used, and GPLv3~\S1
contains those definitions focused on source code.  In this tutorial, those
defined terms are discussed in the section where they are defined and/or
used.

\section{GPLv3~\S1: Understanding CCS}

Ensuring that users have the source code to the software they receive and the
freedom to modify remains the paramount right embodied in the Free Software
Definition (found in \S~\ref{Free Software Definition} of this tutorial).  As
such, GPLv3~\S1 is likely one of the most important sections of GPLv3, as it
contains all the defined terms related to this important software freedom.

\subsection{Source Code Definition}

First, GPLv3\~S1 retains GPLv2's definition of ``source code'' and adds an
explicit definition of ``object code'' as ``any non-source version of a
work''.  Object code is not restricted to a narrow technical meaning and is
understood broadly to include any form of the work other than the preferred
form for making modifications to it.  Object code therefore includes any kind
of transformed version of source code, such as bytecode or minified
Javascript.  The definition of object code also ensures that licensees cannot
escape their obligations under the GPL by resorting to shrouded source or
obfuscated programming.

\subsection{CCS Definition}

The definition of CCS\footnote{Note that the preferred term for those who
  work regularly with both GPLv2 and GPLv3 is ``Complete Corresponding
  Source'', abbreviated to ``CCS''.  Admittedly, the word ``complete'' no
  longer appears in GPLv3 (which uses the word ``all'' instead).  However,
  both GPLv2 and the early drafts of GPLv3 itself used the word ``complete'',
  and early GPLv3 drafts even called this defined term ``Complete
  Corresponding Source''.  Meanwhile, use of the acronym ``CCS'' (sometimes,
  ``C\&CS'') was so widespread among GPL enforcers that its use continues
  even though GPLv3-focused experts tend to say just the defined term of
  ``Corresponding Source''.}, or, as GPLv3 officially calls it,
``Corresponding Source'' in GPLv3~\S1\P4 is possibly the most complex
definition in the license.

The CCS definition is broad so as to protect users' exercise of their rights
under the GPL\@.  The definition includes with particular examples to remove
any doubt that they are to be considered CCS\@.  GPLv3 seeks to make it
completely clear that a licensee cannot avoid complying with the requirements
of the GPL by dynamically linking a subprogram component to the original
version of a program.  The example also clarifies that the shared libraries
and dynamically linked subprograms that are included in Corresponding Source
are those that the work is ``specifically'' designed to require, which
clarifies that they do not include libraries invoked by the work that can be
readily substituted by other existing implementations.  While copyleft
advocates never doubted this was required under GPLv2's definition of CCS,
making it abundantly clear with an extra example.

GPL, as always, seeks to ensure users are truly in a position to install and
run their modified versions of the program; the CCS definition is designed to
be expansive to ensure this software freedom.  However, although the
definition of CCS is expansive, it is not sufficient to protect users'
freedoms in many circumstances.  For example, a GPL'd program, or a modified
version of such a program, might be locked-down and restricted.  The
requirements in GPLv3~\S6 (discussed in Section~\ref{GPLv3s6} of this
tutorial) handle that issue.  (Early drafts of GPLv3 included those
requirements in the definition of CCS; however, given that the lock-down
issue only comes up in distribution of object code, it is more logical to
place those requirements with the parts of GPLv3 dealing directly with object
code distribution).

The penultimate paragraph in GPLv3\S2 notes that GPLv3's CCS definition does
not require source that can be automatically generated.  Many code
generators, preprocessors and take source code as input and sometimes even
have output that is still source code.  Source code should always be whatever
the original programmer preferred to modify.

GPLv3\S1's final paragraph removes any ambiguity about what should be done on
source-only distributions.  Specifically, the right to convey source code
that does not compile, does not work, or otherwise is experimental
in-progress work is fully permitted, \textit{provided that} no object code
form is conveyed as well.  Indeed, when combined with the permissions in
GPLv3\S~5, it is clear that if one conveys \textit{only} source code, one can
never be required to provide more than that.  One always has the right to
modify a source code work by deleting any part of it, and there can be no
requirement that free software source code be a whole functioning program.

\subsection{The System Library Exception}

The previous section skipped over one part of the CCS definition, the
so-called system library exception.  The ``System Libraries'' definition (and
the ``Standard Interface'' and ``Major Component'' definitions, which it
includes) are designed as part

to permit certain distribution arrangements that are considered reasonable by
copyleft advocates.  The system library exception is designed to allow
copylefted software to link with these libraries when such linking would hurt
software freedom more than it would hurt proprietary software.

The system library exception has two parts.  Part (a) rewords the GPLv2
exception for clarity replaces GPLv2's words ``unless that component itself
accompanies the executable'' with ``which is not part of the Major
Component''.  The goal here is to not require disclosure of source code of
certain libraries, such as necessary Microsoft Windows DLLs (which aren't
part of Windows' kernel but accompany it) that are required for functioning
of copylefted programs compiled for Windows.

However, in isolation, (a) would be too permissive, as it would sometimes
allowing distributors to evade important GPL requirements.  Part (b) reigns
in (a).  Specifically, (b) specifies only a few functionalities that a the
system library may provide and still qualify for the exception.  The goal is
to ensure system libraries are truly adjunct to a major essential operating
system component, compiler, or interpreter.  The more low-level the
functionality provided by the library, the more likely it is to be qualified
for this exception.

Admittedly, the system library exception is a frequently discussed topic of
obsessed GPL theorists.  The amount that has been written on the system
library exception (both the GPLv2 and GPLv3 versions of it), if included
herein,  could easily increase this section of the tutorial to a length
greater than all the others.

Like any exception to the copyleft requirements of GPL, would-be GPL
violators frequently look to the system library exception as a potential
software freedom circumvention technique.  When considering whether or not a
library qualifies for the system library exception, here is a pragmatic
thesis to consider, based on the combined decades of experience in GPL
interpretation of this tutorial's authors: the harder and more strained the
reader must study and read the system library exception, the more likely it
is that the library in question does not qualify for it.

\section{GPLv3~\S2: Basic Permissions}

GPLv3~\S2 can roughly be considered as an equivalent to GPLv2~\S0 (discussed
in \S~\ref{GPLv2s0} of this tutorial).  However, the usual style of
improvements found in GPLv3 are found here as well.  For example, the first
sentence of GPLv3~\S2 furthers the goal internationalization.  Under the
copyright laws of some countries, it may be necessary for a copyright license
to include an explicit provision setting forth the duration of the rights
being granted. In other countries, including the USA, such a provision is
unnecessary but permissible.

GPLv3~\S2\P1 also acknowledges that licensees under the GPL enjoy rights of
copyright fair use, or the equivalent under applicable law.  These rights are
compatible with, and not in conflict with, the freedoms that the GPL seeks to
protect, and the GPL cannot and should not restrict them.

However, note that (sadly to some copyleft advocates) the unlimited freedom
to run is confined to the \textit{unmodified} Program.  This confinement is
unfortunately necessary since Programs that do not qualify as a User Product
in GPLv3~\S6 (see \S~\ref{user-product} in this tutorial) might have certain
unfortunate restrictions on the freedom to run\footnote{See
  \S~ref{freedom-to-run} of this tutorial for the details on ``the freedom to
  run''.}

GPLv3~\S2\P2 distinguishes between activities of a licensee that are
permitted without limitation and activities that trigger additional
requirements.  Specifically, GPLv3~\S2\P2 guarantees the basic freedoms of
privately modifying and running the program.

Also, GPLv3~\S2\P2 gives an explicit permission for a client to provide a
copy of its modified software to a contractor exclusively for that contractor
to modify it further, or run it, on behalf of the client.  However, the
client can \textit{only} exercise this control over its own copyrighted
changes to the GPL-covered program.  The parts of the program it obtained
from other contributors must be provided to the contractor with the usual GPL
freedoms.  Thus, GPLv3 permits users to convey covered works to contractors
operating exclusively on the users' behalf, under the users' direction and
control, and to require the contractors to keep the users' copyrighted
changes confidential, but \textit{only if} the contractor is limited to acting
on the users' behalf (just as the users' employees would have to act).

The strict conditions in this ``contractors provision'' are needed so that it
cannot be twisted to fit other activities, such as making a program available
to downstream users or customers.  By making the limits on this provision
very narrow, GPLv3 ensures that, in all other cases, contractors gets the
full freedoms of the GPL that they deserve.

The FSF was specifically asked to add this ``contractors provisions'' by
large enterprise users of Free Software, who often contract with non-employee
developers, working offsite, to make modifications intended for the user's
private or internal use, and often arrange with other companies to operate
their data centers.  Whether GPLv2 permits these activities is not clear and
may depend on variations in copyright law in different jurisdictions.  The
practices seem basically harmless, so FSF decided to make it clear they are
permitted.

GPLv3~\S2's final paragraph includes an explicit prohibition of sublicensing.
This provision ensures that GPL enforcement is always by the copyright
holder.  Usually, sublicensing is regarded as a practical convenience or
necessity for the licensee, to avoid having to negotiate a license with each
licensor in a chain of distribution.  The GPL solves this problem in another
way --- through its automatic licensing provision found in GPLv3\~S10 (which
is discussed in more detail in \S\~ref{GPLv3s10} of this tutorial).

\section{GPLv3's views on DRM and Device Lock-Down}
\label{GPLv3-drm}

The issues of DRM, device lock-down and encryption key disclosure were the
most hotly debated during the GPLv3 process.  FSF's views on this were sadly
frequently misunderstood and, comparing the provisions related to these
issues in the earliest drafts of GPLv3 to  the final version of GPLv3 shows
the FSF's willingness to compromise on tactical issues to reach the larger
goal of software freedom.

Specifically, GPLv3 introduced provisions that respond to the growing
practice of distributing GPL-covered programs in devices that employ
technical means to restrict users from installing and running modified
versions.  This practice thwarts the expectations of developers and users
alike, because the right to modify is one of the core freedoms the GPL is
designed to secure.

Technological measures to defeat users' rights.  These measures are often
described by such Orwellian phrases, such as ``digital rights management,''
which actually means limitation or outright destruction of users' legal
rights, or ``trusted computing,'' which actually means selling people
computers they cannot trust.  However, these measures are alike in one basic
respect.  They all employ technical means to turn the system of copyright law
(where the powers of the copyright holder are limited exceptions to general
freedom) into a virtual prison, where everything not specifically permitted
is utterly forbidden.  This system of ``para-copyright'' was created well
after GPLv2 was written --- initially through legislation in the USA and the
EU, and later in other jurisdictions as well.  This legislation creates
serious civil or even criminal penalties to escape from these restrictions
(commonly and aptly called ``jail-breaking a device''), even where the
purpose in doing so is to restore the users' legal rights that the technology
wrongfully prevents them from exercising.

GPLv2 did not address the use of technical measures to take back the rights
that the GPL granted, because such measures did not exist in 1991, and would
have been irrelevant to the forms in which software was then delivered to
users.  GPLv3 addresses these issues, particularly because copylefted
software is ever more widely embedded in devices that impose technical
limitations on the user's freedom to change it.

However, FSF always made a clear distinction to avoid conflating these
``lock-down'' measures with legitimate applications that give users control,
as by enabling them to choose higher levels of system or data security within
their networks, or by allowing them to protect the security of their
communications using keys they can generate or copy to other devices for
sending or receiving messages.  Such technologies present no obstacles to
software freedom and the goals of copyleft.

The public GPLv3 drafting process sought to balance these positions of
copyleft advocates with various desperate views of the larger
Free-Software-using community.  Ultimately, FSF compromised to the GPLv3\S3
and GPLv3\S6 provisions that, taken together, are a minimalist set of terms
sufficient to protect the software freedom against the threat of invasive
para-copyright.

The compromises made were ultimately quite reasonable.  The primary one is
embodied in GPLv3\S6's ``User Product'' definition (see \S~\ref{user-product}
in this tutorial for details).  Additionally, some readers of early GPLv3
drafts seem to have assumed GPLv3 contained a blanket prohibition on DRM; but
it does not.  In fact, no part of GPLv3 forbids DRM regarding non-GPL'd
works; rather, GPLv3 forbids the use of DRM specifically to lock-down
restrictions on users' ability to install modified versions of the GPL'd
software itself, but again, \textit{only} with regard to User Products.

\section{GPLv3~\S3: What Hath DMCA Wrought}
\label{GPLv3s3}

As discussed in \S~\ref{software-and-non-copyright} of this tutorial,
\href{http://www.law.cornell.edu/uscode/text/17/1201}{17 USC~\S1201} and
relate sections\footnote{These sections of the USC are often referred to as
  the ``Digital Millennium Copyright Act'', or ``DMCA'', as that was the name
  of the bill that so-modified these sections of the USC\@.} prohibits users
from circumventing technological measures that implement DRM\@.  Since this
is part of copyright law and the GPL is primarily a copyright license, and
since what the DMCA calls ``circumvention'' is simply ``modifying the
software'' under the GPL, GPLv3 must disclaim such anti-circumvention
provisions are not applicable to the GPLv3'd software.  GPLv3\S3 shields
users from being subjected to liability under anti-circumvention law for
exercising their rights under the GPL, so far as the GPL can do so.

First, GPLv3\S3\P1 declares that no GPL'd program is part of an effective
technological protection measure, regardless of what the program does.  Early
drafts of GPLv3\S3\P1 referred directly to the DMCA, but the final version
instead includes instead an international legal reference to
anticircumvention laws enacted pursuant to the 1996 WIPO treaty and any
similar laws.  Lawyers outside the USA worried that a USA statutory reference
could be read as indicating a choice for application of USA law to the
license as a whole.  While the FSF did not necessarily agree with that view,
the FSF decided anyway to refer to the WIPO treaty rather than DMCA, since
several national anticircumvention laws were (or will likely be) structured
more similarly to the anticircumvention provisions of the DMCA in their
implementation of WIPO\@.  Furthermore, the addition of ``or similar laws''
provides an appropriate catch-all.

Furthermore, GPLv3\S3\P2 states precisely that a conveying party waives the
power to forbid circumvention of technological measures only to the extent
that such circumvention is accomplished through the exercise of GPL rights in
the conveyed work.  GPLv3\S3\P2 makes clear that the referenced ``legal
rights'' are specifically rights arising under anticircumvention law.  and
refers to both the conveying party's rights and to third party rights, as in
some cases the conveying party will also be the party legally empowered to
enforce or invoke rights arising under anticircumvention law.

These disclaimers by each licensor of any intention to use GPL'd software to
stringently control access to other copyrighted works should effectively
prevent any private or public parties from invoking DMCA-like laws against
users who escape technical restriction measures implemented by GPL'd
software.

\section{GPLv3~\S4: Verbatim Copying}
\label{GPLv3s4}

GPLv3~\S4 is a revision of GPLv2\~S1 (as discussed in \S~\ref{GPLv2s1} of
this tutorial).   There are almost no changes to this section from the
GPLv2\~S1, other than to use the new defined terms.

The only notable change of ``a fee'' to ``any price or no price'' in the
first sentence of GPLv3\S4\P2.  The GPLv2\S1\P1 means that the GPL permits
one to charge money for the distribution of software.  Despite efforts by
copyleft advocates to explain this in GPLv2 itself and in other documents,
there are evidently some people who still believe that GPLv2 allows charging
for services but not for selling copies of software and/or that the GPL
requires downloads to be gratis.  Perhaps this is because GPLv2 referred to
charging a ``fee''; the term ``fee'' is generally used in connection with
services.

GPLv2's wording also referred to ``the physical act of transferring.''  The
intention was to distinguish charging for transfers from attempts to impose
licensing fees on all third parties.  ``Physical'' might be read, however, as
suggesting ``distribution in a physical medium only''.

To address these two issues, GPLv3 says ``price'' in place of ``fee,'' and
removes the term ``physical.''

GPLv3~\S4 has also been revised from its corresponding section in GPLv2 in
light of the GPLv3~\S7 (see \S~\ref{GPLv3s7} in this tutorial for more).
Specifically, a distributor of verbatim copies of the program's source code
must obey any existing additional terms that apply to parts of the program
pursuant to GPLv3~\S7.  In addition, the distributor is required to keep
intact all license notices, including notices of such additional terms.

Finally, there is no harm in explicitly pointing out what ought to be
obvious: that those who convey GPL-covered software may offer commercial
services for the support of that software.

\section{GPLv3~\S5: Modified Source}
\label{GPLv3s5}

GPLv3\S5 is the rewrite of GPLv2\S2, which was discussed in \S~\ref{GPLv2s2}
of this tutorial.  This section discusses the changes found in GPLv3\S5
compared to GPLv2\S2.

GPLv3\S5(a) still requires modified versions be marked with ``relevant
date'', but no longer says ``the date of any change''.  The best practice is
to include the date of the latest and/or most significant changes and who
made those.  Of course, compared to its GPLv2\S2(a), GPLv3\S5(a) slightly
relaxes the requirements regarding notice of changes to the program.  In
particular, the modified files themselves need no longer be marked.  This
reduces administrative burdens for developers of modified versions of GPL'd
software.

GPLv3\S5(b) is a new but simple provision. GPLv3\S5(b)  requires that the
license text itself must be unmodified (except as permitted by GPLv3\S7; see
\S~\ref{GPLv3s7} in this tutorial).  Furthermore, it  removes any perceived
conflict between the words ``keep intact all notices'' in GPLv3\S4, since
operating under GPLv3\S5 still includes all the requirements of GPLv3\S4 by
reference.

GPLv3\S5(c) is the primary source-code-related copyleft provision of GPL. (The
object-code-related copyleft provisions are in GPLv3\S6, discussed in
\S~\ref{GPLv3s6} of this tutorial).  Compared to GPLv2\S2(b), GPLv3\S5(c)
states that the GPL applies to the whole of the work.  Such was stated
already in GPLv2\S2(b), in ``in whole or in part'', but this simplified
wording makes it clear the entire covered work

Another change in GPLv3\S5(c) is the removal of the
words ``at no charge,'' which was often is misunderstood upon na\"{i}ve
reading of in GPLv2\S(b) (as discussed in \S~\ref{GPLv2s2-at-no-charge} of this
tutorial).

%  FIXME-LATER: Write up something on 5d, and related it to Appropriate Legal Notices.


Note that of GPLv2~\S2's penultimate and ante-penultimate paragraphs are now
handled adequately by the definitions in GPLv3\S0 and as such, have no direct
analogs in GPLv3.

GPLv2~\S2's final paragraph, however, is reworded and expanded into the final
paragraph of GPLv3\S5, which now also covers issues related to copyright
compilations (but not compilations into object code --- that's in the next
section!).  The intent and scope is the same as was intended in GPLv2.

\section{GPLv3~\S6: Non-Source and Corresponding Source}
\label{GPLv3s6}

GPLv3~\S6 clarifies and revises GPLv2~\S3.  It requires distributors of GPL'd
object code to provide access to the corresponding source code, in one of
four specified ways.  As noted in \S~\ref{GPLv3s0}, ``object code'' in GPLv3
is defined broadly to mean any non-source version of a work.

% FIXME:  probably mostly still right, needs some updates, though.

GPLv3~\S6(a--b) now apply specifically to distribution of object code in a
physical product.  Physical products include embedded systems, as well as
physical software distribution media such as CDs.  As in GPLv2~\S3 (discussed
in \S~\ref{GPLv2s3} of this tutorial), the distribution of object code may
either be accompanied by the machine-readable source code, or it may be
accompanied by a valid written offer to provide the machine-readable source
code.  However, unlike in GPLv2, that offer cannot be exercised by any third
party; rather, only those ``who possesses the object code'' it can exercised
the offer.  (Note that this is a substantial narrowing of requirements of
offer fulfillment, and is a wonderful counterexample to dispute claims that
the GPLv3 has more requirements than GPLv2.)

% FIXME:  probably mostly still right, needs some updates, though.

GPLv3~\S6(b) further revises the requirements for the written offer to
provide source code. As before, the offer must remain valid for at least
three years. In addition, even after three years, a distributor of a product
containing GPL'd object code must offer to provide source code for as long as
the distributor also continues to offer spare parts or customer support for
the product model.  This is a reasonable and appropriate requirement; a
distributor should be prepared to provide source code if he or she is
prepared to provide support for other aspects of a physical product.

GPLv3~\S6(a--b) clarifies that the medium for software interchange on which
the machine-readable source code is provided must be a durable physical
medium.  GPLv3~\S6(b)(2), however, permits a distributor to instead offer to
provide source code from a network server instead, which is yet another
example GPLv3 looser in its requirements than GPLv2 (see
\S~\ref{GPLv2s3-medium-customarily} for details).

% FIXME-LATER: more information about source provision, cost of physically
% performing, reasonable fees, medium customary clearly being said durable
% connecting back to previous text

GPLv3\S6(c) gives narrower permission than GPLv2\S3(c).  The ``pass along''
option for GPLv3\S6(c)(1) offers is now available only for individual
distribution of object code; moreover, such individual distribution can occur
only ``occasionally and noncommercially.''  A distributor cannot comply with
the GPL merely by making object code available on a publicly-accessible
network server accompanied by a copy of the written offer to provide source
code received from an upstream distributor.

%FIXME-LATER: tie back to the discussion of the occasional offer pass along
%             stuff in GPLv2 this tutorial.

GPLv3~\S6(d) revises and improves GPLv2~\S3's final paragraph.  When object
code is provided by offering access to copy the code from a designated place
(such as by enabling electronic access to a network server), the distributor
must merely offer equivalent access to copy the source code ``in the same way
through the same place''.  This wording also permits a distributor to offer a
third party access to both object code and source code on a single network
portal or web page, even though the access may include links to different
physical servers.  For example, a downstream distributor may provide a link
to an upstream distributor's server and arrange with the operator of that
server to keep the source code available for copying for as long as the
downstream distributor enables access to the object code.  This codifies
formally typical historical interpretation of GPLv2.

% FIXME-LATER: perhaps in enforcement section, but maybe here, note about
% ``slow down'' on source downloads being a compliance problem. 

Furthermore, under GPLv3~\S6(d), distributors may charge for the conveyed
object code; however, those who pay to obtain the object code must be given
equivalent and gratis access to obtain the CCS.  (If distributors convey the
object code gratis, distributors must likewise make CCS available without
charge.)  Those who do not obtain the object code from that distributors
(perhaps because they choose not to pay the fee for object code) are outside
the scope of the provision; distributors are under no specific obligation to
give CCS to someone who has not purchased an object code download under
GPLv3~\S6(d).  (Note: this does not change nor impact any obligations under
GPLv3~\S6(b)(2); GPLv3~\S6(d) is a wholly different provision.)

\subsection{GPLv3~\S6(e): Peer-to-Peer Sharing Networks}

Certain decentralized forms of peer-to-peer file sharing present a challenge
to the unidirectional view of distribution that is implicit in GPLv2 and
Draft 1 of GPLv3.  Identification of an upstream/downstream link in
BitTorrent distribution is neither straightforward nor reasonable; such
distribution is multidirectional, cooperative and anonymous.  In peer-to-peer
distribution systems, participants act both as transmitters and recipients of
blocks of a particular file, but they perceive the experience merely as users
and receivers, and not as distributors in any conventional sense.  At any
given moment of time, most peers will not have the complete file.

Meanwhile, GPLv3~\S6(d) permits distribution of a work in object code form
over a network, provided that the distributor offers equivalent access to
copy the Corresponding Source Code ``in the same way through the same
place''.  This wording might be interpreted to permit peer-to-peer
distribution of binaries \textit{if} they are packaged together with the CCS,
but such packaging impractical, for at least three reasons.  First, even if
the CCS is packaged with the object code, it will only be available to a
non-seeding peer at the end of the distribution process, but the peer will
already have been providing parts of the binary to others in the network.
Second, in practice, peer-to-peer forms of transmission are poorly suited
means for distributing CCS.  In large distributions, packaging CCS with the
object code may result in a substantial increase in file size and
transmission time.  Third, in current practice, CCS packages themselves tend
\textit{not} to be transmitted through BitTorrent --- owing to reduced demand
-- thus, there generally will be too few participants downloading the same
source package at the same time to enable effective seeding and distribution.

GPLv3~\S6(e) addresses this issues.  If a licensee conveys such a work of
object code using peer-to-peer transmission, that licensee is in compliance
with GPLv3~\S6 if the licensee informs other peers where the object code and
its CCS are publicly available at no charge under subsection GPLv3~\S6(d).
The CCS therefore need not be provided through the peer-to-peer system that
was used for providing the binary.

Second, GPLv3\S9 also clarifies that ancillary propagation of a covered work
that occurs as part of the process of peer-to-peer file transmission does not
require acceptance, just as mere receipt and execution of the Program does
not require acceptance.  Such ancillary propagation is permitted without
limitation or further obligation.

% FIXME-LATER: Would be nice to explain much more about interactions between
% the various options of GPLv3~\S6(a-e), which might all be in play at once!

\subsection{User Products, Installation Information and Device Lock-Down}

As discussed in \S~\ref{GPLv3-drm} of this tutorial, GPLv3 seeks thwart
technical measures such as signature checks in hardware to prevent
modification of GPLed software on a device.

To address this issue, GPLv3~\S6 requires that parties distributing object
code provide recipients with the source code through certain means.  When
those distributors pass on the CCS, they are also required to pass on any
information or data necessary to install modified software on the particular
device that included it.  (This strategy is not unlike that used in LGPLv2.1
to enable users to link proprietary programs to modified libraries.)

% FIXME-LATER: LGPLv2.1 section should talk about this explicitly and this
%              should be a forward reference here

\subsubsection{User Products}

\label{user-product}

The scope of these requirements are narrow.  GPLv3~\S6 introduces the concept
of a ``User Product'', which includes devices that are sold for personal,
family, or household use.  Distributors are only required to provide
Installation Information when they convey object code in a User Product.

In brief, the right to convey object code in a defined class of ``User
Products,'' under certain circumstances, on providing whatever information is
required to enable a recipient to replace the object code with a functioning
modified version.

This was a compromise that was difficult for the FSF to agree to during the
GPLv3 drafting process.  However, companies and governments that use
specialized or enterprise-level computer facilities reported that they
actually \textit{want} their systems not to be under their own control.
Rather than agreeing to this as a concession, or bowing to pressure, they ask
for this as a \texit{preference}.  It is not clear that GPL should interfere
here, since the main problem lies elsewhere.

While imposing technical barriers to modification is wrong regardless of
circumstances, the areas where restricted devices are of the greatest
practical concern today fall within the User Product definition.  Most, if
not all, technically-restricted devices running GPL-covered programs are
consumer electronics devices.  Moreover, the disparity in clout between the
manufacturers and these users makes it difficult for the users to reject
technical restrictions through their weak and unorganized market power.  Even
limited to User Products, this provision addresses the fundamental problem.

% FIXME-LATER: link \href to USC 2301

The core of the User Product definition is a subdefinition of ``consumer
product'' adapted from the Magnuson-Moss Warranty Act, a federal
consumer protection law in the USA found in 15~USC~\S2301: ``any tangible
personal property which is normally used for personal, family, or household
purposes.''  The USA has had three decades of experience of liberal
judicial and administrative interpretation of this definition in a manner
favorable to consumer rights.\footnote{The Magnuson-Moss consumer product
  definition itself has been influential in the USA and Canada, having been
  adopted in several state and provincial consumer protection laws.}
Ideally, this body of interpretation\footnote{The FSF, however, was very
  clear that incorporation of such legal interpretation was in no way
  intended work as a general choice of USA law for GPLv3.} will guide
interpretation of the consumer product subdefinition in GPLv3~\S6, and this
will hopefully provide a degree of legal certainty advantageous to device
manufacturers and downstream licensees alike.

One well-established interpretive principle under Magnuson-Moss is that
ambiguities are resolved in favor of coverage.  That is, in cases where
it is not clear whether a product falls under the definition of consumer
product, the product will be treated as a consumer product.\footnote{16
CFR~\S\ 700.1(a); \textit{McFadden v.~Dryvit Systems, Inc.}, 54
UCC~Rep.~Serv.2d 934 (D.~Ore.~2004).}  Moreover, for a given product,
``normally used'' is understood to refer to the typical use of that type
of product, rather than a particular use by a particular buyer.
Products that are commonly used for personal as well as commercial
purposes are consumer products, even if the person invoking rights is a
commercial entity intending to use the product for commercial
purposes.\footnote{16 CFR \S \ 700.1(a).  Numerous court decisions
interpreting Magnuson-Moss are in accord; see, e.g., \textit{Stroebner
Motors, Inc.~v.~Automobili Lamborghini S.p.A.}, 459 F.~Supp.2d 1028,
1033 (D.~Hawaii 2006).}  Even a small amount of ``normal'' personal use
is enough to cause an entire product line to be treated as a consumer
product under Magnuson-Moss\footnote{\textit{Tandy Corp.~v.~Marymac
Industries, Inc.}, 213 U.S.P.Q.~702 (S.D.~Tex.~1981). In this case, the
court concluded that TRS-80 microcomputers were consumer products, where
such computers were designed and advertised for a variety of users,
including small businesses and schools, and had only recently been
promoted for use in the home.}.

However, Magnuson-Moss is not a perfect fit because in the area of components
of dwellings, the settled interpretation under Magnuson-Moss underinclusive.
Depending on how such components are manufactured or sold, they may or may
not be considered Magnuson-Moss consumer products.\footnote{Building
  materials that are purchased directly by a consumer from a retailer, for
  improving or modifying an existing dwelling, are consumer products under
  Magnuson-Moss, but building materials that are integral component parts of
  the structure of a dwelling at the time that the consumer buys the dwelling
  are not consumer products. 16 C.F.R.~\S\S~700.1(c)--(f); Federal Trade
  Commission, Final Action Concerning Review of Interpretations of
  Magnuson-Moss Warranty Act, 64 Fed.~Reg.~19,700 (April 22, 1999); see also,
  e.g., \textit{McFadden}, 54 U.C.C.~Rep.~Serv.2d at 934.}  Therefore, GPLv3
defines User Products as a superset of consumer products that also includes
``anything designed or sold for incorporation into a dwelling.''

Thus, the three sentences in the center of GPLv3's User Product definition
encapsulate the judicial and administrative principles established over the
past three decades in the USA concerning the Magnuson-Moss consumer product
definition.  First, it states that doubtful cases are resolved in favor of
coverage under the definition.  Second, it indicate that the words ``normally
used'' in the consumer product definition refer to a typical or common use of
a class of product, and not the status of a particular user or expected or
actual uses by a particular user.  Third, it clearly states that the
existence of substantial non-consumer uses of a product does not negate a
determination that it is a consumer product, unless such non-consumer uses
represent the only significant mode of use of that product.

It should be clear from these added sentences that it is the general mode of
use of a product that determines objectively whether or not it is a consumer
product.  One could not escape the effects of the User Products provisions by
labeling what is demonstrably a consumer product in ways that suggest it is
``for professionals'', for example.


\subsubsection{Installation Information}

With the User Products definition complete,  The ``Installation Information''
definition uses that to define what those receiving object code inside a User
Product must receive.

Installation Information is information that is ``required to install and
execute modified versions of a covered work \dots from a modified version of
its'' CCS, in the same User Product for which the covered work is conveyed.
GPLv3 provides guidance concerning how much information must be provided: it
``must suffice to ensure that the continued functioning of the modified
object code is in no case prevented or interfered with solely because
modification has been made.''  For example, the information provided would be
insufficient if it enabled a modified version to run only in a disabled
fashion, solely because of the fact of modification (regardless of the actual
nature of the modification).  The information need not consist of
cryptographic keys; Installation Information may be ``any methods,
procedures, authorization keys, or other information''.

Note that GPLv3 does not define ``continued functioning'' further.  However,
GPLv3 does provide some additional guidance concerning the scope of
GPLv3-compliant action or inaction that distributors of
technically-restricted User Products can take with respect to a downstream
recipient who replaces the conveyed object code with a modified version.
First of all, GPLv3 makes clear that GPLv3 implies no obligation ``to
continue to provide support service, warranty, or updates'' for such a work.

Second, most technically-restricted User Products are designed to communicate
across networks.  It is important for both users and network providers to
know when denial of network access to devices running modified versions
becomes a GPL violation.  GPLv3 permits denial of access in two cases: ``when
the modification itself materially and adversely affects the operation of the
network,'' and when the modification itself ``violates the rules and
protocols for communication across the network''.  The second case is
deliberately drawn in general terms, and it serves as a foundation for
reasonable enforcement policies that respect recipients' right to modify
while recognizing the legitimate interests of network providers.

Note that GPLv3 permits the practice of conveying object code in a mode not
practically susceptible to modification by any party, such as code burned in
ROM or embedded in silicon.  The goal of the Installation Information
requirement is to ensure the downstream licensee receives the real right to
modify when the device manufacturer or some other party retains that right.
Accordingly, GPLv3\S6's ante-penultimate paragraph states that the
requirement to provide Installation Information ``does not apply if neither
you nor any third party retains the ability to install modified object code
on the User Product''.

Finally, GPLv3\S6 makes it clear that there is also no requirement to
provide warranty or support for the User Product itself.

\subsection{GPLv3~\S7: Additional Permissions}

The GPL is a statement of permissions, some of which have conditions.
Additional terms --- terms that supplement those of the GPL --- may come to be
placed on, or removed from, GPL-covered code in certain common ways.
Copyleft licensing theorists have generally called
 those added terms ``additional permissions'' if they grant
exceptions from the conditions of the GPL, and ``additional requirements'' if
they add conditions to the basic permissions of the GPL\@. The treatment of
additional permissions and additional requirements under GPLv3 is necessarily
asymmetrical, because they do not raise the same interpretive
issues; in particular, additional requirements, if allowed without careful
limitation, could transform a GPL'd program into a non-free one.

With these principles in the background, GPLv3~\S7  answers the following
questions: 
\begin{enumerate}
\item How do the presence of additional terms on all or part of a GPL'd program
affect users' rights?

\item When and how may a licensee add terms to code being
distributed under the GPL? 

\item When may a licensee remove additional terms?
\end{enumerate}

Additional permissions present the easier case.  Since the mid-1990s,
permissive exceptions often appeared alongside GPLv2 with permissive
exceptions to allow combination
with certain non-free code.  Typically, downstream
stream recipients could remove those exceptions and operate under pure GPLv2.
Similarly, LGPLv2.1 is in essence a permissive variant of GPLv2,
and it permits relicensing under the GPL\@.  

\sectin
These practices are now generalized via GPLv3~\S7.
A licensee may remove any additional permission from
a covered work, whether it was placed by the original author or by an
upstream distributor.  A licensee may also add any kind of additional
permission to any part of a work for which the licensee has, or can give,
appropriate copyright permission. For example, if the licensee has written
that part, the licensee is the copyright holder for that part and can
therefore give additional permissions that are applicable to it.
Alternatively, the part may have been written by someone else and licensed,
with the additional permissions, to that licensee.  Any additional
permissions on that part are, in turn, removable by downstream recipients.
As GPLv3~\S7\P1 explains, the effect of an additional permission depends on
whether the permission applies to the whole work or a part.

% FIXME-LATER: LGPLv3 will have its own section

Indeed, LGPLv3 is itself simply  a list of additional permissions supplementing the
terms of GPLv3.  GPLv3\S7 has thus provided the basis for recasting a
formally complex license as an elegant set of added terms, without changing
any of the fundamental features of the existing LGPL\@.  LGPLv3 is thus  a model for developers wishing to license their works under the
GPL with permissive exceptions.  The removability of additional permissions
under GPLv3\S7 does not alter any existing behavior of the LGPL since the LGPL
has always allowed relicensing under the ordinary GPL\@.

\section{GPLv3~\S7: Understanding License Compatibility}
\label{license-compatibility}

A challenge that faced the Free Software community heavily through out the
early 2000s was the proliferation of incompatible Free Software licenses.  Of
course, the GPL cannot possibly be compatible with all such licenses.
However, GPLv3
contains provisions that are designed to reduce license incompatibility by
making it easier for developers to combine code carrying non-GPL terms with
GPL'd code.

\subsection{GPLv3~\S7: Additional Requirements and License Compatibility}

We broadened the title of section 7 because license compatibility, as it is
conventionally understood, is only one of several facets of the placement of
additional terms on GPL'd code.  The license compatibility issue arises for
three reasons.  First, the GPL is a strong copyleft license, requiring
modified versions to be distributed under the GPL.  Second, the GPL states
that no further restrictions may be placed on the rights of recipients.
Third, all other free software licenses in common use contain certain
requirements, many of which are not conditions made by the GPL.  Thus, when
GPL'd code is modified by combination with code covered by another formal
license that specifies other requirements, and that modified code is then
distributed to others, the freedom of recipients may be burdened by
additional requirements in violation of the GPL.  It can be seen that
additional permissions in other licenses do not raise any problems of license
compatibility.

In GPLv3 we take a new approach to the issue of combining GPL'd code with
code governed by the terms of other free software licenses. Our view, though
it was not explicitly stated in GPLv2 itself, was that GPLv2 allowed such
combinations only if the non-GPL licensing terms permitted distribution under
the GPL and imposed no restrictions on the code that were not also imposed by
the GPL. In practice, we supplemented this policy with a structure of
exceptions for certain kinds of combinations.

% FIXME:  probably mostly still right, needs some updates, though.

Section 7 of GPLv3 implements a more explicit policy on license
compatibility. It formalizes the circumstances under which a licensee may
release a covered work that includes an added part carrying non-GPL terms. We
distinguish between terms that provide additional permissions, and terms that
place additional requirements on the code, relative to the permissions and
requirements established by applying the GPL to the code.

% FIXME:  probably mostly still right, needs some updates, though.

Section 7 first explicitly allows added parts covered by terms with
additional permissions to be combined with GPL'd code. This codifies our
existing practice of regarding such licensing terms as compatible with the
GPL. A downstream user of a combined GPL'd work who modifies such an added
part may remove the additional permissions, in which case the broader
permissions no longer apply to the modified version, and only the terms of
the GPL apply to it.

% FIXME:  probably mostly still right, needs some updates, though.

In its treatment of terms that impose additional requirements, section 7
extends the range of licensing terms with which the GPL is compatible. An
added part carrying additional requirements may be combined with GPL'd code,
but only if those requirements belong to an set enumerated in section 7. We
must, of course, place some limit on the kinds of additional requirements
that we will accept, to ensure that enhanced license compatibility does not
defeat the broader freedoms advanced by the GPL. Unlike terms that grant
additional permissions, terms that impose additional requirements cannot be
removed by a downstream user of the combined GPL'd work, because no such user
would have the right to do so.

% FIXME:  probably mostly still right, needs some updates, though.

Under subsections 7a and 7b, the requirements may include preservation of
copyright notices, information about the origins of the code or alterations
of the code, and different warranty disclaimers. Under subsection 7c, the
requirements may include limitations on the use of names of contributors and
on the use of trademarks for publicity purposes. In general, we permit these
requirements in added terms because many free software licenses include them
and we consider them to be unobjectionable. Because we support trademark fair
use, the limitations on the use of trademarks may seek to enforce only what
is required by trademark law, and may not prohibit what would constitute fair
use.

% FIXME: 7d-f


% FIXME:  removing additional restrictions

% FIXME:  probably mostly still right, needs some updates, though.

Section 7 requires a downstream user of a covered work to preserve the
non-GPL terms covering the added parts just as they must preserve the GPL, as
long as any substantial portion of those parts is present in the user's
version.

% FIXME: minor rewrites needed

Section 7 points out that GPLv3 itself makes no assertion that an additional
requirement is enforceable by the copyright holder.  However, section 7 makes
clear that enforcement of such requirements is expected to be by the
termination procedure given in section 8 of GPLv3.

% FIXME: better context, etc.

Some have questioned whether section 7 is needed, and some have suggested
that it creates complexity that did not previously exist.  We point out to
those readers that there is already GPLv2-licensed code that carries
additional terms.  One of the objectives of section 7 is to rationalize
existing practices of program authors and modifiers by setting clear
guidelines regarding the removal and addition of such terms.  With its
carefully limited list of allowed additional requirements, section 7
accomplishes additional objectives, permitting the expansion of the base of
code available for GPL developers, while also encouraging useful
experimentation with requirements we do not include in the GPL itself.

\section{GPLv3~\S8: A Lighter Termination}

% FIXME:  probably mostly still right, needs some updates, though.

GPLv2 provided for automatic termination of the rights of a person who
copied, modified, sublicensed, or distributed a work in violation of the
license.  Automatic termination can be too harsh for those who have committed
an inadvertent violation, particularly in cases involving distribution of
large collections of software having numerous copyright holders.  A violator
who resumes compliance with GPLv2 would need to obtain forgiveness from all
copyright holders, but even to contact them all might be impossible.

% FIXME: needs to be updated to describe more complex termination

Section 8 of GPLv3 replaces automatic termination with a non-automatic
termination process.  Any copyright holder for the licensed work may opt to
terminate the rights of a violator of the license, provided that the
copyright holder has first given notice of the violation within 60 days of
its most recent occurrence. A violator who has been given notice may make
efforts to enter into compliance and may request that the copyright holder
agree not exercise the right of termination; the copyright holder may choose
to grant or refuse this request.

% FIXME: needs to be updated to describe more complex termination

If a licensee who is in violation of GPLv3 acts to correct the violation and
enter into compliance, and the licensee receives no notice of the past
violation within 60 days, then the licensee need not worry about termination
of rights under the license.

In Draft 3 the termination provision of section 8 has been revised to
indicate that, if a licensee violates the GPL, a contributor may terminate
any patent licenses that it granted under the first paragraph of section 11
to that licensee, in addition to any copyright permissions the contributor
granted to the licensee.  Therefore, a contributor may terminate the patent
licenses it granted to a downstream licensee who brings patent infringement
litigation in violation of section 10.

We have made two substantive changes to section 8.  First, we have clarified
that patent rights granted under the GPL are among the rights that a
copyright holder may terminate under section 8.  Therefore, a contributor who
grants a patent license under the first paragraph of section 11 may terminate
that patent license, just as that contributor may terminate copyright rights,
to a downstream recipient who has violated the license.  We think that this
is a reasonable result, and was already implicit in the wording of the
termination provision in our earlier drafts.  Moreover, this clarification
should encourage patent holders to make contributions to GPL-covered
programs.

Second, we have modified the termination procedure by providing a limited
opportunity to cure license violations, an improvement that was requested by
many different members of our community.  If a licensee has committed a
first-time violation of the GPL with respect to a given copyright holder, but
the licensee cures the violation within 30 days following receipt of notice
of the violation, then any of the licensee's GPL rights that have been
terminated by the copyright holder are ``automatically reinstated.''  The
addition of the cure opportunity achieves a better balance than our earlier
section 8 drafts between facilitating enforcement of the license and
protecting inadvertent violators against unfair results.

We have restructured the form of section 8 by replacing non-automatic
termination with automatic termination coupled with opportunities for
provisional and permanent reinstatement of rights.  The revised wording does
not alter the underlying policy or details of procedure established in the
previous drafts, including the 60-day period of repose and 30-day cure
opportunity for first-time violators.  The restoration of automatic
termination was motivated in part to facilitate enforcement in European
countries.  We also believe the revised wording will be easier to understand
and apply in all jurisdictions.

\section{GPLv3~\S9: Acceptance}

% FIXME: needs some work here

Section 9 means what it says: mere receipt or execution of code neither
requires nor signifies contractual acceptance under the GPL.  Speaking more
broadly, we have intentionally structured our license as a unilateral grant
of copyright permissions, the basic operation of which exists outside of any
law of contract.  Whether and when a contractual relationship is formed
between licensor and licensee under local law do not necessarily matter to
the working of the license.

\section{GPLv3~\S10: Explicit Downstream License}

% FIXME: These don't belong here, but it's closer to where it ought to be now.

It is important to note that section 11, paragraph 3 refers to a work that is
conveyed, and section 10, paragraph 2 refers to a kind of automatic
counterpart to conveying achieved as the result of a transaction. 

% FIXME: needs filled out and more here.

Draft1 removed the words ``at no charge'' from what is now subsection 5b, the
core copyleft provision, for reasons related to our current changes to the
second paragraph of section 4: it had contributed to a misconception that the
GPL did not permit charging for distribution of copies.  The purpose of the
``at no charge'' wording was to prevent attempts to collect royalties from
third parties.  The removal of these words created the danger that the
imposition of licensing fees would no longer be seen as a license
violation.

We therefore have added a new explicit prohibition on imposition of licensing
fees or royalties in section 10.  This section is an appropriate place for
such a clause, since it is a specific consequence of the general requirement
that no further restrictions be imposed on downstream recipients of
GPL-covered code.

Careful readers of the GPL have suggested that its explicit prohibition
against imposition of further restrictions\footnote{GPLv2, section 6; Draft
  3, section 10, third paragraph.} has, or ought to have, implications for
those who assert patents against other licensees.  Draft 2 took some steps to
clarify this point in a manner not specific to patents, by describing the
imposition of ``a license fee, royalty, or other charge'' for exercising GPL
rights as one example of an impermissible further restriction.  In Draft 3 we
have clarified further that the requirement of non-imposition of further
restrictions has specific consequences for litigation accusing GPL-covered
programs of infringement.  Section 10 now states that ``you may not initiate
litigation (including a cross-claim or counterclaim in a lawsuit) alleging
that any patent claim is infringed by making, using, selling, offering for
sale, or importing the Program (or the contribution of any contributor).''
That is to say, a patent holder's licensed permissions to use a work under
GPLv3 may be terminated under section 8 if the patent holder files a lawsuit
alleging that use of the work, or of any upstream GPLv3-licensed work on
which the work is based, infringes a patent.

\section{GPLv3~\S11: Explicit Patent Licensing}
\label{GPLv3s11}

The patent licensing practices that section 7 of GPLv2 (corresponding to
section 12 of GPLv3) was designed to prevent are one of several ways in which
software patents threaten to make free programs non-free and to prevent users
from exercising their rights under the GPL. GPLv3 takes a more comprehensive
approach to combatting the danger of patents.

Software patenting is a harmful and unjust policy, and should be abolished;
recent experience makes this all the more evident. Since many countries grant
patents that can apply to and prohibit software packages, in various guises
and to varying degrees, we seek to protect the users of GPL-covered programs
from those patents, while at the same time making it feasible for patent
holders to contribute to and distribute GPL-covered programs as long as they
do not attack the users of those programs.

It is generally understood that GPLv2 implies some limits on a licensee's
power to assert patent claims against the use of GPL-covered works.

Therefore, we have designed GPLv3 to reduce the patent risks that distort and
threaten the activities of users who make, run, modify and share free
software.  At the same time, we have given due consideration to practical
goals such as certainty and administrability for patent holders that
participate in distribution and development of GPL-covered software.  Our
policy requires each such patent holder to provide appropriate levels of
patent assurance to users, according to the nature of the patent holder's
relationship to the program.

Draft 3 features several significant changes concerning patents.  We have
made improvements to earlier wording, clarified when patent assertion becomes
a prohibited restriction on GPL rights, and replaced a distribution-triggered
non-assertion covenant with a contribution-based patent license grant. We
have also added provisions to block collusion by patent holders with software
distributors that would extend patent licenses in a discriminatory way.


Draft 3 introduces the terms ``contributor'' and ``contribution,'' which are
used in the third paragraph of section 10 and the first paragraph of section
11, discussed successively in the following two subsections.  Section 0
defines a contributor as ``a party who licenses under this License a work on
which the Program is based.'' That work is the ``contribution'' of that
contributor.  In other words, each received GPLv3-covered work is associated
with one or more contributors, making up the finite set of upstream GPLv3
licensors for that work. Viewed from the perspective of a recipient of the
Program, contributors include all the copyright holders for the Program,
other than copyright holders of material originally licensed under non-GPL
terms and later incorporated into a GPL-covered work.  The contributors are
therefore the initial GPLv3 licensors of the Program and all subsequent
upstream licensors who convey, under the terms of section 5, modified works
on which the Program is based.

For a contributor whose contribution is a modified work conveyed under
section 5, the contribution is ``the entire work, as a whole'' which the
contributor is required to license under GPLv3.  The contribution therefore
includes not just the material added or altered by the contributor, but also
the pre-existing material the contributor copied from the upstream version
and retained in the modified version. Our usage of ``contributor'' and
``contribution'' should not be confused with the various other ways in which
those terms are used in certain other free software licenses.\footnote{Cf.,
  e.g., Apache License, version 2.0, section 1; Eclipse Public License,
  version 1.0, section 1; Mozilla Public License, version 1.1, section 1.1.}

The term ``patent license,'' as used in the third through fifth
paragraphs of section 11, is not meant to be confined to agreements
formally identified or classified as patent licenses.  The new second
paragraph of section 11 makes this clear by defining ``patent license,''
for purposes of the subsequent three paragraphs, as ``a patent license,
a covenant not to bring suit for patent infringement, or any other
express agreement or commitment, however denominated, not to enforce a
patent.''  The definition does not include patent licenses that arise by
implication or operation of law, because the third through fifth
paragraphs of section 11 are specifically concerned with explicit
promises that purport to be legally enforceable.

Our previous drafts featured a patent license grant triggered by all
acts of distribution of GPLv3-covered works.\footnote{In Draft 2 we
rewrote the patent license as a covenant not to assert patent claims. We
explain why we reverted to the form of a patent license grant in \S\
\ref{cov}.} Many patent-holding companies objected to this policy. They
have made two objections: (1) the far-reaching impact of the patent
license grant on the patent holder is disproportionate to the act of
merely distributing code without modification or transformation, and (2)
it is unreasonable to expect an owner of vast patent assets to exercise
requisite diligence in reviewing all the GPL-covered software that it
provides to others.  Some expressed particular concern about the
consequences of ``inadvertent'' distribution.

The argument that the impact of the patent license grant would be
``disproportionate,'' that is to say unfair, is not valid. Since
software patents are weapons that no one should have, and using them for
aggression against free software developers is an egregious act,
preventing that act cannot be unfair. 

However, the second argument seems valid in a practical sense.  A
typical GNU/Linux distribution includes thousands of programs. It would
be quite difficult for a redistributor with a large patent portfolio to
review all those programs against that portfolio every time it receives
and passes on a new version of the distribution. Moreover, this question
raises a strategic issue. If the GPLv3 patent license requirements
convince patent-holding companies to remain outside the distribution
path of all GPL-covered software, then these requirements, no matter how
strong, will cover few patents. 

We concluded it would be more effective to make a partial concession
which would lead these companies to feel secure in doing the
distribution themselves, so that the conditions of section 10 would
apply to assertion of their patents.  We therefore made the stricter
section 11 patent license apply only to those distributors that have
modified the program.  The other changes we have made in sections 10 and
11 provide strengthened defenses against patent assertion and compensate
partly for this concession. 

Therefore, in Draft 3, the first paragraph of section 11 states that a
contributor's patent license covers all the essential patent claims
implemented by the whole program as that contributor distributes it.
Contributors of modified works grant a patent license to claims that
read on ``the entire work, as a whole.'' This is the work that the
copyleft clause in section 5 requires the contributor to license under
GPLv3; it includes the material the contributor has copied from the
upstream version that the contributor has modified.  The first paragraph
of section 11 does not apply to those that redistribute the program
without change.\footnote{An implied patent license from the distributor,
however, may arise by operation of law. See the final paragraph of
section 11.  Moreover, distributors are subject to the limits on patent
assertion contained in the third paragraph of section 10.} 

We hope that this decision will result in fairly frequent licensing of
patent claims by contributors.  A contributor is charged with awareness
of the fact that it has modified a work and provided it to others; no
act of contribution should be treated as inadvertent.  Our rule also
requires no more work, for a contributor, than the weaker rule proposed
by the patent holders.  Under their rule, the contributor must always
compare the entire work against its patent portfolio to determine
whether the combination of the modifications with the remainder of the
work cause it to read on any of the contributor's patent claims.



We have made three changes to the definition of ``essential patent
claims'' in section 0.  This definition now serves exclusively to
identify the set of patent claims licensed by a contributor under the
first paragraph of section 11.

First, we have clarified when essential patent claims include
sublicensable claims that have been licensed to the contributor by a
third party.\footnote{This issue is typically handled in other free
software licenses having patent licensing provisions by use of the
unhelpful term ``licensable,'' which is either left undefined or is
given an ambiguous definition.}  Most commercial patent license
agreements that permit sublicensing do so under restrictive terms that
are inconsistent with the requirements of the GPL.  For example, some
patent licenses allow the patent licensee to sublicense but require
collection of royalties from any sublicensees.  The patent licensee
could not distribute a GPL-covered program and grant the recipient a
patent sublicense for the program without violating section 12 of
GPLv3.\footnote{Draft 3 provides a new example in section 12 that makes
this point clear.}  In rare cases, however, a conveying party can freely
grant patent sublicenses to downstream recipients without violating the
GPL.

Draft 3 now defines essential patent claims, for a given party, as a
subset of the claims ``owned or controlled'' by the party.  The
definition states that ``control includes the right to grant sublicenses
in a manner consistent with the requirements of this License.''
Therefore, in the case of a patent license that requires collection of
royalties from sublicensees, essential patent claims would not include
any claims sublicensable under that patent license, because sublicenses
to those claims could not be granted consistent with section 12.

Second, we now state that essential patent claims are those ``that would
be infringed by some manner, permitted by this License, of making,
using, or selling the work.'' This modified wording is intended to make
clear that a patent claim is ``essential'' if some mode of usage would
infringe that claim, even if there are other modes of usage that would
not infringe.

Third, we have clarified that essential patent claims ``do not include
claims that would be infringed only as a consequence of further
modification of the work.''  That is to say, the set of essential patent
claims licensed under the first paragraph of section 11 is fixed by the
the particular version of the work that was contributed.  The claim set
cannot expand as a work is further modified downstream.  (If it could,
then any software patent claim would be included, since any software
patent claim can be infringed by some further modification of the
work.)\footnote{However, ``the work'' should not be understood to be
restricted to a particular mechanical affixation of, or medium for
distributing, a program, where the same program might be provided in
other forms or in other ways that may be captured by other patent claims
held by the contributor.}


The downstream shielding provision of section 11 responds particularly
to the problem of exclusive deals between patent holders and
distributors, which threaten to distort the free software distribution
system in a manner adverse to developers and users. Draft 2 added a
source code availability option to this provision, as a specific
alternative to the general requirement to shield downstream users from
patent claims licensed to the distributor. A distributor conveying a
covered work knowingly relying on a patent license may comply with the
provision by ensuring that the Corresponding Source of the work is
publicly available, free of charge.  We retained the shielding option in
Draft 2 because we did not wish to impose a general requirement to make
source code available to all, which has never been a GPL condition.

The addition of the source code availability option was supported by the
free software vendors most likely to be affected by the downstream
shielding provision.  Enterprises that primarily use and occasionally
distribute free software, however, raised concerns regarding the
continued inclusion of a broadly-worded requirement to ``shield,'' which
appears to have been mistakenly read by those parties as creating an
obligation to indemnify.  To satisfy these concerns, in Draft 3 we have
replaced the option to shield with two specific alternatives to the
source code availability option. The distributor may comply by
disclaiming the patent license it has been granted for the conveyed
work, or by arranging to extend the patent license to downstream
recipients.\footnote{The latter option, if chosen, must be done ``in a
manner consistent with the requirements of this License''; for example,
it is unavailable if extension of the patent license would result in a
violation of section 12. Cf.~the discussion of sublicensable patent
claims in \S\ \ref{epc}.}  The GPL is intended to permit private
distribution as well as public distribution, and the addition of these
options ensures that this remains the case, even though we expect that
distributors in this situation will usually choose the source code
availability option.

Without altering its underlying logic, we have modified the phrasing of
the requirement to make clear that it is activated only if the
Corresponding Source is not already otherwise publicly available.  (Most
often it will, in fact, already be available on some network server
operated by a third party.)  Even if it is not already available, the
option to ``cause the Corresponding Source to be so available'' can then
be satisfied by verifying that a third party has acted to make it
available.  That is to say, the affected distributor need not itself
host the Corresponding Source to take advantage of the source code
availability option.  This subtlety may help the distributor avoid
certain peculiar assumptions of liability.

We have made two other changes to the downstream shielding provision.
The phrase ``knowingly rely'' was left undefined in our earlier drafts;
in Draft 3 we have provided a detailed definition.  We have also deleted
the condition precedent, added in Draft 2, that the relied-upon patent
license be one that is non-sublicensable and ``not generally available
to all''; this was imprecise in Draft 2 and is unnecessary in Draft
3. In nearly all cases in which the ``knowingly relying'' test is met,
the patent license will indeed not be sublicensable or generally
available to all on free terms.  If, on the other hand, the patent
license is generally available under terms consistent with the
requirements of the GPL, the distributor is automatically in compliance,
because the patent license has already been extended to all downstream
recipients.  If the patent license is sublicensable on GPL-consistent
terms, the distributor may choose to grant sublicenses to downstream
recipients instead of causing source code to be publicly available.  In
such a case, if the distributor is also a contributor, it will already
have granted a patent sublicense by operation of the first paragraph of
section 11,\footnote{See \S\ \ref{epc}.} and so it need not do anything
further to comply with the third paragraph.

% FIXME: This probably needs editing

One major goal for GPLv3 is to provide developers with additional protection
from being sued for patent infringement.  After much feedback and cooperation
from the committees, we are now proposing a patent license which closely
resembles those found in other free software licenses.  This will be more
comfortable for everyone in the free software community to use, without
creating undue burdens for distributors.

We have also added new terms to stop distributors from colluding with third
parties to offer selective patent protection, as Microsoft and Novell have
recently done.  The GPL is designed to ensure that all users receive the
same rights; arrangements that circumvent this make a mockery of free
software, and we must do everything in our power to stop them.

Our strategy has two parts.  First, any license that protects some
recipients of GPLed software must be extended to all recipients of the
software.  Second, we prohibit anyone who made such an agreement from
distributing software released under GPLv3.  We are still considering
whether or not this ban should apply when a deal was made before these
terms were written, and we look forward to community input on this issue.

The patent license grant of the first paragraph of section 11 no longer
applies to those who merely distribute works without modification. (We
explain why we made this change in the next subsection.) Such parties are
nonetheless subject to the conditions stated in section 10.  Unlike the
patent license, which establishes a defense for downstream users lasting for
as long as they remain in compliance with the GPL, the commitment not to sue
that arises under section 10 is one that the distributor can end, so long as
the distributor also ceases to distribute.  This is because a party who
initiates patent litigation in violation of section 10 risks termination of
its licensed permissions by the copyright holders of the work.

% FIXME: just brought in words here, needs rewriting.

is rooted in the basic principles of the GPL.
Our license has always stated that distributors may not impose further
restrictions on users' exercise of GPL rights.  To make the suggested
distinction between contribution and distribution is to allow a
distributor to demand patent royalties from a direct or indirect
recipient, based on claims embodied in the distributed code. This
undeniably burdens users with an additional legal restriction on their
rights, in violation of the license.

%FIXME: possible useful text, but maybe not.

In the covenant provided in the revised section 11, the set of claims
that a party undertakes not to assert against downstream users are that
party's ``essential patent claims'' in the work conveyed by the party.
``Essential patent claims,'' a new term defined in section 0, are simply
all claims ``that would be infringed by making, using, or selling the
work.''  We have abandoned the phrase ``reasonably contemplated use.''
This change makes the obligations of distributing patent holders more
predictable.

% FIXME:  probably needs a lot of work, these provisions changed over time.

GPLv3 adds a new section on licensing of patents. GPLv2 relies on an implied
patent license. The doctrine of implied license is one that is recognized
under United States patent law but may not be recognized in other
jurisdictions. We have therefore decided to make the patent license grant
explicit in GPLv3. Under section 11, a redistributor of a GPL'd work
automatically grants a nonexclusive, royalty-free and worldwide license for
any patent claims held by the redistributor, if those claims would be
infringed by the work or a reasonably contemplated use of the work.

% FIXME:  probably needs a lot of work, these provisions changed over time.

The patent license is granted both to recipients of the redistributed work
and to any other users who have received any version of the work. Section 11
therefore ensures that downstream users of GPL'd code and works derived from
GPL'd code are protected from the threat of patent infringement allegations
made by upstream distributors, regardless of which country's laws are held to
apply to any particular aspect of the distribution or licensing of the GPL'd
code.

% FIXME:  probably needs a lot of work, these provisions changed over time.

A redistributor of GPL'd code may benefit from a patent license that has been
granted by a third party, where the third party otherwise could bring a
patent infringement lawsuit against the redistributor based on the
distribution or other use of the code. In such a case, downstream users of
the redistributed code generally remain vulnerable to the applicable patent
claims of the third party. This threatens to defeat the purposes of the GPL,
for the third party could prevent any downstream users from exercising the
freedoms that the license seeks to guarantee.

% FIXME:  probably needs a lot of work, these provisions changed over time.

The second paragraph of section 11 addresses this problem by requiring the
redistributor to act to shield downstream users from these patent claims. The
requirement applies only to those redistributors who distribute knowingly
relying on a patent license. Many companies enter into blanket patent
cross-licensing agreements. With respect to some such agreements, it would
not be reasonable to expect a company to know that a particular patent
license covered by the agreement, but not specifically mentioned in it,
protects the company's distribution of GPL'd code.

% FIXME: does this still fit with the final retaliation provision?

This narrowly-targeted patent retaliation provision is the only form of
patent retaliation that GPLv3 imposes by its own force. We believe that it
strikes a proper balance between preserving the freedom of a user to run and
modify a program, and protecting the rights of other users to run, modify,
copy, and distribute code free from threats by patent holders. It is
particularly intended to discourage a GPL licensee from securing a patent
directed to unreleased modifications of GPL'd code and then suing the
original developers or others for making their own equivalent modifications.

Several other free software licenses include significantly broader patent
retaliation provisions. In our view, too little is known about the
consequences of these forms of patent retaliation. As we explain below,
section 7 permits distribution of a GPL'd work that includes added parts
covered by terms other than those of the GPL. Such terms may include certain
kinds of patent retaliation provisions that are broader than those of section
2.

% FIXME: should we mention Microsoft-Novell at all?

Section 7 of GPLv2 (now section 12 of GPLv3) has seen some success in
deterring conduct that would otherwise result in denial of full downstream
enjoyment of GPL rights.  Experience has shown us that more is necessary,
however, to ensure adequate community safety where companies act in concert
to heighten the anticompetitive use of patents that they hold or license.
Previous drafts of GPLv3 included a ``downstream shielding'' provision in
section 11, which we have further refined in Draft 3; it is now found in the
third paragraph of section 11.  In addition, Draft 3 introduces two new
provisions in section 11, located in the fourth and fifth paragraphs, that
address the problem of collusive extension of patent forbearance promises
that discriminate against particular classes of users and against the
exercise of particular freedoms. This problem has been made more acute by the
recent Microsoft/Novell deal.

We attack the Microsoft-Novell deal from two angles. First, in the sixth
paragraph of section 11, the draft says that if you arrange to provide patent
protection to some of the people who get the software from you, that
protection is automatically extended to everyone who receives the software,
no matter how they get it. This means that the patent protection Microsoft
has extended to Novell's customers would be extended to everyone who uses any
software Novell distributes under GPLv3.

Second, in the seventh paragraph, the draft says that you are prohibited from
distributing software under GPLv3 if you make an agreement like the
Microsoft-Novell deal in the future. This will prevent other distributors
from trying to make other deals like it.

The main reason for this is tactical.  We believe we can do more to
protect the community by allowing Novell to use software under GPL
version 3 than by forbidding it to do so.  This is because of
paragraph 6 of section 11 (corresponding to paragraph 4 in Draft 3).
It will apply, under the Microsoft/Novell deal, because of the coupons
that Microsoft has acquired that essentially commit it to participate
in the distribution of the Novell SLES GNU/Linux system.

Microsoft is scrambling to dispose of as many Novell SLES coupons as
possible prior to the adoption of GPLv3.  Unfortunately for Microsoft,
those coupons bear no expiration date, and paragraph 6 has no cut-off
date.  Through its ongoing distribution of coupons, Microsoft will
have procured the distribution of GPLv3-covered programs as soon as
they are included in Novell SLES distributions, thereby extending
patent defenses to all downstream recipients of that software by
operation of paragraph 6.

A secondary reason is to avoid affecting other kinds of agreements for
other kinds of activities.  We have tried to take care in paragraph 7
to distinguish pernicious deals of the Microsoft/Novell type from
business conduct that is not particularly harmful, but we cannot be
sure we have entirely succeeded.  There remains some risk that other
unchangeable past agreements could fall within its scope.

In future deals, distributors engaging in ordinary business practices
can structure the agreements so that they do not fall under paragraph
7.  However, it will block Microsoft and other patent aggressors from
further such attempts to subvert parts of our community.

A software patent forbids the use of a technique or algorithm, and its
existence is a threat to all software developers and users.  A patent
holder can use a patent to suppress any program which implements the
patented technique, even if thousands of other techniques are
implemented together with it.  Both free software and proprietary
software are threatened with death in this way.  

However, patents threaten free software with a fate worse than death: a
patent holder might also try to use the patent to impose restrictions on
use or distribution of a free program, such as to make users feel they
must pay for permission to use it.  This would effectively make it
proprietary software, exactly what the GPL is intended to prevent.

Novell and Microsoft have recently attempted a new way of using patents
against our community, which involves a narrow and discriminatory
promise by a patent holder not to sue customers of one particular
distributor of a GPL-covered program.  Such deals threaten our community
in several ways, each of which may be regarded as de facto
proprietization of the software.  If users are frightened into paying
that one distributor just to be safe from lawsuits, in effect they are
paying for permission to use the program.  They effectively deny even
these customers the full and safe exercise of some of the freedoms
granted by the GPL.  And they make disfavored free software developers
and distributors more vulnerable to attacks of patent aggression, by
dividing them from another part of our community, the commercial users
that might otherwise come to their defense.

We have added the fourth and fifth paragraphs of section 11 to combat
this threat.  This subsection briefly describes the operation of the new
provisions.  We follow it with a more detailed separate note on the
Microsoft/Novell patent deal, in which we provide an extensive rationale
for these measures.

As noted, one effect of the discriminatory patent promise is to divide
and isolate those who make free software from the commercial users to
whom the promise is extended.  This deprives the noncommercial
developers of the communal defensive measures against patents made
possible by the support of those commercial users.  The fourth paragraph
of section 11 operates to restore effective defenses to the targets of
patent aggression.

A patent holder becomes subject to the fourth paragraph of section 11
when it enters into a transaction or arrangement that involves two acts:
(1) conveying a GPLv3-covered work, and (2) offering to some, but not
all, of the work's eventual users a patent license for particular
activities using specific copies of the covered work.  This paragraph
only operates when the two triggering acts are part of a single
arrangement, because the patent license is part of the arrangement for
conveying, which requires copyright permission.  Under those conditions,
the discriminatory patent license is ``automatically extended to all
recipients of the covered work and works based on it.''

This provision establishes a defense to infringement allegations brought
by the patent holder against any users of the program who are not
covered by the discriminatory patent license.  That is to say, it gives
all recipients the benefit of the patent promise that the patent holder
extended only to some. The effect is to make contributing discriminatory
promises of patent safety to a GPL distribution essentially like
contributing code. In both cases, the operation of the GPL extends
license permission to everyone that receives a copy of the program.


The fourth paragraph of section 11 gives users a defense against patent
aggression brought by the party who made the discriminatory patent
promise that excluded them. By contrast, the fifth paragraph stops free
software vendors from contracting with patent holders to make
discriminatory patent promises.  In effect, the fifth paragraph extends
the principle of section 12 to situations involving collusion between a
patent holder and a distributor.

Under this provision, a distributor conveying a GPL-covered program may
not make an arrangement to get a discriminatory patent promise from a
third party for its customers, covering copies of the program (or
products that contain the program), if the arrangement requires the
distributor to make payment to the third party based on the extent of
its activity in conveying the program, and if the third party is itself
in the business of distributing software. Unlike the fourth paragraph,
which creates a legal defense for targets of patent aggression, the
consequence for violation of the fifth paragraph is termination of GPL
permissions for the distributor.

The business, technical, and patent cooperation agreement between
Microsoft and Novell announced in November 2006 has significantly
affected the development of Draft 3.  The fourth and fifth paragraphs of
section 11 embody our response to the sort of threat represented by the
Microsoft/Novell deal, and are designed to protect users from such
deals, and prevent or deter the making of such deals.

The details of the agreements entered into between Microsoft and Novell,
though subject to eventual public disclosure through the securities
regulation system, have not been fully disclosed to this
point.\footnote{Lawyers employed by the Software Freedom Law Center,
which is counsel to the Free Software Foundation and other relevant free
software clients, were accorded limited access to the terms of the deal
under a non-disclosure agreement between SFLC and Novell.  The reasons
for delay in the application of securities regulations requiring
publication of the relevant contracts are unrelated to the deal between
Microsoft and Novell.}  It is a matter of public knowledge, however,
that the arrangement calls for Novell to pay a portion of the future
gross revenue of one of its divisions to Microsoft, and that (as one
other feature of a complex arrangement) Microsoft has promised Novell's
customers not to bring patent infringement actions against certain
specific copies of Novell's SUSE ``Linux''\footnote{This is a GNU/Linux
distribution, and is properly called SUSE GNU/Linux Enterprise Server.}
Enterprise Server product for which Novell receives revenue from the
user, so long as the user does not make or distribute additional copies
of SLES.

The basic harm that such an agreement can do is to make the free
software subject to it effectively proprietary.  This result occurs to
the extent that users feel compelled, by the threat of the patent, to
get their copies in this way.  So far, the Microsoft/Novell deal does
not seem to have had this result, or at least not very much: users do
not seem to be choosing Novell for this reason.  But we cannot take for
granted that such threats will always fail to harm the community.  We
take the threat seriously, and we have decided to act to block such
threats, and to reduce their potential to do harm.  Such deals also
offer patent holders a crack through which to split the community.
Offering commercial users the chance to buy limited promises of patent
safety in effect invites each of them to make a separate peace with
patent aggressors, and abandon the rest of our community to its fate.

Microsoft has been restrained from patent aggression in the past by the
vocal opposition of its own enterprise customers, who now also use free
software systems to run critical applications.  Public statements by
Microsoft concerning supposed imminent patent infringement actions have
spurred resistance from users Microsoft cannot afford to alienate.  But
if Microsoft can gain royalties from commercial customers by assuring
them that \textit{their} copies of free software have patent licenses
through a deal between Microsoft and specific GNU/Linux vendors,
Microsoft would then be able to pressure each user individually, and
each distributor individually, to treat the software as proprietary.  If
enough users succumb, it might eventually gain a position to terrify
noncommercial developers into abandoning the software entirely.

Preventing these harms is the goal of the new provisions of section 11.
The fourth paragraph deals with the most acute danger posed by
discrimination among customers, by ensuring that any party who
distributes others' GPL-covered programs, and makes promises of patent
safety limited to some but not all recipients of copies of those
specific programs, automatically extends its promises of patent safety
to cover all recipients of all copies of the covered works.  This will
negate part of the harm of the Microsoft/Novell deal, for GPLv3-covered
software.

In addition to the present deal, however, GPLv3 must act to deter
similar future arrangements, and it cannot be assumed that all future
arrangements by Microsoft or other potential patent aggressors will
involve procuring the conveyance of the program by the party that grants
the discriminatory promises of patent safety.  Therefore, we need the
fifth paragraph as well, which is aimed at parties that play the Novell
role in a different range of possible deals.

Drafting this paragraph was difficult because it is necessary to
distinguish between pernicious agreements and other kinds of agreements
which do not have an acutely harmful effect, such as patent
contributions, insurances, customary cross-license promises to
customers, promises incident to ordinary asset transfers, and standard
settlement practices.  We believe that we have achieved this, but it is
hard to be sure, so we are considering making this paragraph apply only
to agreements signed in the future.  If we do that, companies would only
need to structure future agreements in accord with the fifth paragraph,
and would not face problems from past agreements that cannot be changed
now.  We are not yet convinced that this is necessary, and we plan to
ask for more comment on the question. This is why the date-based cutoff
is included in brackets. 

One drawback of this cutoff date is that it would ``let Novell off''
from part of the response to its deal with Microsoft. However, this may
not be a great drawback, because the fourth paragraph will apply to that
deal. We believe it is sufficient to ensure either the deal's voluntary
modification by Microsoft or its reduction to comparative harmlessness.
Novell expected to gain commercial advantage from its patent deal with
Microsoft; the effects of the fourth paragraph in undoing the harm of
that deal will necessarily be visited upon Novell.


\section{GPLv3~\S12: Familiar as GPLv2 \S~7}

% FIXME:  probably mostly still right, needs some updates, though.

The wording in the first sentence of section 12 has been revised
slightly to clarify that an agreement, such as a litigation settlement
agreement or a patent license agreement, is one of the ways in which
conditions may be ``imposed'' on a GPL licensee that may contradict the
conditions of the GPL, but which do not excuse the licensee from
compliance with those conditions.  This change codifies what has been
our interpretation of GPLv2.  

% FIXME:  probably mostly still right, needs some updates, though.

We have removed the limited severability clause of GPLv2 section 7 as a
matter of tactical judgment, believing that this is the best way to ensure
that all provisions of the GPL will be upheld in court. We have also removed
the final sentence of GPLv2 section 7, which we consider to be unnecessary.

\section{GPLv3~\S13: The Great Affero Compromise}

The main purpose of clause 7b4 was to attain GPLv3 compatibility for the
additional condition of version 1 of the Affero GPL, with a view to
achieving compatibility for a future version, since version 1 was
incompatible with GPLv3.\footnote{Version 1 of the Affero GPL contains
its own copyleft clause, worded identically to that in GPLv2, which
conflicts with the copyleft clause in GPLv3.  The Affero GPL permits
relicensing under versions of the GPL later than version 2, but only if
the later version ``includes terms and conditions substantially
equivalent to those of this license'' (Affero GPL, version 1, section
9). The Affero license was written with the expectation that its
additional requirement would be incorporated into the terms of GPLv3
itself, rather than being placeable on parts added to a covered work
through the mechanism of section 7 of GPLv3.}  However, we wrote the
clause broadly enough to cover a range of other possible terms that
would differ from the Affero condition in their details. Draft 3 no
longer pursues the more ambitious goal of allowing compatibility for a
whole category of Affero-like terms.  In place of 7b4, we have added a
new section 13 that simply permits GPLv3-covered code to be linked with
code covered by the forthcoming version 2 of the Affero GPL.

We have made this decision in the face of irreconcilable views from
different parts of our community.  While we had known that many
commercial users of free software were opposed to the inclusion of a
mandatory Affero-like requirement in the body of GPLv3 itself, we were
surprised at their opposition to its availability through section 7.
Free software vendors allied to these users joined in their objections,
as did a number of free software developers arguing on ethical as well
as practical grounds.

Some of this hostility seemed to be based on a misapprehension that
Affero-like terms placed on part of a covered work would somehow extend
to the whole of the work.\footnote{It is possible that the presence of
the GPLv2-derived copyleft clause in the existing Affero GPL contributed
to this misunderstanding.}  Our explanations to the contrary did little
to satisfy these critics; their objections to 7b4 instead evolved into a
broader indictment of the additional requirements scheme of section 7.
It was clear, however, that much of the concern about 7b4 stemmed from
its general formulation.  Many were alarmed at the prospect of GPLv3
compatibility for numerous Affero-like licensing conditions,
unpredictable in their details but potentially having significant
commercial consequences.

On the other hand, many developers, otherwise sympathetic to the policy
goals of the Affero GPL, have objected to the form of the additional
requirement in that license.  These developers were generally
disappointed with our decision to allow Affero-like terms through
section 7, rather than adopt a condition for GPLv3.  Echoing their
concerns about the Affero GPL itself, they found fault with the wording
of the section 7 clause in both of the earlier drafts.  We drafted 7b4
at a higher level than its Draft 1 counterpart based in part on comments
from these developers. They considered the Draft 1 clause too closely
tied to the Affero mechanism of preserving functioning facilities for
downloading source, which they found too restrictive of the right of
modification.  The 7b4 rewording did not satisfy them, however. They
objected to its limitation to terms requiring compliance by network
transmission of source, and to the technically imprecise or inaccurate
use of the phrase ``same network session.''

We have concluded that any redrafting of the 7b4 clause would fail to
satisfy the concerns of both sets of its critics.  The first group
maintains that GPLv3 should do nothing about the problem of public
use. The second group would prefer for GPLv3 itself to have an
Affero-like condition, but that seems to us too drastic. By permitting
GPLv3-covered code to be linked with code covered by version 2 of the
Affero GPL, the new section 13 honors our original commitment to
achieving GPL compatibility for the Affero license.

Version 2 of the Affero GPL is not yet published.  We will work with
Affero, Inc., and with all other interested members of our community, to
complete the drafting of this license following the release of Draft 3,
with a goal of having a final version available by the time of our
adoption of the final version of GPLv3.  We hope the new Affero license
will satisfy those developers who are concerned about the issue of
public use of unconveyed versions but who have concerns about the
narrowness of the condition in the existing Affero license.

As the second sentence in section 13 indicates, when a combined work is
made by linking GPLv3-covered code with Affero-covered code, the
copyleft on one part will not extend to the other part.\footnote{The
plan is that the additional requirement of the new Affero license will
state a reciprocal limitation.} That is to say, in such combinations,
the Affero requirement will apply only to the part that was brought into
the combination under the Affero license.  Those who receive such a
combination and do not wish to use code under the Affero requirement may
remove the Affero-covered portion of the combination.

Those who criticize the permission to link with code under the Affero
GPL should recognize that most other free software licenses also permit
such linking. 

\section{GPLv3~\S14: So, When's GPLv4?}
\label{GPLv3s14}

% FIXME Say more

No substantive change has been made in section 14. The wording of the section
has been revised slightly to make it clearer.

% FIXME; proxy

\section{GPLv3~\S15--17: Warranty Disclaimers and Liability Limitation}

No substantive changes have been made in sections 15 and 16.

% FIXME: more, plus 17

% FIXME: Section header needed here about choice of law.

% FIXME: reword into tutorial

Some have asked us to address the difficulties of internationalization
by including, or permitting the inclusion of, a choice of law
provision.  We maintain that this is the wrong approach.  Free
software licenses should not contain choice of law clauses, for both
legal and pragmatic reasons.  Choice of law clauses are creatures of
contract, but the substantive rights granted by the GPL are defined
under applicable local copyright law. Contractual free software
licenses can operate only to diminish these rights.  Choice of law
clauses also raise complex questions of interpretation when works of
software are created by combination and extension.  There is also the
real danger that a choice of law clause will specify a jurisdiction
that is hostile to free software principles.

% FIXME: reword into tutorial, \ref to section 7.

Our revised version of section 7 makes explicit our view that the
inclusion of a choice of law clause by a licensee is the imposition of
an additional requirement in violation of the GPL.  Moreover, if a
program author or copyright holder purports to supplement the GPL with
a choice of law clause, section 7 now permits any licensee to remove
that clause.


% FIXME: does this need to be a section, describing how it was out then in
% then out then in? :)

We have removed from this draft the appended section on ``How to Apply These
Terms to Your New Programs.'' For brevity, the license document can instead
refer to a web page containing these instructions as a separate document.

%%%%%%%%%%%%%%%%%%%%%%%%%%%%%%%%%%%%%%%%%%%%%%%%%%%%%%%%%%%%%%%%%%%%%%%%%%%%%%%
\chapter{The Lesser GPL}

As we have seen in our consideration of the GPL, its text is specifically
designed to cover all possible derivative works under copyright law. Our
goal in designing GPL was to make sure that any derivative work of GPL'd
software was itself released under GPL when distributed. Reaching as far
as copyright law will allow is the most direct way to reach that goal.

However, while the strategic goal is to bring as much Free Software
into the world as possible, particular tactical considerations
regarding software freedom dictate different means. Extending the
copyleft effect as far as copyright law allows is not always the most
prudent course in reaching the goal. In particular situations, even
those of us with the goal of building a world where all published
software is Free Software realize that full copyleft does not best
serve us. The GNU Lesser General Public License (``GNU LGPL'') was
designed as a solution for such situations.

\section{The First LGPL'd Program}

The first example that FSF encountered where such altered tactics were
needed was when work began on the GNU C Library. The GNU C Library would
become (and today, now is) a drop-in replacement for existing C libraries.
On a Unix-like operating system, C is the lingua franca and the C library
is an essential component for all programs. It is extremely difficult to
construct a program that will run with ease on a Unix-like operating
system without making use of services provided by the C library --- even
if the program is written in a language other than C\@. Effectively, all
user application programs that run on any modern Unix-like system must
make use of the C library.

By the time work began on the GNU implementation of the C libraries, there
were already many C libraries in existence from a variety of vendors.
Every proprietary Unix vendor had one, and many third parties produced
smaller versions for special purpose use. However, our goal was to create
a C library that would provide equivalent functionality to these other C
libraries on a Free Software operating system (which in fact happens today
on modern GNU/Linux systems, which all use the GNU C Library).

Unlike existing GNU application software, however, the licensing
implications of releasing the GNU C Library (``glibc'') under GPL were
somewhat different. Applications released under GPL would never
themselves become part of proprietary software. However, if glibc were
released under GPL, it would require that any application distributed for
the GNU/Linux platform be released under GPL\@.

Since all applications on a Unix-like system depend on the C library, it
means that they must link with that library to function on the system. In
other words, all applications running on a Unix-like system must be
combined with the C library to form a new whole derivative work that is
composed of the original application and the C library. Thus, if glibc
were GPL'd, each and every application distributed for use on GNU/Linux
would also need to be GPL'd, since to even function, such applications
would need to be combined into larger derivative works by linking with
glibc.

At first glance, such an outcome seems like a windfall for Free Software
advocates, since it stops all proprietary software development on
GNU/Linux systems. However, the outcome is a bit more subtle. In a world
where many C libraries already exist, many of which could easily be ported
to GNU/Linux, a GPL'd glibc would be unlikely to succeed. Proprietary
vendors would see the excellent opportunity to license their C libraries
to anyone who wished to write proprietary software for GNU/Linux systems.
The de-facto standard for the C library on GNU/Linux would likely be not
glibc, but the most popular proprietary one.

Meanwhile, the actual goal of releasing glibc under GPL --- to ensure no
proprietary applications on GNU/Linux --- would be unattainable in this
scenario. Furthermore, users of those proprietary applications would also
be users of a proprietary C library, not the Free glibc.

The Lesser GPL was initially conceived to handle this scenario. It was
clear that the existence of proprietary applications for GNU/Linux was
inevitable. Since there were so many C libraries already in existence, a
new one under GPL would not stop that tide. However, if the new C library
were released under a license that permitted proprietary applications
to link with it, but made sure that the library itself remained Free,
an ancillary goal could be met. Users of proprietary applications, while
they would not have the freedom to copy, share, modify and redistribute
the application itself, would have the freedom to do so with respect to
the C library.

There was no way the license of glibc could stop or even slow the creation
of proprietary applications on GNU/Linux. However, loosening the
restrictions on the licensing of glibc ensured that nearly all proprietary
applications at least used a Free C library rather than a proprietary one.
This trade-off is central to the reasoning behind the LGPL\@.

Of course, many people who use the LGPL today are not thinking in these
terms. In fact, they are often choosing the LGPL because they are looking
for a ``compromise'' between the GPL and the X11-style liberal licensing.
However, understanding FSF's reasoning behind the creation of the LGPL is
helpful when studying the license.


\section{What's the Same?}

Much of the text of the LGPL is identical to the GPL\@. As we begin our
discussion of the LGPL, we will first eliminate the sections that are
identical, or that have the minor modification changing the word
``Program'' to ``Library.''

First, LGPLv2.1~\S1, the rules for verbatim copying of source, are
equivalent to those in GPLv2~\S1.

Second, LGPLv2.1~\S8 is equivalent GPLv2~\S4\@. In both licenses, this
section handles termination in precisely the same manner.

LGPLv2.1~\S9 is equivalent to GPLv2~\S5\@. Both sections assert that
the license is a copyright license, and handle the acceptance of those
copyright terms.

LGPLv2.1~\S10 is equivalent to GPLv2~\S6. They both protect the
distribution system of Free Software under these licenses, to ensure that
up, down, and throughout the distribution chain, each recipient of the
software receives identical rights under the license and no other
restrictions are imposed.

LGPLv2.1~\S11 is GPLv2~\S7. As discussed, it is used to ensure that
other claims and legal realities, such as patent licenses and court
judgments, do not trump the rights and permissions granted by these
licenses, and requires that distribution be halted if such a trump is
known to exist.

LGPLv2.1~\S12 adds the same features as GPLv2~\S8. These sections are
used to allow original copyright holders to forbid distribution in
countries with draconian laws that would otherwise contradict these
licenses.

LGPLv2.1~\S13 sets up FSF as the steward of the LGPL, just as GPLv2~\S9
does for GPL. Meanwhile, LGPLv2.1~\S14 reminds licensees that copyright
holders can grant exceptions to the terms of LGPL, just as GPLv2~\S10
reminds licensees of the same thing.

Finally, the assertions of no warranty and limitations of liability are
identical; thus LGPLv2.1~\S15 and LGPLv2.1~\S16 are the same as GPLv2~\S11 and \S
12.

As we see, the entire latter half of the license is identical.
The parts which set up the legal boundaries and meta-rules for the license
are the same. It is our intent that the two licenses operate under the
same legal mechanisms and are enforced precisely the same way.

We strike a difference only in the early portions of the license.
Namely, in the LGPL we go into deeper detail of granting various permissions to
create derivative works, so the redistributors can make
some proprietary derivatives. Since we simply do not allow the
license to stretch as far as copyright law does regarding what
derivative works must be relicensed under the same terms, we must go
further to explain which derivative works we will allow to be
proprietary. Thus, we'll see that the front matter of the LGPL is a
bit more wordy and detailed with regards to the permissions granted to
those who modify or redistribute the software.

\section{Additions to the Preamble}

Most of LGPL's Preamble is identical, but the last seven paragraphs
introduce the concepts and reasoning behind creation of the license,
presenting a more generalized and briefer version of the story with which
we began our consideration of LGPL\@.

In short, FSF designed LGPL for those edge cases where the freedom of the
public can better be served by a more lax licensing system. FSF doesn't
encourage use of LGPL automatically for any software that happens to be a
library; rather, FSF suggests that it only be used in specific cases, such
as the following:

\begin{itemize}

\item To encourage the widest possible use of a Free Software library, so
  it becomes a de-facto standard over similar, although not
  interface-identical, proprietary alternatives

\item To encourage use of a Free Software library that already has
  interface-identical proprietary competitors that are more developed

\item To allow a greater number of users to get freedom, by encouraging
  proprietary companies to pick a Free alternative for its otherwise
  proprietary products

\end{itemize}

LGPL's preamble sets forth the limits to which the license seeks to go in
chasing these goals. LGPL is designed to ensure that users who happen to
acquire software linked with such libraries have full freedoms with
respect to that library. They should have the ability to upgrade to a newer
or modified Free version or to make their own modifications, even if they
cannot modify the primary software program that links to that library.

Finally, the preamble introduces two terms used throughout the license to
clarify between the different types of derivative works: ``works that use
the library,'' and ``works based on the library.''  Unlike GPL, LGPL must
draw some lines regarding derivative works. We do this here in this
license because we specifically seek to liberalize the rights afforded to
those who make derivative works. In GPL, we reach as far as copyright law
allows. In LGPL, we want to draw a line that allows some derivative works
copyright law would otherwise prohibit if the copyright holder exercised
his full permitted controls over the work.

\section{An Application: A Work that Uses the Library}

In the effort to allow certain proprietary derivative works and prohibit
others, LGPL distinguishes between two classes of derivative works:
``works based on the library,'' and ``works that use the library.''  The
distinction is drawn on the bright line of binary (or runtime) derivative
works and source code derivatives. We will first consider the definition
of a ``work that uses the library,'' which is set forth in LGPLv2.1~\S5.

We noted in our discussion of GPLv2~\S3 (discussed in
Section~\ref{GPLv2s3} of this document) that binary programs when
compiled and linked with GPL'd software are derivative works of that GPL'd
software. This includes both linking that happens at compile-time (when
the binary is created) or at runtime (when the binary -- including library
and main program both -- is loaded into memory by the user). In GPL,
binary derivative works are controlled by the terms of the license (in GPLv2~\S3),
and distributors of such binary derivatives must release full
corresponding source\@.

In the case of LGPL, these are precisely the types of derivative works
we wish to permit. This scenario, defined in LGPL as ``a work that uses
the library,'' works as follows:

\newcommand{\workl}{$\mathcal{L}$}
\newcommand{\lplusi}{$\mathcal{L\!\!+\!\!I}$}

\begin{itemize}

\item A new copyright holder creates a separate and independent work,
  \worki{}, that makes interface calls (e.g., function calls) to the
  LGPL'd work, called \workl{}, whose copyright is held by some other
  party. Note that since \worki{} and \workl{} are separate and
  independent works, there is no copyright obligation on this new copyright
  holder with regard to the licensing of \worki{}, at least with regard to
  the source code.

\item The new copyright holder, for her software to be useful, realizes
  that it cannot run without combining \worki{} and \workl{}.
  Specifically, when she creates a running binary program, that running
  binary must be a derivative work, called \lplusi{}, that the user can
  run.

\item Since \lplusi{} is a derivative work of both \worki{} and \workl{},
  the license of \workl{} (the LGPL) can put restrictions on the license
  of \lplusi{}. In fact, this is what LGPL does.

\end{itemize}

We will talk about the specific restrictions LGPLv2.1 places on ``works
that use the library'' in detail in Section~\ref{lgpl-section-6}. For
now, focus on the logic related to how the LGPLv2.1 places requirements on
the license of \lplusi{}. Note, first of all, the similarity between
this explanation and that in Section~\ref{separate-and-independent},
which discussed the combination of otherwise separate and independent
works with GPL'd code. Effectively, what LGPLv2.1 does is say that when a
new work is otherwise separate and independent, but has interface
calls out to an LGPL'd library, then it is considered a ``work that
uses the library.''

In addition, the only reason that LGPLv2.1 has any control over the licensing
of a ``work that uses the library'' is for the same reason that GPL has
some say over separate and independent works. Namely, such controls exist
because the {\em binary combination\/} (\lplusi{}) that must be created to
make the separate work (\worki{}) at all useful is a derivative work of
the LGPLv2.1'd software (\workl{}).

Thus, a two-question test that will help indicate if a particular work is
a ``work that uses the library'' under LGPLv2.1 is as follows:

\begin{enumerate}

\item Is the source code of the new copyrighted work, \worki{}, a
  completely independent work that stands by itself, and includes no
  source code from \workl{}?

\item When the source code is compiled, does it create a derivative work
  by combining with \workl{}, either by static (compile-time) or dynamic
  (runtime) linking, to create a new binary work, \lplusi{}?
\end{enumerate}

If the answers to both questions are ``yes,'' then \worki{} is most likely
a ``work that uses the library.''  If the answer to the first question
``yes,'' but the answer to the second question is ``no,'' then most likely
\worki{} is neither a ``work that uses the library'' nor a ``work based on
the library.''  If the answer to the first question is ``no,'' but the
answer to the second question is ``yes,'' then an investigation into
whether or not \worki{} is in fact a ``work based on the library'' is
warranted.

\section{The Library, and Works Based On It}

In short, a ``work based on the library'' could be defined as any
derivative work of LGPL'd software that cannot otherwise fit the
definition of a ``work that uses the library.''  A ``work based on the
library'' extends the full width and depth of copyright derivative works,
in the same sense that GPL does.

Most typically, one creates a ``work based on the library'' by directly
modifying the source of the library. Such a work could also be created by
tightly integrating new software with the library. The lines are no doubt
fuzzy, just as they are with GPL'd works, since copyright law gives us no
litmus test for derivative works of a software program.

Thus, the test to use when considering whether something is a ``work
based on the library'' is as follows:

\begin{enumerate}

\item Is the new work, when in source form, a derivative work under
  copyright law of the LGPL'd work?

\item Is there no way in which the new work fits the definition of a
  ``work that uses the library''?
\end{enumerate}


If the answer is ``yes'' to both these questions, then you most likely
have a ``work based on the library.''  If the answer is ``no'' to the
first but ``yes'' to the second, you are in a gray area between ``work
based on the library'' and a ``work that uses the library.''

In our years of work with the LGPLv2.1, however, we have never seen a work
of software that was not clearly one or the other; the line is quite
bright. At times, though, we have seen cases where a derivative work
appeared in some ways to be a work that used the library and in other
ways a work based on the library. We overcame this problem by
dividing the work into smaller subunits. It was soon discovered that
what we actually had were three distinct components: the original
LGPL'd work, a specific set of works that used that library, and a
specific set of works that were based on the library. Once such
distinctions are established, the licensing for each component can be
considered independently and the LGPLv2.1 applied to each work as
prescribed.


\section{Subtleties in Defining the Application}

In our discussion of the definition of ``works that use the library,'' we
left out a few more complex details that relate to lower-level programming
details. The fourth paragraph of LGPLv2.1~\S5 covers these complexities,
and it has been a source of great confusion. Part of the confusion comes
because a deep understanding of how compiler programs work is nearly
mandatory to grasp the subtle nature of what LGPLv2.1~\S5, \P 4 seeks to
cover. It helps some to note that this is a border case that we cover in
the license only so that when such a border case is hit, the implications
of using LGPL continue in the expected way.

To understand this subtle point, we must recall the way that a compiler
operates. The compiler first generates object code, which are the binary
representations of various programming modules. Each of those modules is
usually not useful by itself; it becomes useful to a user of a full program
when those modules are {\em linked\/} into a full binary executable.

As we have discussed, the assembly of modules can happen at compile-time
or at runtime. Legally, there is no distinction between the two --- both
create a derivative work by copying and combining portions of one work and
mixing them with another. However, under LGPL, there is a case in the
compilation process where the legal implications are different.
Specifically, while we know that a ``work that uses the library'' is one
whose final binary is a derivative work, but whose source is not, there
are cases where the object code --- that intermediate step between source
and final binary --- is a derivative work created by copying verbatim code
from the LGPL'd software.

For efficiency, when a compiler turns source code into object code, it
sometimes places literal portions of the copyrighted library code into the
object code for an otherwise separate independent work. In the normal
scenario, the derivative would not be created until final assembly and
linking of the executable occurred. However, when the compiler does this
efficiency optimization, at the intermediate object code step, a
derivative work is created.

LGPLv2.1~\S5\P4 is designed to handle this specific case. The intent of
the license is clearly that simply compiling software to ``make use'' of
the library does not in itself cause the compiled work to be a ``work
based on the library.''  However, since the compiler copies verbatim,
copyrighted portions of the library into the object code for the otherwise
separate and independent work, it would actually cause that object file to be a
``work based on the library.''  It is not FSF's intent that a mere
compilation idiosyncrasy would change the requirements on the users of the
LGPLv2.1'd software. This paragraph removes that restriction, allowing the
implications of the license to be the same regardless of the specific
mechanisms the compiler uses underneath to create the ``work that uses the
library.''

As it turns out, we have only once had anyone worry about this specific
idiosyncrasy, because that particular vendor wanted to ship object code
(rather than final binaries) to their customers and was worried about
this edge condition. The intent of clarifying this edge condition is
primarily to quell the worries of software engineers who understand the
level of verbatim code copying that a compiler often does, and to help
them understand that the full implications of LGPLv2.1 are the same regardless
of the details of the compilation progress.

\section{LGPLv2.1~\S6 \& LGPLv2.1~\S5: Combining the Works}
\label{lgpl-section-6}
Now that we have established a good working definition of works that
``use'' and works that ``are based on'' the library, we will consider the
rules for distributing these two different works.

The rules for distributing ``works that use the library'' are covered in
LGPLv2.1~\S6\@. LGPLv2.1~\S6 is much like GPLv2~\S3, as it requires the release
of source when a binary version of the LGPL'd software is released. Of
course, it only requires that source code for the library itself be made
available. The work that ``uses'' the library need not be provided in
source form. However, there are also conditions in LGPLv2.1~\S6 to make sure
that a user who wishes to modify or update the library can do so.

LGPLv2.1~\S6 lists five choices with regard to supplying library source
and granting the freedom to modify that library source to users. We
will first consider the option given by \S~6(b), which describes the
most common way currently used for LGPLv2.1 compliance on a ``work that
uses the library.''

LGPLv2.1~\S6(b) allows the distributor of a ``work that uses the library'' to
simply use a dynamically linked, shared library mechanism to link with the
library. This is by far the easiest and most straightforward option for
distribution. In this case, the executable of the work that uses the
library will contain only the ``stub code'' that is put in place by the
shared library mechanism, and at runtime the executable will combine with
the shared version of the library already resident on the user's computer.
If such a mechanism is used, it must allow the user to upgrade and
replace the library with interface-compatible versions and still be able
to use the ``work that uses the library.''  However, all modern shared
library mechanisms function as such, and thus LGPLv2.1~\S6(b) is the simplest
option, since it does not even require that the distributor of the ``work
2based on the library'' ship copies of the library itself.

LGPLv2.1~\S6(a) is the option to use when, for some reason, a shared library
mechanism cannot be used. It requires that the source for the library be
included, in the typical GPL fashion, but it also has a requirement beyond
that. The user must be able to exercise her freedom to modify the library
to its fullest extent, and that means recombining it with the ``work based
on the library.''  If the full binary is linked without a shared library
mechanism, the user must have available the object code for the ``work
based on the library,'' so that the user can relink the application and
build a new binary.

The remaining options in LGPLv2.1~\S6 are very similar to the other choices
provided by GPLv2~\S3. There are some additional options, but time does
not permit us in this course to go into those additional options. In
almost all cases of distribution under LGPL, either LGPLv2.1~\S6(a) or LGPLv2.1~\S6(b) are
exercised.

\section{Distribution of the Combined Works}

Essentially, ``works based on the library'' must be distributed under the
same conditions as works under full GPL\@. In fact, we note that 
LGPLv2.1~\S2 is nearly identical in its terms and requirements to GPLv2~\S2.
There are again subtle differences and additions, which time does not
permit us to cover in this course.

\section{And the Rest}

The remaining variations between LGPL and GPL cover the following
conditions:

\begin{itemize}

\item Allowing a licensing ``upgrade'' from LGPL to GPL\@ (in LGPLv2.1~\S3)

\item Binary distribution of the library only, covered in LGPLv2.1~\S4,
  which is effectively equivalent to LGPLv2.1~\S3

\item Creating aggregates of libraries that are not derivative works of
  each other, and distributing them as a unit (in LGPLv2.1~\S7)

\end{itemize}


Due to time constraints, we cannot cover these additional terms in detail,
but they are mostly straightforward. The key to understanding LGPLv2.1 is
understanding the difference between a ``work based on the library'' and a
``work that uses the library.''  Once that distinction is clear, the
remainder of LGPLv2.1 is close enough to GPL that the concepts discussed in
our more extensive GPL unit can be directly applied.

%%%%%%%%%%%%%%%%%%%%%%%%%%%%%%%%%%%%%%%%%%%%%%%%%%%%%%%%%%%%%%%%%%%%%%%%%%%%%%%
\chapter{Integrating the GPL into Business Practices}

Since GPL'd software is now extremely prevalent through the industry, it
is useful to have some basic knowledge about using GPL'd software in
business and how to build business models around GPL'd software.

\section{Using GPL'd Software In-House}

As discussed in Sections~\ref{GPLv2s0} and~\ref{GPLv2s5} of this tutorial,
the GPL only governs the activities of copying, modifying and
distributing software programs that are not governed by the license.
Thus, in FSF's view, simply installing the software on a machine and
using it is not controlled or limited in any way by GPL\@. Using Free
Software in general requires substantially fewer agreements and less
license compliance activity than any known proprietary software.

Even if a company engages heavily in copying the software throughout the
enterprise, such copying is not only permitted by GPLv2~\S\S1 and 3, but it is
encouraged!  If the company simply deploys unmodified (or even modified)
Free Software throughout the organization for its employees to use, the
obligations under the license are very minimal. Using Free Software has a
substantially lower cost of ownership --- both in licensing fees and in
licensing checking and handling -- than the proprietary software
equivalents.

\section{Business Models}
\label{Business Models}

Using Free Software in house is certainly helpful, but a thriving
market for Free Software-oriented business models also exists. There is the
traditional model of selling copies of Free Software distributions.
Many companies, including IBM and Red Hat, make substantial revenue
from this model. IBM primarily chooses this model because they have
found that for higher-end hardware, the cost of the profit made from
proprietary software licensing fees is negligible. The real profit is
in the hardware, but it is essential that software be stable, reliable
and dependable, and the users be allowed to have unfettered access to
it. Free Software, and GPL'd software in particular (because IBM can
be assured that proprietary versions of the same software will not
exists to compete on their hardware) is the right choice.

Red Hat has actually found that a ``convenience fee'' for Free Software,
when set at a reasonable price (around \$60 or so), can produce some
profit. Even though Red Hat's system is fully downloadable on their
Web site, people still go to local computer stores and buy copies of their
box set, which is simply a printed version of the manual (available under
a Free license as well) and the Free Software system it documents.

\medskip

However, custom support, service, and software improvement contracts
are the most widely used models for GPL'd software. The GPL is
central to their success, because it ensures that the code base
remains common, and that large and small companies are on equal
footing for access to the technology. Consider, for example, the GNU
Compiler Collection (GCC). Cygnus Solutions, a company started in the
early 1990s, was able to grow steadily simply by providing services
for GCC --- mostly consisting of new ports of GCC to different or new,
embedded targets. Eventually, Cygnus was so successful that
it was purchased by Red Hat where it remains a profitable division.

However, there are very small companies like CodeSourcery, as well as
other medium-sized companies like MontaVista and OpenTV that compete in
this space. Because the code-base is protect by GPL, it creates and
demands industry trust. Companies can cooperate on the software and
improve it for everyone. Meanwhile, companies who rely on GCC for their
work are happy to pay for improvements, and for ports to new target
platforms. Nearly all the changes fold back into the standard
versions, and those forks that exist remain freely available.

\medskip

\label{Proprietary Relicensing}

A final common business model that is perhaps the most controversial is
proprietary relicensing of a GPL'd code base. This is only an option for
software in which a particular entity is the sole copyright holder. As
discussed earlier in this tutorial, a copyright holder is permitted under
copyright law to license a software system under her copyright as many
different ways as she likes to as many different parties as she wishes.

Some companies, such as MySQL AB and TrollTech, use this to their
financial advantage with regard to a GPL'd code base. The standard
version is available from the company under the terms of the GPL\@.
However, parties can purchase separate proprietary software licensing for
a fee.

This business model is problematic because it means that the GPL'd code
base must be developed in a somewhat monolithic way, because volunteer
Free Software developers may be reluctant to assign their copyrights to
the company because it will not promise to always and forever license the
software as Free Software. Indeed, the company will surely use such code
contributions in proprietary versions licensed for fees.

\section{Ongoing Compliance}

GPL compliance is in fact a very simple matter -- much simpler than
typical proprietary software agreements and EULAs. Usually, the most
difficult hurdle is changing from a proprietary software mindset to one
that seeks to foster a community of sharing and mutual support. Certainly
complying with the GPL from a users' perspective gives substantially fewer
headaches than proprietary license compliance.

For those who go into the business of distributing {\em modified\\}
versions of GPL'd software, the burden is a bit higher, but not by
much. The glib answer is that by releasing the whole product as Free
Software, it is always easy to comply with the GPL. However,
admittedly to the dismay of FSF, many modern and complex software
systems are built using both proprietary and GPL'd components that are
not legally derivative works of each other. Sometimes, it is easier simply to
improve existing GPL'd application than to start from scratch. In
exchange for that benefit, the license requires that the modifier give
back to the commons that made the work easier in the first place. It is a
reasonable trade-off and a way to help build a better world while also
making a profit.

Note that FSF does provide services to assist companies who need
assistance in complying with the GPL. You can contact FSF's GPL
Compliance Labs at $<$compliance@fsf.org$>$.

If you are particularly interested in matters of GPL compliance, we
recommend the second course in this series, {\em GPL Compliance Case
  Studies and Legal Ethics in Free Software Licensing\/}, in which we
discuss some real GPL violation cases that FSF has worked to resolve.
Consideration of such cases can help give insight on how to handle GPL
compliance in new situations.


% =====================================================================
% END OF FIRST DAY SEMINAR SECTION
% =====================================================================


% compliance-guide.tex                            -*- LaTeX -*-

\part{A Practical Guide to GPL Compliance}
\label{gpl-compliance-guide}

{\parindent 0in
This part is: \\
\begin{tabbing}
Copyright \= \copyright{} 2014 \= \hspace{.2in} Bradley M. Kuhn. \\
Copyright \> \copyright{} 2008 \> \hspace{.2in} Software Freedom Law Center. \\
\end{tabbing}

\vspace{1in}

\begin{center}
Authors of this part are: \\

Bradley M. Kuhn \\
Aaron Williamson \\
Karen M. Sandler \\

\vspace{1in}

Copy editors of this part include: \\
Martin Michlmayr

\vspace{3in}

The copyright holders of this part hereby grant the freedom to copy, modify,
convey, Adapt, and/or redistribute this work under the terms of the Creative
Commons Attribution Share Alike 4.0 International License.  A copy of that
license is available at
\verb=https://creativecommons.org/licenses/by-sa/4.0/legalcode=. 
\end{center}
}

\bigskip

\chapter*{Executive Summary}

This is a guide to effective compliance with the GNU General Public
License (GPL) and related licenses.  Copyleft advocates
usually seek to assist the community with
GPL compliance cooperatively.   This guide focuses on complying from the
start, so that readers can learn to avoid enforcement actions entirely, or, at
least, minimize  the negative impact when enforcement actions occur.
This guide  introduces and explains basic legal concepts related to the GPL and its
enforcement by copyright holders. It also outlines business practices and
methods that lead to better GPL compliance.  Finally, it recommends proper
post-violation responses to the concerns of copyright holders.

\chapter{Background}

Early GPL enforcement efforts began soon after the GPL was written by
Richard M.~Stallman (RMS) in 1989, and consisted of informal community efforts,
often in public Usenet discussions.\footnote{One example is the public
  outcry over NeXT's attempt to make the Objective-C front-end to GCC
  proprietary.  RMS, in fact, handled this enforcement action personally and
  the Objective-C front-end is still part of upstream GCC today.}  Over the next decade, the Free Software Foundation (FSF),
which holds copyrights in many GNU programs, was the only visible entity
actively enforcing its GPL'd copyrights on behalf of the software freedom
community.
FSF's enforcement
was generally a private process; the FSF contacted violators
confidentially and helped them to comply with the license.  Most
violations were pursued this way until the early 2000's.

By that time, Linux-based systems such as GNU/Linux and BusyBox/Linux had become very common, particularly in
embedded devices such as wireless routers.  During this period, public
ridicule of violators in the press and on Internet fora supplemented
ongoing private enforcement and increased pressure on businesses to
comply.  In 2003, the FSF formalized its efforts into the GPL Compliance
Lab, increased the volume of enforcement, and built community coalitions
to encourage copyright holders to together settle amicably with violators.
Beginning in 2004, Harald Welte took a more organized public enforcement
approach and launched \verb0gpl-violations.org0, a website and mailing
list for collecting reports of GPL violations.  On the basis of these
reports, Welte successfully pursued many enforcements in Europe, including
formal legal action.  Harald earns the permanent fame as the first copyright
holder to bring legal action in a court regarding GPL compliance.

In 2007, two copyright holders in BusyBox, in conjunction with the
Software Freedom Conservancy (``Conservancy''), filed the first copyright infringement lawsuit
based on a violation of the GPL\@ in the USA. While  lawsuits are of course
quite public, the vast majority of Conservancy's enforcement actions 
are resolved privately via
cooperative communications with violators.  As both FSF and Conservancy have worked to bring
individual companies into compliance, both organizations have encountered numerous
violations resulting from preventable problems such as inadequate
attention to licensing of upstream software, misconceptions about the
GPL's terms, and poor communication between software developers and their
management.  This document highlights these problems and describe
best practices to encourage corporate Free Software users to reevaluate their
approach to GPL'd software and avoid future violations.

Both FSF and Conservancy continue GPL enforcement and compliance efforts
for software under the GPL, the GNU Lesser
Public License (LGPL) and other copyleft licenses.  In doing so, both organizations have
found that most violations stem from a few common, avoidable mistakes.  All copyleft advocates  hope to educate the community of
commercial distributors, redistributors, and resellers on how to avoid
violations in the first place, and to respond adequately and appropriately
when a violation occurs.

\chapter{Best Practices to Avoid Common Violations}
\label{best-practices}

Unlike highly permissive licenses (such as the ISC license), which
typically only require preservation of copyright notices, licensees face many
important requirements from the GPL.  These requirements are
carefully designed to uphold certain values and standards of the software
freedom community.  While the GPL's requirements may initially appear
counter-intuitive to those more familiar with proprietary software
licenses, by comparison, its terms are in fact clear and quite favorable to
licensees.  Indeed, the GPL's terms actually simplify compliance when
violations occur.

GPL violations occur (or, are compounded) most often when companies lack sound
practices for the incorporation of GPL'd components into their
internal development environment.  This section introduces some best
practices for software tool selection, integration and distribution,
inspired by and congruent with software freedom methodologies.  Companies should
establish such practices before building a product based on GPL'd
software.\footnote{This document addresses compliance with GPLv2,
  GPLv3, LGPLv2, and LGPLv3.  Advice on avoiding the most common
  errors differs little for compliance with these four licenses.
  \S~\ref{lgpl} discusses the key differences between GPL and LGPL
  compliance.}

\section{Evaluate License Applicability}
\label{derivative-works}
Political discussion about the GPL often centers around the ``copyleft''
requirements of the license.  Indeed, the license was designed primarily
to embody this licensing feature.  Most companies adding non-trivial
features (beyond mere porting and bug-fixing) to GPL'd software (and
thereby invoking these requirements) are already well aware of their
more complex obligations under the license.\footnote{There has been much legal
  discussion regarding copyleft and derivative works.  In practical
  reality, this issue is not relevant to the vast majority of companies
  distributing GPL'd software.  Those interested in this issue should study
  \tutorialpartsplit{\textit{Detailed Analysis of the GNU GPL and Related
      Licenses}'s Section on derivative works}{\S~\ref{derivative-works} of
    this tutorial}.}

However, experienced  GPL enforcers find that few redistributors'
compliance challenges relate directly to combined work issues in copyleft.
Instead, the distributions of GPL'd
systems most often encountered typically consist of a full operating system
including components under the GPL (e.g., Linux, BusyBox) and components
under the LGPL (e.g., the GNU C Library).  Sometimes, these programs have
been patched or slightly improved by direct modification of their sources,
resulting unequivocally in a derivative work.  Alongside these programs,
companies often distribute fully independent, proprietary programs,
developed from scratch, which are designed to run on the Free Software operating
system but do not combine with, link to, modify, derive from, or otherwise
create a combined work with
the GPL'd components.\footnote{However, these programs do often combine
  with LGPL'd libraries. This is discussed in detail in \S~\ref{lgpl}.}
In the latter case, where the work is unquestionably a separate work of
creative expression, no copyleft provisions are invoked.
The core compliance issue faced, thus, in such a situation, is not an discussion of what is or is not a
combined or derivative work, but rather, issues related to distribution and
conveyance of binary works based on GPL'd source, but without Complete,
Corresponding Source.  This tutorial therefore focuses primarily on that issue.

Admittedly, a tiny
minority of compliance situations relate to question of derivative and
combined words.  Those
situations are so rare, and the details from situation to situation differ
greatly.  Thus, such situations require a highly
fact-dependent analysis and cannot be addressed in a general-purpose
document such as this one.

\medskip

Most companies accused of violations lack a basic understanding
of how to comply even in the straightforward scenario.  This document
provides those companies with the fundamental and generally applicable prerequisite knowledge.
For answers to rarer and more complicated legal questions, such as whether
your software is a derivative or combined work of some copylefted software, consult
with an attorney.\footnote{If you would like more information on the
  application of derivative works doctrine to software, a detailed legal
  discussion is presented in our colleague Dan Ravicher's article,
  \textit{Software Derivative Work: A Circuit Dependent Determination} and in
  \tutorialpartsplit{\textit{Detailed Analysis of the GNU GPL and Related
      Licenses}'s Section on derivative works}{\S~\ref{derivative-works} of
    this tutorial}.}

This discussion thus assumes that you have already identified the
``work'' covered by the license, and that any components not under the GPL
(e.g., applications written entirely by your developers that merely happen
to run on a Linux-based operating system) distributed in conjunction with
those works are separate works within the meaning of copyright law and the GPL\@.  In
such a case, the GPL requires you to provide complete corresponding
source (CCS)\footnote{For more on CCS,  see
\tutorialpartsplit{\textit{Detailed Analysis of the GNU GPL and Related
      Licenses}'s Section on GPLv2~\S2 and GPLv3~\S1.}{\S~\ref{GPLv2s2} and \S~\ref{GPLv3s1} of
    this tutorial}.}
for the GPL'd components and your modifications thereto, but not
for independent proprietary applications.  The procedures described in
this document address this typical scenario.

\section{Monitor Software Acquisition}

Software engineers deserve the freedom to innovate and import useful
software components to improve products.  However, along with that
freedom should come rules and reporting procedures to make sure that you
are aware of what software that you include with your product.

The most typical response to an initial enforcement action is: ``We
didn't know there was GPL'd stuff in there''.  This answer indicates
failure in the software acquisition and procurement process.  Integration
of third-party proprietary software typically requires a formal
arrangement and management/legal oversight before the developers
incorporate the software.  By contrast, developers often obtain and
integrate Free Software without intervention nor oversight. That ease of acquisition, however,
does not mean the oversight is any less necessary.  Just as your legal
and/or management team negotiates terms for inclusion of any proprietary
software, they should gently facilitate all decisions to bring Free Software into your
product.

Simple, engineering-oriented rules help provide a stable foundation for
Free Software integration.  For example, simply ask your software developers to send an email to a
standard place describing each new Free Software component they add to the system,
and have them include a brief description of how they will incorporate it
into the product.  Further, make sure developers use a revision control
system (such as Git or Mercurial), and
store the upstream versions of all software in a ``vendor branch'' or
similar mechanism, whereby they can easily track and find the main version
of the software and, separately, any local changes.

Such procedures are best instituted at your project's launch.  Once 
chaotic and poorly-sourced development processes begin, cataloging the
presence of GPL'd components  becomes challenging.

Such a situation often requires use of a tool to ``catch up'' your knowledge
about what software your product includes.  Most commonly, companies choose
some software licensing scanning tool to inspect the codebase.  However,
there are few tools that are themselves Free Software.  Thus, GPL enforcers
usually recommend the GPL'd
\href{http://fossology.org/}{FOSSology system}, which analyzes a
source code base and produces a list of Free Software licenses that may apply to
the code.  FOSSology can help you build a catalog of the sources you have
already used to build your product.  You can then expand that into a more
structured inventory and process.

\section{Track Your Changes and Releases}

As explained in further detail below, the most important component of GPL
compliance is the one most often ignored: proper inclusion of CCS in all
distributions  of GPL'd
software.  To comply with GPL's CCS requirements, the distributor
\textit{must} always know precisely what sources generated a given binary
distribution.

In an unfortunately large number of our enforcement cases, the violating
company's engineering team had difficulty reconstructing the CCS
for binaries distributed by the company.  Here are three simple rules to
follow to decrease the likelihood of this occurrence:

\begin{itemize}

\item Ensure that your
developers are using revision control systems properly.

\item Have developers mark or ``tag'' the full source tree corresponding to
  builds distributed to customers.

\item Check that your developers store all parts of the software
development in the revision control system, including {\sc readme}s, build
scripts, engineers' notes, and documentation.
\end{itemize}

Your developers will benefit anyway from these rules.  Developers will be
happier in their jobs if their tools already track the precise version of
source that corresponds to any deployed binary.

\section{Avoid the ``Build Guru''}

Too many software projects rely on only one or a very few team members who
know how to build and assemble the final released product.  Such knowledge
centralization not only creates engineering redundancy issues, but also
thwarts GPL compliance.  Specifically, CCS does not just require source code,
but scripts and other material that explain how to control compilation and
installation of the executable and object code.

Thus, avoid relying on a ``build guru'', a single developer who is the only one
who knows how to produce your final product. Make sure the build process
is well defined.  Train every developer on the build process for the final
binary distribution, including (in the case of embedded software)
generating a final firmware image suitable for distribution to the
customer.  Require developers to use revision control for build processes.
Make a rule that adding new components to the system without adequate
build instructions (or better yet, scripts) is unacceptable engineering
practice.

\chapter{Details of Compliant Distribution}

This section explains the specific requirements placed upon
distributors of GPL'd software.  Note that this section refers heavily to
specific provisions and language in
\href{http://www.gnu.org/licenses/old-licenses/gpl-2.0.html#section3}{GPLv2}
and \href{http://www.fsf.org/licensing/licenses/gpl.html#section6}{GPLv3}.
It may be helpful to have a copy of each license open while reading this
section.

\section{Binary Distribution Permission}
\label{binary-distribution-permission}

% be careful below, you cannot refill the \if section, so don't refill
% this paragraph without care.

The various versions of the GPL are copyright licenses that grant
permission to make certain uses of software that are otherwise restricted
by copyright law.  This permission is conditioned upon compliance with the
GPL's requirements.

This section walks through the requirements (of both GPLv2 and GPLv3) that
apply when you distribute GPL'd programs in binary (i.e., executable or
object code) form, which is typical for embedded applications.  Because a
binary application derives from a program's original sources, you need
permission from the copyright holder to distribute it.  \S~3 of GPLv2 and
\S~6 of GPLv3 contain the permissions and conditions related to binary
distributions of GPL'd programs.\footnote{These sections cannot be fully
  understood in isolation; read the entire license thoroughly before
  focusing on any particular provision.  However, once you have read and
  understood the entire license, look to these sections to guide
  compliance for binary distributions.}

GPL's binary distribution sections offer a choice of compliance methods,
each of which we consider in turn.  Each option refers to the
``Corresponding Source'' code for the binary distribution, which includes
the source code from which the binary was produced.  This abbreviated and
simplified definition is sufficient for the binary distribution discussion
in this section, but you may wish to refer back to this section after
reading the thorough discussion of ``Corresponding Source'' that appears
in \S~\ref{corresponding-source}.

\subsection{Option (a): Source Alongside Binary}

GPLv2~\S~3(a) and v3~\S~6(a) embody the easiest option for providing
source code: including Corresponding Source with every binary
distribution.  While other options appear initially less onerous, this
option invariably minimizes potential compliance problems, because when
you distribute Corresponding Source with the binary, \emph{your GPL
  obligations are satisfied at the time of distribution}.  This is not
true of other options, and for this reason, we urge you to seriously
consider this option.  If you do not, you may extend the duration of your
obligations far beyond your last binary distribution.

Compliance under this option is straightforward.  If you ship a product
that includes binary copies of GPL'd software (e.g., in firmware, or on a
hard drive, CD, or other permanent storage medium), you can store the
Corresponding Source alongside the binaries.  Alternatively, you can
include the source on a CD or other removable storage medium in the box
containing the product.

GPLv2 refers to the various storage mechanisms as ``medi[a] customarily
used for software interchange''.  While the Internet has attained primacy
as a means of software distribution where super-fast Internet connections
are available, GPLv2 was written at a time when downloading software was
not practical (and was often impossible).  For much of the world, this
condition has not changed since GPLv2's publication, and the Internet
still cannot be considered ``a medium customary for software
interchange''.  GPLv3 clarifies this matter, requiring that source be
``fixed on a durable physical medium customarily used for software
interchange''.  This language affirms that option (a) requires binary
redistributors to provide source on a physical medium.

Please note that while selection of option (a) requires distribution on a
physical medium, voluntary distribution via the Internet is very useful.  This
is discussed in detail in \S~\ref{offer-with-internet}.

\subsection{Option (b): The Offer}
\label{offer-for-source}

Many distributors prefer to ship only an offer for source with the binary
distribution, rather than the complete source package.  This
option has value when the cost of source distribution is a true
per-unit cost.  For example, this option might be a good choice for
embedded products with permanent storage too small to fit the source, and
which are not otherwise shipped with a CD but \emph{are} shipped with a
manual or other printed material.

However, this option increases the duration of your obligations
dramatically.  An offer for source must be good for three full years from
your last binary distribution (under GPLv2), or your last binary or spare
part distribution (under GPLv3).  Your source code request and
provisioning system must be designed to last much longer than your product
life cycle.

In addition, if you are required to comply with the terms of GPLv2, you
{\bf cannot} use a network service to provide the source code.  For GPLv2,
the source code offer is fulfilled only with physical media.  This usually
means that you must continue to produce an up-to-date ``source code CD''
for years after the product's end-of-life.

\label{offer-with-internet}

Under GPLv2, it is acceptable and advisable for your offer for source code
to include an Internet link for downloadable source \emph{in addition} to
offering source on a physical medium.  This practice enables those with
fast network connections to get the source more quickly, and typically
decreases the number of physical media fulfillment requests.
(GPLv3~\S~6(b) permits provision of source with a public
network-accessible distribution only and no physical media.  We discuss
this in detail at the end of this section.)

The following is a suggested compliant offer for source under GPLv2 (and
is also acceptable for GPLv3) that you would include in your printed
materials accompanying each binary distribution:

\begin{quote}
The software included in this product contains copyrighted software that
is licensed under the GPL\@.  A copy of that license is included in this
document on page $X$\@.  You may obtain the complete Corresponding Source
code from us for a period of three years after our last shipment of this
product, which will be no earlier than 2011-08-01, by sending a money
order or check for \$5 to: \\
GPL Compliance Division \\
Our Company \\
Any Town, US 99999 \\
\\
Please write ``source for product $Y$'' in the memo line of your
payment.

You may also find a copy of the source at
\verb0http://www.example.com/sources/Y/0.

This offer is valid to anyone in receipt of this information.
\end{quote}

There are a few important details about this offer.  First, it requires a
copying fee.  GPLv2 permits ``a charge no more than your cost of
physically performing source distribution''.  This fee must be reasonable.
If your cost of copying and mailing a CD is more than around \$10, you
should perhaps find a cheaper CD stock and shipment method.  It is simply
not in your interest to try to overcharge the community.  Abuse of this
provision in order to make a for-profit enterprise of source code
provision will likely trigger enforcement action.

Second, note that the last line makes the offer valid to anyone who
requests the source.  This is because v2~\S~3(b) requires that offers be
``to give any third party'' a copy of the Corresponding Source.  GPLv3 has
a similar requirement, stating that an offer must be valid for ``anyone
who possesses the object code''.  These requirements indicated in
v2~\S~3(c) and v3~\S~6(c) are so that noncommercial redistributors may
pass these offers along with their distributions.  Therefore, the offers
must be valid not only to your customers, but also to anyone who received
a copy of the binaries from them.  Many distributors overlook this
requirement and assume that they are only required to fulfill a request
from their direct customers.

The option to provide an offer for source rather than direct source
distribution is a special benefit to companies equipped to handle a
fulfillment process.  GPLv2~\S~3(c) and GPLv3~\S~6(c) avoid burdening
noncommercial, occasional redistributors with fulfillment request
obligations by allowing them to pass along the offer for source as they
received it.

Note that commercial redistributors cannot avail themselves of the option
(c) exception, and so while your offer for source must be good to anyone
who receives the offer (under v2) or the object code (under v3), it
\emph{cannot} extinguish the obligations of anyone who commercially
redistributes your product.  The license terms apply to anyone who
distributes GPL'd software, regardless of whether they are the original
distributor.  Take the example of Vendor $V$, who develops a software
platform from GPL'd sources for use in embedded devices.  Manufacturer $M$
contracts with $V$ to install the software as firmware in $M$'s device.
$V$ provides the software to $M$, along with a compliant offer for source.
In this situation, $M$ cannot simply pass $V$'s offer for source along to
its customers.  $M$ also distributes the GPL'd software commercially, so
$M$ too must comply with the GPL and provide source (or $M$'s \emph{own}
offer for source) to $M$'s customers.

This situation illustrates that the offer for source is often a poor
choice for products that your customers will likely redistribute.  If you
include the source itself with the products, then your distribution to
your customers is compliant, and their (unmodified) distribution to their
customers is likewise compliant, because both include source.  If you
include only an offer for source, your distribution is compliant but your
customer's distribution does not ``inherit'' that compliance, because they
have not made their own offer to accompany their distribution.

The terms related to the offer for source are quite different if you
distribute under GPLv3.  Under v3, you may make source available only over
a network server, as long as it is available to the general public and
remains active for three years from the last distribution of your product
or related spare part.  Accordingly, you may satisfy your fulfillment
obligations via Internet-only distribution.  This makes the ``offer for
source'' option less troublesome for v3-only distributions, easing
compliance for commercial redistributors.  However, before you switch to a
purely Internet-based fulfillment process, you must first confirm that you
can actually distribute \emph{all} of the software under GPLv3.  Some
programs are indeed licensed under ``GPLv2, \emph{or any later version}''
(often abbreviated ``GPLv2-or-later'').  Such licensing gives you the
option to redistribute under GPLv3.  However, a few popular programs are
only licensed under GPLv2 and not ``or any later version''
(``GPLv2-only'').  You cannot provide only Internet-based source request
fulfillment for the latter programs.

If you determine that all GPL'd works in your whole product allow upgrade
to GPLv3 (or were already GPLv3'd to start), your offer for source may be
as simple as this:

\begin{quote}
The software included in this product contains copyrighted software that
is licensed under the GPLv3\@.  A copy of that license is included in this
document on page $X$\@.  You may obtain the complete Corresponding Source
code from us for a period of three years after our last shipment of this
product and/or spare parts therefor, which will be no earlier than
2011-08-01, on our website at
\verb0http://www.example.com/sources/productnum/0.
\end{quote}

\medskip

Under both GPLv2 and GPLv3, source offers must be accompanied by a copy of
the license itself, either electronically or in print, with every
distribution.
 
Finally, it is unacceptable to use option (b) merely because you do not have
Corresponding Source ready.  We find that some companies choose this option
because writing an offer is easy, but producing a source distribution as
an afterthought to a hasty development process is difficult.  The offer
for source does not exist as a stop-gap solution for companies rushing to
market with an out-of-compliance product.  If you ship an offer for source
with your product but cannot actually deliver \emph{immediately} on that
offer when your customers request it, you should expect an enforcement
action.

\subsection{Option (c): Noncommercial Offers}

As discussed in the last section, GPLv2~\S~3(c) and GPLv3~\S~6(c) apply
only to noncommercial use.  These options are not available to businesses
distributing GPL'd software.  Consequently, companies that redistribute
software packaged for them by an upstream vendor cannot merely pass along
the offer they received from the vendor; they must provide their own offer
or corresponding source to their distributees.  We talk in detail about
upstream software providers in \S~\ref{upstream}.

\subsection{Option 6(d) in GPLv3: Internet Distribution}

Under GPLv2, your formal provisioning options for Corresponding Source
ended with \S~3(c).  But even under GPLv2, pure Internet source
distribution was a common practice and generally considered to be
compliant.  GPLv2 mentions Internet-only distribution almost as aside in
the language, in text at the end of the section after the three
provisioning options are listed.  To quote that part of GPLv2~\S~3:
\begin{quote}
If distribution of executable or object code is made by offering access to
copy from a designated place, then offering equivalent access to copy the
source code from the same place counts as distribution of the source code,
even though third parties are not compelled to copy the source along with
the object code.
\end{quote}

When that was written in 1991, Internet distribution of software was the
exception, not the rule.  Some FTP sites existed, but generally software
was sent on magnetic tape or CDs.  GPLv2 therefore mostly assumed that
binary distribution happened on some physical media.  By contrast,
GPLv3~\S~6(d) explicitly gives an option for this practice that the
community has historically considered GPLv2-compliant.

Thus, you may fulfill your source-provision obligations by providing the
source code in the same way and from the same location.  When exercising
this option, you are not obligated to ensure that users download the
source when they download the binary, and you may use separate servers as
needed to fulfill the requests as long as you make the source as
accessible as the binary.  However, you must ensure that users can easily
find the source code at the time they download the binary. GPLv3~\S~6(d)
thus clarifies a point that has caused confusion about source provision in
v2.  Indeed, many such important clarifications are included in v3 which
together provide a compelling reason for authors and redistributors alike
to adopt GPLv3.

\subsection{Option 6(e) in GPLv3: Software Torrents}

Peer-to-peer file sharing arose well after GPLv2 was written, and does not
easily fit any of the v2 source provision options.  GPLv3~\S~6(e)
addresses this issue, explicitly allowing for distribution of source and
binary together on a peer-to-peer file sharing network.  If you distribute
solely via peer-to-peer networks, you can exercise this option.  However,
peer-to-peer source distribution \emph{cannot} fulfill your source
provision obligations for non-peer-to-peer binary distributions.  Finally,
you should ensure that binaries and source are equally seeded upon initial
peer-to-peer distribution.

\section{Preparing Corresponding Source}
\label{corresponding-source}

Most enforcement cases involve companies that have unfortunately not
implemented procedures like our \S~\ref{best-practices} recommendations
and have no source distribution arranged at all.  These companies must
work backwards from a binary distribution to come into compliance.  Our
recommendations in \S~\ref{best-practices} are designed to make it easy to
construct a complete and Corresponding Source release from the outset.  If
you have followed those principles in your development, you can meet the
following requirements with ease.  If you have not, you may have
substantial reconstruction work to do.

\subsection{Assemble the Sources}

For every binary that you produce, you should collect and maintain a copy
of the sources from which it was built.  A large system, such as an
embedded firmware, will probably contain many GPL'd and LGPL'd components
for which you will have to provide source.  The binary distribution may
also contain proprietary components which are separate and independent
works that are covered by neither the GPL nor LGPL\@.

The best way to separate out your sources is to have a subdirectory for
each component in your system.  You can then easily mark some of them as
required for your Corresponding Source releases.  Collecting
subdirectories of GPL'd and LGPL'd components is the first step toward
preparing your release.

\subsection{Building the Sources}

Few distributors, particularly of embedded systems, take care to read the
actual definition of Corresponding Source in the GPL\@.  Consider
carefully the definition, from GPLv3:
\begin{quote}
  The ``Corresponding Source'' for a work in object code form means all
  the source code needed to generate, install, and (for an executable
  work) run the object code and to modify the work, including scripts to
  control those activities.
\end{quote}

and the definition from GPLv2:
\begin{quote}
The source code for a work means the preferred form of the work for making
modifications to it.  For an executable work, complete source code means
all the source code for all modules it contains, plus any associated
interface definition files, plus the scripts used to control compilation
and installation of the executable.
\end{quote}

Note that you must include ``scripts used to control compilation and
installation of the executable'' and/or anything ``needed to generate,
install, and (for an executable work) run the object code and to modify
the work, including scripts to control those activities''.  These phrases
are written to cover different types of build environments and systems.
Therefore, the details of what you need to provide with regard to scripts
and installation instructions vary depending on the software details.  You
must provide all information necessary such that someone generally skilled
with computer systems could produce a binary similar to the one provided.

Take as an example an embedded wireless device.  Usually, a company
distributes a firmware, which includes a binary copy of
Linux\footnote{``Linux'' refers only to the kernel, not the larger system
  as a whole.} and a filesystem.  That filesystem contains various binary
programs, including some GPL'd binaries, alongside some proprietary
binaries that are separate works (i.e., not derived from, nor based on
freely-licensed sources).  Consider what, in this case, constitutes adequate
``scripts to control compilation and installation'' or items ``needed to
generate, install and run'' the GPL'd programs.

Most importantly, you must provide some sort of roadmap that allows
technically sophisticated users to build your software.  This can be
complicated in an embedded environment.  If your developers use scripts to
control the entire compilation and installation procedure, then you can
simply provide those scripts to users along with the sources they act
upon.  Sometimes, however, scripts were never written (e.g., the
information on how to build the binaries is locked up in the mind of your
``build guru'').  In that case, we recommend that you write out build
instructions in a natural language as a detailed, step-by-step {\sc
  readme}.

No matter what you offer, you need to give those who receive source a
clear path from your sources to binaries similar to the ones you ship.  If
you ship a firmware (kernel plus filesystem), and the filesystem contains
binaries of GPL'd programs, then you should provide whatever is necessary
to enable a reasonably skilled user to build any given GPL'd source
program (and modified versions thereof), and replace the given binary in
your filesystem.  If the kernel is Linux, then the users must have the
instructions to do the same with the kernel.  The best way to achieve this
is to make available to your users whatever scripts or process your
engineers would use to do the same.

These are the general details for how installation instructions work.
Details about what differs when the work is licensed under LGPL is
discussed in \S~\ref{lgpl}, and specific details that are unique to
GPLv3's installation instructions are in \S~\ref{user-products}.

\subsection{What About the Compiler?}

The GPL contains no provision that requires distribution of the compiler
used to build the software.  While companies are encouraged to make it as
easy as possible for their users to build the sources, inclusion of the
compiler itself is not normally considered mandatory.  The Corresponding
Source definition -- both in GPLv2 and GPLv3 -- has not been typically
read to include the compiler itself, but rather things like makefiles,
build scripts, and packaging scripts.

Nonetheless, in the interest of goodwill and the spirit of the GPL, most
companies do provide the compiler itself when they are able, particularly
when the compiler is based on GCC\@ or another copylefted compiler.  If you have
a GCC-based system, it is your prerogative to redistribute that GCC
version (binaries plus sources) to your customers.  We in the software freedom
community encourage you to do this, since it often makes it easier for
users to exercise their software freedom.  However, if you chose to take
this recommendation, ensure that your GCC distribution is itself
compliant.

If you have used a proprietary, third-party compiler to build the
software, then you probably cannot ship it to your customers.  We consider
the name of the compiler, its exact version number, and where it can be
acquired as information that \emph{must} be provided as part of the
Corresponding Source.  This information is essential to anyone who wishes
to produce a binary.  It is not the intent of the GPL to require you to
distribute third-party software tools to your customer (provided the tools
themselves are not based on the GPL'd software shipped), but we do believe
it requires that you give the user all the essential non-proprietary facts
that you had at your disposal to build the software.  Therefore, if you
choose not to distribute the compiler, you should include a {\sc readme}
about where you got it, what version it was, and who to contact to acquire
it, regardless of whether your compiler is Free Software, proprietary, or
internally developed.

\section{Best Practices and Corresponding Source}

\S~\ref{best-practices} and \S~\ref{corresponding-source} above are
closely related.  If you follow the best practices outlined above, you
will find that preparing your Corresponding Source release is an easier
task, perhaps even a trivial one.

Indeed, the enforcement process itself has historically been useful to
software development teams.  Development on a deadline can lead
organizations to cut corners in a way that negatively impacts its
development processes.  We have frequently been told by violators that
they experience difficulty when determining the exact source for a binary
in production (in some cases because their ``build guru'' quit during the
release cycle).  When management rushes a development team to ship a
release, they are less likely to keep release sources tagged and build
systems well documented.

We suggest that, if contacted about a violation, product builders use GPL
enforcement as an opportunity to improve their development practices.  No
developer would argue that their system is better for having a mysterious
build system and no source tracking.  Address these issues by installing a
revision system, telling your developers to use it, and requiring your
build guru to document his or her work!

\chapter{When The Letter Comes}

Unfortunately, many GPL violators ignore their obligations until they are
contacted by a copyright holder or the lawyer of a copyright holder.  You
should certainly contact your own lawyer if you have received a letter
alleging that you have infringed copyrights that were licensed to you
under the GPL\@.  This section outlines a typical enforcement case and
provides some guidelines for response.  These discussions are
generalizations and do not all apply to every alleged violation.

\section{Understanding Who's Enforcing}
\label{compliance-understanding-whos-enforcing}
% FIXME-LATER: this text needs work.

Both  FSF and Conservancy has, as part their mission,  to spread software
freedom. When FSF or Conservancy
enforces GPL, the goal is to bring the violator back into compliance as
quickly as possible, and redress the damage caused by the violation.
That is FSF's steadfast position in a violation negotiation --- comply
with the license and respect freedom.

However, other entities who do not share the full ethos of software freedom
as institutionalized by FSF and Conservancy pursue GPL violations differently.  Oracle, a
company that produces the GPL'd MySQL database, upon discovering GPL
violations typically negotiates a proprietary software license separately for
a fee.  While this practice is not one that FSF nor Conservancy would ever
consider undertaking or even endorsing, it is a legal way for copyright
holders to proceed.

Generally, GPL enforcers come in two varieties.  First, there are
Conservancy, FSF, and other ``community enforcers'', who primarily seek the
policy goals of GPL (software freedom), and see financial compensation as
ultimately secondary to those goals.  Second, there are ``for-profit
enforcers'' who use the GPL either as a crippleware license, or sneakily
induce infringement merely to gain proprietary licensing revenue.

Note that the latter model \textit{only} works for companies that hold 100\% of
the copyrights in the infringed work.  As such, multi-copyright-held works
are fully insulated from these tactics.


\section{Communication Is Key}

GPL violations are typically only escalated when a company ignores the
copyright holder's initial communication or fails to work toward timely
compliance.  Accused violators should respond very promptly to the
initial request.  As the process continues, violators should follow up weekly with the
copyright holders to make sure everyone agrees on targets and deadlines
for resolving the situation.

Ensure that any staff who might receive communications regarding alleged
GPL violations understands how to channel the communication appropriately
within your organization.  Often, initial contact is addressed for general
correspondence (e.g., by mail to corporate headquarters or by e-mail to
general informational or support-related addresses).  Train the staff that
processes such communications to escalate them to someone with authority
to take action.  An unknowledgable response to such an inquiry (e.g., from
a first-level technical support person) can cause negotiations to fail
prematurely.

Answer promptly by multiple means (paper letter, telephone call, and
email), even if your response merely notifies the sender that you are
investigating the situation and will respond by a certain date.  Do not
let the conversation lapse until the situation is fully resolved.
Proactively follow up with synchronous communication means to be sure
communications sent by non-reliable means (such as email) were received.

Remember that the software freedom community generally values open communication and
cooperation, and these values extend to GPL enforcement.  You will
generally find that software freedom developers and their lawyers are willing to
have a reasonable dialogue and will work with you to resolve a violation
once you open the channels of communication in a friendly way.

\section{Termination}

Many redistributors overlook the GPL's termination provision (GPLv2~\S~4 and
GPLv3~\S~8).  Under v2, violators forfeit their rights to redistribute and
modify the GPL'd software until those rights are explicitly reinstated by
the copyright holder.  In contrast, v3 allows violators to rapidly resolve
some violations without consequence.

If you have redistributed an application under GPLv2\footnote{This applies
  to all programs licensed to you under only GPLv2 (``GPLv2-only'').
  However, most so-called GPLv2 programs are actually distributed with
  permission to redistribute under GPLv2 \emph{or any later version of the
    GPL} (``GPLv2-or-later'').  In the latter cases, the redistributor can
  choose to redistribute under GPLv2, GPLv3, GPLv2-or-later or even
  GPLv3-or-later.  Where the redistributor has chosen v2 explicitly, the
  v2 termination provision will always apply.  If the redistributor has
  chosen v3, the v3 termination provision will always apply.  If the
  redistributor has chosen GPLv2-or-later, then the redistributor may want
  to narrow to GPLv3-only upon violation, to take advantage of the
  termination provisions in v3.}, but have violated the terms of GPLv2,
you must request a reinstatement of rights from the copyright holders
before making further distributions, or else cease distribution and
modification of the software forever.  Different copyright holders
condition reinstatement upon different requirements, and these
requirements can be (and often are) wholly independent of the GPL\@.  The
terms of your reinstatement will depend upon what you negotiate with the
copyright holder of the GPL'd program.

Since your rights under GPLv2 terminate automatically upon your initial
violation, \emph{all your subsequent distributions} are violations and
infringements of copyright.  Therefore, even if you resolve a violation on
your own, you must still seek a reinstatement of rights from the copyright
holders whose licenses you violated, lest you remain liable for
infringement for even compliant distributions made subsequent to the
initial violation.

GPLv3 is more lenient.  If you have distributed only v3-licensed programs,
you may be eligible under v3~\S~8 for automatic reinstatement of rights.
You are eligible for automatic reinstatement when:
\begin{itemize}
\item you correct the violation and are not contacted by a copyright
  holder about the violation within sixty days after the correction, or

\item you receive, from a copyright holder, your first-ever contact
  regarding a GPL violation, and you correct that violation within thirty
  days of receipt of copyright holder's notice.
\end{itemize}

In addition to these permanent reinstatements provided under v3, violators
who voluntarily correct their violation also receive provisional
permission to continue distributing until they receive contact from the
copyright holder.  If sixty days pass without contact, that reinstatement
becomes permanent.  Nonetheless, you should be prepared to cease
distribution during those initial sixty days should you receive a
termination notice from the copyright holder.

Given that much discussion of v3 has focused on its so-called more
complicated requirements, it should be noted that v3 is, in this regard,
more favorable to violators than v2.

\chapter{Standard Requests}

As we noted above, different copyright holders have different requirements
for reinstating a violator's distribution rights.  Upon violation, you no
longer have a license under the GPL\@.  Copyright holders can therefore
set their own requirements outside the license before reinstatement of
rights.  We have collected below a list of reinstatement demands that
copyright holders often require.

\begin{itemize}

\item {\bf Compliance on all Free Software copyrights}.  Copyright holders of Free Software
  often want a company to demonstrate compliance for all GPL'd software in
  a distribution, not just their own.  A copyright holder may refuse to
  reinstate your right to distribute one program unless and until you
  comply with the licenses of all Free Software in your distribution.
 
\item {\bf Notification to past recipients}.  Users to whom you previously
  distributed non-compliant software should receive a communication
  (email, letter, bill insert, etc.) indicating the violation, describing
  their rights under the GPL, and informing them how to obtain a gratis source
  distribution.  If a customer list does not exist (such as in reseller
  situations), an alternative form of notice may be required (such as a
  magazine advertisement).

\item {\bf Appointment of a GPL Compliance Officer.}  The software freedom community
  values personal accountability when things go wrong.  Copyright holders
  often require that you name someone within the violating company
  officially responsible for Free Software license compliance, and that this
  individual serve as the key public contact for the community when
  compliance concerns arise.

\item {\bf Periodic Compliance Reports.}  Many copyright holders wish to
  monitor future compliance for some period of time after the violation.
  For some period, your company may be required to send regular reports on
  how many distributions of binary and source have occurred.
\end{itemize}

These are just a few possible requirements for reinstatement.  In the
context of a GPL violation, and particularly under v2's termination
provision, the copyright holder may have a range of requests in exchange
for reinstatement of rights.  These software developers are talented
professionals from whose work your company has benefited.  Indeed, you are
unlikely to find a better value or more generous license terms for similar
software elsewhere.  Treat the copyright holders with the same respect you
treat your corporate partners and collaborators.

\chapter{Special Topics in Compliance}

There are several other issues that are less common, but also relevant in
a GPL compliance situation.  To those who face them, they tend to be of
particular interest.

\section{LGPL Compliance}
\label{lgpl}

GPL compliance and LGPL compliance mostly involve the same issues.  As we
discussed in \S~\ref{derivative-works}, questions of modified versions of
software are highly fact-dependant and cannot be easily addressed in any
overview document.  The LGPL adds some additional complexity to the
analysis.  Namely, the various LGPL versions permit proprietary licensing
of certain types of modified versions.  These issues are well beyond the
scope of this document, but as a rule of thumb, once you have determined
(in accordance with LGPLv3) what part of the work is the ``Application''
and what portions of the source are ``Minimal Corresponding Source'', then
you can usually proceed to follow the GPL compliance rules that we
discussed, replacing our discussion of ``Corresponding Source'' with
``Minimal Corresponding Source''.

LGPL also requires that you provide a mechanism to combine the Application
with a modified version of the library, and outlines some options for
this.  Also, the license of the whole work must permit ``reverse
engineering for debugging such modifications'' to the library.  Therefore,
you should take care that the EULA used for the Application does not
contradict this permission.

\section{Upstream Providers}
\label{upstream}

With ever-increasing frequency, software development (particularly for
embedded devices) is outsourced to third parties.  If you rely on an
upstream provider for your software, note that you \emph{cannot ignore
  your GPL compliance requirements} simply because someone else packaged
the software that you distribute.  If you redistribute GPL'd software
(which you do, whenever you ship a device with your upstream's software in
it), you are bound by the terms of the GPL\@.  No distribution (including
redistribution) is permissible absent adherence to the license terms.

Therefore, you should introduce a due diligence process into your software
acquisition plans.  This is much like the software-oriented
recommendations we make in \S~\ref{best-practices}.  Implementing
practices to ensure that you are aware of what software is in your devices
can only improve your general business processes.  You should ask a clear
list of questions of all your upstream providers and make sure the answers
are complete and accurate.  The following are examples of questions you
should ask:
\begin{itemize}

\item What are all the licenses that cover the software in this device?

\item From which upstream vendors, be they companies or individuals, did
  \emph{you} receive your software before distributing it to us?

\item What are your GPL compliance procedures?

\item If there is GPL'd software in your distribution, we will be
  redistributors of this GPL'd software.  What mechanisms do you have in
  place to aid us with compliance?

\item If we follow your recommended compliance procedures, will you
  formally indemnify us in case we are nonetheless found to be in
  violation of the GPL?

\end{itemize}

This last point is particularly important.  Many GPL enforcements are
escalated because of petty finger-pointing between the distributor and its
upstream.  In our experience, agreements regarding GPL compliance issues
and procedures are rarely negotiated up front.  However, when they are,
violations are resolved much more smoothly (at least from the point of
view of the redistributor).

Consider the cost of potential violations in your acquisition process.
Using Free Software allows software vendors to reduce costs significantly, but be
wary of vendors who have done so without regard for the licenses.  If your
vendor's costs seem ``too good to be true,'' you may ultimately bear the
burden of the vendor's inattention to GPL compliance.  Ask the right
questions, demand an account of your vendors' compliance procedures, and
seek indemnity from them.

\section{User Products and Installation Information}
\label{user-products}

GPLv3 requires you to provide ``Installation Information'' when v3
software is distributed in a ``User Product.''  During the drafting of v3,
the debate over this requirement was contentious.  However, the provision
as it appears in the final license is reasonable and easy to understand.

If you put GPLv3'd software into a User Product (as defined by the
license) and \emph{you} have the ability to install modified versions onto
that device, you must provide information that makes it possible for the
user to install functioning, modified versions of the software.  Note that
if no one, including you, can install a modified version, this provision
does not apply.  For example, if the software is burned onto an
non-field-upgradable ROM chip, and the only way that chip can be upgraded
is by producing a new one via a hardware factory process, then it is
acceptable that the users cannot electronically upgrade the software
themselves.

Furthermore, you are permitted to refuse support service, warranties, and
software updates to a user who has installed a modified version.  You may
even forbid network access to devices that behave out of specification due
to such modifications.  Indeed, this permission fits clearly with usual
industry practice.  While it is impossible to provide a device that is
completely unmodifiable\footnote{Consider that the iPhone, a device
  designed primarily to restrict users' freedom to modify it, was unlocked
  and modified within 48 hours of its release.}, users are generally on
notice that they risk voiding their warranties and losing their update and
support services when they make modifications.\footnote{A popular t-shirt
  in the software freedom community reads: ``I void warranties.''.  Our community is
  well-known for modifying products with full knowledge of the
  consequences.  GPLv3's ``Installation Instructions'' section merely
  confirms that reality, and makes sure GPL rights can be fully exercised,
  even if users exercise those rights at their own peril.}

GPLv3 is in many ways better for distributors who seek some degree of
device lock-down.  Technical processes are always found for subverting any
lock-down; pursuing it is a losing battle regardless.  With GPLv3, unlike
with GPLv2, the license gives you clear provisions that you can rely on
when you are forced to cut off support, service or warranty for a customer
who has chosen to modify.

\chapter{Conclusion}

GPL compliance need not be an onerous process.  Historically, struggles
have been the result of poor development methodologies and communications,
rather than any unexpected application of the GPL's source code disclosure
requirements.

Compliance is straightforward when the entirety of your enterprise is
well-informed and well-coordinated.  The receptionists should know how to
route a GPL source request or accusation of infringement.  The lawyers
should know the basic provisions of Free Software licenses and your source
disclosure requirements, and should explain those details to the software
developers.  The software developers should use a version control system
that allows them to associate versions of source with distributed
binaries, have a well-documented build process that anyone skilled in the
art can understand, and inform the lawyers when they bring in new
software.  Managers should build systems and procedures that keep everyone
on target.  With these practices in place, any organization can comply
with the GPL without serious effort, and receive the substantial benefits
of good citizenship in the software freedom community, and lots of great code
ready-made for their products.

\vfill

% LocalWords:  redistributors NeXT's Slashdot Welte gpl ISC embedders BusyBox
% LocalWords:  someone's downloadable subdirectory subdirectories filesystem
% LocalWords:  roadmap README upstream's Ravicher's FOSSology readme CDs iPhone
% LocalWords:  makefiles violator's


%      Tutorial Text for the Detailed Study and Analysis of GPL and LGPL course
%
% Copyright (C) 2003, 2004 Free Software Foundation, Inc.

% License: CC-By-SA-4.0

% The copyright holders hereby grant the freedom to copy, modify, convey,
% Adapt, and/or redistribute this work under the terms of the Creative
% Commons Attribution Share Alike 4.0 International License.

% This text is distributed in the hope that it will be useful, but
% WITHOUT ANY WARRANTY; without even the implied warranty of
% MERCHANTABILITY or FITNESS FOR A PARTICULAR PURPOSE.

% You should have received a copy of the license with this document in
% a file called 'CC-By-SA-4.0.txt'.  If not, please visit
% https://creativecommons.org/licenses/by-sa/4.0/legalcode to receive
% the license text.


\part{Case Studies in GPL Enforcement}

{\parindent 0in
This part is: \\
\begin{tabbing}
Copyright \= \copyright{} 2003, 2004 \= \hspace{.2in} Free Software Foundation, Inc. \\
\end{tabbing}

\vspace{1in}

\begin{center}
Authors of this part are: \\

Bradley M. Kuhn \\
John Sullivan
\vspace{3in}

The copyright holders hereby grant the freedom to copy, modify, convey,
Adapt, and/or redistribute this work under the terms of the Creative Commons
Attribution Share Alike 4.0 International License.  A copy of that license is
available at \verb=https://creativecommons.org/licenses/by-sa/4.0/legalcode=.
\end{center}
}
% =====================================================================
% START OF SECOND DAY SEMINAR SECTION
% =====================================================================

\chapter*{Preface}

This one-day course presents the details of five different GPL
compliance cases handled by FSF's GPL Compliance Laboratory. Each case
offers unique insights into problems that can arise when the terms of
the GPL are not properly followed, and how diplomatic negotiation between
the violator and the copyright holder can yield positive results for
both parties.

Attendees should have successfully completely the course, a ``Detailed
Study and Analysis of the GPL and LGPL,'' as the material from that
course forms the building blocks for this material.

This course is of most interest to lawyers who have clients or
employers that deal with Free Software on a regular basis. However,
technical managers and executives whose businesses use or distribute
Free Software will also find the course very helpful.

\bigskip

These course materials are merely a summary of the highlights of the
course presented. Please be aware that during the actual GPL course, class
discussion supplements this printed curriculum. Simply reading it is
not equivalent to attending the course.

%FIXME-LATER: write these

%\chapter{Not All GPL Enforcement is Created Equal}

%\section{For-Profit Enforcement}

%\section{Community and Non-Profit Enforcement}

\chapter{Overview of Community Enforcement}

The GPL is a Free Software license with legal teeth. Unlike licenses like
the X11-style or various BSD licenses, the GPL (and by extension, the LGPL) is
designed to defend as well as grant freedom. We saw in the last course
that the GPL uses copyright law as a mechanism to grant all the key freedoms
essential in Free Software, but also to ensure that those freedoms
propagate throughout the distribution chain of the software.

\section{Termination Begins Enforcement}

As we have learned, the assurance that Free Software under the GPL remains
Free Software is accomplished through various terms of the GPL: \S 3 ensures
that binaries are always accompanied with source; \S 2 ensures that the
sources are adequate, complete and usable; \S 6 and \S 7 ensure that the
license of the software is always the GPL for everyone, and that no other
legal agreements or licenses trump the GPL. It is \S 4, however, that ensures
that the GPL can be enforced.

Thus, \S 4 is where we begin our discussion of GPL enforcement. This
clause is where the legal teeth of the license are rooted. As a copyright
license, the GPL governs only the activities governed by copyright law ---
copying, modifying and redistributing computer software. Unlike most
copyright licenses, the GPL gives wide grants of permission for engaging with
these activities. Such permissions continue, and all parties may exercise
them until such time as one party violates the terms of the GPL\@. At the
moment of such a violation (i.e., the engaging of copying, modifying or
redistributing in ways not permitted by the GPL) \S 4 is invoked. While other
parties may continue to operate under the GPL, the violating party loses their
rights.

Specifically, \S 4 terminates the violators' rights to continue
engaging in the permissions that are otherwise granted by the GPL\@.
Effectively, their rights revert to the copyright defaults ---
no permission is granted to copy, modify, nor redistribute the work.
Meanwhile, \S 5 points out that if the violator has no rights under
the GPL, they are prohibited by copyright law from engaging in the
activities of copying, modifying and distributing. They have lost
these rights because they have violated the GPL, and no other license
gives them permission to engage in these activities governed by copyright law.

\section{Ongoing Violations}

In conjunction with \S 4's termination of violators' rights, there is
one final industry fact added to the mix: rarely, does one engage in a
single, solitary act of copying, distributing or modifying software.
Almost always, a violator will have legitimately acquired a copy of a
GPL'd program, either making modifications or not, and then begun
distributing that work. For example, the violator may have put the
software in boxes and sold them at stores. Or perhaps the software
was put up for download on the Internet. Regardless of the delivery
mechanism, violators almost always are engaged in {\em ongoing\/}
violation of the GPL\@.

In fact, when we discover a GPL violation that occurred only once --- for
example, a user group who distributed copies of a GNU/Linux system without
source at one meeting --- we rarely pursue it with a high degree of
tenacity. In our minds, such a violation is an educational problem, and
unless the user group becomes a repeat offender (as it turns out, they
never do), we simply forward along a FAQ entry that best explains how user
groups can most easily comply with the GPL, and send them on their merry way.

It is only the cases of {\em ongoing\/} GPL violation that warrant our
active attention. We vehemently pursue those cases where dozens, hundreds
or thousands of customers are receiving software that is out of
compliance, and where the company continually offers for sale (or
distributes gratis as a demo) software distributions that include GPL'd
components out of compliance. Our goal is to maximize the impact of
enforcement and educate industries who are making such a mistake on a
large scale.

In addition, such ongoing violation shows that a particular company is
committed to a GPL'd product line. We are thrilled to learn that someone
is benefiting from Free Software, and we understand that sometimes they
become confused about the rules of the road. Rather than merely
giving us a postmortem to perform on a past mistake, an ongoing violation
gives us an active opportunity to educate a new contributor to the GPL'd
commons about proper procedures to contribute to the community.

Our central goal is not, in fact, to merely clear up a particular violation.
In fact, over time, we hope that our compliance lab will be out of
business. We seek to educate the businesses that engage in commerce
related to GPL'd software to obey the rules of the road and allow them to
operate freely under them. Just as a traffic officer would not revel in
reminding people which side of the road to drive on, so we do not revel in
violations. By contrast, we revel in the successes of educating an
ongoing violator about the GPL so that GPL compliance becomes a second-nature
matter, allowing that company to join the GPL ecosystem as a contributor.

\section{How are Violations Discovered?}

Our enforcement of the GPL is not a fund-raising effort; in fact, FSF's GPL
Compliance Lab runs at a loss (in other words, it is subsided by our
donors). Our violation reports come from volunteers, who have encountered,
in their business or personal life, a device or software product that
appears to contain GPL'd software. These reports are almost always sent
via email to $<$license-violation@fsf.org$>$.

Our first order of business, upon receiving such a report, is to seek
independent confirmation. When possible, we get a copy of the software
product. For example, if it is an offering that is downloadable from a
Web site, we download it and investigate ourselves. When it is not
possible for us to actually get a copy of the software, we ask the
reporter to go through the same process we would use in examining the
software.

By rough estimation, about 95\% of violations at this stage can be
confirmed by simple commands. Almost all violators have merely made an
error and have no nefarious intentions. They have made no attempt to
remove our copyright notices from the software. Thus, given the
third-party binary, {\tt tpb}, usually, a simple command (on a GNU/Linux
system) such as the following will find a Free Software copyright notice
and GPL reference:
\begin{quotation}
{\tt strings tpb | grep Copyright}
\end{quotation}
In other words, it is usually more than trivial to confirm that GPL'd
software is included.

Once we have confirmed that a violation has indeed occurred, we must then
determine whose copyright has been violated. Contrary to popular belief,
FSF does not have the power to enforce the GPL in all cases. Since the GPL
operates under copyright law, the powers of enforcement --- to seek
redress once \S 4 has been invoked --- lie with the copyright holder of
the software. FSF is one of the largest copyright holders in the world of
GPL'd software, but we are by no means the only one. Thus, we sometimes
discover that while GPL'd code is present in the software, there is no
software copyrighted by FSF present.

In cases where FSF does not hold copyright interest in the software, but
we have confirmed a violation, we contact the copyright holders of the
software, and encourage them to enforce the GPL\@. We offer our good offices
to help negotiate compliance on their behalf, and many times, we help as a
third party to settle such GPL violations. However, what we will describe
primarily in this course is FSF's first-hand experience enforcing its own
copyrights and the GPL\@.

\section{First Contact}

The Free Software community is built on a structure of voluntary
cooperation and mutual help. Our community has learned that cooperation
works best when you assume the best of others, and only change policy,
procedures and attitudes when some specific event or occurrence indicates
that a change is necessary. We treat the process of GPL enforcement in
the same way. Our goal is to encourage violators to join the cooperative
community of software sharing, so we want to open our hand in friendship.

Therefore, once we have confirmed a violation, our first assumption is
that the violation is an oversight or otherwise a mistake due to confusion
about the terms of the license. We reach out to the violator and ask them
to work with us in a collaborative way to bring the product into
compliance. We have received the gamut of possible reactions to such
requests, and in this course, we examine four specific examples of such
compliance work.


%%%%%%%%%%%%%%%%%%%%%%%%%%%%%%%%%%%%%%%%%%%%%%%%%%%%%%%%%%%%%%%%%%%%%%%%%%%%%%%
\chapter{Bortez: Modified GCC SDK}

In our first case study, we will consider Bortez, a company that
produces software and hardware toolkits to assist OEM vendors, makers
of consumer electronic devices.

\section{Facts}

One of Bortez's key products is a Software Development Kit (``SDK'')
designed to assist developers building software for a specific class of
consumer electronics devices.

FSF received a report that the SDK may be based on the GNU Compiler
Collection (which is an FSF-copyrighted collection of tools for software
development in C, C++ and other popular languages). FSF investigated the
claim, but was unable to confirm the violation. The violation reporter
was unresponsive to follow-up requests for more information.

Since FSF was unable to confirm the violation, we did not pursue it any
further. Bogus reports do happen, and we do not want to burden companies
with specious GPL violation complaints. FSF shelved the matter until
more evidence was discovered.

FSF was later able to confirm the violation when two additional reports
surfaced from other violation reporters, both of whom had used the SDK
professionally and noticed clear similarities to FSF's GNU GCC\@. FSF's
Compliance Engineer asked the reporters to run standard tests to confirm
the violation, and it was confirmed that Bortez's SDK was indeed a
derivative work of GCC\@. Bortez had ported to Windows and added a number
of features, including support for a specific consumer device chipset and
additional features to aid in the linking process (``LP'') for those
specific devices. FSF explained the rights that the GPL afforded these
customers and pointed out, for example, that Bortez only needed to provide
source to those in possession of the binaries, and that the users may need
to request that source (if \S 3(b) was exercised). The violators
confirmed that such requests were not answered.

FSF brought the matter to the attention of Bortez, who immediately
escalated the matter to their attorneys. After a long negotiation,
Bortez acknowledged that their SDK was indeed a derivative work of
GCC\@. Bortez released most of the source, but some disagreement
occurred over whether LP was a derivative work of GCC\@. After repeated
FSF inquiries, Bortez reaudited the source to discover that FSF's
analysis was correct. Bortez determined that LP included a number of
source files copied from the GCC code-base.

\label{davrik-build-problems}
Once the full software release was made available, FSF asked the violation
reporters if it addressed the problem. Reports came back that the source
did not properly build. FSF asked Bortez to provide better build
instructions with the software, and such build instructions were
incorporated into the next software release.

At FSF's request as well, Bortez informed customers who had previously
purchased the product that the source was now available by announcing
the availability on its Web site and via a customer newsletter.

Bortez did have some concerns regarding patents. They wished to include a
statement with the software release that made sure they were not granting
any patent permission other than what was absolutely required by the GPL\@.
They understood that their patent assertions could not trump any rights
granted by the GPL\@. The following language was negotiated into the release:

\begin{quotation}
Subject to the qualifications stated below, Bortez, on behalf of itself
and its Subsidiaries, agrees not to assert the Claims against you for your
making, use, offer for sale, sale, or importation of the Bortez's GNU
Utilities or derivative works of the Bortez's GNU Utilities
(``Derivatives''), but only to the extent that any such Derivatives are
licensed by you under the terms of the GNU General Public License. The
Claims are the claims of patents that Bortez or its Subsidiaries have
standing to enforce that are directly infringed by the making, use, or
sale of an Bortez Distributed GNU Utilities in the form it was distributed
by Bortez and that do not include any limitation that reads on hardware;
the Claims do not include any additional patent claims held by Bortez that
cover any modifications of, derivative works based on or combinations with
the Bortez's GNU Utilities, even if such a claim is disclosed in the same
patent as a Claim. Subsidiaries are entities that are wholly owned by
Bortez.

This statement does not negate, limit or restrict any rights you already
have under the GNU General Public License version 2.
\end{quotation}

This quelled Bortez's concerns about other patent licensing they sought to
do outside of the GPL'd software, and satisfied FSF's concerns that Bortez
give proper permissions to exercise teachings of patents that were
exercised in their GPL'd software release.

Finally, a GPL Compliance Officer inside Bortez was appointed to take
responsibility for all matters of GPL compliance inside the company.
Darvik is responsible for informing FSF if the position is given to
someone else inside the company, and making sure that FSF has direct
contact with Darvik's Compliance Officer.

\section{Lessons}

This case introduces a number of concepts regarding GPL enforcement.

\begin{enumerate}

\item {\bf Enforcement should not begin until the evidence is confirmed.}
  Most companies who distribute GPL'd software do so in compliance, and at
  times, violation reports are mistaken. Even with extensive efforts in
  GPL education, many users do not fully understand their rights and the
  obligations that companies have. By working through the investigation
  with reporters, the violation can be properly confirmed, and {\bf the
    user of the software can be educated about what to expect with GPL'd
    software}. When users and customers of GPL'd products know their
  rights, what to expect, and how to properly exercise their rights
  (particularly under \S 3(b)), it reduces the chances for user
  frustration and inappropriate community outcry about an alleged GPL
  violation.

\item {\bf GPL compliance requires friendly negotiation and cooperation.}
  Often, attorneys and managers are legitimately surprised to find out
  GPL'd software is included in their company's products. Engineers
  sometimes include GPL'd software without understanding the requirements.
  This does not excuse companies from their obligations under the license,
  but it does mean that care and patience are essential for reaching GPL
  compliance. We want companies to understand that participating and
  benefiting from a collaborative Free Software community is not a burden,
  so we strive to make the process of coming into compliance as smooth as
  possible.

\item {\bf Confirming compliance is a community effort.}  The whole point
  of making sure that software distributors respect the terms of the GPL is to
  allow a thriving software sharing community to benefit and improve the
  work. FSF is not the expert on how a compiler for consumer electronic
  devices should work. We therefore inform the community who originally
  brought the violation to our attention and ask them to assist in
  evaluation and confirmation of the product's compliance. Of course, FSF
  coordinates and oversees the process, but we do not want compliance for
  compliance's sake; rather, we wish to foster a cooperating community of
  development around the Free Software in question, and encourage the
  once-violator to begin participating in that community.

\item {\bf Informing the harmed community is part of compliance.} FSF asks
  violators to make some attempt --- such as via newsletters and the
  company's Web site --- to inform those who already have the products as
  to their rights under the GPL\@. One of the key thrusts of the GPL's \S 1 and
  \S 3 is to {\em make sure the user knows she has these rights\/}. If a
  product was received out of compliance by a customer, she may never
  actually discover that she has such rights. Informing customers, in a
  way that is not burdensome but has a high probability of successfully
  reaching those who would seek to exercise their freedoms, is essential
  to properly remedy the mistake.

\item {\bf Lines between various copyright, patent, and other legal
  mechanisms must be precisely defined and considered.}  The most
  difficult negotiation point of the Bortez case was drafting language
  that simultaneously protected Bortez's patent rights outside of the
  GPL'd source, but was consistent with the implicit patent grant in
  the GPL\@. As we discussed in the first course of this series, there is
  indeed an implicit patent grant with the GPL, thanks to \S 6 and \S 7.
  However, many companies become nervous and wish to make the grant
  explicit to assure themselves that the grant is sufficiently narrow for
  their needs. We understand that there is no reasonable way to determine
  what patent claims read on a company's GPL holdings and which do not, so
  we do not object to general language that explicitly narrows the patent
  grant to only those patents that were, in fact, exercised by the GPL'd
  software as released by the company.

\end{enumerate}

%%%%%%%%%%%%%%%%%%%%%%%%%%%%%%%%%%%%%%%%%%%%%%%%%%%%%%%%%%%%%%%%%%%%%%%%%%%%%%%
\chapter{Bracken: a Minor Violation in a GNU/Linux Distribution}

In this case study, we consider a minor violation made by a company whose
knowledge of the Free Software community and its functions is deep.

\section{The Facts} 

Bracken produces a GNU/Linux operating system product that is sold
primarily to OEM vendors to be placed in appliance devices used for a
single purpose, such as an Internet-browsing-only device. The product
is almost 100\% Free Software, mostly licensed under the GPL and related
Free Software licenses.

FSF found out about this violation through a report first posted on a
  Slashdot\footnote{Slashdot is a popular news and discussion site for
  technical readers.} comment, and then it was brought to our attention again
  by another Free Software copyright holder who had discovered the
  same violation.

Bracken's GNU/Linux product is delivered directly from their Web site.
This allowed FSF engineers to directly download and confirm the
violation quickly. Two primary problems were discovered with the
online distribution:

\begin{itemize}

\item No source code nor offer for source code was provided for a number
  of components for the distributed GNU/Linux system; only binaries were
  available

\item An End User License Agreement (``EULA'') was included that
  contradicted the permissions granted by the GPL\@

\end{itemize}

FSF contacted Bracken and gave them the details of the violation. Bracken
immediately ceased distribution of the product temporarily and set forth
a plan to bring themselves back into compliance. This plan included the
following steps:

\begin{itemize}

\item Bracken attorneys would rewrite the EULA to comply with the GPL and
  would vet the new EULA through FSF before use

\item Bracken engineers would provide source side-by-side with the
  binaries for the GNU/Linux distribution on the site (and on CD's, if
  ever they distributed that way)

\item Bracken attorneys would run an internal seminar for its engineers
  regarding proper GPL compliance to help ensure that such oversights
  regarding source releases would not occur in the future

\item Bracken would resume distribution of the product only after FSF
  formally restored Bracken's distribution rights
\end{itemize}

This case was completed in about a month. FSF approved the new EULA
text. The key portion in the EULA relating to the GPL read as follows:

\begin{quotation}
Many of the Software Programs included in Bracken Software are distributed
under the terms of agreements with Third Parties (``Third Party
Agreements'') which may expand or limit the Licensee's rights to use
certain Software Programs as set forth in [this EULA]. Certain Software
Programs may be licensed (or sublicensed) to Licensee under the GNU
General Public License and other similar license agreements listed in part
in this section which, among other rights, permit the Licensee to copy,
modify and redistribute certain Software Programs, or portions thereof,
and have access to the source code of certain Software Programs, or
portions thereof. In addition, certain Software Programs, or portions
thereof, may be licensed (or sublicensed) to Licensee under terms stricter
than those set forth in [this EULA]. The Licensee must review the
electronic documentation that accompanies certain Software Programs, or
portions thereof, for the applicable Third Party Agreements. To the
extent any Third Party Agreements require that Bracken provide rights to
use, copy or modify a Software Program that are broader than the rights
granted to the Licensee in [this EULA], then such rights shall take
precedence over the rights and restrictions granted in this Agreement
solely for such Software Programs.
\end{quotation}

FSF restored Bracken's distribution rights shortly after the work was
completed as described.

\section{Lessons Learned}

This case was probably the most quickly and easily resolved of all GPL
violations in the history of FSF's Compliance Lab. The ease with which
the problem was resolved shows a number of cultural factors that play a
role in GPL compliance.

\begin{enumerate}

\item {\bf Companies that understand Free Software culture better have an
  easier time with compliance.}  Bracken's products were designed and
  built around the GNU/Linux system and Free Software components. Their
  engineers were deeply familiar with the Free Software ecosystem, and
  their lawyers had seen and reviewed the GPL before. The violation was
  completely an honest mistake. Since the culture inside the company had
  already adapted to the cooperative style of resolution in the Free
  Software world, there was very little work for either party to bring the
  product into compliance.

\item {\bf When people in key positions understand the Free Software
  nature of their software products, compliance concerns are as
  mundane as minor software bugs.}  Even the most functional system or
  structure has its problems, and successful business often depends on
  agile response to the problems that do come up; avoiding problems
  altogether is a pipe dream. Minor GPL violations can and do happen
  even with well-informed redistributors. However, resolution is
  reached quickly when the company --- and in particular, the lawyers,
  managers, and engineers working on the Free Software product lines
  --- have adapted to Free Software culture that the lower-level
  engineer already understood

\item {\bf Legally, distribution must stop when a violation is
  identified.}  In our opinion, Bracken went above and beyond the call of
  duty by ceasing distribution while the violation was being resolved.
  Under GPL \S 4, the redistributor loses the right to distribute the
  software, and thus they are in ongoing violation of copyright law if
  they distribute before rights are restored. It is FSF's policy to
  temporarily allow distribution while compliance negotiations are ongoing
  and only in the most extreme cases (where the other party appears to be
  negotiating in bad faith) does FSF even threaten an injunction on
  copyright grounds. However, Bracken --- as a good Free Software citizen
  --- chose to be on the safe side and do the legally correct thing while
  the violation case was pending. From start to finish, it took less
  than a month to resolve. This lapse in distribution did not, to FSF's
  knowledge, impact Bracken's business in any way.

\item {\bf EULAs are a common area for GPL problems.}  Often, EULAs
  are drafted from boilerplate text that a company uses for all its
  products. Even the most diligent attorneys forget or simply do not
  know that a product contains software licensed under the GPL and other
  Free Software licenses. Drafting a EULA that accounts for such
  licenses is straightforward; the text quoted above works just fine.
  The EULA must be designed so that it does not trump rights and
  permissions already granted by the GPL\@. The EULA must clearly state
  that if there is a conflict between it and the GPL, with regard to GPL'd
  code, the GPL is the overriding license.

\item {\bf Compliance Officers are rarely necessary when companies are
  educated about GPL compliance.}  As we saw in the Bortez case, FSF asks
  that a formal ``GPL Compliance Officer'' be appointed inside a
  previously violating organization to shepherd the organization to a
  cooperative approach to GPL compliance. However, when FSF
  sees that an organization already has such an approach, there is no
  need to request that such an officer be appointed.

\end{enumerate}


%%%%%%%%%%%%%%%%%%%%%%%%%%%%%%%%%%%%%%%%%%%%%%%%%%%%%%%%%%%%%%%%%%%%%%%%%%%%%%%
\chapter{Vigorien: Security, Export Controls, and GPL Compliance}

This case study introduces how concerns of ``security through obscurity''
and regulatory problems can impact GPL compliance matters.

\section{The Facts}

Vigorien distributes a back-up solution product that allows system
administrators to create encrypted backups of file-systems on
Unix-like computers. The product is based on GNU tar, a backup utility
that replaces the standard Unix utility simply called tar, but has
additional features.

Vigorien's backup solution added cryptographic features to GNU tar, and
included a suite of utilities and graphical user interfaces surrounding
GNU tar to make backups convenient.

FSF discovered the violation from a user report, and determined that the
cryptographic features were the only part of the product that constituted
a derivative work of GNU tar; the extraneous utilities merely made
shell calls out to GNU tar. FSF requested that Vigorien come into
compliance with the GPL by releasing the source of GNU tar, with the
cryptographic modifications, to its customers.

Vigorien released the original GNU tar sources, but kept the cryptographic
modifications proprietary. They argued that the security of their system
depending on keeping the software proprietary and that regardless, USA
export restrictions on cryptographic software prohibited such a release.
FSF disputed the first claim, pointing out that Vigorien had only one
option if they did not want to release the source: they would have to
remove GNU tar from the software and not distribute it further. Vigorien
rejected this suggestion, since GNU tar was an integral part of the
product, and the security changes were useless without GNU tar.

Regarding the export control claims, FSF proposed a number of options,
including release of the source from one of Vigorien's divisions overseas
where no such restrictions occurred, but Vigorien argued that the problem
was insoluble because they operated primarily in the USA\@.

The deadlock on the second issue was resolved when those cryptographic
export restrictions were lifted shortly thereafter, and FSF again raised
the matter with Vigorien. At that point, they dropped the first claim and
agreed to release the remaining source module to their customers. They
did so, and the violation was resolved.


\section{Lessons Learned}

\begin{enumerate}

\item {\bf Removing the GPL'd portion of the product is always an
  option.}  Many violators' first response is to simply refuse to
  release the source code as the GPL requires. FSF offers the option to
  simply remove the GPL'd portions from the product and continue along
  without them. Every case where this has been suggested has led to
  the same conclusion. Like Vigorien, the violator argues that the
  product cannot function without the GPL'd components, and they
  cannot effectively replace them.

  Such an outcome is simply further evidence that the combined work in
  question is indeed a derivative work of the original GPL'd component.
  If the other components cannot stand on their own and be useful without
  the GPL'd portions, then one cannot effectively argue that the work as a
  whole is not a derivative of the GPL'd portions.

\item {\bf The whole product is not always covered.}  In this case,
  Vigorien had additional works aggregated. The backup system was a suite
  of utilities, some of which were the GPL and some of which were not. While
  the cryptographic routines were tightly coupled with GNU tar and clearly
  derivative works, the various GUI utilities were separate and
  independent works merely aggregated with the distribution of the
  GNU-tar-based product.


\item {\bf ``Security'' concerns do not exonerate a distributor from GPL
  obligations, and ``security through obscurity'' does not work anyway.}
  The argument that ``this is security software, so it cannot be released
  in source form'' is not a valid defense for explaining why the terms of
  the GPL are ignored. If companies do not want to release source code
  for some reason, then they should not base the work on GPL'd software.
  No external argument for noncompliance can hold weight if the work as
  whole is indeed a derivative work of a GPL'd program.

  The ``security concerns'' argument is often floated as a reason to keep
  software proprietary, but the computer security community has on
  numerous occasions confirmed that such arguments are entirely specious.
  Security experts have found --- since the beginnings of the field of
  cryptography in the ancient world --- that sharing results about systems
  and having such systems withstand peer review and scrutiny builds the
  most secure systems. While full disclosure may help some who wish to
  compromise security, it helps those who want to fix problems even more
  by identifying them early.

\item {\bf External regulatory problems can be difficult to resolve.}
  The GPL, though grounded in copyright law, does not have the power to trump
  regulations like export controls. While Vigorien's ``security
  concerns'' were specious, their export control concerns were not. It is
  indeed a difficult problem that FSF acknowledges. We want compliance
  with the GPL and respect for users' freedoms, but we certainly do not expect
  companies to commit criminal offenses for the sake of compliance. We
  will see more about this issue in our next case study.
\end{enumerate}


%%%%%%%%%%%%%%%%%%%%%%%%%%%%%%%%%%%%%%%%%%%%%%%%%%%%%%%%%%%%%%%%%%%%%%%%%%%%%%%
\chapter{Haxil, Polgara, and Thesulac: Mergers, Upstream Providers and Radio Devices}

This case study considers an ongoing (at the time of writing) violation
that has occurred. By the end of the investigation period, three
companies were involved and many complex issues arose.

\section{The Facts}

Haxil produced a consumer electronics device which included a mini
GNU/Linux distribution to control the device. The device was of interest
to many technically-minded consumers, who purchased the device and very
quickly discovered that Free Software was included without source.
Mailing lists throughout the Free Software community erupted with
complaints about the problem, and FSF quickly investigated.

FSF confirmed that FSF-copyrighted GPL'd software was included. In
addition, the whole distribution included GPL'd works from hundreds of
individual copyright holders, many of whom were, at this point, up in
arms about the violation.

Meanwhile, Haxil was in the midst of being acquired by Polgara. Polgara
was as surprised as everyone else to discover the product was based on
GPL'd software; this fact had not been part of the disclosures made during
acquisition. FSF contacted Haxil, Polgara, and the product managers
who had transitioned into the ``Haxil division'' of the newly-merged
Polgara company. Polgara's General Counsel's office worked with FSF on
the matter.

FSF formed a coalition with the other primary copyright holders
to pursue the enforcement effort on their behalf. FSF communicated
directly with Polgara's representatives to begin working through the
issues on behalf of itself and the Free Software community at large.

Polgara pointed out that the software distribution they used was mostly
contributed by an upstream provider, Thesulac, and Haxil's changes to that
code base were minimal. Polgara negotiated with Thesulac to obtain the
source, although the issue moved very slowly in the channels between
Polgara and Thesulac.

FSF encouraged a round-table meeting so that high bandwidth communication
could occur between FSF, Polgara and Thesulac. Polgara and Thesulac
agreed, and that discussion began. Thesulac provided nearly complete
sources to Polgara, and Polgara made a full software release on their
Web site. At the time of writing, that software still has some build
problems (similar to those that occurred with Bortez, as described in
Section~\ref{davrik-build-problems}). FSF continues to negotiate with
Polgara and Thesulac to resolve these problems, which have a clear path to
a solution and are expected to resolve.

Similar to the Vigorien case, Thesulac has regulatory concerns. In this
case, it is not export controls --- an issue that has since been resolved
--- but radio spectrum regulation. Since this consumer electronic device
contains a software-programmable radio transmitter, regulations in (at
least) the USA and Japan prohibit release of those portions of the code
that operate the device. Since this is a low-level programming issue, the
changes to operate the device are a derivative work of the kernel named
Linux. This situation remains unresolved at the time of writing, although
FSF continues to negotiation with Thesulac and the Linux community
regarding the problem.

\section{Lessons Learned}

\begin{enumerate}

\item {\bf Community outrage, while justified, can often make negotiation
  more difficult.}  FSF has a strong policy never to publicize names of
  GPL violators if they are negotiating in a friendly way and operating in
  good faith toward compliance. Most violations are honest mistakes, and
  FSF sees no reason to publicly admonish violators who genuinely want to
  come into compliance with the GPL and to work hard staying in compliance.

  This case was so public in the Free Software community that both Haxil's
  and Polgara's representatives were nearly shell-shocked by the time FSF
  began negotiations. There was much work required to diffuse the
  situation. We empathize with our community and their outrage about GPL
  violations, but we also want to follow a path that leads expediently
  to compliance. In our experience, public outcry works best as a last
  resort, not the first.

\item {\bf For software companies, GPL compliance belongs on a corporate
  acquisition checklist. }  Polgara was truly amazed that Haxil had used
  GPL'd software in a major new product line but never informed Polgara
  during the acquisition process. While GPL compliance is not a
  particularly difficult matter, it is an additional obligation that comes
  along with the product line. When planning mergers and joint ventures,
  one should include lists of GPL'd components contained in the products
  discussed.

\item {\bf Compliance problems of upstream providers do not excuse a
  violation for the downstream distributor.}  To paraphrase \S 6, upstream
  providers are not responsible for enforcing compliance of their
  downstream, nor are downstream distributors responsible for compliance
  problems of upstream providers. However, engaging in distribution of
  GPL'd works out of compliance is still just that: a compliance problem.
  When FSF carries out enforcement, we are patient and sympathetic when
  the problem appears to be upstream. In fact, we urge the violator to
  point us to the upstream provider so we may talk to them directly. In
  this case, we were happy to begin negotiations with Thesulac. However,
  Polgara still has an obligation to bring their product into compliance,
  regardless of Thesulac's response.

\item {\bf It behooves upstream providers to advise downstream
  distributors about compliance matters.}  FSF has encouraged Thesulac to
  distribute a ``good practices for GPL compliance'' document with their
  product. Polgara added various software components to Thesulac's
  product, and it is conceivable that such additions can introduce
  compliance. In FSF's opinion, Thesulac is in no way legally responsible
  for such a violation introduced by their customer, but it behooves them
  from a marketing standpoint to educate their customers about using the
  product. We can argue whether or not it is your coffee vendor's fault
  if you burn yourself with their product, but (likely) no one on either
  side would dispute the prudence of placing a ``caution: hot'' label on
  the cup.

\item {\bf FSF enforcement often avoids redundant enforcement cases from
  many parties.}  Most Free Software systems have hundreds of copyright
  holders. Some have thousands. FSF is in a unique position as one of
  the largest single copyright holders on GPL'd software and as a
  respected umpire in the community, neutrally enforcing the rules of the
  GPL road. FSF works hard in the community to convince copyright
  holders that consolidating GPL claims through FSF is better for them,
  and more likely to yield positive compliance results.

  A few copyright holders engage in the ``proprietary relicensing''
  business, so they use GPL enforcement as a sales channel for that
  business. FSF, as a community-oriented, not-for-profit organization,
  seeks only to preserve the freedom of Free Software in its enforcement
  efforts. As it turns out, most of the community of copyright holders
  of Free Software want the same thing. Share and share alike is a
  simple rule to follow, and following that rule to FSF's satisfaction
  usually means you are following it to the satisfaction of the entire
  Free Software community.

\end{enumerate}

%%%%%%%%%%%%%%%%%%%%%%%%%%%%%%%%%%%%%%%%%%%%%%%%%%%%%%%%%%%%%%%%%%%%%%%%%%%%%%%
% COMMENT OUT THIS CHAPTER.
% FIXME: is this material moot now that we include the compliance guide?
% Either way, it should be merged into compliance guide.
%\chapter{Good Practices for Compliance}

Generally, from the experience of GPL enforcement, we glean the following
general practices that can help in GPL compliance for organizations that
distribute products based on GPL'd software:

\begin{itemize}

\item Talk to your software engineers and ask them where they got the
  components they use in the products they build. Find out if GPL'd
  components are present.

\item Teach your engineering staff to pay attention to license documents.
  Give them easy-to-follow policies to get approval for using a Free
  Software component.

\item Build a ``Free Software Licensing'' committee that handles requests
  and questions about the GPL and other Free Software licenses.

\item Add ``What parts of your products are under the GPL or other Free
  Software licenses?'' to your checklist of questions to ask when you
  consider mergers, acquisitions, or joint ventures.

\item Encourage your engineers to participate collaboratively with GPL'd
  software development. The more knowledge about the Free Software world
  your organization has, the better equipped it is to deal with this
  rapidly changing field.

\item When someone points out a potential GPL violation in one of your
  products, do not assume the product line is doomed. The GPL is not a virus;
  merely having GPL'd code in one part of a product does not necessarily
  mean that every related product must also be GPL'd. And, even if some
  software needs to be released that was not before, the product will
  surely survive. In FSF's enforcement efforts, we have not yet
  seen a product line die because source was released to customers in
  compliance with the GPL.

\end{itemize}

%%%%%%%%%%%%%%%%%%%%%%%%%%%%%%%%%%%%%%%%%%%%%%%%%%%%%%%%%%%%%%%%%%%%%%%%%%%%%%%
% LocalWords:  proprietarize redistributors sublicense yyyy Gnomovision EULAs
% LocalWords:  Yoyodyne FrontPage improvers Berne copyrightable Stallman's GPLs
% LocalWords:  Lessig Lessig's UCITA pre PDAs CDs reshifts GPL's Gentoo glibc
% LocalWords:  TrollTech administrivia LGPL's MontaVista OpenTV Mitek Arce DVD
% LocalWords:  unprotectable protectable Unfreedonia chipset CodeSourcery Iqtel
% LocalWords:  impermissibly Bateman faire minimis Borland uncopyrightable Mgmt
% LocalWords:  franca downloadable Bortez Bortez's Darvik
% LocalWords:  Slashdot sublicensed Vigorien Vigorien's Haxil Polgara
% LocalWords:  Thesulac Polgara's Haxil's Thesulac's SDK CD's


\appendix

% license-texts.tex                                                  -*- LaTeX -*-
%      Tutorial Text for the Detailed Study and Analysis of GPL and LGPL course
%
% Copyright (C) 1989, 1991, 1999, 2002 Free Software Foundation, Inc.

\part{Full Texts of the GNU GPL and Related Licenses}

In this appendix, we include a full copy of GPLv2, GPLv3, LGPLv2.1,
LGPLv3, and AGPLv3.  These are the most commonly used licenses in the GPL
family of licenses.

\chapter{The GNU General Public License, version 2}

\begin{center}
{\parindent 0in

Version 2, June 1991

Copyright \copyright\ 1989, 1991 Free Software Foundation, Inc.

\bigskip

59 Temple Place - Suite 330, Boston, MA  02111-1307, USA

\bigskip

Everyone is permitted to copy and distribute verbatim copies
of this license document, but changing it is not allowed.
}
\end{center}

\begin{center}
{\bf\large Preamble}
\end{center}


The licenses for most software are designed to take away your freedom
to share and change it. By contrast, the GNU General Public License is
intended to guarantee your freedom to share and change Free
Software---to make sure the software is free for all its users. This
General Public License applies to most of the Free Software
Foundation's software and to any other program whose authors commit to
using it. (Some other Free Software Foundation software is covered by
the GNU Library General Public License instead.) You can apply it to
your programs, too.

When we speak of Free Software, we are referring to freedom, not price.
Our General Public Licenses are designed to make sure that you have the
freedom to distribute copies of Free Software (and charge for this service
if you wish), that you receive source code or can get it if you want it,
that you can change the software or use pieces of it in new Free programs;
and that you know you can do these things.

To protect your rights, we need to make restrictions that forbid anyone to
deny you these rights or to ask you to surrender the rights. These
restrictions translate to certain responsibilities for you if you
distribute copies of the software, or if you modify it.

For example, if you distribute copies of such a program, whether gratis or
for a fee, you must give the recipients all the rights that you have. You
must make sure that they, too, receive or can get the source code. And
you must show them these terms so they know their rights.

We protect your rights with two steps: (1) copyright the software, and (2)
offer you this license which gives you legal permission to copy,
distribute and/or modify the software.

Also, for each author's protection and ours, we want to make certain that
everyone understands that there is no warranty for this Free Software. If
the software is modified by someone else and passed on, we want its
recipients to know that what they have is not the original, so that any
problems introduced by others will not reflect on the original authors'
reputations.

Finally, any Free program is threatened constantly by software patents.
We wish to avoid the danger that redistributors of a Free program will
individually obtain patent licenses, in effect making the program
proprietary. To prevent this, we have made it clear that any patent must
be licensed for everyone's free use or not licensed at all.

The precise terms and conditions for copying, distribution and
modification follow.

\begin{center}
{\Large \sc Terms and Conditions For Copying, Distribution and
  Modification}
\end{center}


\begin{enumerate}

\addtocounter{enumi}{-1}
\item

This License applies to any program or other work which contains a notice
placed by the copyright holder saying it may be distributed under the
terms of this General Public License. The ``Program,'' below, refers to
any such program or work, and a ``work based on the Program'' means either
the Program or any derivative work under copyright law: that is to say, a
work containing the Program or a portion of it, either verbatim or with
modifications and/or translated into another language. (Hereinafter,
translation is included without limitation in the term ``modification.'')
Each licensee is addressed as ``you.''

Activities other than copying, distribution and modification are not
covered by this License; they are outside its scope. The act of
running the Program is not restricted, and the output from the Program
is covered only if its contents constitute a work based on the
Program (independent of having been made by running the Program).
Whether that is true depends on what the Program does.

\item You may copy and distribute verbatim copies of the Program's source
  code as you receive it, in any medium, provided that you conspicuously
  and appropriately publish on each copy an appropriate copyright notice
  and disclaimer of warranty; keep intact all the notices that refer to
  this License and to the absence of any warranty; and give any other
  recipients of the Program a copy of this License along with the Program.

You may charge a fee for the physical act of transferring a copy, and you
may at your option offer warranty protection in exchange for a fee.

\item

You may modify your copy or copies of the Program or any portion
of it, thus forming a work based on the Program, and copy and
distribute such modifications or work under the terms of Section 1
above, provided that you also meet all of these conditions:

\begin{enumerate}

\item

You must cause the modified files to carry prominent notices stating that
you changed the files and the date of any change.

\item

You must cause any work that you distribute or publish, that in
whole or in part contains or is derived from the Program or any
part thereof, to be licensed as a whole at no charge to all third
parties under the terms of this License.

\item
If the modified program normally reads commands interactively
when run, you must cause it, when started running for such
interactive use in the most ordinary way, to print or display an
announcement including an appropriate copyright notice and a
notice that there is no warranty (or else, saying that you provide
a warranty) and that users may redistribute the program under
these conditions, and telling the user how to view a copy of this
License. (Exception: if the Program itself is interactive but
does not normally print such an announcement, your work based on
the Program is not required to print an announcement.)

\end{enumerate}


These requirements apply to the modified work as a whole. If
identifiable sections of that work are not derived from the Program,
and can be reasonably considered independent and separate works in
themselves, then this License, and its terms, do not apply to those
sections when you distribute them as separate works. But when you
distribute the same sections as part of a whole which is a work based
on the Program, the distribution of the whole must be on the terms of
this License, whose permissions for other licensees extend to the
entire whole, and thus to each and every part regardless of who wrote it.

Thus, it is not the intent of this section to claim rights or contest
your rights to work written entirely by you; rather, the intent is to
exercise the right to control the distribution of derivative or
collective works based on the Program.

In addition, mere aggregation of another work not based on the Program
with the Program (or with a work based on the Program) on a volume of
a storage or distribution medium does not bring the other work under
the scope of this License.

\item
You may copy and distribute the Program (or a work based on it,
under Section 2) in object code or executable form under the terms of
Sections 1 and 2 above provided that you also do one of the following:

\begin{enumerate}

\item

Accompany it with the complete corresponding machine-readable
source code, which must be distributed under the terms of Sections
1 and 2 above on a medium customarily used for software interchange; or,

\item

Accompany it with a written offer, valid for at least three
years, to give any third party, for a charge no more than your
cost of physically performing source distribution, a complete
machine-readable copy of the corresponding source code, to be
distributed under the terms of Sections 1 and 2 above on a medium
customarily used for software interchange; or,

\item

Accompany it with the information you received as to the offer
to distribute corresponding source code. (This alternative is
allowed only for noncommercial distribution and only if you
received the program in object code or executable form with such
an offer, in accord with Subsection b above.)

\end{enumerate}


The source code for a work means the preferred form of the work for
making modifications to it. For an executable work, complete source
code means all the source code for all modules it contains, plus any
associated interface definition files, plus the scripts used to
control compilation and installation of the executable. However, as a
special exception, the source code distributed need not include
anything that is normally distributed (in either source or binary
form) with the major components (compiler, kernel, and so on) of the
operating system on which the executable runs, unless that component
itself accompanies the executable.

If distribution of executable or object code is made by offering
access to copy from a designated place, then offering equivalent
access to copy the source code from the same place counts as
distribution of the source code, even though third parties are not
compelled to copy the source along with the object code.

\item
You may not copy, modify, sublicense, or distribute the Program
except as expressly provided under this License. Any attempt
otherwise to copy, modify, sublicense or distribute the Program is
void, and will automatically terminate your rights under this License.
However, parties who have received copies, or rights, from you under
this License will not have their licenses terminated so long as such
parties remain in full compliance.

\item
You are not required to accept this License, since you have not
signed it. However, nothing else grants you permission to modify or
distribute the Program or its derivative works. These actions are
prohibited by law if you do not accept this License. Therefore, by
modifying or distributing the Program (or any work based on the
Program), you indicate your acceptance of this License to do so, and
all its terms and conditions for copying, distributing or modifying
the Program or works based on it.

\item
Each time you redistribute the Program (or any work based on the
Program), the recipient automatically receives a license from the
original licensor to copy, distribute or modify the Program subject to
these terms and conditions. You may not impose any further
restrictions on the recipients' exercise of the rights granted herein.
You are not responsible for enforcing compliance by third parties to
this License.

\item
If, as a consequence of a court judgment or allegation of patent
infringement or for any other reason (not limited to patent issues),
conditions are imposed on you (whether by court order, agreement or
otherwise) that contradict the conditions of this License, they do not
excuse you from the conditions of this License. If you cannot
distribute so as to satisfy simultaneously your obligations under this
License and any other pertinent obligations, then as a consequence you
may not distribute the Program at all. For example, if a patent
license would not permit royalty-free redistribution of the Program by
all those who receive copies directly or indirectly through you, then
the only way you could satisfy both it and this License would be to
refrain entirely from distribution of the Program.

If any portion of this section is held invalid or unenforceable under
any particular circumstance, the balance of the section is intended to
apply and the section as a whole is intended to apply in other
circumstances.

It is not the purpose of this section to induce you to infringe any
patents or other property right claims or to contest validity of any
such claims; this section has the sole purpose of protecting the
integrity of the Free Software distribution system, which is
implemented by public license practices. Many people have made
generous contributions to the wide range of software distributed
through that system in reliance on consistent application of that
system; it is up to the author/donor to decide if he or she is willing
to distribute software through any other system and a licensee cannot
impose that choice.

This section is intended to make thoroughly clear what is believed to
be a consequence of the rest of this License.

\item
If the distribution and/or use of the Program is restricted in
certain countries either by patents or by copyrighted interfaces, the
original copyright holder who places the Program under this License
may add an explicit geographical distribution limitation excluding
those countries, so that distribution is permitted only in or among
countries not thus excluded. In such case, this License incorporates
the limitation as if written in the body of this License.

\item
The Free Software Foundation may publish revised and/or new versions
of the General Public License from time to time. Such new versions will
be similar in spirit to the present version, but may differ in detail to
address new problems or concerns.

Each version is given a distinguishing version number. If the Program
specifies a version number of this License which applies to it and ``any
later version,'' you have the option of following the terms and conditions
either of that version or of any later version published by the Free
Software Foundation. If the Program does not specify a version number of
this License, you may choose any version ever published by the Free Software
Foundation.

\item
If you wish to incorporate parts of the Program into other free
programs whose distribution conditions are different, write to the author
to ask for permission. For software which is copyrighted by the Free
Software Foundation, write to the Free Software Foundation; we sometimes
make exceptions for this. Our decision will be guided by the two goals
of preserving the free status of all derivatives of our Free Software and
of promoting the sharing and reuse of software generally.

\begin{center}
{\Large\sc
No Warranty
}
\end{center}

\item
{\sc Because the program is licensed free of charge, there is no warranty
for the program, to the extent permitted by applicable law. Except when
otherwise stated in writing the copyright holders and/or other parties
provide the program ``as is'' without warranty of any kind, either expressed
or implied, including, but not limited to, the implied warranties of
merchantability and fitness for a particular purpose. The entire risk as
to the quality and performance of the program is with you. Should the
program prove defective, you assume the cost of all necessary servicing,
repair or correction.}

\item {\sc In no event unless required by applicable law or agreed to
    in writing will any copyright holder, or any other party who may
    modify and/or redistribute the program as permitted above, be
    liable to you for damages, including any general, special,
    incidental or consequential damages arising out of the use or
    inability to use the program (including but not limited to loss of
    data or data being rendered inaccurate or losses sustained by you
    or third parties or a failure of the program to operate with any
    other programs), even if such holder or other party has been
    advised of the possibility of such damages.}

\end{enumerate}


\begin{center}
{\Large\sc End of Terms and Conditions}
\end{center}
\vfill

\pagebreak[4]

\section*{Appendix: How to Apply These Terms to Your New Programs}

If you develop a new program, and you want it to be of the greatest
possible use to the public, the best way to achieve this is to make it
Free Software which everyone can redistribute and change under these
terms.

  To do so, attach the following notices to the program. It is safest to
  attach them to the start of each source file to most effectively convey
  the exclusion of warranty; and each file should have at least the
  ``copyright'' line and a pointer to where the full notice is found.

\begin{quote}
one line to give the program's name and a brief idea of what it does. \\
Copyright (C) yyyy  name of author \\

This program is Free Software; you can redistribute it and/or modify
it under the terms of the GNU General Public License as published by
the Free Software Foundation; either version 2 of the License, or
(at your option) any later version.

This program is distributed in the hope that it will be useful,
but WITHOUT ANY WARRANTY; without even the implied warranty of
MERCHANTABILITY or FITNESS FOR A PARTICULAR PURPOSE. See the
GNU General Public License for more details.

You should have received a copy of the GNU General Public License
along with this program; if not, write to the Free Software
Foundation, Inc., 59 Temple Place - Suite 330, Boston, MA  02111-1307, USA.
\end{quote}

Also add information on how to contact you by electronic and paper mail.

If the program is interactive, make it output a short notice like this
when it starts in an interactive mode:

\begin{quote}
Gnomovision version 69, Copyright (C) yyyy  name of author \\
Gnomovision comes with ABSOLUTELY NO WARRANTY; for details type `show w'. \\
This is Free Software, and you are welcome to redistribute it
under certain conditions; type `show c' for details.
\end{quote}


The hypothetical commands {\tt show w} and {\tt show c} should show the
appropriate parts of the General Public License. Of course, the commands
you use may be called something other than {\tt show w} and {\tt show c};
they could even be mouse-clicks or menu items---whatever suits your
program.

You should also get your employer (if you work as a programmer) or your
school, if any, to sign a ``copyright disclaimer'' for the program, if
necessary. Here is a sample; alter the names:

\begin{quote}
Yoyodyne, Inc., hereby disclaims all copyright interest in the program \\
`Gnomovision' (which makes passes at compilers) written by James Hacker. \\

signature of Ty Coon, 1 April 1989 \\
Ty Coon, President of Vice
\end{quote}


This General Public License does not permit incorporating your program
into proprietary programs. If your program is a subroutine library, you
may consider it more useful to permit linking proprietary applications
with the library. If this is what you want to do, use the GNU Library
General Public License instead of this License.


\chapter{The GNU Lesser General Public License, version 2.1}

\begin{center}
{\parindent 0in

Version 2.1, February 1999

Copyright \copyright\ 1991, 1999 Free Software Foundation, Inc.

\bigskip

59 Temple Place - Suite 330, Boston, MA  02111-1307, USA

\bigskip

Everyone is permitted to copy and distribute verbatim copies
of this license document, but changing it is not allowed.

\bigskip

[This is the first released version of the Lesser GPL. It also counts
 as the successor of the GNU Library Public License version 2, hence
 the version number 2.1.]
}

\end{center}

\begin{center}
{\bf\large Preamble}
\end{center}

The licenses for most software are designed to take away your freedom to
share and change it. By contrast, the GNU General Public Licenses are
intended to guarantee your freedom to share and change Free Software---to
make sure the software is free for all its users.

This license, the Lesser General Public License, applies to some specially
designated software packages---typically libraries---of the Free Software
Foundation and other authors who decide to use it. You can use it too,
but we suggest you first think carefully about whether this license or the
ordinary General Public License is the better strategy to use in any
particular case, based on the explanations below.

When we speak of Free Software, we are referring to freedom of use, not
price. Our General Public Licenses are designed to make sure that you
have the freedom to distribute copies of Free Software (and charge for
this service if you wish); that you receive source code or can get it if
you want it; that you can change the software and use pieces of it in new
Free programs; and that you are informed that you can do these things.

To protect your rights, we need to make restrictions that forbid
distributors to deny you these rights or to ask you to surrender these
rights. These restrictions translate to certain responsibilities for you
if you distribute copies of the library or if you modify it.

For example, if you distribute copies of the library, whether gratis or
for a fee, you must give the recipients all the rights that we gave you.
You must make sure that they, too, receive or can get the source code. If
you link other code with the library, you must provide complete object
files to the recipients, so that they can relink them with the library
after making changes to the library and recompiling it. And you must show
them these terms so they know their rights.

We protect your rights with a two-step method: (1) we copyright the
library, and (2) we offer you this license, which gives you legal
permission to copy, distribute and/or modify the library.

To protect each distributor, we want to make it very clear that there is
no warranty for the Free library. Also, if the library is modified by
someone else and passed on, the recipients should know that what they have
is not the original version, so that the original author's reputation will
not be affected by problems that might be introduced by others.

Finally, software patents pose a constant threat to the existence of any
Free program. We wish to make sure that a company cannot effectively
restrict the users of a Free program by obtaining a restrictive license
from a patent holder. Therefore, we insist that any patent license
obtained for a version of the library must be consistent with the full
freedom of use specified in this license.

Most GNU software, including some libraries, is covered by the ordinary
GNU General Public License. This license, the GNU Lesser General Public
License, applies to certain designated libraries, and is quite different
from the ordinary General Public License. We use this license for certain
libraries in order to permit linking those libraries into non-Free
programs.

When a program is linked with a library, whether statically or using a
shared library, the combination of the two is legally speaking a combined
work, a derivative of the original library. The ordinary General Public
License therefore permits such linking only if the entire combination fits
its criteria of freedom. The Lesser General Public License permits more
lax criteria for linking other code with the library.

We call this license the ``Lesser'' General Public License because it does
Less to protect the user's freedom than the ordinary General Public
License. It also provides other Free Software developers Less of an
advantage over competing non-Free programs. These disadvantages are the
reason we use the ordinary General Public License for many libraries.
However, the Lesser license provides advantages in certain special
circumstances.

For example, on rare occasions, there may be a special need to encourage
the widest possible use of a certain library, so that it becomes a
de-facto standard. To achieve this, non-Free programs must be allowed to
use the library. A more frequent case is that a Free library does the
same job as widely used non-Free libraries. In this case, there is little
to gain by limiting the Free library to Free Software only, so we use the
Lesser General Public License.

In other cases, permission to use a particular library in non-Free
programs enables a greater number of people to use a large body of Free
software. For example, permission to use the GNU C Library in non-Free
programs enables many more people to use the whole GNU operating system,
as well as its variant, the GNU/Linux operating system.

Although the Lesser General Public License is Less protective of the
users' freedom, it does ensure that the user of a program that is linked
with the library has the freedom and the wherewithal to run that program
using a modified version of the library.

The precise terms and conditions for copying, distribution and
modification follow. Pay close attention to the difference between a
``work based on the library'' and a ``work that uses the library.''  The
former contains code derived from the library, whereas the latter must be
combined with the library in order to run.

\begin{center}
{\Large \sc GNU Lesser General Public License} \\
{\Large \sc Terms and Conditions For Copying, Distribution and
  Modification}
\end{center}

\begin{enumerate}

\addtocounter{enumi}{-1}

\item

This License Agreement applies to any software library or other program
which contains a notice placed by the copyright holder or other authorized
party saying it may be distributed under the terms of this Lesser General
Public License (also called ``this License''). Each licensee is addressed
as ``you.''

A ``library'' means a collection of software functions and/or data
prepared so as to be conveniently linked with application programs (which
use some of those functions and data) to form executables.

The ``library,'' below, refers to any such software library or work which
has been distributed under these terms. A ``work based on the library''
means either the library or any derivative work under copyright law: that
is to say, a work containing the library or a portion of it, either
verbatim or with modifications and/or translated straightforwardly into
another language. (Hereinafter, translation is included without
limitation in the term ``modification.'')

``Source code'' for a work means the preferred form of the work for making
modifications to it. For a library, complete source code means all the
source code for all modules it contains, plus any associated interface
definition files, plus the scripts used to control compilation and
installation of the library.

Activities other than copying, distribution and modification are not
covered by this License; they are outside its scope. The act of running a
program using the library is not restricted, and output from such a
program is covered only if its contents constitute a work based on the
library (independent of the use of the library in a tool for writing it).
Whether that is true depends on what the library does and what the program
that uses the library does.
  
\item 

You may copy and distribute verbatim copies of the library's complete
source code as you receive it, in any medium, provided that you
conspicuously and appropriately publish on each copy an appropriate
copyright notice and disclaimer of warranty; keep intact all the notices
that refer to this License and to the absence of any warranty; and
distribute a copy of this License along with the library.

You may charge a fee for the physical act of transferring a copy,
and you may at your option offer warranty protection in exchange for a
fee.

\item

You may modify your copy or copies of the library or any portion of it,
thus forming a work based on the library, and copy and distribute such
modifications or work under the terms of Section 1 above, provided that
you also meet all of these conditions:

\begin{enumerate}

  \item

    The modified work must itself be a software library.

  \item

    You must cause the files modified to carry prominent notices stating
    that you changed the files and the date of any change.

  \item

    You must cause the whole of the work to be licensed at no charge to
    all third parties under the terms of this License.

  \item 
    If a facility in the modified library refers to a function or a table
    of data to be supplied by an application program that uses the
    facility, other than as an argument passed when the facility is
    invoked, then you must make a good faith effort to ensure that, in the
    event an application does not supply such function or table, the
    facility still operates, and performs whatever part of its purpose
    remains meaningful.

(For example, a function in a library to compute square roots has a
purpose that is entirely well-defined independent of the application.
Therefore, Subsection 2d requires that any application-supplied function
or table used by this function must be optional: if the application does
not supply it, the square root function must still compute square roots.)
\end{enumerate}

These requirements apply to the modified work as a whole. If identifiable
sections of that work are not derived from the library, and can be
reasonably considered independent and separate works in themselves, then
this License, and its terms, do not apply to those sections when you
distribute them as separate works. But when you distribute the same
sections as part of a whole which is a work based on the library, the
distribution of the whole must be on the terms of this License, whose
permissions for other licensees extend to the entire whole, and thus to
each and every part regardless of who wrote it.

Thus, it is not the intent of this section to claim rights or contest your
rights to work written entirely by you; rather, the intent is to exercise
the right to control the distribution of derivative or collective works
based on the library.

In addition, mere aggregation of another work not based on the library
with the library (or with a work based on the library) on a volume of a
storage or distribution medium does not bring the other work under the
scope of this License.

\item

You may opt to apply the terms of the ordinary GNU General Public License
instead of this License to a given copy of the library. To do this, you
must alter all the notices that refer to this License, so that they refer
to the ordinary GNU General Public License version 2, instead of to this
License. (If a newer version than version 2 of the ordinary GNU General
Public License has appeared, then you can specify that version instead if
you wish.)  Do not make any other change in these notices.

Once this change is made in a given copy, it is irreversible for that
copy, so the ordinary GNU General Public License applies to all subsequent
copies and derivative works made from that copy.

This option is useful when you wish to copy part of the code of the
library into a program that is not a library.

\item

You may copy and distribute the library (or a portion or derivative of it,
under Section 2) in object code or executable form under the terms of
Sections 1 and 2 above provided that you accompany it with the complete
corresponding machine-readable source code, which must be distributed
under the terms of Sections 1 and 2 above on a medium customarily used for
software interchange.

If distribution of object code is made by offering access to copy from a
designated place, then offering equivalent access to copy the source code
from the same place satisfies the requirement to distribute the source
code, even though third parties are not compelled to copy the source along
with the object code.

\item

A program that contains no derivative of any portion of the library, but
is designed to work with the library by being compiled or linked with it,
is called a ``work that uses the library.''  Such a work, in isolation, is
not a derivative work of the library, and therefore falls outside the
scope of this License.

However, linking a ``work that uses the library'' with the library creates
an executable that is a derivative of the library (because it contains
portions of the library), rather than a ``work that uses the library.''
The executable is therefore covered by this License. Section 6 states
terms for distribution of such executables.

When a ``work that uses the library'' uses material from a header file
that is part of the library, the object code for the work may be a
derivative work of the library even though the source code is not.
Whether this is true is especially significant if the work can be linked
without the library, or if the work is itself a library. The threshold
for this to be true is not precisely defined by law.

If such an object file uses only numerical parameters, data structure
layouts and accessors, and small macros and small inline functions (ten
lines or less in length), then the use of the object file is unrestricted,
regardless of whether it is legally a derivative work. (Executables
containing this object code plus portions of the library will still fall
under Section 6.)

Otherwise, if the work is a derivative of the library, you may distribute
the object code for the work under the terms of Section 6. Any
executables containing that work also fall under Section 6, whether or not
they are linked directly with the library itself.

\item

As an exception to the Sections above, you may also combine or link a
``work that uses the library'' with the library to produce a work
containing portions of the library, and distribute that work under terms
of your choice, provided that the terms permit modification of the work
for the customer's own use and reverse engineering for debugging such
modifications.

You must give prominent notice with each copy of the work that the library
is used in it and that the library and its use are covered by this
License. You must supply a copy of this License. If the work during
execution displays copyright notices, you must include the copyright
notice for the library among them, as well as a reference directing the
user to the copy of this License. Also, you must do one of these things:

\begin{enumerate}

  \item

    Accompany the work with the complete corresponding machine-readable
    source code for the library including whatever changes were used in
    the work (which must be distributed under Sections 1 and 2 above);
    and, if the work is an executable linked with the library, with the
    complete machine-readable ``work that uses the library,'' as object
    code and/or source code, so that the user can modify the library and
    then relink to produce a modified executable containing the modified
    library. (It is understood that the user who changes the contents of
    definitions files in the library will not necessarily be able to
    recompile the application to use the modified definitions.)

  \item

    Use a suitable shared library mechanism for linking with the library.
    A suitable mechanism is one that (1) uses at run time a copy of the
    library already present on the user's computer system, rather than
    copying library functions into the executable, and (2) will operate
    properly with a modified version of the library, if the user installs
    one, as long as the modified version is interface-compatible with the
    version that the work was made with.

  \item

    Accompany the work with a written offer, valid for at least three
    years, to give the same user the materials specified in Subsection 6a,
    above, for a charge no more than the cost of performing this
    distribution.

  \item

    If distribution of the work is made by offering access to copy from a
    designated place, offer equivalent access to copy the above specified
    materials from the same place.

  \item

    Verify that the user has already received a copy of these materials or
    that you have already sent this user a copy.
\end{enumerate}

For an executable, the required form of the ``work that uses the library''
must include any data and utility programs needed for reproducing the
executable from it. However, as a special exception, the materials to be
distributed need not include anything that is normally distributed (in
either source or binary form) with the major components (compiler, kernel,
and so on) of the operating system on which the executable runs, unless
that component itself accompanies the executable.

It may happen that this requirement contradicts the license restrictions
of other proprietary libraries that do not normally accompany the
operating system. Such a contradiction means you cannot use both them and
the library together in an executable that you distribute.

\item

You may place library facilities that are a work based on the library
side-by-side in a single library together with other library facilities
not covered by this License, and distribute such a combined library,
provided that the separate distribution of the work based on the library
and of the other library facilities is otherwise permitted, and provided
that you do these two things:

\begin{enumerate}

   \item

     Accompany the combined library with a copy of the same work based on
     the library, uncombined with any other library facilities. This must
     be distributed under the terms of the Sections above.

   \item

     Give prominent notice with the combined library of the fact that part
     of it is a work based on the library, and explaining where to find
     the accompanying uncombined form of the same work.
\end{enumerate}

\item

  You may not copy, modify, sublicense, link with, or distribute the
  library except as expressly provided under this License. Any attempt
  otherwise to copy, modify, sublicense, link with, or distribute the
  library is void, and will automatically terminate your rights under this
  License. However, parties who have received copies, or rights, from you
  under this License will not have their licenses terminated so long as
  such parties remain in full compliance.

\item  

  You are not required to accept this License, since you have not signed
  it. However, nothing else grants you permission to modify or distribute
  the library or its derivative works. These actions are prohibited by
  law if you do not accept this License. Therefore, by modifying or
  distributing the library (or any work based on the library), you
  indicate your acceptance of this License to do so, and all its terms and
  conditions for copying, distributing or modifying the library or works
  based on it.

\item

  Each time you redistribute the library (or any work based on the
  library), the recipient automatically receives a license from the
  original licensor to copy, distribute, link with or modify the library
  subject to these terms and conditions. You may not impose any further
  restrictions on the recipients' exercise of the rights granted herein.
  You are not responsible for enforcing compliance by third parties with
  this License.

\item

  If, as a consequence of a court judgment or allegation of patent
  infringement or for any other reason (not limited to patent issues),
  conditions are imposed on you (whether by court order, agreement or
  otherwise) that contradict the conditions of this License, they do not
  excuse you from the conditions of this License. If you cannot
  distribute so as to satisfy simultaneously your obligations under this
  License and any other pertinent obligations, then as a consequence you
  may not distribute the library at all. For example, if a patent license
  would not permit royalty-free redistribution of the library by all those
  who receive copies directly or indirectly through you, then the only way
  you could satisfy both it and this License would be to refrain entirely
  from distribution of the library.

  If any portion of this section is held invalid or unenforceable under
  any particular circumstance, the balance of the section is intended to
  apply, and the section as a whole is intended to apply in other
  circumstances.

  It is not the purpose of this section to induce you to infringe any
  patents or other property right claims or to contest validity of any
  such claims; this section has the sole purpose of protecting the
  integrity of the Free Software distribution system which is implemented
  by public license practices. Many people have made generous
  contributions to the wide range of software distributed through that
  system in reliance on consistent application of that system; it is up to
  the author/donor to decide if he or she is willing to distribute
  software through any other system and a licensee cannot impose that
  choice.

  This section is intended to make thoroughly clear what is believed to be
  a consequence of the rest of this License.


% \pagebreak[4]


\item 

  If the distribution and/or use of the library is restricted in certain
  countries either by patents or by copyrighted interfaces, the original
  copyright holder who places the library under this License may add an
  explicit geographical distribution limitation excluding those countries,
  so that distribution is permitted only in or among countries not thus
  excluded. In such case, this License incorporates the limitation as if
  written in the body of this License.

\item 

  The Free Software Foundation may publish revised and/or new versions of
  the Lesser General Public License from time to time. Such new versions
  will be similar in spirit to the present version, but may differ in
  detail to address new problems or concerns.

  Each version is given a distinguishing version number. If the library
  specifies a version number of this License which applies to it and ``any
  later version,'' you have the option of following the terms and
  conditions either of that version or of any later version published by
  the Free Software Foundation. If the library does not specify a license
  version number, you may choose any version ever published by the Free
  Software Foundation.


\item
  

  If you wish to incorporate parts of the library into other Free programs
  whose distribution conditions are incompatible with these, write to the
  author to ask for permission. For software which is copyrighted by the
  Free Software Foundation, write to the Free Software Foundation; we
  sometimes make exceptions for this. Our decision will be guided by the
  two goals of preserving the Free status of all derivatives of our Free
  software and of promoting the sharing and reuse of software generally.


\begin{center}
{\Large\sc
No Warranty
}
\end{center}

\item

{\sc Because the library is licensed free of charge, there is no
warranty for the library, to the extent permitted by applicable law.
Except when otherwise stated in writing the copyright holders and/or
other parties provide the library ``as is'' without warranty of any
kind, either expressed or implied, including, but not limited to, the
implied warranties of merchantability and fitness for a particular
purpose. The entire risk as to the quality and performance of the
library is with you. should the library prove defective, you assume
the cost of all necessary servicing, repair or correction.}

% \pagebreak[4]

\item

{\sc In no event unless required by applicable law or agreed to in writing
  will any copyright holder, or any other party who may modify and/or
  redistribute the library as permitted above, be liable to you for
  damages, including any general, special, incidental or consequential
  damages arising out of the use or inability to use the library
  (including but not limited to loss of data or data being rendered
  inaccurate or losses sustained by you or third parties or a failure of
  the library to operate with any other software), even if such holder or
  other party has been advised of the possibility of such damages.}

\end{enumerate}

\begin{center}
{\Large\sc End of Terms and Conditions}
\end{center}
\vfill

\pagebreak[4]

\section*{How to Apply These Terms to Your New Libraries}
           
If you develop a new library, and you want it to be of the greatest
possible use to the public, we recommend making it Free Software that
everyone can redistribute and change. You can do so by permitting
redistribution under these terms (or, alternatively, under the terms of
the ordinary General Public License).

To apply these terms, attach the following notices to the library. It is
safest to attach them to the start of each source file to most effectively
convey the exclusion of warranty; and each file should have at least the
``copyright'' line and a pointer to where the full notice is found.

\begin{quote}
one line to give the library's name and a brief idea of what it does. \\
Copyright (C) year  name of author \\

This library is Free Software; you can redistribute it and/or modify it
under the terms of the GNU Lesser General Public License as published by
the Free Software Foundation; either version 2.1 of the License, or (at
your option) any later version.

This library is distributed in the hope that it will be useful, but
WITHOUT ANY WARRANTY; without even the implied warranty of MERCHANTABILITY
or FITNESS FOR A PARTICULAR PURPOSE. See the GNU Lesser General Public
License for more details.

You should have received a copy of the GNU Lesser General Public License
along with this library; if not, write to the Free Software Foundation,
Inc., 59 Temple Place, Suite 330, Boston, MA 02111-1307 USA
\end{quote}

Also add information on how to contact you by electronic and paper mail.

You should also get your employer (if you work as a programmer) or your
school, if any, to sign a ``copyright disclaimer'' for the library, if
necessary. Here is a sample; alter the names:

\begin{quote}
Yoyodyne, Inc., hereby disclaims all copyright interest in the program \\
`Gnomovision' (which makes passes at compilers) written by James Hacker. \\

signature of Ty Coon, 1 April 1990 \\
Ty Coon, President of Vice
\end{quote}

\chapter{The GNU General Public License, version 3}
\begin{center}
{\parindent 0in

Version 3, 29 June 2007

Copyright \copyright\  2007 Free Software Foundation, Inc. \texttt{http://fsf.org/}

\bigskip
Everyone is permitted to copy and distribute verbatim copies of this

license document, but changing it is not allowed.}

\end{center}

\begin{center}
{\bf\large Preamble}
\end{center}

The GNU General Public License is a free, copyleft license for
software and other kinds of works.

The licenses for most software and other practical works are designed
to take away your freedom to share and change the works.  By contrast,
the GNU General Public License is intended to guarantee your freedom to
share and change all versions of a program--to make sure it remains free
software for all its users.  We, the Free Software Foundation, use the
GNU General Public License for most of our software; it applies also to
any other work released this way by its authors.  You can apply it to
your programs, too.

When we speak of free software, we are referring to freedom, not
price.  Our General Public Licenses are designed to make sure that you
have the freedom to distribute copies of free software (and charge for
them if you wish), that you receive source code or can get it if you
want it, that you can change the software or use pieces of it in new
free programs, and that you know you can do these things.

To protect your rights, we need to prevent others from denying you
these rights or asking you to surrender the rights.  Therefore, you have
certain responsibilities if you distribute copies of the software, or if
you modify it: responsibilities to respect the freedom of others.

For example, if you distribute copies of such a program, whether
gratis or for a fee, you must pass on to the recipients the same
freedoms that you received.  You must make sure that they, too, receive
or can get the source code.  And you must show them these terms so they
know their rights.

Developers that use the GNU GPL protect your rights with two steps:
(1) assert copyright on the software, and (2) offer you this License
giving you legal permission to copy, distribute and/or modify it.

For the developers' and authors' protection, the GPL clearly explains
that there is no warranty for this free software.  For both users' and
authors' sake, the GPL requires that modified versions be marked as
changed, so that their problems will not be attributed erroneously to
authors of previous versions.

Some devices are designed to deny users access to install or run
modified versions of the software inside them, although the manufacturer
can do so.  This is fundamentally incompatible with the aim of
protecting users' freedom to change the software.  The systematic
pattern of such abuse occurs in the area of products for individuals to
use, which is precisely where it is most unacceptable.  Therefore, we
have designed this version of the GPL to prohibit the practice for those
products.  If such problems arise substantially in other domains, we
stand ready to extend this provision to those domains in future versions
of the GPL, as needed to protect the freedom of users.

Finally, every program is threatened constantly by software patents.
States should not allow patents to restrict development and use of
software on general-purpose computers, but in those that do, we wish to
avoid the special danger that patents applied to a free program could
make it effectively proprietary.  To prevent this, the GPL assures that
patents cannot be used to render the program non-free.

The precise terms and conditions for copying, distribution and
modification follow.

\begin{center}
{\Large \sc Terms and Conditions}
\end{center}


\begin{enumerate}

\addtocounter{enumi}{-1}

\item Definitions.

``This License'' refers to version 3 of the GNU General Public License.

``Copyright'' also means copyright-like laws that apply to other kinds of
works, such as semiconductor masks.

``The Program'' refers to any copyrightable work licensed under this
License.  Each licensee is addressed as ``you''.  ``Licensees'' and
``recipients'' may be individuals or organizations.

To ``modify'' a work means to copy from or adapt all or part of the work
in a fashion requiring copyright permission, other than the making of an
exact copy.  The resulting work is called a ``modified version'' of the
earlier work or a work ``based on'' the earlier work.

A ``covered work'' means either the unmodified Program or a work based
on the Program.

To ``propagate'' a work means to do anything with it that, without
permission, would make you directly or secondarily liable for
infringement under applicable copyright law, except executing it on a
computer or modifying a private copy.  Propagation includes copying,
distribution (with or without modification), making available to the
public, and in some countries other activities as well.

To ``convey'' a work means any kind of propagation that enables other
parties to make or receive copies.  Mere interaction with a user through
a computer network, with no transfer of a copy, is not conveying.

An interactive user interface displays ``Appropriate Legal Notices''
to the extent that it includes a convenient and prominently visible
feature that (1) displays an appropriate copyright notice, and (2)
tells the user that there is no warranty for the work (except to the
extent that warranties are provided), that licensees may convey the
work under this License, and how to view a copy of this License.  If
the interface presents a list of user commands or options, such as a
menu, a prominent item in the list meets this criterion.

\item Source Code.

The ``source code'' for a work means the preferred form of the work
for making modifications to it.  ``Object code'' means any non-source
form of a work.

A ``Standard Interface'' means an interface that either is an official
standard defined by a recognized standards body, or, in the case of
interfaces specified for a particular programming language, one that
is widely used among developers working in that language.

The ``System Libraries'' of an executable work include anything, other
than the work as a whole, that (a) is included in the normal form of
packaging a Major Component, but which is not part of that Major
Component, and (b) serves only to enable use of the work with that
Major Component, or to implement a Standard Interface for which an
implementation is available to the public in source code form.  A
``Major Component'', in this context, means a major essential component
(kernel, window system, and so on) of the specific operating system
(if any) on which the executable work runs, or a compiler used to
produce the work, or an object code interpreter used to run it.

The ``Corresponding Source'' for a work in object code form means all
the source code needed to generate, install, and (for an executable
work) run the object code and to modify the work, including scripts to
control those activities.  However, it does not include the work's
System Libraries, or general-purpose tools or generally available free
programs which are used unmodified in performing those activities but
which are not part of the work.  For example, Corresponding Source
includes interface definition files associated with source files for
the work, and the source code for shared libraries and dynamically
linked subprograms that the work is specifically designed to require,
such as by intimate data communication or control flow between those
subprograms and other parts of the work.

The Corresponding Source need not include anything that users
can regenerate automatically from other parts of the Corresponding
Source.

The Corresponding Source for a work in source code form is that
same work.

\item Basic Permissions.

All rights granted under this License are granted for the term of
copyright on the Program, and are irrevocable provided the stated
conditions are met.  This License explicitly affirms your unlimited
permission to run the unmodified Program.  The output from running a
covered work is covered by this License only if the output, given its
content, constitutes a covered work.  This License acknowledges your
rights of fair use or other equivalent, as provided by copyright law.

You may make, run and propagate covered works that you do not
convey, without conditions so long as your license otherwise remains
in force.  You may convey covered works to others for the sole purpose
of having them make modifications exclusively for you, or provide you
with facilities for running those works, provided that you comply with
the terms of this License in conveying all material for which you do
not control copyright.  Those thus making or running the covered works
for you must do so exclusively on your behalf, under your direction
and control, on terms that prohibit them from making any copies of
your copyrighted material outside their relationship with you.

Conveying under any other circumstances is permitted solely under
the conditions stated below.  Sublicensing is not allowed; section 10
makes it unnecessary.

\item Protecting Users' Legal Rights From Anti-Circumvention Law.

No covered work shall be deemed part of an effective technological
measure under any applicable law fulfilling obligations under article
11 of the WIPO copyright treaty adopted on 20 December 1996, or
similar laws prohibiting or restricting circumvention of such
measures.

When you convey a covered work, you waive any legal power to forbid
circumvention of technological measures to the extent such circumvention
is effected by exercising rights under this License with respect to
the covered work, and you disclaim any intention to limit operation or
modification of the work as a means of enforcing, against the work's
users, your or third parties' legal rights to forbid circumvention of
technological measures.

\item Conveying Verbatim Copies.

You may convey verbatim copies of the Program's source code as you
receive it, in any medium, provided that you conspicuously and
appropriately publish on each copy an appropriate copyright notice;
keep intact all notices stating that this License and any
non-permissive terms added in accord with section 7 apply to the code;
keep intact all notices of the absence of any warranty; and give all
recipients a copy of this License along with the Program.

You may charge any price or no price for each copy that you convey,
and you may offer support or warranty protection for a fee.

\item Conveying Modified Source Versions.

You may convey a work based on the Program, or the modifications to
produce it from the Program, in the form of source code under the
terms of section 4, provided that you also meet all of these conditions:
  \begin{enumerate}
  \item The work must carry prominent notices stating that you modified
  it, and giving a relevant date.

  \item The work must carry prominent notices stating that it is
  released under this License and any conditions added under section
  7.  This requirement modifies the requirement in section 4 to
  ``keep intact all notices''.

  \item You must license the entire work, as a whole, under this
  License to anyone who comes into possession of a copy.  This
  License will therefore apply, along with any applicable section 7
  additional terms, to the whole of the work, and all its parts,
  regardless of how they are packaged.  This License gives no
  permission to license the work in any other way, but it does not
  invalidate such permission if you have separately received it.

  \item If the work has interactive user interfaces, each must display
  Appropriate Legal Notices; however, if the Program has interactive
  interfaces that do not display Appropriate Legal Notices, your
  work need not make them do so.
\end{enumerate}
A compilation of a covered work with other separate and independent
works, which are not by their nature extensions of the covered work,
and which are not combined with it such as to form a larger program,
in or on a volume of a storage or distribution medium, is called an
``aggregate'' if the compilation and its resulting copyright are not
used to limit the access or legal rights of the compilation's users
beyond what the individual works permit.  Inclusion of a covered work
in an aggregate does not cause this License to apply to the other
parts of the aggregate.

\item Conveying Non-Source Forms.

You may convey a covered work in object code form under the terms
of sections 4 and 5, provided that you also convey the
machine-readable Corresponding Source under the terms of this License,
in one of these ways:
  \begin{enumerate}
  \item Convey the object code in, or embodied in, a physical product
  (including a physical distribution medium), accompanied by the
  Corresponding Source fixed on a durable physical medium
  customarily used for software interchange.

  \item Convey the object code in, or embodied in, a physical product
  (including a physical distribution medium), accompanied by a
  written offer, valid for at least three years and valid for as
  long as you offer spare parts or customer support for that product
  model, to give anyone who possesses the object code either (1) a
  copy of the Corresponding Source for all the software in the
  product that is covered by this License, on a durable physical
  medium customarily used for software interchange, for a price no
  more than your reasonable cost of physically performing this
  conveying of source, or (2) access to copy the
  Corresponding Source from a network server at no charge.

  \item Convey individual copies of the object code with a copy of the
  written offer to provide the Corresponding Source.  This
  alternative is allowed only occasionally and noncommercially, and
  only if you received the object code with such an offer, in accord
  with subsection 6b.

  \item Convey the object code by offering access from a designated
  place (gratis or for a charge), and offer equivalent access to the
  Corresponding Source in the same way through the same place at no
  further charge.  You need not require recipients to copy the
  Corresponding Source along with the object code.  If the place to
  copy the object code is a network server, the Corresponding Source
  may be on a different server (operated by you or a third party)
  that supports equivalent copying facilities, provided you maintain
  clear directions next to the object code saying where to find the
  Corresponding Source.  Regardless of what server hosts the
  Corresponding Source, you remain obligated to ensure that it is
  available for as long as needed to satisfy these requirements.

  \item Convey the object code using peer-to-peer transmission, provided
  you inform other peers where the object code and Corresponding
  Source of the work are being offered to the general public at no
  charge under subsection 6d.
  \end{enumerate}

A separable portion of the object code, whose source code is excluded
from the Corresponding Source as a System Library, need not be
included in conveying the object code work.

A ``User Product'' is either (1) a ``consumer product'', which means any
tangible personal property which is normally used for personal, family,
or household purposes, or (2) anything designed or sold for incorporation
into a dwelling.  In determining whether a product is a consumer product,
doubtful cases shall be resolved in favor of coverage.  For a particular
product received by a particular user, ``normally used'' refers to a
typical or common use of that class of product, regardless of the status
of the particular user or of the way in which the particular user
actually uses, or expects or is expected to use, the product.  A product
is a consumer product regardless of whether the product has substantial
commercial, industrial or non-consumer uses, unless such uses represent
the only significant mode of use of the product.

``Installation Information'' for a User Product means any methods,
procedures, authorization keys, or other information required to install
and execute modified versions of a covered work in that User Product from
a modified version of its Corresponding Source.  The information must
suffice to ensure that the continued functioning of the modified object
code is in no case prevented or interfered with solely because
modification has been made.

If you convey an object code work under this section in, or with, or
specifically for use in, a User Product, and the conveying occurs as
part of a transaction in which the right of possession and use of the
User Product is transferred to the recipient in perpetuity or for a
fixed term (regardless of how the transaction is characterized), the
Corresponding Source conveyed under this section must be accompanied
by the Installation Information.  But this requirement does not apply
if neither you nor any third party retains the ability to install
modified object code on the User Product (for example, the work has
been installed in ROM).

The requirement to provide Installation Information does not include a
requirement to continue to provide support service, warranty, or updates
for a work that has been modified or installed by the recipient, or for
the User Product in which it has been modified or installed.  Access to a
network may be denied when the modification itself materially and
adversely affects the operation of the network or violates the rules and
protocols for communication across the network.

Corresponding Source conveyed, and Installation Information provided,
in accord with this section must be in a format that is publicly
documented (and with an implementation available to the public in
source code form), and must require no special password or key for
unpacking, reading or copying.

\item Additional Terms.

``Additional permissions'' are terms that supplement the terms of this
License by making exceptions from one or more of its conditions.
Additional permissions that are applicable to the entire Program shall
be treated as though they were included in this License, to the extent
that they are valid under applicable law.  If additional permissions
apply only to part of the Program, that part may be used separately
under those permissions, but the entire Program remains governed by
this License without regard to the additional permissions.

When you convey a copy of a covered work, you may at your option
remove any additional permissions from that copy, or from any part of
it.  (Additional permissions may be written to require their own
removal in certain cases when you modify the work.)  You may place
additional permissions on material, added by you to a covered work,
for which you have or can give appropriate copyright permission.

Notwithstanding any other provision of this License, for material you
add to a covered work, you may (if authorized by the copyright holders of
that material) supplement the terms of this License with terms:
  \begin{enumerate}
  \item Disclaiming warranty or limiting liability differently from the
  terms of sections 15 and 16 of this License; or

  \item Requiring preservation of specified reasonable legal notices or
  author attributions in that material or in the Appropriate Legal
  Notices displayed by works containing it; or

  \item Prohibiting misrepresentation of the origin of that material, or
  requiring that modified versions of such material be marked in
  reasonable ways as different from the original version; or

  \item Limiting the use for publicity purposes of names of licensors or
  authors of the material; or

  \item Declining to grant rights under trademark law for use of some
  trade names, trademarks, or service marks; or

  \item Requiring indemnification of licensors and authors of that
  material by anyone who conveys the material (or modified versions of
  it) with contractual assumptions of liability to the recipient, for
  any liability that these contractual assumptions directly impose on
  those licensors and authors.
  \end{enumerate}

All other non-permissive additional terms are considered ``further
restrictions'' within the meaning of section 10.  If the Program as you
received it, or any part of it, contains a notice stating that it is
governed by this License along with a term that is a further
restriction, you may remove that term.  If a license document contains
a further restriction but permits relicensing or conveying under this
License, you may add to a covered work material governed by the terms
of that license document, provided that the further restriction does
not survive such relicensing or conveying.

If you add terms to a covered work in accord with this section, you
must place, in the relevant source files, a statement of the
additional terms that apply to those files, or a notice indicating
where to find the applicable terms.

Additional terms, permissive or non-permissive, may be stated in the
form of a separately written license, or stated as exceptions;
the above requirements apply either way.

\item Termination.

You may not propagate or modify a covered work except as expressly
provided under this License.  Any attempt otherwise to propagate or
modify it is void, and will automatically terminate your rights under
this License (including any patent licenses granted under the third
paragraph of section 11).

However, if you cease all violation of this License, then your
license from a particular copyright holder is reinstated (a)
provisionally, unless and until the copyright holder explicitly and
finally terminates your license, and (b) permanently, if the copyright
holder fails to notify you of the violation by some reasonable means
prior to 60 days after the cessation.

Moreover, your license from a particular copyright holder is
reinstated permanently if the copyright holder notifies you of the
violation by some reasonable means, this is the first time you have
received notice of violation of this License (for any work) from that
copyright holder, and you cure the violation prior to 30 days after
your receipt of the notice.

Termination of your rights under this section does not terminate the
licenses of parties who have received copies or rights from you under
this License.  If your rights have been terminated and not permanently
reinstated, you do not qualify to receive new licenses for the same
material under section 10.

\item Acceptance Not Required for Having Copies.

You are not required to accept this License in order to receive or
run a copy of the Program.  Ancillary propagation of a covered work
occurring solely as a consequence of using peer-to-peer transmission
to receive a copy likewise does not require acceptance.  However,
nothing other than this License grants you permission to propagate or
modify any covered work.  These actions infringe copyright if you do
not accept this License.  Therefore, by modifying or propagating a
covered work, you indicate your acceptance of this License to do so.

\item Automatic Licensing of Downstream Recipients.

Each time you convey a covered work, the recipient automatically
receives a license from the original licensors, to run, modify and
propagate that work, subject to this License.  You are not responsible
for enforcing compliance by third parties with this License.

An ``entity transaction'' is a transaction transferring control of an
organization, or substantially all assets of one, or subdividing an
organization, or merging organizations.  If propagation of a covered
work results from an entity transaction, each party to that
transaction who receives a copy of the work also receives whatever
licenses to the work the party's predecessor in interest had or could
give under the previous paragraph, plus a right to possession of the
Corresponding Source of the work from the predecessor in interest, if
the predecessor has it or can get it with reasonable efforts.

You may not impose any further restrictions on the exercise of the
rights granted or affirmed under this License.  For example, you may
not impose a license fee, royalty, or other charge for exercise of
rights granted under this License, and you may not initiate litigation
(including a cross-claim or counterclaim in a lawsuit) alleging that
any patent claim is infringed by making, using, selling, offering for
sale, or importing the Program or any portion of it.

\item Patents.

A ``contributor'' is a copyright holder who authorizes use under this
License of the Program or a work on which the Program is based.  The
work thus licensed is called the contributor's ``contributor version''.

A contributor's ``essential patent claims'' are all patent claims
owned or controlled by the contributor, whether already acquired or
hereafter acquired, that would be infringed by some manner, permitted
by this License, of making, using, or selling its contributor version,
but do not include claims that would be infringed only as a
consequence of further modification of the contributor version.  For
purposes of this definition, ``control'' includes the right to grant
patent sublicenses in a manner consistent with the requirements of
this License.

Each contributor grants you a non-exclusive, worldwide, royalty-free
patent license under the contributor's essential patent claims, to
make, use, sell, offer for sale, import and otherwise run, modify and
propagate the contents of its contributor version.

In the following three paragraphs, a ``patent license'' is any express
agreement or commitment, however denominated, not to enforce a patent
(such as an express permission to practice a patent or covenant not to
sue for patent infringement).  To ``grant'' such a patent license to a
party means to make such an agreement or commitment not to enforce a
patent against the party.

If you convey a covered work, knowingly relying on a patent license,
and the Corresponding Source of the work is not available for anyone
to copy, free of charge and under the terms of this License, through a
publicly available network server or other readily accessible means,
then you must either (1) cause the Corresponding Source to be so
available, or (2) arrange to deprive yourself of the benefit of the
patent license for this particular work, or (3) arrange, in a manner
consistent with the requirements of this License, to extend the patent
license to downstream recipients.  ``Knowingly relying'' means you have
actual knowledge that, but for the patent license, your conveying the
covered work in a country, or your recipient's use of the covered work
in a country, would infringe one or more identifiable patents in that
country that you have reason to believe are valid.

If, pursuant to or in connection with a single transaction or
arrangement, you convey, or propagate by procuring conveyance of, a
covered work, and grant a patent license to some of the parties
receiving the covered work authorizing them to use, propagate, modify
or convey a specific copy of the covered work, then the patent license
you grant is automatically extended to all recipients of the covered
work and works based on it.

A patent license is ``discriminatory'' if it does not include within
the scope of its coverage, prohibits the exercise of, or is
conditioned on the non-exercise of one or more of the rights that are
specifically granted under this License.  You may not convey a covered
work if you are a party to an arrangement with a third party that is
in the business of distributing software, under which you make payment
to the third party based on the extent of your activity of conveying
the work, and under which the third party grants, to any of the
parties who would receive the covered work from you, a discriminatory
patent license (a) in connection with copies of the covered work
conveyed by you (or copies made from those copies), or (b) primarily
for and in connection with specific products or compilations that
contain the covered work, unless you entered into that arrangement,
or that patent license was granted, prior to 28 March 2007.

Nothing in this License shall be construed as excluding or limiting
any implied license or other defenses to infringement that may
otherwise be available to you under applicable patent law.

\item No Surrender of Others' Freedom.

If conditions are imposed on you (whether by court order, agreement or
otherwise) that contradict the conditions of this License, they do not
excuse you from the conditions of this License.  If you cannot convey a
covered work so as to satisfy simultaneously your obligations under this
License and any other pertinent obligations, then as a consequence you may
not convey it at all.  For example, if you agree to terms that obligate you
to collect a royalty for further conveying from those to whom you convey
the Program, the only way you could satisfy both those terms and this
License would be to refrain entirely from conveying the Program.

\item Use with the GNU Affero General Public License.

Notwithstanding any other provision of this License, you have
permission to link or combine any covered work with a work licensed
under version 3 of the GNU Affero General Public License into a single
combined work, and to convey the resulting work.  The terms of this
License will continue to apply to the part which is the covered work,
but the special requirements of the GNU Affero General Public License,
section 13, concerning interaction through a network will apply to the
combination as such.

\item Revised Versions of this License.

The Free Software Foundation may publish revised and/or new versions of
the GNU General Public License from time to time.  Such new versions will
be similar in spirit to the present version, but may differ in detail to
address new problems or concerns.

Each version is given a distinguishing version number.  If the
Program specifies that a certain numbered version of the GNU General
Public License ``or any later version'' applies to it, you have the
option of following the terms and conditions either of that numbered
version or of any later version published by the Free Software
Foundation.  If the Program does not specify a version number of the
GNU General Public License, you may choose any version ever published
by the Free Software Foundation.

If the Program specifies that a proxy can decide which future
versions of the GNU General Public License can be used, that proxy's
public statement of acceptance of a version permanently authorizes you
to choose that version for the Program.

Later license versions may give you additional or different
permissions.  However, no additional obligations are imposed on any
author or copyright holder as a result of your choosing to follow a
later version.

\item Disclaimer of Warranty.

\begin{sloppypar}
 THERE IS NO WARRANTY FOR THE PROGRAM, TO THE EXTENT PERMITTED BY
 APPLICABLE LAW.  EXCEPT WHEN OTHERWISE STATED IN WRITING THE
 COPYRIGHT HOLDERS AND/OR OTHER PARTIES PROVIDE THE PROGRAM ``AS IS''
 WITHOUT WARRANTY OF ANY KIND, EITHER EXPRESSED OR IMPLIED,
 INCLUDING, BUT NOT LIMITED TO, THE IMPLIED WARRANTIES OF
 MERCHANTABILITY AND FITNESS FOR A PARTICULAR PURPOSE.  THE ENTIRE
 RISK AS TO THE QUALITY AND PERFORMANCE OF THE PROGRAM IS WITH YOU.
 SHOULD THE PROGRAM PROVE DEFECTIVE, YOU ASSUME THE COST OF ALL
 NECESSARY SERVICING, REPAIR OR CORRECTION.
\end{sloppypar}

\item Limitation of Liability.

 IN NO EVENT UNLESS REQUIRED BY APPLICABLE LAW OR AGREED TO IN
 WRITING WILL ANY COPYRIGHT HOLDER, OR ANY OTHER PARTY WHO MODIFIES
 AND/OR CONVEYS THE PROGRAM AS PERMITTED ABOVE, BE LIABLE TO YOU FOR
 DAMAGES, INCLUDING ANY GENERAL, SPECIAL, INCIDENTAL OR CONSEQUENTIAL
 DAMAGES ARISING OUT OF THE USE OR INABILITY TO USE THE PROGRAM
 (INCLUDING BUT NOT LIMITED TO LOSS OF DATA OR DATA BEING RENDERED
 INACCURATE OR LOSSES SUSTAINED BY YOU OR THIRD PARTIES OR A FAILURE
 OF THE PROGRAM TO OPERATE WITH ANY OTHER PROGRAMS), EVEN IF SUCH
 HOLDER OR OTHER PARTY HAS BEEN ADVISED OF THE POSSIBILITY OF SUCH
 DAMAGES.

\item Interpretation of Sections 15 and 16.

If the disclaimer of warranty and limitation of liability provided
above cannot be given local legal effect according to their terms,
reviewing courts shall apply local law that most closely approximates
an absolute waiver of all civil liability in connection with the
Program, unless a warranty or assumption of liability accompanies a
copy of the Program in return for a fee.

\begin{center}
{\Large\sc End of Terms and Conditions}

\bigskip
How to Apply These Terms to Your New Programs
\end{center}

If you develop a new program, and you want it to be of the greatest
possible use to the public, the best way to achieve this is to make it
free software which everyone can redistribute and change under these terms.

To do so, attach the following notices to the program.  It is safest
to attach them to the start of each source file to most effectively
state the exclusion of warranty; and each file should have at least
the ``copyright'' line and a pointer to where the full notice is found.

{\footnotesize
\begin{verbatim}
<one line to give the program's name and a brief idea of what it does.>

Copyright (C) <textyear>  <name of author>

This program is free software: you can redistribute it and/or modify
it under the terms of the GNU General Public License as published by
the Free Software Foundation, either version 3 of the License, or
(at your option) any later version.

This program is distributed in the hope that it will be useful,
but WITHOUT ANY WARRANTY; without even the implied warranty of
MERCHANTABILITY or FITNESS FOR A PARTICULAR PURPOSE.  See the
GNU General Public License for more details.

You should have received a copy of the GNU General Public License
along with this program.  If not, see <http://www.gnu.org/licenses/>.
\end{verbatim}
}

Also add information on how to contact you by electronic and paper mail.

If the program does terminal interaction, make it output a short
notice like this when it starts in an interactive mode:

{\footnotesize
\begin{verbatim}
<program>  Copyright (C) <year>  <name of author>

This program comes with ABSOLUTELY NO WARRANTY; for details type `show w'.
This is free software, and you are welcome to redistribute it
under certain conditions; type `show c' for details.
\end{verbatim}
}

The hypothetical commands {\tt show w} and {\tt show c} should show
the appropriate
parts of the General Public License.  Of course, your program's commands
might be different; for a GUI interface, you would use an ``about box''.

You should also get your employer (if you work as a programmer) or
school, if any, to sign a ``copyright disclaimer'' for the program, if
necessary.  For more information on this, and how to apply and follow
the GNU GPL, see \texttt{http://www.gnu.org/licenses/}.

The GNU General Public License does not permit incorporating your
program into proprietary programs.  If your program is a subroutine
library, you may consider it more useful to permit linking proprietary
applications with the library.  If this is what you want to do, use
the GNU Lesser General Public License instead of this License.  But
first, please read \texttt{http://www.gnu.org/philosophy/why-not-lgpl.html}.

\end{enumerate}

\end{document}

\chapter{The Affero General Public License, version 3}
% FIXME, this is version 1 below.

\begin{center}
{\parindent 0in

Version 1, March 2002

Copyright \copyright\ 2002 Affero, Inc.

\bigskip

510 Third Street - Suite 225, San Francisco, CA 94107, USA

\bigskip

This license is a modified version of the GNU General Public License
copyright (C) 1989, 1991 Free Software Foundation, Inc. made with
their permission. Section 2(d) has been added to cover use of software
over a computer network.

Everyone is permitted to copy and distribute verbatim copies
of this license document, but changing it is not allowed.
}
\end{center}

\begin{center}
{\bf\large Preamble}
\end{center}



The licenses for most software are designed to take away your freedom
to share and change it. By contrast, the Affero General Public License
is intended to guarantee your freedom to share and change free
software--to make sure the software is free for all its users. This
Public License applies to most of Affero's software and to any other
program whose authors commit to using it. (Some other Affero software
is covered by the GNU Library General Public License instead.) You can
apply it to your programs, too.


When we speak of free software, we are referring to freedom, not price. This General Public License is designed to make sure that you have the freedom to distribute copies of free software (and charge for this service if you wish), that you receive source code or can get it if you want it, that you can change the software or use pieces of it in new free programs; and that you know you can do these things.

To protect your rights, we need to make restrictions that forbid anyone to deny you these rights or to ask you to surrender the rights. These restrictions translate to certain responsibilities for you if you distribute copies of the software, or if you modify it.

For example, if you distribute copies of such a program, whether gratis or for a fee, you must give the recipients all the rights that you have. You must make sure that they, too, receive or can get the source code. And you must show them these terms so they know their rights.

We protect your rights with two steps: (1) copyright the software, and (2) offer you this license which gives you legal permission to copy, distribute and/or modify the software.

Also, for each author's protection and ours, we want to make certain that everyone understands that there is no warranty for this free software. If the software is modified by someone else and passed on, we want its recipients to know that what they have is not the original, so that any problems introduced by others will not reflect on the original authors' reputations.

Finally, any free program is threatened constantly by software patents. We wish to avoid the danger that redistributors of a free program will individually obtain patent licenses, in effect making the program proprietary. To prevent this, we have made it clear that any patent must be licensed for everyone's free use or not licensed at all.

The precise terms and conditions for copying, distribution and modification follow.

\begin{center}
{\Large \sc Terms and Conditions For Copying, Distribution and
  Modification}
\end{center}


\begin{enumerate}

\addtocounter{enumi}{-1}
\item

This License applies to any program or other work which contains a
notice placed by the copyright holder saying it may be distributed
under the terms of this Affero General Public License.  The
``Program'', below, refers to any such program or work, and a ``work
based on the Program'' means either the Program or any derivative work
under copyright law: that is to say, a work containing the Program or
a portion of it, either verbatim or with modifications and/or
translated into another language.  (Hereinafter, translation is
included without limitation in the term ``modification''.)  Each
licensee is addressed as ``you''.

Activities other than copying, distribution and modification are not
covered by this License; they are outside its scope.  The act of
running the Program is not restricted, and the output from the Program
is covered only if its contents constitute a work based on the
Program (independent of having been made by running the Program).
Whether that is true depends on what the Program does.

\item You may copy and distribute verbatim copies of the Program's source
  code as you receive it, in any medium, provided that you conspicuously
  and appropriately publish on each copy an appropriate copyright notice
  and disclaimer of warranty; keep intact all the notices that refer to
  this License and to the absence of any warranty; and give any other
  recipients of the Program a copy of this License along with the Program.

You may charge a fee for the physical act of transferring a copy, and you
may at your option offer warranty protection in exchange for a fee.

\item

You may modify your copy or copies of the Program or any portion
of it, thus forming a work based on the Program, and copy and
distribute such modifications or work under the terms of Section 1
above, provided that you also meet all of these conditions:

\begin{enumerate}

\item

You must cause the modified files to carry prominent notices stating that
you changed the files and the date of any change.

\item

You must cause any work that you distribute or publish, that in
whole or in part contains or is derived from the Program or any
part thereof, to be licensed as a whole at no charge to all third
parties under the terms of this License.

\item
If the modified program normally reads commands interactively
when run, you must cause it, when started running for such
interactive use in the most ordinary way, to print or display an
announcement including an appropriate copyright notice and a
notice that there is no warranty (or else, saying that you provide
a warranty) and that users may redistribute the program under
these conditions, and telling the user how to view a copy of this
License.  (Exception: if the Program itself is interactive but
does not normally print such an announcement, your work based on
the Program is not required to print an announcement.)

\item
\textbf{If the Program as you received it is intended to interact with users
through a computer network and if, in the version you received, any
user interacting with the Program was given the opportunity to request
transmission to that user of the Program's complete source code, you
must not remove that facility from your modified version of the
Program or work based on the Program, and must offer an equivalent
opportunity for all users interacting with your Program through a
computer network to request immediate transmission by HTTP of the
complete source code of your modified version or other derivative
work.}

\end{enumerate}


These requirements apply to the modified work as a whole.  If
identifiable sections of that work are not derived from the Program,
and can be reasonably considered independent and separate works in
themselves, then this License, and its terms, do not apply to those
sections when you distribute them as separate works.  But when you
distribute the same sections as part of a whole which is a work based
on the Program, the distribution of the whole must be on the terms of
this License, whose permissions for other licensees extend to the
entire whole, and thus to each and every part regardless of who wrote it.

Thus, it is not the intent of this section to claim rights or contest
your rights to work written entirely by you; rather, the intent is to
exercise the right to control the distribution of derivative or
collective works based on the Program.

In addition, mere aggregation of another work not based on the Program
with the Program (or with a work based on the Program) on a volume of
a storage or distribution medium does not bring the other work under
the scope of this License.

\item
You may copy and distribute the Program (or a work based on it,
under Section 2) in object code or executable form under the terms of
Sections 1 and 2 above provided that you also do one of the following:

\begin{enumerate}

\item

Accompany it with the complete corresponding machine-readable
source code, which must be distributed under the terms of Sections
1 and 2 above on a medium customarily used for software interchange; or,

\item

Accompany it with a written offer, valid for at least three
years, to give any third party, for a charge no more than your
cost of physically performing source distribution, a complete
machine-readable copy of the corresponding source code, to be
distributed under the terms of Sections 1 and 2 above on a medium
customarily used for software interchange; or,

\item

Accompany it with the information you received as to the offer
to distribute corresponding source code.  (This alternative is
allowed only for noncommercial distribution and only if you
received the program in object code or executable form with such
an offer, in accord with Subsection b above.)

\end{enumerate}


The source code for a work means the preferred form of the work for
making modifications to it.  For an executable work, complete source
code means all the source code for all modules it contains, plus any
associated interface definition files, plus the scripts used to
control compilation and installation of the executable.  However, as a
special exception, the source code distributed need not include
anything that is normally distributed (in either source or binary
form) with the major components (compiler, kernel, and so on) of the
operating system on which the executable runs, unless that component
itself accompanies the executable.

If distribution of executable or object code is made by offering
access to copy from a designated place, then offering equivalent
access to copy the source code from the same place counts as
distribution of the source code, even though third parties are not
compelled to copy the source along with the object code.

\item
You may not copy, modify, sublicense, or distribute the Program
except as expressly provided under this License.  Any attempt
otherwise to copy, modify, sublicense or distribute the Program is
void, and will automatically terminate your rights under this License.
However, parties who have received copies, or rights, from you under
this License will not have their licenses terminated so long as such
parties remain in full compliance.

\item
You are not required to accept this License, since you have not
signed it.  However, nothing else grants you permission to modify or
distribute the Program or its derivative works.  These actions are
prohibited by law if you do not accept this License.  Therefore, by
modifying or distributing the Program (or any work based on the
Program), you indicate your acceptance of this License to do so, and
all its terms and conditions for copying, distributing or modifying
the Program or works based on it.

\item
Each time you redistribute the Program (or any work based on the
Program), the recipient automatically receives a license from the
original licensor to copy, distribute or modify the Program subject to
these terms and conditions.  You may not impose any further
restrictions on the recipients' exercise of the rights granted herein.
You are not responsible for enforcing compliance by third parties to
this License.

\item
If, as a consequence of a court judgment or allegation of patent
infringement or for any other reason (not limited to patent issues),
conditions are imposed on you (whether by court order, agreement or
otherwise) that contradict the conditions of this License, they do not
excuse you from the conditions of this License.  If you cannot
distribute so as to satisfy simultaneously your obligations under this
License and any other pertinent obligations, then as a consequence you
may not distribute the Program at all.  For example, if a patent
license would not permit royalty-free redistribution of the Program by
all those who receive copies directly or indirectly through you, then
the only way you could satisfy both it and this License would be to
refrain entirely from distribution of the Program.

If any portion of this section is held invalid or unenforceable under
any particular circumstance, the balance of the section is intended to
apply and the section as a whole is intended to apply in other
circumstances.

It is not the purpose of this section to induce you to infringe any
patents or other property right claims or to contest validity of any
such claims; this section has the sole purpose of protecting the
integrity of the free software distribution system, which is
implemented by public license practices.  Many people have made
generous contributions to the wide range of software distributed
through that system in reliance on consistent application of that
system; it is up to the author/donor to decide if he or she is willing
to distribute software through any other system and a licensee cannot
impose that choice.

This section is intended to make thoroughly clear what is believed to
be a consequence of the rest of this License.

\item
If the distribution and/or use of the Program is restricted in
certain countries either by patents or by copyrighted interfaces, the
original copyright holder who places the Program under this License
may add an explicit geographical distribution limitation excluding
those countries, so that distribution is permitted only in or among
countries not thus excluded.  In such case, this License incorporates
the limitation as if written in the body of this License.

\item
\textbf{Affero Inc. may publish revised and/or new versions of the Affero
General Public License from time to time. Such new versions will be
similar in spirit to the present version, but may differ in detail to
address new problems or concerns.}

\textbf{Each version is given a distinguishing version number. If the Program
specifies a version number of this License which applies to it and
``any later version'', you have the option of following the terms and
conditions either of that version or of any later version published by
Affero, Inc. If the Program does not specify a version number of this
License, you may choose any version ever published by Affero, Inc.}

\textbf{You may also choose to redistribute modified versions of this program
under any version of the Free Software Foundation's GNU General Public
License version 3 or higher, so long as that version of the GNU GPL
includes terms and conditions substantially equivalent to those of
this license.}

\item
If you wish to incorporate parts of the Program into other free
programs whose distribution conditions are different, write to the
author to ask for permission. For software which is copyrighted by
Affero, Inc., write to us; we sometimes make exceptions for this. Our
decision will be guided by the two goals of preserving the free status
of all derivatives of our free software and of promoting the sharing
and reuse of software generally.

\begin{center}
{\Large\sc
No Warranty
}
\end{center}

\item
{\sc Because the program is licensed free of charge, there is no warranty
for the program, to the extent permitted by applicable law.  Except when
otherwise stated in writing the copyright holders and/or other parties
provide the program ``as is'' without warranty of any kind, either expressed
or implied, including, but not limited to, the implied warranties of
merchantability and fitness for a particular purpose.  The entire risk as
to the quality and performance of the program is with you.  Should the
program prove defective, you assume the cost of all necessary servicing,
repair or correction.}

\item
{\sc In no event unless required by applicable law or agreed to in writing
will any copyright holder, or any other party who may modify and/or
redistribute the program as permitted above, be liable to you for damages,
including any general, special, incidental or consequential damages arising
out of the use or inability to use the program (including but not limited
to loss of data or data being rendered inaccurate or losses sustained by
you or third parties or a failure of the program to operate with any other
programs), even if such holder or other party has been advised of the
possibility of such damages.}

\end{enumerate}

That's all there is to it!



\end{document}
