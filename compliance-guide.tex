% compliance-guide.tex                            -*- LaTeX -*-

\part{A Practical Guide to GPL Compliance}
\label{gpl-compliance-guide}

{\parindent 0in
This part is: \\
\begin{tabbing}
Copyright \= \copyright{} 2014 \= \hspace{.2in} Bradley M. Kuhn. \\
Copyright \> \copyright{} 2008 \> \hspace{.2in} Software Freedom Law Center. \\
\end{tabbing}

\vspace{1in}

\begin{center}
Authors of this part are: \\

Bradley M. Kuhn \\
Aaron Williamson \\
Karen M. Sandler \\

\vspace{3in}


The copyright holders of this part hereby grant the freedom to copy, modify,
convey, Adapt, and/or redistribute this work under the terms of the Creative
Commons Attribution Share Alike 4.0 International License.  A copy of that
license is available at
\verb=https://creativecommons.org/licenses/by-sa/4.0/legalcode=. 
\end{center}
}

\bigskip

\chapter*{Executive Summary}

This is a guide to effective compliance with the GNU General Public
License (GPL) and related licenses.  Copyleft advocates
usually seek to assist the community with
GPL compliance cooperatively.   This guide focuses on complying from the
start, so that readers can learn to avoid enforcement actions entirely, or, at
least, minimize  the negative impact when enforcement actions occur.
This guide  introduces and explains basic legal concepts related to the GPL and its
enforcement by copyright holders. It also outlines business practices and
methods that lead to better GPL compliance.  Finally, it recommends proper
post-violation responses to the concerns of copyright holders.

\chapter{Background}

Early GPL enforcement efforts began soon after the GPL was written by
Richard M.~Stallman (RMS) in 1989, and consisted of informal community efforts,
often in public Usenet discussions.\footnote{One example is the public
  outcry over NeXT's attempt to make the Objective-C front-end to GCC
  proprietary.  RMS, in fact, handled this enforcement action personally and
  the Objective-C front-end is still part of upstream GCC today.}  Over the next decade, the Free Software Foundation (FSF),
which holds copyrights in many GNU programs, was the only visible entity
actively enforcing its GPL'd copyrights on behalf of the software freedom
community.
FSF's enforcement
was generally a private process; the FSF contacted violators
confidentially and helped them to comply with the license.  Most
violations were pursued this way until the early 2000's.

By that time, Linux-based systems such as GNU/Linux and BusyBox/Linux had become very common, particularly in
embedded devices such as wireless routers.  During this period, public
ridicule of violators in the press and on Internet fora supplemented
ongoing private enforcement and increased pressure on businesses to
comply.  In 2003, the FSF formalized its efforts into the GPL Compliance
Lab, increased the volume of enforcement, and built community coalitions
to encourage copyright holders to together settle amicably with violators.
Beginning in 2004, Harald Welte took a more organized public enforcement
approach and launched \verb0gpl-violations.org0, a website and mailing
list for collecting reports of GPL violations.  On the basis of these
reports, Welte successfully pursued many enforcements in Europe, including
formal legal action.  Harald earns the permanent fame as the first copyright
holder to bring legal action in a Court regarding GPL compliance. 

In 2007, two copyright holders in BusyBox, in conjunction with the
Software Freedom Conservancy (``Conservancy''), filed the first copyright infringement lawsuit
based on a violation of the GPL\@ in the USA. While  lawsuits are of course
quite public, the vast majority of Conservancy's enforcement actions 
are resolved privately via
cooperative communications with violators.  As both FSF and Conservancy has worked to bring
individual companies into compliance, both organizations have encountered numerous
violations resulting from preventable problems such as inadequate
attention to licensing of upstream software, misconceptions about the
GPL's terms, and poor communication between software developers and their
management.  This document highlights these problems and describe
best practices to encourage corporate Free Software users to reevaluate their
approach to GPL'd software and avoid future violations.

Both FSF and Conservancy continue GPL enforcement and compliance efforts
for software under the GPL, the GNU Lesser
Public License (LGPL) and other copyleft licenses.  In doing so, both organizations have
found that most violations stem from a few common mistakes that can be,
for the most part, easily avoided.  All copyleft advocates  hope to educate the community of
commercial distributors, redistributors, and resellers on how to avoid
violations in the first place, and to respond adequately and appropriately
when a violation occurs.

\chapter{Best Practices to Avoid Common Violations}
\label{best-practices}

Unlike highly permissive licenses (such as the ISC license), which
typically only require preservation of copyright notices, licensees face many
important requirements from the GPL.  These requirements are
carefully designed to uphold certain values and standards of the software
freedom community.  While the GPL's requirements may appear initially
counter-intuitive to those more familiar with proprietary software
licenses, by comparison, its terms are in fact clear and quite favorable to
licensees.  Indeed, the GPL's terms actually simplify compliance when
violations occur.

GPL violations occur (or, are compounded) most often when companies lack sound
practices for the incorporation of GPL'd components into their
internal development environment.  This section introduces some best
practices for software tool selection, integration and distribution,
inspired by and congruent with software freedom methodologies.  Companies should
establish such practices before building a product based on GPL'd
software.\footnote{This document addresses compliance with GPLv2,
  GPLv3, LGPLv2, and LGPLv3.  Advice on avoiding the most common
  errors differs little for compliance with these four licenses.
  \S~\ref{lgpl} discusses the key differences between GPL and LGPL
  compliance.}

\section{Evaluate License Applicability}
\label{derivative-works}
Political discussion about the GPL often centers around the ``copyleft''
requirements of the license.  Indeed, the license was designed primarily
to embody this licensing feature.  Most companies adding non-trivial
features (beyond mere porting and bug-fixing) to GPL'd software (and
thereby invoking these requirements) are already well aware of their
more complex obligations under the license.\footnote{While, there has been much legal
  discussion regarding copyleft and derivative works.  In practical
  reality, this issue is not relevant to the vast majority of companies
  distributing GPL'd software.  Those interested in this issue should study
  \tutorialpartsplit{\texit{Detailed Analysis of the GNU GPL and Related
      Licenses}'s Section on derivative works}{\S~\ref{derivative-works} of
    this tutorial}.}

However, experienced  GPL enforcers find that few redistributors'
compliance challenges relate directly to the copyleft provisions; this is
particularly true for most embedders.  Instead, the distributions of GPL'd
systems most often encountered typically consist of a full operating system
including components under the GPL (e.g., Linux, BusyBox) and components
under the LGPL (e.g., the GNU C Library).  Sometimes, these programs have
been patched or slightly improved by direct modification of their sources,
resulting unequivocally in a derivative work.  Alongside these programs,
companies often distribute fully independent, proprietary programs,
developed from scratch, which are designed to run on the Free Software operating
system but do not combine with, link to, modify, derive from, or otherwise
create a combined work with
the GPL'd components.\footnote{However, these programs do often combine
  with LGPL'd libraries. This is discussed in detail in \S~\ref{lgpl}.}
In the latter case, where the work is unquestionably a separate work of
creative expression, no copyleft provisions are invoked.

Admittedly, a tiny
minority of situations which lie outside these two categories, and thus
do involve close questions about derivative and combined works.  Those
situations admittedly do require a highly
fact-dependent analysis and cannot be addressed in a general-purpose
document, anyway.

\medskip

Most companies accused of violations lack a basic understanding
of how to comply even in the former straightforward scenario.  This document
provides those companies with the fundamental and generally applicable prerequisite knowledge.
For answers to rarer and more complicated legal questions, such as whether
your software is a derivative or combined work of some copylefted software, consult
with an attorney.\footnote{If you would like more information on the
  application of derivative works doctrine to software, a detailed legal
  discussion is presented in our colleague Dan Ravicher's article,
  \textit{Software Derivative Work: A Circuit Dependent Determination} and in
  \tutorialpartsplit{\texit{Detailed Analysis of the GNU GPL and Related
      Licenses}'s Section on derivative works}{\S~\ref{derivative-works} of
    this tutorial}.}

This discussion thus assumes that you have already identified the
``work'' covered by the license, and that any components not under the GPL
(e.g., applications written entirely by your developers that merely happen
to run on a Linux-based operating system) distributed in conjunction with
those works are separate works within the meaning of copyright law and the GPL\@.  In
such a case, the GPL requires you to provide complete corresponding
source (CCS)\footnote{For more on CCS,  see
\tutorialpartsplit{\texit{Detailed Analysis of the GNU GPL and Related
      Licenses}'s Section on GPLv2~\S2 and GPLv3~\S1.}{\S~\ref{GPLv2s2} and \S~\ref{GPLv3s1} of
    this tutorial}.
for the GPL'd components and your modifications thereto, but not
for independent proprietary applications.  The procedures described in
this document address this typical scenario.

\section{Monitor Software Acquisition}

Software engineers deserve the freedom to innovate and import useful
software components to improve products.  However, along with that
freedom should come rules and reporting procedures to make sure that you
are aware of what software that you include with your product.

The most typical response to an initial enforcement action is: ``We
didn't know there was GPL'd stuff in there''.  This answer indicates
failure in the software acquisition and procurement process.  Integration
of third-party proprietary software typically requires a formal
arrangement and management/legal oversight before the developers
incorporate the software.  By contrast, developers often obtain and
integrate Free Software without intervention nor oversight. That ease of acquisition, however,
does not mean the oversight is any less necessary.  Just as your legal
and/or management team negotiates terms for inclusion of any proprietary
software, they should gently facilitate all decisions to bring Free Software into your
product.

Simple, engineering-oriented rules help provide a stable foundation for
Free Software integration.  For example, simply ask your software developers to send an email to a
standard place describing each new Free Software component they add to the system,
and have them include a brief description of how they will incorporate it
into the product.  Further, make sure developers use a revision control
system (such as Git or Mercurial), and have
store the upstream versions of all software in a ``vendor branch'' or
similar mechanism, whereby they can easily track and find the main version
of the software and, separately, any local changes.

Such procedures are best instituted at your project's launch.  Once 
chaotic and poorly-sourced development processes begin, cataloging the
presence of GPL'd components  becomes challenging.

Such a situation often requires use of a tool to ``catch up'' your knowledge
about what software your product includes.  Most commonly, companies choose
some software licensing scanning tool to inspect the codebase.  However,
there are few tools that are themselves Free Software.  Thus, GPL enforcers
usually recommend the GPL'd
\href{http://fossology.org/}{Fossology system}, which analyzes a
source code base and produces a list of Free Software licenses that may apply to
the code.  Fossology can help you build a catalog of the sources you have
already used to build your product.  You can then expand that into a more
structured inventory and process.

\section{Track Your Changes and Releases}

As explained in further detail below, the most important component of GPL
compliance is the one most often ignored: proper inclusion of CCS in all
distributions  of GPL'd
software.  To comply with GPL's CCS requirements, the distributor
\textit{must} always know precisely what sources generated a given binary
distribution.

In an unfortunately large number of our enforcement cases, the violating
company's engineering team had difficulty reconstructing the CCS
for binaries distributed by the company.  Here are three simple rules to
follow to decrease the likelihood of this occurance:

\begin{itemize}

\item Ensure that your
developers are using revision control systems properly.

\item Have developers mark or ``tag'' the full source tree corresponding to
  builds distributed to customers

\item Check that your developers store all parts of the software
development in the revision control system, including {\sc readme}s, build
scripts, engineers' notes, and documentation.
\end{itemize}

Your developers will benefit anyway from these rules.  Developers will be
happier in their jobs if their tools already track the precise version of
source that corresponds to any deployed binary.

\section{Avoid the ``Build Guru''}

Too many software projects rely on only one or a very few team members who
know how to build and assemble the final released product.  Such knowledge
centralization not only creates engineering redundancy issues, but also
thwarts GPL compliance.  Specifically, CCS does not just require source code,
but scripts and other material that explain how to control compilation and
installation of the executable and object code.

Thus, avoid relying on a ``build guru'', a single developer who is the only one
who knows how to produce your final product. Make sure the build process
is well defined.  Train every developer on the build process for the final
binary distribution, including (in the case of embedded software)
generating a final firmware image suitable for distribution to the
customer.  Require developers to use revision control for build processes.
Make a rule that adding new components to the system without adequate
build instructions (or better yet, scripts) is unacceptable engineering
practice.

\chapter{Details of Compliant Distribution}

This section explains the specific requirements placed upon
distributors of GPL'd software.  Note that this section refers heavily to
specific provisions and language in
\href{http://www.gnu.org/licenses/old-licenses/gpl-2.0.html#section3}{GPLv2}
and \href{http://www.fsf.org/licensing/licenses/gpl.html#section6}{GPLv3}.
It may be helpful to have a copy of each license open while reading this
section.

\section{Binary Distribution Permission}
\label{binary-distribution-permission}

% be careful below, you cannot refill the \if section, so don't refill
% this paragraph without care.

The various versions of the GPL are copyright licenses that grant
permission to make certain uses of software that are otherwise restricted
by copyright law.  This permission is conditioned upon compliance with the
GPL's requirements.

This section walks through the requirements (of both GPLv2 and GPLv3) that
apply when you distribute GPL'd programs in binary (i.e., executable or
object code) form, which is typical for embedded applications.  Because a
binary application derives from a program's original sources, you need
permission from the copyright holder to distribute it.  \S~3 of GPLv2 and
\S~6 of GPLv3 contain the permissions and conditions related to binary
distributions of GPL'd programs.\footnote{These sections cannot be fully
  understood in isolation; read the entire license thoroughly before
  focusing on any particular provision.  However, once you have read and
  understood the entire license, look to these sections to guide
  compliance for binary distributions.}

GPL's binary distribution sections offer a choice of compliance methods,
each of which we consider in turn.  Each option refers to the
``Corresponding Source'' code for the binary distribution, which includes
the source code from which the binary was produced.  This abbreviated and
simplified definition is sufficient for the binary distribution discussion
in this section, but you may wish to refer back to this section after
reading the thorough discussion of ``Corresponding Source'' that appears
in \S~\ref{corresponding-source}.

\subsection{Option (a): Source Alongside Binary}

GPLv2~\S~3(a) and v3~\S~6(a) embody the easiest option for providing
source code: including Corresponding Source with every binary
distribution.  While other options appear initially less onerous, this
option invariably minimizes potential compliance problems, because when
you distribute Corresponding Source with the binary, \emph{your GPL
  obligations are satisfied at the time of distribution}.  This is not
true of other options, and for this reason, we urge you to seriously
consider this option.  If you do not, you may extend the duration of your
obligations far beyond your last binary distribution.

Compliance under this option is straightforward.  If you ship a product
that includes binary copies of GPL'd software (e.g., in firmware, or on a
hard drive, CD, or other permanent storage medium), you can store the
Corresponding Source alongside the binaries.  Alternatively, you can
include the source on a CD or other removable storage medium in the box
containing the product.

GPLv2 refers to the various storage mechanisms as ``medi[a] customarily
used for software interchange''.  While the Internet has attained primacy
as a means of software distribution where super-fast Internet connections
are available, GPLv2 was written at a time when downloading software was
not practical (and was often impossible).  For much of the world, this
condition has not changed since GPLv2's publication, and the Internet
still cannot be considered ``a medium customary for software
interchange''.  GPLv3 clarifies this matter, requiring that source be
``fixed on a durable physical medium customarily used for software
interchange''.  This language affirms that option (a) requires binary
redistributors to provide source on a physical medium.

Please note that while selection of option (a) requires distribution on a
physical medium, voluntary distribution via the Internet is very useful.  This
is discussed in detail in \S~\ref{offer-with-internet}.

\subsection{Option (b): The Offer}
\label{offer-for-source}

Many distributors prefer to ship only an offer for source with the binary
distribution, rather than the complete source package.  This
option has value when the cost of source distribution is a true
per-unit cost.  For example, this option might be a good choice for
embedded products with permanent storage too small to fit the source, and
which are not otherwise shipped with a CD but \emph{are} shipped with a
manual or other printed material.

However, this option increases the duration of your obligations
dramatically.  An offer for source must be good for three full years from
your last binary distribution (under GPLv2), or your last binary or spare
part distribution (under GPLv3).  Your source code request and
provisioning system must be designed to last much longer than your product
life cycle.

In addition, if you are required to comply with the terms of GPLv2, you
{\bf cannot} use a network service to provide the source code.  For GPLv2,
the source code offer is fulfilled only with physical media.  This usually
means that you must continue to produce an up-to-date ``source code CD''
for years after the product's end-of-life.

\label{offer-with-internet}

Under GPLv2, it is acceptable and advisable for your offer for source code
to include an Internet link for downloadable source \emph{in addition} to
offering source on a physical medium.  This practice enables those with
fast network connections to get the source more quickly, and typically
decreases the number of physical media fulfillment requests.
(GPLv3~\S~6(b) permits provision of source with a public
network-accessible distribution only and no physical media.  We discuss
this in detail at the end of this section.)

The following is a suggested compliant offer for source under GPLv2 (and
is also acceptable for GPLv3) that you would include in your printed
materials accompanying each binary distribution:

\begin{quote}
The software included in this product contains copyrighted software that
is licensed under the GPL\@.  A copy of that license is included in this
document on page $X$\@.  You may obtain the complete Corresponding Source
code from us for a period of three years after our last shipment of this
product, which will be no earlier than 2011-08-01, by sending a money
order or check for \$5 to: \\
GPL Compliance Division \\
Our Company \\
Any Town, US 99999 \\
\\
Please write ``source for product $Y$'' in the memo line of your
payment.

You may also find a copy of the source at
\verb0http://www.example.com/sources/Y/0.

This offer is valid to anyone in receipt of this information.
\end{quote}

There are a few important details about this offer.  First, it requires a
copying fee.  GPLv2 permits ``a charge no more than your cost of
physically performing source distribution''.  This fee must be reasonable.
If your cost of copying and mailing a CD is more than around \$10, you
should perhaps find a cheaper CD stock and shipment method.  It is simply
not in your interest to try to overcharge the community.  Abuse of this
provision in order to make a for-profit enterprise of source code
provision will likely trigger enforcement action.

Second, note that the last line makes the offer valid to anyone who
requests the source.  This is because v2~\S~3(b) requires that offers be
``to give any third party'' a copy of the Corresponding Source.  GPLv3 has
a similar requirement, stating that an offer must be valid for ``anyone
who possesses the object code''.  These requirements indicated in
v2~\S~3(c) and v3~\S~6(c) are so that non-commercial redistributors may
pass these offers along with their distributions.  Therefore, the offers
must be valid not only to your customers, but also to anyone who received
a copy of the binaries from them.  Many distributors overlook this
requirement and assume that they are only required to fulfill a request
from their direct customers.

The option to provide an offer for source rather than direct source
distribution is a special benefit to companies equipped to handle a
fulfillment process.  GPLv2~\S~3(c) and GPLv3~\S~6(c) avoid burdening
noncommercial, occasional redistributors with fulfillment request
obligations by allowing them to pass along the offer for source as they
received it.

Note that commercial redistributors cannot avail themselves of the option
(c) exception, and so while your offer for source must be good to anyone
who receives the offer (under v2) or the object code (under v3), it
\emph{cannot} extinguish the obligations of anyone who commercially
redistributes your product.  The license terms apply to anyone who
distributes GPL'd software, regardless of whether they are the original
distributor.  Take the example of Vendor $V$, who develops a software
platform from GPL'd sources for use in embedded devices.  Manufacturer $M$
contracts with $V$ to install the software as firmware in $M$'s device.
$V$ provides the software to $M$, along with a compliant offer for source.
In this situation, $M$ cannot simply pass $V$'s offer for source along to
its customers.  $M$ also distributes the GPL'd software commercially, so
$M$ too must comply with the GPL and provide source (or $M$'s \emph{own}
offer for source) to $M$'s customers.

This situation illustrates that the offer for source is often a poor
choice for products that your customers will likely redistribute.  If you
include the source itself with the products, then your distribution to
your customers is compliant, and their (unmodified) distribution to their
customers is likewise compliant, because both include source.  If you
include only an offer for source, your distribution is compliant but your
customer's distribution does not ``inherit'' that compliance, because they
have not made their own offer to accompany their distribution.

The terms related to the offer for source are quite different if you
distribute under GPLv3.  Under v3, you may make source available only over
a network server, as long as it is available to the general public and
remains active for three years from the last distribution of your product
or related spare part.  Accordingly, you may satisfy your fulfillment
obligations via Internet-only distribution.  This makes the ``offer for
source'' option less troublesome for v3-only distributions, easing
compliance for commercial redistributors.  However, before you switch to a
purely Internet-based fulfillment process, you must first confirm that you
can actually distribute \emph{all} of the software under GPLv3.  Some
programs are indeed licensed under ``GPLv2, \emph{or any later version}''
(often abbreviated ``GPLv2-or-later'').  Such licensing gives you the
option to redistribute under GPLv3.  However, a few popular programs are
only licensed under GPLv2 and not ``or any later version''
(``GPLv2-only'').  You cannot provide only Internet-based source request
fulfillment for the latter programs.

If you determine that all GPL'd works in your whole product allow upgrade
to GPLv3 (or were already GPLv3'd to start), your offer for source may be
as simple as this:

\begin{quote}
The software included in this product contains copyrighted software that
is licensed under the GPLv3\@.  A copy of that license is included in this
document on page $X$\@.  You may obtain the complete Corresponding Source
code from us for a period of three years after our last shipment of this
product and/or spare parts therefor, which will be no earlier than
2011-08-01, on our website at
\verb0http://www.example.com/sources/productnum/0.
\end{quote}

\medskip

Under both GPLv2 and GPLv3, source offers must be accompanied by a copy of
the license itself, either electronically or in print, with every
distribution.
 
Finally, it is unacceptable to use option (b) merely because you do not have
Corresponding Source ready.  We find that some companies chose this option
because writing an offer is easy, but producing a source distribution as
an afterthought to a hasty development process is difficult.  The offer
for source does not exist as a stop-gap solution for companies rushing to
market with an out-of-compliance product.  If you ship an offer for source
with your product but cannot actually deliver \emph{immediately} on that
offer when your customers receive it, you should expect an enforcement
action.

\subsection{Option (c): Noncommercial Offers}

As discussed in the last section, GPLv2~\S~3(c) and GPLv3~\S~6(c) apply
only to noncommercial use.  These options are not available to businesses
distributing GPL'd software.  Consequently, companies who redistribute
software packaged for them by an upstream vendor cannot merely pass along
the offer they received from the vendor; they must provide their own offer
or corresponding source to their distributees.  We talk in detail about
upstream software providers in \S~\ref{upstream}.

\subsection{Option 6(d) in GPLv3: Internet Distribution}

Under GPLv2, your formal provisioning options for Corresponding Source
ended with \S~3(c).  But even under GPLv2, pure Internet source
distribution was a common practice and generally considered to be
compliant.  GPLv2 mentions Internet-only distribution almost as aside in
the language, in text at the end of the section after the three
provisioning options are listed.  To quote that part of GPLv2~\S~3:
\begin{quote}
If distribution of executable or object code is made by offering access to
copy from a designated place, then offering equivalent access to copy the
source code from the same place counts as distribution of the source code,
even though third parties are not compelled to copy the source along with
the object code.
\end{quote}

When that was written in 1991, Internet distribution of software was the
exception, not the rule.  Some FTP sites existed, but generally software
was sent on magnetic tape or CDs.  GPLv2 therefore mostly assumed that
binary distribution happened on some physical media.  By contrast,
GPLv3~\S~6(d) explicitly gives an option for this practice that the
community has historically considered GPLv2-compliant.

Thus, you may fulfill your source-provision obligations by providing the
source code in the same way and from the same location.  When exercising
this option, you are not obligated to ensure that users download the
source when they download the binary, and you may use separate servers as
needed to fulfill the requests as long as you make the source as
accessible as the binary.  However, you must ensure that users can easily
find the source code at the time they download the binary. GPLv3~\S~6(d)
thus clarifies a point that has caused confusion about source provision in
v2.  Indeed, many such important clarifications are included in v3 which
together provide a compelling reason for authors and redistributors alike
to adopt GPLv3.

\subsection{Option 6(e) in GPLv3: Software Torrents}

Peer-to-peer file sharing arose well after GPLv2 was written, and does not
easily fit any of the v2 source provision options.  GPLv3~\S~6(e)
addresses this issue, explicitly allowing for distribution of source and
binary together on a peer-to-peer file sharing network.  If you distribute
solely via peer-to-peer networks, you can exercise this option.  However,
peer-to-peer source distribution \emph{cannot} fulfill your source
provision obligations for non-peer-to-peer binary distributions.  Finally,
you should ensure that binaries and source are equally seeded upon initial
peer-to-peer distribution.

\section{Preparing Corresponding Source}
\label{corresponding-source}

Most enforcement cases involve companies that have unfortunately not
implemented procedures like our \S~\ref{best-practices} recommendations
and have no source distribution arranged at all.  These companies must
work backwards from a binary distribution to come into compliance.  Our
recommendations in \S~\ref{best-practices} are designed to make it easy to
construct a complete and Corresponding Source release from the outset.  If
you have followed those principles in your development, you can meet the
following requirements with ease.  If you have not, you may have
substantial reconstruction work to do.

\subsection{Assemble the Sources}

For every binary that you produce, you should collect and maintain a copy
of the sources from which it was built.  A large system, such as an
embedded firmware, will probably contain many GPL'd and LGPL'd components
for which you will have to provide source.  The binary distribution may
also contain proprietary components which are separate and independent
works that are covered by neither the GPL nor LGPL\@.

The best way to separate out your sources is to have a subdirectory for
each component in your system.  You can then easily mark some of them as
required for your Corresponding Source releases.  Collecting
subdirectories of GPL'd and LGPL'd components is the first step toward
preparing your release.

\subsection{Building the Sources}

Few distributors, particularly of embedded systems, take care to read the
actual definition of Corresponding Source in the GPL\@.  Consider
carefully the definition, from GPLv3:
\begin{quote}
  The ``Corresponding Source'' for a work in object code form means all
  the source code needed to generate, install, and (for an executable
  work) run the object code and to modify the work, including scripts to
  control those activities.
\end{quote}

and the definition from GPLv2:
\begin{quote}
The source code for a work means the preferred form of the work for making
modifications to it.  For an executable work, complete source code means
all the source code for all modules it contains, plus any associated
interface definition files, plus the scripts used to control compilation
and installation of the executable.
\end{quote}

Note that you must include ``scripts used to control compilation and
installation of the executable'' and/or anything ``needed to generate,
install, and (for an executable work) run the object code and to modify
the work, including scripts to control those activities''.  These phrases
are written to cover different types of build environments and systems.
Therefore, the details of what you need to provide with regard to scripts
and installation instructions vary depending on the software details.  You
must provide all information necessary such that someone generally skilled
with computer systems could produce a binary similar to the one provided.

Take as an example an embedded wireless device.  Usually, a company
distributes a firmware, which includes a binary copy of
Linux\footnote{``Linux'' refers only to the kernel, not the larger system
  as a whole.} and a filesystem.  That filesystem contains various binary
programs, including some GPL'd binaries, alongside some proprietary
binaries that are separate works (i.e., not derived from, nor based on
freely-licensed sources).  Consider what, in this case, constitutes adequate
``scripts to control compilation and installation'' or items ``needed to
generate, install and run'' the GPL'd programs.

Most importantly, you must provide some sort of roadmap that allows
technically sophisticated users to build your software.  This can be
complicated in an embedded environment.  If your developers use scripts to
control the entire compilation and installation procedure, then you can
simply provide those scripts to users along with the sources they act
upon.  Sometimes, however, scripts were never written (e.g., the
information on how to build the binaries is locked up in the mind of your
``build guru'').  In that case, we recommend that you write out build
instructions in a natural language as a detailed, step-by-step {\sc
  readme}.

No matter what you offer, you need to give those who receive source a
clear path from your sources to binaries similar to the ones you ship.  If
you ship a firmware (kernel plus filesystem), and the filesystem contains
binaries of GPL'd programs, then you should provide whatever is necessary
to enable a reasonably skilled user to build any given GPL'd source
program (and modified versions thereof), and replace the given binary in
your filesystem.  If the kernel is Linux, then the users must have the
instructions to do the same with the kernel.  The best way to achieve this
is to make available to your users whatever scripts or process your
engineers would use to do the same.

These are the general details for how installation instructions work.
Details about what differs when the work is licensed under LGPL is
discussed in \S~\ref{lgpl}, and specific details that are unique to
GPLv3's installation instructions are in \S~\ref{user-products}.

\subsection{What About the Compiler?}

The GPL contains no provision that requires distribution of the compiler
used to build the software.  While companies are encouraged to make it as
easy as possible for their users to build the sources, inclusion of the
compiler itself is not normally considered mandatory.  The Corresponding
Source definition -- both in GPLv2 and GPLv3 -- has not been typically
read to include the compiler itself, but rather things like makefiles,
build scripts, and packaging scripts.

Nonetheless, in the interest of goodwill and the spirit of the GPL, most
companies do provide the compiler itself when they are able, particularly
when the compiler is based on GCC\@ or another copylefted compiler.  If you have
a GCC-based system, it is your prerogative to redistribute that GCC
version (binaries plus sources) to your customers.  We in the software freedom
community encourage you to do this, since it often makes it easier for
users to exercise their software freedom.  However, if you chose to take
this recommendation, ensure that your GCC distribution is itself
compliant.

If you have used a proprietary, third-party compiler to build the
software, then you probably cannot ship it to your customers.  We consider
the name of the compiler, its exact version number, and where it can be
acquired as information that \emph{must} be provided as part of the
Corresponding Source.  This information is essential to anyone who wishes
to produce a binary.  It is not the intent of the GPL to require you to
distribute third-party software tools to your customer (provided the tools
themselves are not based on the GPL'd software shipped), but we do believe
it requires that you give the user all the essential non-proprietary facts
that you had at your disposal to build the software.  Therefore, if you
choose not to distribute the compiler, you should include a {\sc readme}
about where you got it, what version it was, and who to contact to acquire
it, regardless of whether your compiler is Free Software, proprietary, or
internally developed.

\section{Best Practices and Corresponding Source}

\S~\ref{best-practices} and \S~\ref{corresponding-source} above are
closely related.  If you follow the best practices outlined above, you
will find that preparing your Corresponding Source release is an easier
task, perhaps even a trivial one.

Indeed, the enforcement process itself has historically been useful to
software development teams.  Development on a deadline can lead
organizations to cut corners in a way that negatively impacts its
development processes.  We have frequently been told by violators that
they experience difficulty when determining the exact source for a binary
in production (in some cases because their ``build guru'' quit during the
release cycle).  When management rushes a development team to ship a
release, they are less likely to keep release sources tagged and build
systems well documented.

We suggest that, if contacted about a violation, product builders use GPL
enforcement as an opportunity to improve their development practices.  No
developer would argue that their system is better for having a mysterious
build system and no source tracking.  Address these issues by installing a
revision system, telling your developers to use it, and requiring your
build guru to document his or her work!

\chapter{When The Letter Comes}

Unfortunately, many GPL violators ignore their obligations until they are
contacted by a copyright holder or the lawyer of a copyright holder.  You
should certainly contact your own lawyer if you have received a letter
alleging that you have infringed copyrights that were licensed to you
under the GPL\@.  This section outlines a typical enforcement case and
provides some guidelines for response.  These discussions are
generalizations and do not all apply to every alleged violation.

\section{Understanding Who's Enforcing}
\label{compliance-understanding-whos-enforcing}
% FIXME-LATER: this text needs work.

Both  FSF and Conservancy has, as part their mission,  to spread software
freedom. When FSF or Conservancy
enforces GPL, the goal is to bring the violator back into compliance as
quickly as possible, and redress the damage caused by the violation.
That is FSF's steadfast position in a violation negotiation --- comply
with the license and respect freedom.

However, other entities who do not share the full ethos of software freedom
as institutionalized by FSF pursue GPL violations differently.  Oracle, a
company that produces the GPL'd MySQL database, upon discovering GPL
violations typically negotiates a proprietary software license separately for
a fee.  While this practice is not one that FSF nor Conservancy would ever
consider undertaking or even endorsing, it is a legally way for copyright
holders to proceed.

Generally, GPL enforcers come in two varieties.  First, there are
Conservancy, FSF, and other ``community enforcers'', who primary seek the
policy goals of GPL (software freedom), and see financial compensation as
ultimately secondary to those goals.  Second, there are ``for-profit
enforcers'' who use the GPL as a either a crippleware license, or sneakily
induce infringement merely to gain proprietary licensing revenue.

Note that the latter model \texit{only} works for companies who hold 100\% of
the copyrights in the infringed work.  As such, multi-copyright-held works
are fully insulated from these tactics.


\section{Communication Is Key}

GPL violations are typically only escalated when a company ignores the
copyright holder's initial communication or fails to work toward timely
compliance.  Accused violators should respond very promptly to the
initial request.  As the process continues, violators should follow up weekly with the
copyright holders to make sure everyone agrees on targets and deadlines
for resolving the situation.

Ensure that any staff who might receive communications regarding alleged
GPL violations understands how to channel the communication appropriately
within your organization.  Often, initial contact is addressed for general
correspondence (e.g., by mail to corporate headquarters or by e-mail to
general informational or support-related addresses).  Train the staff that
processes such communications to escalate them to someone with authority
to take action.  An unknowledgable response to such an inquiry (e.g., from
a first-level technical support person) can cause negotiations to fail
prematurely.

Answer promptly by multiple means (paper letter, telephone call, and
email), even if your response merely notifies the sender that you are
investigating the situation and will respond by a certain date.  Do not
let the conversation lapse until the situation is fully resolved.
Proactively follow up with synchronous communication means to be sure
communications sent by non-reliable means (such as email) were received.

Remember that the software freedom community generally values open communication and
cooperation, and these values extend to GPL enforcement.  You will
generally find that software freedom developers and their lawyers are willing to
have a reasonable dialogue and will work with you to resolve a violation
once you open the channels of communication in a friendly way.

\section{Termination}

Many redistributors overlook GPL's termination provision (GPLv2~\S~4 and
GPLv3~\S~8).  Under v2, violators forfeit their rights to redistribute and
modify the GPL'd software until those rights are explicitly reinstated by
the copyright holder.  In contrast, v3 allows violators to rapidly resolve
some violations without consequence.

If you have redistributed an application under GPLv2\footnote{This applies
  to all programs licensed to you under only GPLv2 (``GPLv2-only'').
  However, most so-called GPLv2 programs are actually distributed with
  permission to redistribute under GPLv2 \emph{or any later version of the
    GPL} (``GPLv2-or-later'').  In the latter cases, the redistributor can
  choose to redistribute under GPLv2, GPLv3, GPLv2-or-later or even
  GPLv3-or-later.  Where the redistributor has chosen v2 explicitly, the
  v2 termination provision will always apply.  If the redistributor has
  chosen v3, the v3 termination provision will always apply.  If the
  redistributor has chosen GPLv2-or-later, then the redistributor may want
  to narrow to GPLv3-only upon violation, to take advantage of the
  termination provisions in v3.}, but have violated the terms of GPLv2,
you must request a reinstatement of rights from the copyright holders
before making further distributions, or else cease distribution and
modification of the software forever.  Different copyright holders
condition reinstatement upon different requirements, and these
requirements can be (and often are) wholly independent of the GPL\@.  The
terms of your reinstatement will depend upon what you negotiate with the
copyright holder of the GPL'd program.

Since your rights under GPLv2 terminate automatically upon your initial
violation, \emph{all your subsequent distributions} are violations and
infringements of copyright.  Therefore, even if you resolve a violation on
your own, you must still seek a reinstatement of rights from the copyright
holders whose licenses you violated, lest you remain liable for
infringement for even compliant distributions made subsequent to the
initial violation.

GPLv3 is more lenient.  If you have distributed only v3-licensed programs,
you may be eligible under v3~\S~8 for automatic reinstatement of rights.
You are eligible for automatic reinstatement when:
\begin{itemize}
\item you correct the violation and are not contacted by a copyright
  holder about the violation within sixty days after the correction, or

\item you receive, from a copyright holder, your first-ever contact
  regarding a GPL violation, and you correct that violation within thirty
  days of receipt of copyright holder's notice.
\end{itemize}

In addition to these permanent reinstatements provided under v3, violators
who voluntarily correct their violation also receive provisional
permission to continue distributing until they receive contact from the
copyright holder.  If sixty days pass without contact, that reinstatement
becomes permanent.  Nonetheless, you should be prepared to cease
distribution during those initial sixty days should you receive a
termination notice from the copyright holder.

Given that much discussion of v3 has focused on its so-called more
complicated requirements, it should be noted that v3 is, in this regard,
more favorable to violators than v2.

\chapter{Standard Requests}

As we noted above, different copyright holders have different requirements
for reinstating a violator's distribution rights.  Upon violation, you no
longer have a license under the GPL\@.  Copyright holders can therefore
set their own requirements outside the license before reinstatement of
rights.  We have collected below a list of reinstatement demands that
copyright holders often require.

\begin{itemize}

\item {\bf Compliance on all Free Software copyrights}.  Copyright holders of Free Software
  often want a company to demonstrate compliance for all GPL'd software in
  a distribution, not just their own.  A copyright holder may refuse to
  reinstate your right to distribute one program unless and until you
  comply with the licenses of all Open Source and Free Software in your distribution.
 
\item {\bf Notification to past recipients}.  Users to whom you previously
  distributed non-compliant software should receive a communication
  (email, letter, bill insert, etc.) indicating the violation, describing
  their rights under GPL, and informing them how to obtain a gratis source
  distribution.  If a customer list does not exist (such as in reseller
  situations), an alternative form of notice may be required (such as a
  magazine advertisement).

\item {\bf Appointment of a GPL Compliance Officer.}  The software freedom community
  values personal accountability when things go wrong.  Copyright holders
  often require that you name someone within the violating company
  officially responsible for Open Source and Free Software license compliance, and that this
  individual serve as the key public contact for the community when
  compliance concerns arise.

\item {\bf Periodic Compliance Reports.}  Many copyright holders wish to
  monitor future compliance for some period of time after the violation.
  For some period, your company may be required to send regular reports on
  how many distributions of binary and source have occurred.
\end{itemize}

These are just a few possible requirements for reinstatement.  In the
context of a GPL violation, and particularly under v2's termination
provision, the copyright holder may have a range of requests in exchange
for reinstatement of rights.  These software developers are talented
professionals from whose work your company has benefited.  Indeed, you are
unlikely to find a better value or more generous license terms for similar
software elsewhere.  Treat the copyright holders with the same respect you
treat your corporate partners and collaborators.

\chapter{Special Topics in Compliance}

There are several other issues that are less common, but also relevant in
a GPL compliance situation.  To those who face them, they tend to be of
particular interest.

\section{LGPL Compliance}
\label{lgpl}

GPL compliance and LGPL compliance mostly involve the same issues.  As we
discussed in \S~\ref{derivative-works}, questions of modified versions of
software are highly fact-dependant and cannot be easily addressed in any
overview document.  The LGPL adds some additional complexity to the
analysis.  Namely, the various LGPL versions permit proprietary licensing
of certain types of modified versions.  These issues are well beyond the
scope of this document, but as a rule of thumb, once you have determined
(in accordance with LGPLv3) what part of the work is the ``Application''
and what portions of the source are ``Minimal Corresponding Source'', then
you can usually proceed to follow the GPL compliance rules that we
discussed, replacing our discussion of ``Corresponding Source'' with
``Minimal Corresponding Source''.

LGPL also requires that you provide a mechanism to combine the Application
with a modified version of the library, and outlines some options for
this.  Also, the license of the whole work must permit ``reverse
engineering for debugging such modifications'' to the library.  Therefore,
you should take care that the EULA used for the Application does not
contradict this permission.

\section{Upstream Providers}
\label{upstream}

With ever-increasing frequency, software development (particularly for
embedded devices) is outsourced to third parties.  If you rely on an
upstream provider for your software, note that you \emph{cannot ignore
  your GPL compliance requirements} simply because someone else packaged
the software that you distribute.  If you redistribute GPL'd software
(which you do, whenever you ship a device with your upstream's software in
it), you are bound by the terms of the GPL\@.  No distribution (including
redistribution) is permissible absent adherence to the license terms.

Therefore, you should introduce a due diligence process into your software
acquisition plans.  This is much like the software-oriented
recommendations we make in \S~\ref{best-practices}.  Implementing
practices to ensure that you are aware of what software is in your devices
can only improve your general business processes.  You should ask a clear
list of questions of all your upstream providers and make sure the answers
are complete and accurate.  The following are examples of questions you
should ask:
\begin{itemize}

\item What are all the licenses that cover the software in this device?

\item From which upstream vendors, be they companies or individuals, did
  \emph{you} receive your software from before distributing it to us?

\item What are your GPL compliance procedures?

\item If there is GPL'd software in your distribution, we will be
  redistributors of this GPL'd software.  What mechanisms do you have in
  place to aid us with compliance?

\item If we follow your recommended compliance procedures, will you
  formally indemnify us in case we are nonetheless found to be in
  violation of the GPL?

\end{itemize}

This last point is particularly important.  Many GPL enforcements are
escalated because of petty finger-pointing between the distributor and its
upstream.  In our experience, agreements regarding GPL compliance issues
and procedures are rarely negotiated up front.  However, when they are,
violations are resolved much more smoothly (at least from the point of
view of the redistributor).

Consider the cost of potential violations in your acquisition process.
Using Free Software allows software vendors to reduce costs significantly, but be
wary of vendors who have done so without regard for the licenses.  If your
vendor's costs seem ``too good to be true,'' you may ultimately bear the
burden of the vendor's inattention to GPL compliance.  Ask the right
questions, demand an account of your vendors' compliance procedures, and
seek indemnity from them.

\section{User Products and Installation Information}
\label{user-products}

GPLv3 requires you to provide ``Installation Information'' when v3
software is distributed in a ``User Product.''  During the drafting of v3,
the debate over this requirement was contentious.  However, the provision
as it appears in the final license is reasonable and easy to understand.

If you put GPLv3'd software into a User Product (as defined by the
license) and \emph{you} have the ability to install modified versions onto
that device, you must provide information that makes it possible for the
user to install functioning, modified versions of the software.  Note that
if no one, including you, can install a modified version, this provision
does not apply.  For example, if the software is burned onto an
non-field-upgradable ROM chip, and the only way that chip can be upgraded
is by producing a new one via a hardware factory process, then it is
acceptable that the users cannot electronically upgrade the software
themselves.

Furthermore, you are permitted to refuse support service, warranties, and
software updates to a user who has installed a modified version.  You may
even forbid network access to devices that behave out of specification due
to such modifications.  Indeed, this permission fits clearly with usual
industry practice.  While it is impossible to provide a device that is
completely unmodifiable\footnote{Consider that the iPhone, a device
  designed primarily to restrict users' freedom to modify it, was unlocked
  and modified within 48 hours of its release.}, users are generally on
notice that they risk voiding their warranties and losing their update and
support services when they make modifications.\footnote{A popular t-shirt
  in the software freedom community reads: ``I void warranties.''.  Our community is
  well-known for modifying products with full knowledge of the
  consequences.  GPLv3's ``Installation Instructions'' section merely
  confirms that reality, and makes sure GPL rights can be fully exercised,
  even if users exercise those rights at their own peril.}

GPLv3 is in many ways better for distributors who seek some degree of
device lock-down.  Technical processes are always found for subverting any
lock-down; pursuing it is a losing battle regardless.  With GPLv3, unlike
with GPLv2, the license gives you clear provisions that you can rely on
when you are forced to cut off support, service or warranty for a customer
who has chosen to modify.

\chapter{Conclusion}

GPL compliance need not be an onerous process.  Historically, struggles
have been the result of poor development methodologies and communications,
rather than any unexpected application of the GPL's source code disclosure
requirements.

Compliance is straightforward when the entirety of your enterprise is
well-informed and well-coordinated.  The receptionists should know how to
route a GPL source request or accusation of infringement.  The lawyers
should know the basic provisions of Free Software licenses and your source
disclosure requirements, and should explain those details to the software
developers.  The software developers should use a version control system
that allows them to associate versions of source with distributed
binaries, have a well-documented build process that anyone skilled in the
art can understand, and inform the lawyers when they bring in new
software.  Managers should build systems and procedures that keep everyone
on target.  With these practices in place, any organization can comply
with the GPL without serious effort, and receive the substantial benefits
of good citizenship in the software freedom community, and lots of great code
ready-made for their products.

\vfill

% LocalWords:  redistributors NeXT's Slashdot Welte gpl ISC embedders BusyBox
% LocalWords:  someone's downloadable subdirectory subdirectories filesystem
% LocalWords:  roadmap README upstream's Ravicher's Fossology readme CDs iPhone
% LocalWords:  makefiles violator's
