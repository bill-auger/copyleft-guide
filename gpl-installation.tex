\documentstyle[twocolumn]{article}
\pagestyle{empty}
\begin{document}

%don't want date printed
\date{}

%make title bold and 14 pt font (Latex default is non-bold, 16 pt)

\title{\Large\bf A Comprehensive Consideration of Installation Requirements of the GPL}

%for two authors (this is what is printed)

\author{\begin{tabular}[t]{c@{\extracolsep{8em}}c@{\extracolsep{8em}}c}
    Bradley M. Kuhn & Behan Webster \\
    Software Freedom Conservancy, Inc. & Converse In Code
\end{tabular}
}

\thispagestyle{empty}

\maketitle

\subsection*{\centering Abstract}

The GNU General License (``GPL'') is the most widely used \textit{copyleft}
license for software.  Copyleft licenses function as copyright license in
atypical manner: rather than restricting permission to copy, modify and
redistribute the software, copyleft licenses encourage and enable such
activities.  However, these license have strict requirements that mandate
further software sharing by enabling downstream users to easily improve,
modify, and upgrade the copylefted software on their own.

GPL has two versions in common use: version 2 (``GPLv2'') and version 3
(``GPLv3'').  Both versions require those who redistribute the software to
provide information related to the installation of software modified by
downstream.  These installation requirements, however, differ somewhat in
their details.  While some business practices around license compliance
efforts have reached adequate sophistication to address simpler compliance
problems, firms have generally given inadequate attention to the installation
requirements of both common versions of GPL\@.  Misunderstanding of these
clauses is often common, and violations related to installation instructions
remain prevalent.

Furthermore, perceived differences in the requirements, and lack of rigorous
study of the Installation Information requirements of GPLv3\S6 has allowed
rumor and impression, rather than a textually grounded adherence to the
written rules, to govern industry response in adoption of software licensed
under GPLv3.  The resulting scenario often causes redistributors to assume
the GPLv2 has \textbf{no} requirements regarding installation information,
and that GPLv3's requirements in this regard are impossible to meet,
particularly in security-conscious embedded products.

This paper explores the installation provisions of both common versions of
GPL, discusses historical motivations and context for each, and suggests best
practices regarding installation information for firms that redistribute
software under both licenses.

\end{document}
