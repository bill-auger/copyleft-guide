% case-study-ethics.tex                                           -*- LaTeX -*-

%      Tutorial Text for GPL Compliance Case Studies
%                    and Legal Ethics in Free Software Licensing
%
% Copyright (C) 2004 Free Software Foundation, Inc.

% Verbatim copying and distribution of this entire document is permitted in
% any medium, provided this notice is preserved.

\documentclass[12pt]{report}
% FILTER_PS:  \usepackage[dvips]{graphicx}

\newcommand\href[2]{#2}

%\newcommand\textcolor[2]{#2}

\input{/usr/share/texmf/tex/plain/dvips/colordvi.tex}

% FILTER_PDF: \usepackage[pdftex]{graphicx}
\usepackage[
pdftex=true,
latex2html=false,
pdftitle={The GNU General Public License for Businesspeople and Developers},
pdfauthor={Bradley M. Kuhn},
pdfsubject={GNU General Public License},
pdfkeywords={computer, science, free, software, freedom, licensing, licenses, GPL, GNU, general, public, license}
]{hyperref}

% I could not get this to work!
%\usepackage[pdftex,usenames]{color}
% So I used this:

\input{/usr/share/texmf/pdftex/plain/misc/pdfcolor.tex}

%\usepackage[ref]{backref}

% FILTER_HTML: 
%\usepackage{html}

%\usepackage[dvips]{graphicx}

%\usepackage[latex2html]{hyperref}
\newcommand\href[2]{\htmladdnormallink{#1}{#2}}
%\renewcommand\cite[1]{\hypercite{#1}}
%\renewcommand\ref[1]{\hyperref{#1}}

\usepackage[usenames]{color}

%\input{/usr/share/texmf/tex/plain/dvips/colordvi.tex}

% one-inch-margins.tex                                            -*- LaTeX -*-
%      code to create one inch margins in LaTeX
%
% COPYRIGHT (C) 1994 Bradley M. Kuhn
%
% Written   :   Bradley M. Kuhn         Loyola College
%   By


%% This code creates one inch margins for a LaTeX document

\oddsidemargin 0in
\evensidemargin 0in
\textwidth 6.5in

\topmargin 0in
\textheight 8.0in


%\setlength\parskip{0.7em}
%\setlength\parindent{0pt}

\newcommand{\defn}[1]{\emph{#1}}

%\pagestyle{empty}

\begin{document}

\begin{titlepage}


\begin{center}

\vspace{.5in}

{\Large {\sc GPL Compliance Case Studies and Legal Ethics in Free Software
    Licensing } \\

\vspace{.7in}

Sponsored by the Free Software Foundation \\


\vspace{.3in}

Columbia Law School, New York, NY, USA \\
\vspace{.1in}
Wednesday 21 January 2003 
}

\vspace{.7in}

{\large
Bradley M. Kuhn

Executive Director

Free Software Foundation
}

\vspace{.3in}


{\large
Daniel Ravicher

Senior Counsel 

Free Software Foundation
}

\end{center}

\vfill

{\parindent 0in
Copyright \copyright{} 2004 \hspace{.2in} Free Software Foundation, Inc.

\vspace{.3in}

Verbatim copying and distribution of this entire document is permitted in
any medium, provided this notice is preserved.
}

\end{titlepage}

\pagestyle{plain}
\pagenumbering{roman}

\begin{abstract}


This one-day course presents the details of five different GPL compliance
cases handled by FSF's GPL Compliance Laboratory.  Each case offers unique
insights into problems that can arise when the terms of GPL are not
properly followed, and how diplomatic negotiation between the violator and
the copyright holder can yield positive results for both parties.

This course also includes a unit on the ethical considerations for
attorneys who want to represent clients that make use of or sell Free
Software products.

Attendees should have successfully completely the course, a ``Detailed
Study and Analysis of GPL and LGPL'', as the material from that course
forms the building blocks for this material.

The course is of most interest to lawyers who have clients or employers
that deal with Free Software on a regular basis.  However, technical
managers and executives whose businesses use or distribute Free Software
will also find the course very helpful.

\end{abstract}

\tableofcontents

\pagebreak

\pagenumbering{arabic}

%%%%%%%%%%%%%%%%%%%%%%%%%%%%%%%%%%%%%%%%%%%%%%%%%%%%%%%%%%%%%%%%%%%%%%%%%%%%%%%
\chapter{Overview of FSF's GPL Compliance Lab}

The GPL is a Free Software license with legal teeth.  Unlike licenses like
the X11-style or various BSD licenses, GPL (and by extention, the LGPL) is
designed to defend as well as grant freedom.  We saw in the last course
that GPL uses copyright law as a mechanism to grant all the key freedoms
essential in Free Software, but also to ensure that those freedoms
propogate throughout the distribution chain of the software.

\section{Termination Begins Enforcement}

As we have learned, the assurance that Free Software under GPL remains
Free Software is accomplished through various terms of GPL: \S 3 ensures
that binaries are always accompanied with source; \S 2 ensures that the
sources are adequate, complete and usable; \S 6 and \S 7 ensures that the
license of the software is always GPL for everyone, and that no other
legal agreements or licenses trump GPL; \S 4 ensures that the GPL can be
enforced.

In fact, \S 4 is where we begin our discussion of GPL enforcement.  This
clause is where the legal teeth of the license are rooted.  As a copyright
license, GPL governs only the activities governed by copyright law ---
copying, modifying and redistributing computer software.  Unlike most
copyright licenses, GPL gives wide grants of permission for engaging with
these activities.  Such permissions continue and all parties may exercise
until such time as one party violates the terms of GPL\@.  At the moment
of such a violation --- the engaging of copying, modifying or
redistributing in ways not permitted by GPL --- \S 4 is invoked.

Specifically, \S 4 terminates the violators rights to continue engaging
in the permissions that otherwise granted by GPL\@.  Effectively, their
permission go back to the copyright defaults --- no permission to copy,
modify, or redistribute the work.  Meanwhile, \S 5 points out that if
if the violator has no rights under GPL --- as they will not once they
have violated it --- then they otherwise have no right and are prohibited
by copyright law from engaging in the activities of copying, modifying
and distributing.

\section{Ongoing Violations}

In conjuction with \S 4's termination of violators' rights, there is one
final industry fact is added to the mix: rarely, does on engage in a
single, solitary act of copying, distributing or modifying software.
Almost always, a violator will have legitimately acquired a copy a GPL'd
program --- either made modifications or not --- and then begun a ongoing
activity of distributing that work.  For example, the violator may have
put the software in boxes and sold them at stores.  Or perhaps the
software was put up for download on the Internet.  Regardless of the
delivery mechanism, violators almost always are engaged in {\em ongoing\/}
violation of GPL\@.

In fact, when we discover a GPL violation that occured only once --- for
example, a user group who distributed copies of a GNU/Linux system without
source at a meeting once --- we rarely pursue it with a high degree of
dilligence.  In our minds, that is an educational problem, and unless the
user group becomes a repeat offender (as it turns out, the never do) we
simply send an FAQ entry that best explains how user groups can most
easily comply with GPL, and send them on there merry way.

It is only the cases of {\em ongoing\/} GPL violation that warrant our
active attention.  We vehemently pursue those cases where dozens, hundreds
or thousands of customers are receiving software that is out of
compliance, and the company continually puts for sale (or distributes
gratis as a demo) software distributions that include GPL'd components out
of compliance.  Our goal is to maximize the impact of enforcement and
educate industries who are making a mistake on a large scale.

In addition, such ongoing violation shows that a particular company is
committed to a GPL'd product line.  We are thrilled to learn that someone
is benefitting from Free Software, and we understand that sometimes they
have become confused about the rules of the road.  Rather than merely
giving us a post mortem to perform on a past mistake, an ongoing violation
gives us an active opportunity to educate a new contributor the GPL'd
commons about proper procedures to contribute to the community.

Our central goal is not, in fact, to merely clear up particular violation.
Over time, we hope that our compliance lab will be out of business.  We
seek to educate the businesses that engage in commerce related to GPL'd
software to obey the rules of the road and allow them to operate freely
under them.  Just as a traffic officer would not revel in reminding people
which side of the road to drive in, so we do not revel in violations.  By
contrast, we revel in the successes of educating an ongoing violator about
GPL so that GPL compliance becomes a second-nature matter, and they join
the GPL ecosystem as contributors.

\section{First Contact}

The Free Software community is built on a structure of voluntary
cooperation and mutual help.



%%%%%%%%%%%%%%%%%%%%%%%%%%%%%%%%%%%%%%%%%%%%%%%%%%%%%%%%%%%%%%%%%%%%%%%%%%%%%%%
\chapter{Case Study A}

%%%%%%%%%%%%%%%%%%%%%%%%%%%%%%%%%%%%%%%%%%%%%%%%%%%%%%%%%%%%%%%%%%%%%%%%%%%%%%%
\chapter{Case Study B}

%%%%%%%%%%%%%%%%%%%%%%%%%%%%%%%%%%%%%%%%%%%%%%%%%%%%%%%%%%%%%%%%%%%%%%%%%%%%%%%
\chapter{Case Study C}

%%%%%%%%%%%%%%%%%%%%%%%%%%%%%%%%%%%%%%%%%%%%%%%%%%%%%%%%%%%%%%%%%%%%%%%%%%%%%%%
\chapter{Case Study D}

Reminder about how organizations themselves work.  We don't have to
educate the organization, just call their attention to something.

Working on DVD cases -- interested in the question on how one plays DVD
on one ligitimate owns, if one uses GNU/Linux give the licensing
structure of DVD content scrambling system.

An article from the IBM guy who had arranged to have DVD player
application by a vendor for includsion with IBM distributed based T20s.

They shimed the kernel, it was a GPL problem.

Couple of weeks, we've looked into it, and we're going back to the
contractor and having them redo the thing to comply with GPL.

contaminate a video output port with MacroVision.

kernel mods 

%%%%%%%%%%%%%%%%%%%%%%%%%%%%%%%%%%%%%%%%%%%%%%%%%%%%%%%%%%%%%%%%%%%%%%%%%%%%%%%
\chapter{Good Practices for Compliance}
%%%%%%%%%%%%%%%%%%%%%%%%%%%%%%%%%%%%%%%%%%%%%%%%%%%%%%%%%%%%%%%%%%%%%%%%%%%%%%%
\chapter{Ethical Considerations for the Attorney Practicing Free Software}

\end{document}

% LocalWords:  proprietarize redistributors sublicense yyyy Gnomovision EULAs
% LocalWords:  Yoyodyne FrontPage improvers Berne copyrightable Stallman's GPLs
% LocalWords:  Lessig Lessig's UCITA pre PDAs CDs reshifts GPL's Gentoo glibc
% LocalWords:  TrollTech administrivia LGPL's MontaVista OpenTV
