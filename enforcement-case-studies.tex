%      Tutorial Text for the Detailed Study and Analysis of GPL and LGPL course

% License: CC-By-SA-4.0

% The copyright holders hereby grant the freedom to copy, modify, convey,
% Adapt, and/or redistribute this work under the terms of the Creative
% Commons Attribution Share Alike 4.0 International License.

% This text is distributed in the hope that it will be useful, but
% WITHOUT ANY WARRANTY; without even the implied warranty of
% MERCHANTABILITY or FITNESS FOR A PARTICULAR PURPOSE.

% You should have received a copy of the license with this document in
% a file called 'CC-By-SA-4.0.txt'.  If not, please visit
% https://creativecommons.org/licenses/by-sa/4.0/legalcode to receive
% the license text.


\part{Case Studies in GPL Enforcement}

{\parindent 0in
This part is: \\
\begin{tabbing}
Copyright \= \copyright{} 2003, 2004, 2014 \hspace{1mm} \= \hspace{1.mm} \=  \kill
Copyright \> \copyright{} 2014 \>  Bradley M. Kuhn. \\
Copyright \> \copyright{} 2014 \>  Denver Gingerich \\
Copyright \> \copyright{} 2003, 2004, 2014 \> Free Software Foundation, Inc. \\
\end{tabbing}

\vspace{1in}

\begin{center}
Authors of this part are: \\

Bradley M. Kuhn \\
John Sullivan
\vspace{3in}

Copy editors of this part include: \\
Martin Michlmayr

\vspace{3in}

The copyright holders hereby grant the freedom to copy, modify, convey,
Adapt, and/or redistribute this work under the terms of the Creative Commons
Attribution Share Alike 4.0 International License.  A copy of that license is
available at \url{https://creativecommons.org/licenses/by-sa/4.0/legalcode}.
\end{center}
}
% =====================================================================
% START OF SECOND DAY SEMINAR SECTION
% =====================================================================

\chapter*{Preface}

This one-day course presents the details of five different GPL
compliance cases handled by FSF's GPL Compliance Laboratory. Each case
offers unique insights into problems that can arise when the terms of
the GPL are not properly followed, and how diplomatic negotiation between
the violator and the copyright holder can yield positive results for
both parties.

Attendees should have successfully completely the course, a ``Detailed
Study and Analysis of the GPL and LGPL,'' as the material from that
course forms the building blocks for this material.

This course is of most interest to lawyers who have clients or
employers that deal with Free Software on a regular basis. However,
technical managers and executives whose businesses use or distribute
Free Software will also find the course very helpful.

\bigskip

These course materials are merely a summary of the highlights of the
course presented. Please be aware that during the actual GPL course, class
discussion supplements this printed curriculum. Simply reading it is
not equivalent to attending the course.

%FIXME-LATER: write these

%\chapter{Not All GPL Enforcement is Created Equal}

%\section{For-Profit Enforcement}

%\section{Community and Non-Profit Enforcement}

\chapter{Overview of Community Enforcement}

The GPL is a Free Software license with legal teeth. Unlike licenses like
the X11-style or various BSD licenses, the GPL (and by extension, the LGPL) is
designed to defend as well as grant freedom. We saw in the last course
that the GPL uses copyright law as a mechanism to grant all the key freedoms
essential in Free Software, but also to ensure that those freedoms
propagate throughout the distribution chain of the software.

\section{Termination Begins Enforcement}

As we have learned, the assurance that Free Software under the GPL remains
Free Software is accomplished through various terms of the GPL: \S 3 ensures
that binaries are always accompanied with source; \S 2 ensures that the
sources are adequate, complete and usable; \S 6 and \S 7 ensure that the
license of the software is always the GPL for everyone, and that no other
legal agreements or licenses trump the GPL. It is \S 4, however, that ensures
that the GPL can be enforced.

Thus, \S 4 is where we begin our discussion of GPL enforcement. This
clause is where the legal teeth of the license are rooted. As a copyright
license, the GPL governs only the activities governed by copyright law ---
copying, modifying and redistributing computer software. Unlike most
copyright licenses, the GPL gives wide grants of permission for engaging with
these activities. Such permissions continue, and all parties may exercise
them until such time as one party violates the terms of the GPL\@. At the
moment of such a violation (i.e., the engaging of copying, modifying or
redistributing in ways not permitted by the GPL) \S 4 is invoked. While other
parties may continue to operate under the GPL, the violating party loses their
rights.

Specifically, \S 4 terminates the violators' rights to continue
engaging in the permissions that are otherwise granted by the GPL\@.
Effectively, their rights revert to the copyright defaults ---
no permission is granted to copy, modify, nor redistribute the work.
Meanwhile, \S 5 points out that if the violator has no rights under
the GPL, they are prohibited by copyright law from engaging in the
activities of copying, modifying and distributing. They have lost
these rights because they have violated the GPL, and no other license
gives them permission to engage in these activities governed by copyright law.

\section{Ongoing Violations}

In conjunction with \S 4's termination of violators' rights, there is
one final industry fact added to the mix: rarely does one engage in a
single, solitary act of copying, distributing or modifying software.
Almost always, a violator will have legitimately acquired a copy of a
GPL'd program, either making modifications or not, and then begun
distributing that work. For example, the violator may have put the
software in boxes and sold them at stores. Or perhaps the software
was put up for download on the Internet. Regardless of the delivery
mechanism, violators almost always are engaged in {\em ongoing\/}
violation of the GPL\@.

In fact, when we discover a GPL violation that occurred only once --- for
example, a user group who distributed copies of a GNU/Linux system without
source at one meeting --- we rarely pursue it with a high degree of
tenacity. In our minds, such a violation is an educational problem, and
unless the user group becomes a repeat offender (as it turns out, they
never do), we simply forward along a FAQ entry that best explains how user
groups can most easily comply with the GPL, and send them on their merry way.

It is only the cases of {\em ongoing\/} GPL violation that warrant our
active attention. We vehemently pursue those cases where dozens, hundreds
or thousands of customers are receiving software that is out of
compliance, and where the company continually offers for sale (or
distributes gratis as a demo) software distributions that include GPL'd
components out of compliance. Our goal is to maximize the impact of
enforcement and educate industries who are making such a mistake on a
large scale.

In addition, such ongoing violation shows that a particular company is
committed to a GPL'd product line. We are thrilled to learn that someone
is benefiting from Free Software, and we understand that sometimes they
become confused about the rules of the road. Rather than merely
giving us a postmortem to perform on a past mistake, an ongoing violation
gives us an active opportunity to educate a new contributor to the GPL'd
commons about proper procedures to contribute to the community.

Our central goal is not, in fact, to merely clear up a particular violation.
In fact, over time, we hope that our compliance lab will be out of
business. We seek to educate the businesses that engage in commerce
related to GPL'd software to obey the rules of the road and allow them to
operate freely under them. Just as a traffic officer would not revel in
reminding people which side of the road to drive on, so we do not revel in
violations. By contrast, we revel in the successes of educating an
ongoing violator about the GPL so that GPL compliance becomes a second-nature
matter, allowing that company to join the GPL ecosystem as a contributor.

\section{How are Violations Discovered?}

Our enforcement of the GPL is not a fund-raising effort; in fact, FSF's GPL
Compliance Lab runs at a loss (in other words, it is subsided by our
donors). Our violation reports come from volunteers, who have encountered,
in their business or personal life, a device or software product that
appears to contain GPL'd software. These reports are almost always sent
via email to $<$license-violation@fsf.org$>$.

Our first order of business, upon receiving such a report, is to seek
independent confirmation. When possible, we get a copy of the software
product. For example, if it is an offering that is downloadable from a
Web site, we download it and investigate ourselves. When it is not
possible for us to actually get a copy of the software, we ask the
reporter to go through the same process we would use in examining the
software.

By rough estimation, about 95\% of violations at this stage can be
confirmed by simple commands. Almost all violators have merely made an
error and have no nefarious intentions. They have made no attempt to
remove our copyright notices from the software. Thus, given the
third-party binary, {\tt tpb}, usually, a simple command (on a GNU/Linux
system) such as the following will find a Free Software copyright notice
and GPL reference:
\begin{quotation}
{\tt strings tpb | grep Copyright}
\end{quotation}
In other words, it is usually more than trivial to confirm that GPL'd
software is included.

Once we have confirmed that a violation has indeed occurred, we must then
determine whose copyright has been violated. Contrary to popular belief,
FSF does not have the power to enforce the GPL in all cases. Since the GPL
operates under copyright law, the powers of enforcement --- to seek
redress once \S 4 has been invoked --- lie with the copyright holder of
the software. FSF is one of the largest copyright holders in the world of
GPL'd software, but we are by no means the only one. Thus, we sometimes
discover that while GPL'd code is present in the software, there is no
software copyrighted by FSF present.

In cases where FSF does not hold copyright interest in the software, but
we have confirmed a violation, we contact the copyright holders of the
software, and encourage them to enforce the GPL\@. We offer our good offices
to help negotiate compliance on their behalf, and many times, we help as a
third party to settle such GPL violations. However, what we will describe
primarily in this course is FSF's first-hand experience enforcing its own
copyrights and the GPL\@.

\section{First Contact}

The Free Software community is built on a structure of voluntary
cooperation and mutual help. Our community has learned that cooperation
works best when you assume the best of others, and only change policy,
procedures and attitudes when some specific event or occurrence indicates
that a change is necessary. We treat the process of GPL enforcement in
the same way. Our goal is to encourage violators to join the cooperative
community of software sharing, so we want to open our hand in friendship.

Therefore, once we have confirmed a violation, our first assumption is
that the violation is an oversight or otherwise a mistake due to confusion
about the terms of the license. We reach out to the violator and ask them
to work with us in a collaborative way to bring the product into
compliance. We have received the gamut of possible reactions to such
requests, and in this course, we examine four specific examples of such
compliance work.

% FIXME: make this section properly TeX-formatted
\chapter{ThinkPenguin Wireless Router: Excellent CCS}

Too often, case studies examine failure and mistakes.  Indeed, most of the
chapters that follow herein will consider the myriad difficulties discovered
in community-oriented GPL enforcement for the last two decades.  However, to
begin, we offer a study in how copyleft compliance can be done correctly.

This example is, in fact, more than ten years in the making.  Since almost
the inception of for-profit corporate adoption of Free Software, companies
have requested a clear example of a model citizen to emulate.  Sadly, while
community-oriented enforcers have vetted uncounted thousands of ``Complete,
Corresponding Source'' CCS candidates from hundreds of companies, the CCS
release describes the first one CCS experts have declared a ``pristine
example''.

% FIXME (above): link to a further discussion of CCS in the compliance guide
% when a good spot exists, then (below) link to a ``CCS iteration''
% discussion in compliance-guide.tex when one exists.  (the ``iteration
% process'' is discussed in~\ref{} of this guide)

Of course, most CCS examined for the last decade has (eventually) complied
with the GPL, perhaps after many iterations of review by the enforcer.
However, in the experience of the two primary community-oriented enforcers,
Conservancy and the FSF, such CCS results routinely fix the description of
``barely complies with GPL's requirements''.  To use an academic analogy:
while a ``C'' is certainly a passing grade, any instructor prefers to
disseminate to the class an exemplar sample that earned an ``A''.

Fortunately, thanks in large part to the FSF's
``Respects Your Freedom'' (RYF) certification campaign\footnote{\href{http://www.fsf.org/resources/hw/endorsement/respects-your-freedom}{RYF is
    a campaign by FSF to certify products that truly meet the principles of
    software freedom}.  Products must meet
  \href{http://www.fsf.org/resources/hw/endorsement/criteria}{strict
    standards for RYF certification}, and among them is a pristine example of
  CCS\@.}, electronics products have begun to appear on the market that are
held to a higher standard of copyleft compliance.  As such, for the first
time in the history of copyleft, CCS experts have pristine examples to study
and present as exemplars worthy of emulation.

This case study therefore examines the entire life-cycle of a GPL compliance
investigation: from product purchase, to source request, to CCS review.
Specifically, this chapter discusses the purchase, CCS provision, and a
step-by-step build and installation analysis of a specific, physical,
embedded electronics product.  The product in question is
\href{https://www.thinkpenguin.com/gnu-linux/free-software-wireless-n-broadband-router-gnu-linux-tpe-nwifirouter}{the
  ``TPE-NWIFIROUTER'' wireless router by ThinkPenguin}.\footnote{The FSF of
  course performed a thorough CCS check as part of the certification process.
  The analysis discussed herein was independently performed by Software
  Freedom Conservancy without reviewing any findings of the FSF, and thus the
  analysis provides a ``true to form'' analysis as it occurs when Conservancy
  investigates a potential GPL violation.  In this case, obviously, no
  violation was uncovered.}

\section{Consumer Purchase and Unboxing}

The process for copyleft compliance investigation, when properly conducted,
determines whether users inclined to exercise their rights under a copyleft
license will be successful in their attempt.  Therefore, at every stage, the
investigator seeks to take actions that reasonably technically knowledgeable
users would during the ordinary course of their acquisition and use of
products.  As such, the investigator typically purchases the device on the
open market to verify that distribution of the copylefted software therein
complies with binary distribution requirements (such as those
\tutorialpartsplit{discussed in \textit{Detailed Analysis of the GNU GPL and
    Related Licenses}}{discussed here in \S~\ref{GPLv2s3} and
  \S~\ref{GPLv3s6}}).

% FIXME: Above is my only use of \tutorialpartsplit in this chapter.  I just
% got lazy and that should be fixed by someone.

\label{thinkpenguin-included-ccs}

Therefore, the investigator first purchased the TPE-NWIFIROUTER through an
online order, and when the package arrived, examined the contents of the box.
The investigator immediately discovered that ThinkPenguin had taken advice
from \S~\ref{offer-for-source} in this guide, and had chosen to use
\hyperref[GPLv2s3a]{GPLv2\S3(a)} and \hyperref[GPLv3s6]{GPLv3s6}, rather than
using the \hyperref[offer-for-source]{problematic offer for source
  provisions}.  This choice not only speeds up the investigation (since there
is no CCS offer to test), but also simplifies the compliance requirements for
ThinkPenguin.

\section{Root Filesystem and Kernel Compilation}

The CD found in the box was labeled ``libreCMC v1.2.1 source code'', and
contained 407 megabytes of data.  The investigator copied this ISO and
examined its contents.  Upon doing so, the investigator immediately found a
file called ``README'' at the top-level directory:

\lstset{tabsize=2}
\begin{lstlisting}[language=bash]
$ dd if=/dev/cdrom of=libreCMC_v1.2.1_SRC.iso
$ mkdir libCMC
$ sudo mount -o loop ./libreCMC_v1.2.1_SRC.iso libCMC
mount: block device /path/to/libreCMC_v1.2.1_SRC.iso
       is write-protected, mounting read-only
$ ls -1 libCMC
bin
librecmc-u-boot.tar.bz2
librecmc-v1.2.1.tar.bz2
README
u-boot_reflash
$ cat libCMC/README
...
\end{lstlisting}
\label{thinkpenguin-toplevel-readme}
The investigator therefore knew immediately to begin the CCS check by
studying the contents of the ``README'', which contained the appropriate
details to get started with a build:
\begin{quotation}

In order to build firmware images for your router,the following needs to be
installed:

gcc, binutils, bzip2, flex, python, perl, make, find, grep, diff, unzip,
gawk, getopt, libz-dev and libc headers.

Please use ``make menuconfig'' to configure your appreciated configuration
for the toolchain and firmware. Please note that the default configuration is
what was used to build the firmware image for your router. It is advised that
you use this configuration.

Simply running ``make'' will build your firmware.  The build system will
download all sources, build the cross-compile toolchain, the kernel and all
chosen applications.

To build your own firmware you need to have access to a GNU/Linux system
(case-sensitive filesystem required).
\end{quotation}

In other words, the first ``script'' that investigator ran in building
testing this CCS candidate was the above, which ran on the investigator's own
brain --- like a script of a play.  Less glibly, instructions written in
English are particularly necessary for parts of the build and installation
process that require some amount of actual intelligence to complete.
In this case, the investigator was able to determine the requirements for the
host system to use when constructing the firmware for the embedded device.

GPL does not, of course, give specific guidance on the form or location of
such instructions.  Community-oriented GPL enforcers generally use a
reasonableness standard to evaluate such instructions.  If an investigator of
average skill in embedded firmware construction can surmise the proper
procedures to build and install a replacement firmware, the instructions are
likely sufficient to meet GPL's requirements.  However, in this case, the
instructions are more abundant and give more detail.

These instructions are more general than typical.  Often, top-level build
instructions will specifically name a host distribution to use, such as
``Debian 7 installed on an amd64 system with the following packages
installed''.  If the build will not complete on any other system,
instructions should have such details.  However, in this case, the CCS can
build on a wide range of distributions, and thus no specific distribution was
specified.

\label{thinkpenguin-specific-host-system}

In this specific case, the developers of the libreCMC project (on which the
TPE-NWIFIROUTER is based) have clearly made an effort to ensure the CCS builds
on a variety of host systems.  The investigator was in fact dubious upon
seeing these instructions, since finicky embedded build processes usually
require a very specific host system.   Even in this case, a
\hyperref[thinkpenguin-glibc-214-issue]{minor annoyance was found that more
  detailed instructions would address}.

Anyway, since these instructions did not specify a specific host system, the
investigator simply used his own amd64 Debian 6 desktop system.  Before
beginning, the investigator used the following command:

\lstset{tabsize=2}
\begin{lstlisting}[language=bash]
$ dpkg --list | egrep '^iii' | less
\end{lstlisting}

to verify that the required packages listed in the README were
installed\footnote{The ``dpkg'' command is a Debian-specific way of
  finding installed packages.}.


Next, the investigator then extracted the primary source package with the
following command:

\lstset{tabsize=2}
\begin{lstlisting}[language=bash]
$ tar --posix -jxpf libCMC/librecmc-v1.2.1.tar.bz2
\end{lstlisting}

The investigator did notice an additional source release, entitled
``librecmc-u-boot.tar.bz2''.  The investigator concluded upon simple
inspection that the instructions found in ``u-boot\verb0_0reflash'' were
specific instructions for that part of the CCS\@.  This was a minor
annoyance, and ideally the ``README'' would list that fact, but the existing
layout met the reasonable standard that community-oriented GPL enforcers
typically apply, since the skilled investigator could determine the correct
course of action with a few moments of study.

The investigator then noted the additional step offered by the ``README'',
which read:
\begin{quotation}
Please use ``make menuconfig'' to configure your appreciated configuration
for the toolchain and firmware. Please note that the default configuration is
what was used to build the firmware image for your router. It is advised that
you use this configuration.
\end{quotation}

This instruction actually goes above and beyond the requirements of GPL\@.
Specifically, the instruction guides users in their first step toward
exercising the freedom to modify the software.  While the GPL does contain
requirements that facilitate the freedom to modify (such as ensuring the CCS is
in the ``preferred form \ldots for making modifications to it'' form), it
does not require that you write specific instructions explaining how
modifications might be undertaken.  This instruction therefore exemplifies
the exceptional quality of this particular CCS\@.

%FIXME: add a \hyperref to some ``preferred for for modification'' stuff above.

However, for purposes of the CCS verification process, typically the
investigator avoids any unnecessary changes to the source code during the
build process, lest the investigator err and cause the build to fail through
his own modification, and thus incorrectly identify the CCS as inadequate.
Therefore, the investigator proceeded to simply run:

\lstset{tabsize=2}
\begin{lstlisting}[language=bash]
$ cd libCMC
$ make
\end{lstlisting}

and waited approximately 40 minutes for the build to complete\footnote{Build
  times will likely vary widely on various host systems.}.  The investigator
kept a
\href{https://gitorious.org/copyleft-org/tutorial/source/master:enforcement-case-studies_log-output/thinkpenguin_librecmc-complete.log}{full
  log of the build}, which is not included herein due its size (approximately
7.2K of text).
\label{thinkpenguin-main-build}

Upon completion of the ``make'' process, the investigator immediately found
(almost to his surprise) several large firmware files in the ``bin/ar71xx''
directory.  Typically, this step in the CCS verification process is
harrowing.  In most cases, the ``make'' step will fail due to a missing
package or because toolchain paths are not setup correctly.

From experience, the investigator is sure that ThinkPenguin's engineers did
the most important step in self-CCS verification: use one's own instructions
on a clean system.  Ideally, an employee with similar skills but
unfamiliar with the specific product can most easily verify CCS  and identify
problems before a violation occurs.

% FIXME: Is there stuff about the above in the compliance guide?  If so, link
% to it.  If not, write it, then link to it. :)

However, upon completing the ``make'', the investigator was unclear which
filesystem and kernel images to install on the TPE-NWIFIROUTER hardware.
Ideally, the original ``README'' would indicate which image is appropriate
for the included hardware.  However, this was ultimately an annoyance rather
than a compliance issue due to other information available.  Specifically,
the web UI (see next section) on the TPE-NWIFIROUTER performs firmware image
installation.  Additionally, the router's version number was specified on the
bottom of the device, which indicated which of the differently-versioned images
we should install.  It would be ideal to find
\href{http://librecmc.org/librecmc/wiki?name=Tp+MR3020}{instructions similar
  to these} in the README itself.  However, application of the reasonableness
standard indicates compliance, since a knowledgeable user was able to
determine the proper course of action.


\section{U-Boot Compilation}

%FIXME: link to u-boot reflash, maybe put it in log-output dir?

The investigator then turned his attention to the file,
``u-boot\verb0_0reflash'' instructions.  These instructions explained how to
build and install the bootloader for the device.

The investigator followed the instructions for compiling U-Boot, and found
them quite straight-forward.  The investigator discovered two minor
annoyances, however, while building U-Boot:

\begin{itemize}

 \item The variable \verb0$U-BOOT_SRC0 was used as a placeholder for the name
   of the extracted source directory.  This was easy to surmise and was not a
   compliance issue (per the reasonableness standard), but explicitly stating
   that at the top of the instructions would be helpful.

\label{thinkpenguin-glibc-214-issue}
\item Toolchain binaries were included and used by default by the build
  process.  These binaries were not the appropriate ones for the
  investigator's host system, and the build failed with the following error:

\lstset{tabsize=2}
\begin{lstlisting}
mips-librecmc-linux-uclibc-gcc.bin: /lib/libc.so.6:
   version `GLIBC`_2.14' not found
   (required by mips-librecmc-linux-uclibc-gcc.bin)
\end{lstlisting}

   (The
\href{https://gitorious.org/copyleft-org/tutorial/source/master:enforcement-case-studies_log-output/thinkpenguin_u-boot-build_fail.log}{complete
  log output from the failure} is too lengthy to include herein.)

   This issue is an annoyance, not a compliance problem.  It was clear from
   context that these binaries were simply for a different architecture, and
   the investigator simply removed ``toolchain/bin'' and used a symlink to
   utilize the toolchain already built earlier (during the compilation
   discussed in \S~\ref{thinkpenguin-main-build}):

\lstset{tabsize=2}
\begin{lstlisting}
$ ln -s \
  ../../staging_dir/toolchain-mips_34kc_gcc-4.6-linaro_uClibc-0.9.33.2/bin \
  toolchain/bin
\end{lstlisting}


   After this change, the U-Boot build completed successfully.
\end{itemize}

The
\href{https://gitorious.org/copyleft-org/tutorial/source/master:enforcement-case-studies_log-output/thinkpenguin_u-boot-finish_build.log}{full
  log of the build} is not included herein due its size (approximately 3.8K
of text).  After that, the investigator found a new U-Boot image in the
``bin'' directory.

\section{Root Filesystem and Kernel Installation}

The investigator next tested installation of the firmware.  In particular,
the investigator connected the TPE-NWIFIROUTER to a local network, and
visited \url{http://192.168.10.1/}, logged in, and chose the option sequence:
``System $\Rightarrow$ Backup / Flash Firmware''.

From there, the investigator chose the ``Flash new firmware image'' section
and selected the
``librecmc-ar71xx-generic-tl-wr841n-v8-squashfs-sysupgrade.bin'' image from
the ``bin/ar71xx'' directory.  The investigator chose the ``v8'' image upon
verifying the physical router read ``v8.2'' on its bottom.  The investigator
chose the ``sysupgrade'' version of the image because this was clearly a
system upgrade (as a firmware already came preinstalled on the
TPE-NWIFIROUTER).

Upon clicking ``Flash image\ldots'', the web interface prompted the
investigator to confirm the MD5 hash of the image to flash.  The investigator
did so, and then clicked ``Proceed'' to flash the image.  The process took
about one minute, at which point the web page refreshed to the login screen.
Upon logging in, the investigator was able to confirm in ``Kernel Log''
section of the interface that the newly built copy of Linux had indeed been
installed.

The investigator confirmed that a new version of ``busybox'' had also been
installed by using SSH to connect to the router and ran the command
``busybox'', which showed the newly-compiled version (via its date of
compilation).

%FIXME: dg: can you get me  a screen shot for the Kernel Log above, and paste
%in the output of running busybox ?
%FIXME: bkuhn: the screen shot for the Kernel Log is in the log output dir at
%thinkpenguin_librecmc-built-kernel_log.png and the BusyBox output is in the
%same directory at thinkpenguin_librecmc-built-busybox_output.log - you may want
%to only use part of the BusyBox output (maybe even just the login) for brevity

\section{U-Boot Installation}

The U-Boot installation process is substantially more complicated than the
firmware update.  The investigator purchased the optional serial cable
along with the TPE-NWIFIROUTER, in order to complete the U-Boot installation
per the instructions in ``u-boot\verb0_0reflash'' in its section ``Installing
u-boot to your router'', which reads:

\begin{quotation}
  \begin{enumerate}

    \item Install and configure any TFTP server on your PC (tftp-hpa).

       Set a fixed IP address on your PC \ldots and connect it to the router,
       using RJ45 network cable \ldots

 \item Connect USB to UART adapter to the router and start any application to
   communicate with it, like PuTTY. \ldots

   \item Power on the router, wait for a line like one of the following and
     interrupt the process of loading a kernel:
\begin{verbatim}
    Autobooting in 1 seconds (for most TP-Link routers, you should enter tpl at this point)
    Hit ESC key to stop autoboot: 1 (for 8devices Carambola 2, use ESC key)
    Hit any key to stop autoboot: 1 (for D-Link DIR-505, use any key)
\end{verbatim}
\item   Set ipaddr and serverip environment variables:
\lstset{tabsize=2}
\begin{lstlisting}
    hornet> setenv ipaddr 192.168.1.1
    hornet> setenv serverip 192.168.1.2
\end{lstlisting}

  \end{enumerate}
\end{quotation}


%FIXME: image of the serial cable available anywhere to put here:
%  https://www.adafruit.com/images/970x728/954-02.jpg

The investigator used the purchased serial cable, which was a USB serial
adapter with a male USB type A connector to 4 female jumper wires.  The
female jumper wires were red, black, white, and green.

The instructions here were slightly incomplete, since they did not specify
how to connect the wires to the router.  However, the investigator found
general information available online at
\url{http://wiki.openwrt.org/toh/tp-link/tl-wr841nd#serial.console} which
described the proper procedure.  While the ``power'' and ``ground'' cables
were obvious, some trial and error was necessary to find the RX and TX
cables, but this was easily determined since miswiring TX and RX yields no
I/O and proper wiring yields the output as expected.  Using the pin gender
changer included with the adapter, the investigator was able to stably wire
the pins for use once the proper RX and TX connections were determined.

The investigator then used the recommended 115200 8N1 for serial console
settings, leaving all flow control off, and tested both with the
\verb0minicom0 and \verb0screen0 commands.  The investigator found that if
all 4 wires were connected on the USB serial adapter that the router would
start without additional power and the console would receive the startup
messages.  The investigator could replicate the same behavior by omitting the
power cable from the USB serial adapter (red wire) and connecting the main
power adapter to the router instead.

At this point, the on-screen messages as described in the installation
instructions appeared, but the investigator found that no key events sent via
the serial port appeared to reach the U-Boot console.  In other words, while
the investigator saw both U-Boot and kernel boot messages in the serial
console, the investigator was unable to interrupt the boot process as
instructed by ``u-boot\verb0_0reflash''.  Hitting a key simply did
\textit{not} interrupt the boot process and yield the \verb0hornet>0 prompt.

After additional trial and error over a period of hours, the investigator had
finally to consider this question for the first time during the process:
``Has ThinkPenguin violated the GPL?'' More specifically, the immediate
question was: ``Given this failure, has the distributor met
\hyperref{GPLv2s3-build-scripts}{the requirements for `scripts used to
  control \ldots installation of the executable' (GPLv2)} and
\hyperref[GPLv3-installation-information]{necessary `Installation
  Information' (GPLv3)}?''

The answer to the question; however, is (at this specific stage), ``possibly,
but more information is needed''.  Embedded installation and configuration is
a tricky and complex technical process.  While the GPL requires documentation
and clear instructions for this process, immediately blaming the distributor
for honest, correctable mistakes, or issues legitimately explained by
feasible alternative theories.

In this case, upon remembering the issues of wiring, the investigator wonder
if perhaps the power pin was accidentally connected for a moment to the RX
pin while live.  Such action could easily fry the RX pin, and yield the
observed behavior.  Since the investigator could not rule out the possibility
of accidental connection of power to the RX pin mentioned, the investigator
had to assume the instructions would work properly if he had not made that
error.

That conclusion, while correct, also left the investigator with only two
option to complete the final verification of the CCS:

\begin{itemize}

   \item Purchase a new router and cable anew, and reattempt the installation
     process while taking extra care not to miswire any cables.

   \item Seek assistance from the libreCMC community to find an alternative
     method of installation.

\end{itemize}

The investigator chose the latter and then contacted a libreCMC developers
familiar with the product.  That developer, who  agreed the the RX pin was
likely ruined, described an alternative method for console access using the
{\tt netcat}.  The method described was:

\begin{quotation}

  \begin{itemize}

  \item Change the IP address of the router to 192.168.1.1.

  \item Change the IP address of a desktop GNU/Linux system to 192.168.1.2.

  \item Power on the router while holding the reset button for 7 seconds.

  \item Use the {\tt netcat} command (as below) on the desktop, and press
    enter to receive U-Boot's prompt:
    
\lstset{tabsize=2}
\begin{lstlisting}[language=bash]
$ nc -u -p 6666 192.168.1.1 6666
uboot>
\end{lstlisting}
  \end{itemize}
\end{quotation}

Upon following this procedure, the investigator was able to confirm the
(original)  version:
\begin{lstlisting}[language=bash]
$ nc -u -p 6666 192.168.1.1 6666
uboot> version
U-Boot 1.1.4 (Jul 28 2014)
\end{lstlisting}

Thereafter, the investigator followed the instructions as original specified
in ``u-boot\verb0_0reflash''.  The investigator configured a TFTP server,
placed the newly built firmware into \texttt{/srv/tftp}.  The investigator
then followed the remaining instructions in ``u-boot\verb0_0reflash'' as
written, using the \texttt{netcat} console rather than the serial console, and
used U-Boot's \texttt{reset} command to reboot the router.

Upon reboot, the serial console (still connect with working output) showed
the message \texttt{U-Boot 1.1.4  (Oct 17 2014)}, and thus confirmed a
successful reflash of the U-Boot image built by the investigator.

\section{Firmware Comparison}

To ensure the CCS did indeed correspond to the firmware original
installed on the TPE-NWIFIROUTER, the investigator compared the built
firmware image with the filesystem originally found on the device itself.
The comparison steps were as follows:

\begin{enumerate}
  
\item Extract the filesystem from the image we built by running
  \href{https://gitorious.org/copyleft-org/gpl-compliance-scripts/source/master:find-firmware.pl}{find-firmware.pl}
  on ``bin/ar71xx/librecmc-ar71xx-generic-tl-wr841n-v8-squashfs-factory.bin''
  and then running
  \href{http://www.binaryanalysis.org/en/content/show/download}{bat-extratools}'
  ``squashfs4.2/squashfs-tools/bat-unsquashfs42'' on the resulting
  morx0.squash, using the filesystem in the new squashfs-root directory for
  comparison.

\item Login to the router's web interface (at \url{http://192.168.10.1/ }) from a computer that is
  connected to the router.
  
\item Set a password using the provided link at the top (since the router's
  UI warns that no password is set and asks the user to change it).
  
\item Login to the router via SSH, using the root user with the
  aforementioned password.
  
\item Compare representative directory listings and binaries to ensure the set of
  included files (on the router) is similar to those found in the firmware image
  we created (whose contents are now in the local squashfs-root directory).  In
  particular, we did the following comparisons:

  \begin{enumerate}
  \item List the /bin folder (``ls -l /bin'') and confirm the list of files is the same
    and that the file sizes are similar.
    
  \item Check the ``strings'' output of ``/bin/busybox'' to confirm it is similar in both
   places (similar number of lines and content of lines).  (One cannot directly
   compare the binaries because the slight compilation variations will cause
   some bits to be different.)
 \item Do the above two steps for ``/lib/modules'', ``/usr/bin'', and other directories with
   a significant number of binaries.
   
 \item Check that the kernel is sufficiently similar.  The investigator
   compared the "dmesg" output both before and after flashing the new
   firmware.  As the investigator expected, the kernel version string was
   similar, but had a different build date and user@host indicator.  (The
   kernel binary itself is not easily accessible from an SSH login, but was
   retrievable using the U-Boot console (the start address of the kernel in
   flash appears to be 0x9F020000, based on the boot messages seen in the
   serial console).
  \end{enumerate}
\end{enumerate}

\section{Minor Annoyances}

As discussed in detail above, there were a few minor annoyances, none of
which were GPL violations.  Rather, the annoyances briefly impeded the
build and installation.  However, the investigator, as a reasonably skilled
build engineer for embedded devices, was able to complete the process with
the instructions provided.

To summarize, no GPL compliance issues were found, and the CCS release was
one of the best ever reviewed by any investigator at any community-oriented
enforcement organization.  However, the following annoyances were discovered:

\begin{itemize}
\item Failure to explain how to extract the source tarball and then where to run the
  ``make'' command.
\item Failure to explain how to install the kernel and root filesystem on the
  device; the user must assume the web UI must be used.

\item Including pre-built toolchain binaries that don't work on all systems,
  and failure to put built toolchain binaries in the right location.

\item Failure to include information in the U-Boot installation instructions for
  wiring the serial cable.

\item Ideally, the U-Boot installation instructions would also include the
  {\tt netcat} method of installation.

\item Finally, the instructions should note that the new U-Boot firmware
  should be placed into \texttt{/srv/tftp} when using TFTP on most GNU/Linux
  desktops.
\end{itemize}

Thus, no CCS is absolutely perfect, but GPL violation investigators always
give the distributors the benefit of any doubts and seek to work with the
vendors and improve their CCS for the betterment of their users, customers,
and the entire software freedom community.

\section{Lessons Learned}

Companies that seek to redistribute copylefted software can benefit greatly
from ThinkPenguin's example.  Here are just a few of the many lessons that
can be learned here:

\begin{enumerate}

\item Even though copyleft licenses have them,
  \hyperref[thinkpenguin-included-ccs]{\bf avoid the offer-for-source
    provisions.}  Not only does including the CCS alongside binary
  distribution make violation investigation and compliance confirmation
  substantially easier, but more importantly it also
  \hyperref[offer-for-source]{completes the distributor's CCS compliance
    obligations at the time of distribution} (provided, of course, that the
  distributor is otherwise in compliance with copyleft).
  
\item {\bf Include top-level build instructions in a natural language (such
  as English) in a \hyperref[thinkpenguin-toplevel-readme]{clear and
    conspicuous place}.}  Copyleft licenses require that someone reasonably
  skilled in the art can reproduce your work.  Ultimately, sometimes
  instructions written in English are necessary, and often easier than trying
  to write programmed scripts to do everything.  The ``script'' included can
  certainly be more like the script of a play and less like a Bash script.

\item {\bf Write build/install instructions to the appropriate level of
  specificity}.  The upstream engineers
  in this case study \hyperref[thinkpenguin-specific-host-system]{clearly did
    additional work to ensure functionality on a wide variety of host build
    systems}; this is quite rare.  When in doubt, include the maximum level
  of detail build engineers can provide with the CCS instructions.

\item {\bf Seek to adhere to the spirit of copyleft, not just the letter of
  the license}.  ThinkPenguin uses encouragement of  users to improve and
  make their devices better as a commercial differentiator.  Copyleft advocates
  remain baffled as to why other companies have not realized how the large the
  market for
  users who seek hackable devices continues to grow.  By going beyond the
  mere minimal requirements of GPL, companies can immediately reap the
  benefits in that target market.

  \item Community-oriented enforcement organizations (for lack of a better
    phrase) do not play ``gotcha'' with distributors regarding GPL
    violations.  The goal in the GPL enforcement process is to achieve
    compliance and correct mistakes and annoyances.  Such organizations
    therefore do not wish to assume a violation, and simply work with the
    vendor to correct issues.  (This can be almost directly contrasted with
    the \hyperref[Proprietary Relicensing]{discussion in \S~\ref*{Proprietary Relicensing} about the
      proprietary relicensing} business model.)
    
\end{enumerate}

%%%%%%%%%%%%%%%%%%%%%%%%%%%%%%%%%%%%%%%%%%%%%%%%%%%%%%%%%%%%%%%%%%%%%%%%%%%%%%%
\chapter{Bortez: Modified GCC SDK}

In our first case study, we will consider Bortez, a company that
produces software and hardware toolkits to assist OEM vendors, makers
of consumer electronic devices.

\section{Facts}

One of Bortez's key products is a Software Development Kit (``SDK'')
designed to assist developers building software for a specific class of
consumer electronics devices.

FSF received a report that the SDK may be based on the GNU Compiler
Collection (which is an FSF-copyrighted collection of tools for software
development in C, C++ and other popular languages). FSF investigated the
claim, but was unable to confirm the violation. The violation reporter
was unresponsive to follow-up requests for more information.

Since FSF was unable to confirm the violation, we did not pursue it any
further. Bogus reports do happen, and we do not want to burden companies
with specious GPL violation complaints. FSF shelved the matter until
more evidence was discovered.

FSF was later able to confirm the violation when two additional reports
surfaced from other violation reporters, both of whom had used the SDK
professionally and noticed clear similarities to FSF's GNU GCC\@. FSF's
Compliance Engineer asked the reporters to run standard tests to confirm
the violation, and it was confirmed that Bortez's SDK was indeed a
modified version of GCC\@. Bortez had ported to Windows and added a number
of features, including support for a specific consumer device chipset and
additional features to aid in the linking process (``LP'') for those
specific devices. FSF explained the rights that the GPL afforded these
customers and pointed out, for example, that Bortez only needed to provide
source to those in possession of the binaries, and that the users may need
to request that source (if \S 3(b) was exercised). The violators
confirmed that such requests were not answered.

FSF brought the matter to the attention of Bortez, who immediately
escalated the matter to their attorneys. After a long negotiation,
Bortez acknowledged that their SDK was indeed a modified version of
GCC\@. Bortez released most of the source, but some disagreement
occurred over whether LP was also derivative of GCC\@. After repeated
FSF inquiries, Bortez reaudited the source to discover that FSF's
analysis was correct. Bortez determined that LP included a number of
source files copied from the GCC code-base.

\label{davrik-build-problems}
Once the full software release was made available, FSF asked the violation
reporters if it addressed the problem. Reports came back that the source
did not properly build. FSF asked Bortez to provide better build
instructions with the software, and such build instructions were
incorporated into the next software release.

At FSF's request as well, Bortez informed customers who had previously
purchased the product that the source was now available by announcing
the availability on its Web site and via a customer newsletter.

Bortez did have some concerns regarding patents. They wished to include a
statement with the software release that made sure they were not granting
any patent permission other than what was absolutely required by the GPL\@.
They understood that their patent assertions could not trump any rights
granted by the GPL\@. The following language was negotiated into the release:

\begin{quotation}
Subject to the qualifications stated below, Bortez, on behalf of itself
and its Subsidiaries, agrees not to assert the Claims against you for your
making, use, offer for sale, sale, or importation of the Bortez's GNU
Utilities or derivative works of the Bortez's GNU Utilities
(``Derivatives''), but only to the extent that any such Derivatives are
licensed by you under the terms of the GNU General Public License. The
Claims are the claims of patents that Bortez or its Subsidiaries have
standing to enforce that are directly infringed by the making, use, or
sale of an Bortez Distributed GNU Utilities in the form it was distributed
by Bortez and that do not include any limitation that reads on hardware;
the Claims do not include any additional patent claims held by Bortez that
cover any modifications of, derivative works based on or combinations with
the Bortez's GNU Utilities, even if such a claim is disclosed in the same
patent as a Claim. Subsidiaries are entities that are wholly owned by
Bortez.

This statement does not negate, limit or restrict any rights you already
have under the GNU General Public License version 2.
\end{quotation}

This quelled Bortez's concerns about other patent licensing they sought to
do outside of the GPL'd software, and satisfied FSF's concerns that Bortez
give proper permissions to exercise teachings of patents that were
exercised in their GPL'd software release.

Finally, a GPL Compliance Officer inside Bortez was appointed to take
responsibility for all matters of GPL compliance inside the company.
Bortez is responsible for informing FSF if the position is given to
someone else inside the company, and making sure that FSF has direct
contact with Bortez's Compliance Officer.

\section{Lessons}

This case introduces a number of concepts regarding GPL enforcement.

\begin{enumerate}

\item {\bf Enforcement should not begin until the evidence is confirmed.}
  Most companies that distribute GPL'd software do so in compliance, and at
  times, violation reports are mistaken. Even with extensive efforts in
  GPL education, many users do not fully understand their rights and the
  obligations that companies have. By working through the investigation
  with reporters, the violation can be properly confirmed, and {\bf the
    user of the software can be educated about what to expect with GPL'd
    software}. When users and customers of GPL'd products know their
  rights, what to expect, and how to properly exercise their rights
  (particularly under \S 3(b)), it reduces the chances for user
  frustration and inappropriate community outcry about an alleged GPL
  violation.

\item {\bf GPL compliance requires friendly negotiation and cooperation.}
  Often, attorneys and managers are legitimately surprised to find out
  GPL'd software is included in their company's products. Engineers
  sometimes include GPL'd software without understanding the requirements.
  This does not excuse companies from their obligations under the license,
  but it does mean that care and patience are essential for reaching GPL
  compliance. We want companies to understand that participating and
  benefiting from a collaborative Free Software community is not a burden,
  so we strive to make the process of coming into compliance as smooth as
  possible.

\item {\bf Confirming compliance is a community effort.}  The whole point
  of making sure that software distributors respect the terms of the GPL is to
  allow a thriving software sharing community to benefit and improve the
  work. FSF is not the expert on how a compiler for consumer electronic
  devices should work. We therefore inform the community who originally
  brought the violation to our attention and ask them to assist in
  evaluation and confirmation of the product's compliance. Of course, FSF
  coordinates and oversees the process, but we do not want compliance for
  compliance's sake; rather, we wish to foster a cooperating community of
  development around the Free Software in question, and encourage the
  once-violator to begin participating in that community.

\item {\bf Informing the harmed community is part of compliance.} FSF asks
  violators to make some attempt --- such as via newsletters and the
  company's Web site --- to inform those who already have the products as
  to their rights under the GPL\@. One of the key thrusts of the GPL's \S 1 and
  \S 3 is to {\em make sure the user knows she has these rights\/}. If a
  product was received out of compliance by a customer, she may never
  actually discover that she has such rights. Informing customers, in a
  way that is not burdensome but has a high probability of successfully
  reaching those who would seek to exercise their freedoms, is essential
  to properly remedy the mistake.

\item {\bf Lines between various copyright, patent, and other legal
  mechanisms must be precisely defined and considered.}  The most
  difficult negotiation point of the Bortez case was drafting language
  that simultaneously protected Bortez's patent rights outside of the
  GPL'd source, but was consistent with the implicit patent grant in
  the GPL\@. As we discussed in the first course of this series, there is
  indeed an implicit patent grant with the GPL, thanks to \S 6 and \S 7.
  However, many companies become nervous and wish to make the grant
  explicit to assure themselves that the grant is sufficiently narrow for
  their needs. We understand that there is no reasonable way to determine
  what patent claims read on a company's GPL holdings and which do not, so
  we do not object to general language that explicitly narrows the patent
  grant to only those patents that were, in fact, exercised by the GPL'd
  software as released by the company.

\end{enumerate}

%%%%%%%%%%%%%%%%%%%%%%%%%%%%%%%%%%%%%%%%%%%%%%%%%%%%%%%%%%%%%%%%%%%%%%%%%%%%%%%
\chapter{Bracken: a Minor Violation in a GNU/Linux Distribution}

In this case study, we consider a minor violation made by a company whose
knowledge of the Free Software community and its functions is deep.

\section{The Facts} 

Bracken produces a GNU/Linux operating system product that is sold
primarily to OEM vendors to be placed in appliance devices used for a
single purpose, such as an Internet-browsing-only device. The product
is almost 100\% Free Software, mostly licensed under the GPL and related
Free Software licenses.

FSF found out about this violation through a report first posted on a
  Slashdot\footnote{Slashdot is a popular news and discussion site for
  technical readers.} comment, and then it was brought to our attention again
  by another Free Software copyright holder who had discovered the
  same violation.

Bracken's GNU/Linux product is delivered directly from their Web site.
This allowed FSF engineers to directly download and confirm the
violation quickly. Two primary problems were discovered with the
online distribution:

\begin{itemize}

\item No source code nor offer for source code was provided for a number
  of components for the distributed GNU/Linux system; only binaries were
  available

\item An End User License Agreement (``EULA'') was included that
  contradicted the permissions granted by the GPL\@

\end{itemize}

FSF contacted Bracken and gave them the details of the violation. Bracken
immediately ceased distribution of the product temporarily and set forth
a plan to bring themselves back into compliance. This plan included the
following steps:

\begin{itemize}

\item Bracken attorneys would rewrite the EULA to comply with the GPL and
  would vet the new EULA through FSF before use

\item Bracken engineers would provide source side-by-side with the
  binaries for the GNU/Linux distribution on the site (and on CD's, if
  ever they distributed that way)

\item Bracken attorneys would run an internal seminar for its engineers
  regarding proper GPL compliance to help ensure that such oversights
  regarding source releases would not occur in the future

\item Bracken would resume distribution of the product only after FSF
  formally restored Bracken's distribution rights
\end{itemize}

This case was completed in about a month. FSF approved the new EULA
text. The key portion in the EULA relating to the GPL read as follows:

\begin{quotation}
Many of the Software Programs included in Bracken Software are distributed
under the terms of agreements with Third Parties (``Third Party
Agreements'') which may expand or limit the Licensee's rights to use
certain Software Programs as set forth in [this EULA]. Certain Software
Programs may be licensed (or sublicensed) to Licensee under the GNU
General Public License and other similar license agreements listed in part
in this section which, among other rights, permit the Licensee to copy,
modify and redistribute certain Software Programs, or portions thereof,
and have access to the source code of certain Software Programs, or
portions thereof. In addition, certain Software Programs, or portions
thereof, may be licensed (or sublicensed) to Licensee under terms stricter
than those set forth in [this EULA]. The Licensee must review the
electronic documentation that accompanies certain Software Programs, or
portions thereof, for the applicable Third Party Agreements. To the
extent any Third Party Agreements require that Bracken provide rights to
use, copy or modify a Software Program that are broader than the rights
granted to the Licensee in [this EULA], then such rights shall take
precedence over the rights and restrictions granted in this Agreement
solely for such Software Programs.
\end{quotation}

FSF restored Bracken's distribution rights shortly after the work was
completed as described.

\section{Lessons Learned}

This case was probably the most quickly and easily resolved of all GPL
violations in the history of FSF's Compliance Lab. The ease with which
the problem was resolved shows a number of cultural factors that play a
role in GPL compliance.

\begin{enumerate}

\item {\bf Companies that understand Free Software culture better have an
  easier time with compliance.}  Bracken's products were designed and
  built around the GNU/Linux system and Free Software components. Their
  engineers were deeply familiar with the Free Software ecosystem, and
  their lawyers had seen and reviewed the GPL before. The violation was
  completely an honest mistake. Since the culture inside the company had
  already adapted to the cooperative style of resolution in the Free
  Software world, there was very little work for either party to bring the
  product into compliance.

\item {\bf When people in key positions understand the Free Software
  nature of their software products, compliance concerns are as
  mundane as minor software bugs.}  Even the most functional system or
  structure has its problems, and successful business often depends on
  agile response to the problems that do come up; avoiding problems
  altogether is a pipe dream. Minor GPL violations can and do happen
  even with well-informed redistributors. However, resolution is
  reached quickly when the company --- and in particular, the lawyers,
  managers, and engineers working on the Free Software product lines
  --- have adapted to Free Software culture that the lower-level
  engineer already understood

\item {\bf Legally, distribution must stop when a violation is
  identified.}  In our opinion, Bracken went above and beyond the call of
  duty by ceasing distribution while the violation was being resolved.
  Under GPL \S 4, the redistributor loses the right to distribute the
  software, and thus they are in ongoing violation of copyright law if
  they distribute before rights are restored. It is FSF's policy to
  temporarily allow distribution while compliance negotiations are ongoing
  and only in the most extreme cases (where the other party appears to be
  negotiating in bad faith) does FSF even threaten an injunction on
  copyright grounds. However, Bracken --- as a good Free Software citizen
  --- chose to be on the safe side and do the legally correct thing while
  the violation case was pending. From start to finish, it took less
  than a month to resolve. This lapse in distribution did not, to FSF's
  knowledge, impact Bracken's business in any way.

\item {\bf EULAs are a common area for GPL problems.}  Often, EULAs
  are drafted from boilerplate text that a company uses for all its
  products. Even the most diligent attorneys forget or simply do not
  know that a product contains software licensed under the GPL and other
  Free Software licenses. Drafting a EULA that accounts for such
  licenses is straightforward; the text quoted above works just fine.
  The EULA must be designed so that it does not trump rights and
  permissions already granted by the GPL\@. The EULA must clearly state
  that if there is a conflict between it and the GPL, with regard to GPL'd
  code, the GPL is the overriding license.

\item {\bf Compliance Officers are rarely necessary when companies are
  educated about GPL compliance.}  As we saw in the Bortez case, FSF asks
  that a formal ``GPL Compliance Officer'' be appointed inside a
  previously violating organization to shepherd the organization to a
  cooperative approach to GPL compliance. However, when FSF
  sees that an organization already has such an approach, there is no
  need to request that such an officer be appointed.

\end{enumerate}


%%%%%%%%%%%%%%%%%%%%%%%%%%%%%%%%%%%%%%%%%%%%%%%%%%%%%%%%%%%%%%%%%%%%%%%%%%%%%%%
\chapter{Vigorien: Security, Export Controls, and GPL Compliance}

This case study introduces how concerns of ``security through obscurity''
and regulatory problems can impact GPL compliance matters.

\section{The Facts}

Vigorien distributes a back-up solution product that allows system
administrators to create encrypted backups of file-systems on
Unix-like computers. The product is based on GNU tar, a backup utility
that replaces the standard Unix utility simply called tar, but has
additional features.

Vigorien's backup solution added cryptographic features to GNU tar, and
included a suite of utilities and graphical user interfaces surrounding
GNU tar to make backups convenient.

FSF discovered the violation from a user report, and determined that the
cryptographic features were the only part of the product that constituted
a derivative work of GNU tar; the extraneous utilities merely made
shell calls out to GNU tar. FSF requested that Vigorien come into
compliance with the GPL by releasing the source of GNU tar, with the
cryptographic modifications, to its customers.

Vigorien released the original GNU tar sources, but kept the cryptographic
modifications proprietary. They argued that the security of their system
depending on keeping the software proprietary and that regardless, USA
export restrictions on cryptographic software prohibited such a release.
FSF disputed the first claim, pointing out that Vigorien had only one
option if they did not want to release the source: they would have to
remove GNU tar from the software and not distribute it further. Vigorien
rejected this suggestion, since GNU tar was an integral part of the
product, and the security changes were useless without GNU tar.

Regarding the export control claims, FSF proposed a number of options,
including release of the source from one of Vigorien's divisions overseas
where no such restrictions occurred, but Vigorien argued that the problem
was insoluble because they operated primarily in the USA\@.

The deadlock on the second issue was resolved when those cryptographic
export restrictions were lifted shortly thereafter, and FSF again raised
the matter with Vigorien. At that point, they dropped the first claim and
agreed to release the remaining source module to their customers. They
did so, and the violation was resolved.


\section{Lessons Learned}

\begin{enumerate}

\item {\bf Removing the GPL'd portion of the product is always an
  option.}  Many violators' first response is to simply refuse to
  release the source code as the GPL requires. FSF offers the option to
  simply remove the GPL'd portions from the product and continue along
  without them. Every case where this has been suggested has led to
  the same conclusion. Like Vigorien, the violator argues that the
  product cannot function without the GPL'd components, and they
  cannot effectively replace them.

  Such an outcome is simply further evidence that the combined work in
  question is indeed a modified version of the original GPL'd component.
  If the other components cannot stand on their own and be useful without
  the GPL'd portions, then one cannot effectively argue that the work as a
  whole is not a based on the GPL'd portions.

\item {\bf The whole product is not always covered.}  In this case,
  Vigorien had additional works aggregated. The backup system was a suite
  of utilities, some of which were the GPL and some of which were not. While
  the cryptographic routines were tightly coupled with GNU tar and clearly
  made a whole new combined work of both components, the various GUI utilities were separate and
  independent works merely aggregated with the distribution of the
  GNU-tar-based product.


\item {\bf ``Security'' concerns do not exonerate a distributor from GPL
  obligations, and ``security through obscurity'' does not work anyway.}
  The argument that ``this is security software, so it cannot be released
  in source form'' is not a valid defense for explaining why the terms of
  the GPL are ignored. If companies do not want to release source code
  for some reason, then they should not base the work on GPL'd software.
  No external argument for noncompliance can hold weight if the work as
  a whole is indeed a modified version of a GPL'd program.

  The ``security concerns'' argument is often floated as a reason to keep
  software proprietary, but the computer security community has on
  numerous occasions confirmed that such arguments are entirely specious.
  Security experts have found --- since the beginnings of the field of
  cryptography in the ancient world --- that sharing results about systems
  and having such systems withstand peer review and scrutiny builds the
  most secure systems. While full disclosure may help some who wish to
  compromise security, it helps those who want to fix problems even more
  by identifying them early.

\item {\bf External regulatory problems can be difficult to resolve.}
  The GPL, though grounded in copyright law, does not have the power to trump
  regulations like export controls. While Vigorien's ``security
  concerns'' were specious, their export control concerns were not. It is
  indeed a difficult problem that FSF acknowledges. We want compliance
  with the GPL and respect for users' freedoms, but we certainly do not expect
  companies to commit criminal offenses for the sake of compliance. We
  will see more about this issue in our next case study.
\end{enumerate}


%%%%%%%%%%%%%%%%%%%%%%%%%%%%%%%%%%%%%%%%%%%%%%%%%%%%%%%%%%%%%%%%%%%%%%%%%%%%%%%
\chapter{Haxil, Polgara, and Thesulac: Mergers, Upstream Providers and Radio Devices}

This case study considers an ongoing (at the time of writing) violation
that has occurred. By the end of the investigation period, three
companies were involved and many complex issues arose.

\section{The Facts}

Haxil produced a consumer electronics device which included a mini
GNU/Linux distribution to control the device. The device was of interest
to many technically-minded consumers, who purchased the device and very
quickly discovered that Free Software was included without source.
Mailing lists throughout the Free Software community erupted with
complaints about the problem, and FSF quickly investigated.

FSF confirmed that FSF-copyrighted GPL'd software was included. In
addition, the whole distribution included GPL'd works from hundreds of
individual copyright holders, many of whom were, at this point, up in
arms about the violation.

Meanwhile, Haxil was in the midst of being acquired by Polgara. Polgara
was as surprised as everyone else to discover the product was based on
GPL'd software; this fact had not been part of the disclosures made during
acquisition. FSF contacted Haxil, Polgara, and the product managers
who had transitioned into the ``Haxil division'' of the newly-merged
Polgara company. Polgara's General Counsel's office worked with FSF on
the matter.

FSF formed a coalition with the other primary copyright holders
to pursue the enforcement effort on their behalf. FSF communicated
directly with Polgara's representatives to begin working through the
issues on behalf of itself and the Free Software community at large.

Polgara pointed out that the software distribution they used was mostly
contributed by an upstream provider, Thesulac, and Haxil's changes to that
code base were minimal. Polgara negotiated with Thesulac to obtain the
source, although the issue moved very slowly in the channels between
Polgara and Thesulac.

FSF encouraged a round-table meeting so that high bandwidth communication
could occur between FSF, Polgara and Thesulac. Polgara and Thesulac
agreed, and that discussion began. Thesulac provided nearly complete
sources to Polgara, and Polgara made a full software release on their
Web site. At the time of writing, that software still has some build
problems (similar to those that occurred with Bortez, as described in
Section~\ref{davrik-build-problems}). FSF continues to negotiate with
Polgara and Thesulac to resolve these problems, which have a clear path to
a solution and are expected to resolve.

Similar to the Vigorien case, Thesulac has regulatory concerns. In this
case, it is not export controls --- an issue that has since been resolved
--- but radio spectrum regulation. Since this consumer electronic device
contains a software-programmable radio transmitter, regulations in (at
least) the USA and Japan prohibit release of those portions of the code
that operate the device. Since this is a low-level programming issue, the
changes to operate the device form a single combined work with the kernel named
Linux.  A decade later, this situation remains largely unresolved.

\section{Lessons Learned}

\begin{enumerate}

\item {\bf Community outrage, while justified, can often make negotiation
  more difficult.}  FSF has a strong policy never to publicize names of
  GPL violators if they are negotiating in a friendly way and operating in
  good faith toward compliance. Most violations are honest mistakes, and
  FSF sees no reason to publicly admonish violators who genuinely want to
  come into compliance with the GPL and to work hard staying in compliance.

  This case was so public in the Free Software community that both Haxil's
  and Polgara's representatives were nearly shell-shocked by the time FSF
  began negotiations. There was much work required to diffuse the
  situation. We empathize with our community and their outrage about GPL
  violations, but we also want to follow a path that leads expediently
  to compliance. In our experience, public outcry works best as a last
  resort, not the first.

\item {\bf For software companies, GPL compliance belongs on a corporate
  acquisition checklist. }  Polgara was truly amazed that Haxil had used
  GPL'd software in a major new product line but never informed Polgara
  during the acquisition process. While GPL compliance is not a
  particularly difficult matter, it is an additional obligation that comes
  along with the product line. When planning mergers and joint ventures,
  one should include lists of GPL'd components contained in the products
  discussed.

\item {\bf Compliance problems of upstream providers do not excuse a
  violation for the downstream distributor.}  To paraphrase \S 6, upstream
  providers are not responsible for enforcing compliance of their
  downstream, nor are downstream distributors responsible for compliance
  problems of upstream providers. However, engaging in distribution of
  GPL'd works out of compliance is still just that: a compliance problem.
  When FSF carries out enforcement, we are patient and sympathetic when
  the problem appears to be upstream. In fact, we urge the violator to
  point us to the upstream provider so we may talk to them directly. In
  this case, we were happy to begin negotiations with Thesulac. However,
  Polgara still has an obligation to bring their product into compliance,
  regardless of Thesulac's response.

\item {\bf It behooves upstream providers to advise downstream
  distributors about compliance matters.}  FSF has encouraged Thesulac to
  distribute a ``good practices for GPL compliance'' document with their
  product. Polgara added various software components to Thesulac's
  product, and it is conceivable that such additions can introduce
  compliance. In FSF's opinion, Thesulac is in no way legally responsible
  for such a violation introduced by their customer, but it behooves them
  from a marketing standpoint to educate their customers about using the
  product. We can argue whether or not it is your coffee vendor's fault
  if you burn yourself with their product, but (likely) no one on either
  side would dispute the prudence of placing a ``caution: hot'' label on
  the cup.

\item {\bf FSF enforcement often avoids redundant enforcement cases from
  many parties.}  Most Free Software systems have hundreds of copyright
  holders. Some have thousands. FSF is in a unique position as one of
  the largest single copyright holders on GPL'd software and as a
  respected umpire in the community, neutrally enforcing the rules of the
  GPL road. FSF works hard in the community to convince copyright
  holders that consolidating GPL claims through FSF is better for them,
  and more likely to yield positive compliance results.

  A few copyright holders engage in the ``proprietary relicensing''
  business, so they use GPL enforcement as a sales channel for that
  business. FSF, as a community-oriented, not-for-profit organization,
  seeks only to preserve the freedom of Free Software in its enforcement
  efforts. As it turns out, most of the community of copyright holders
  of Free Software want the same thing. Share and share alike is a
  simple rule to follow, and following that rule to FSF's satisfaction
  usually means you are following it to the satisfaction of the entire
  Free Software community.

\end{enumerate}

%%%%%%%%%%%%%%%%%%%%%%%%%%%%%%%%%%%%%%%%%%%%%%%%%%%%%%%%%%%%%%%%%%%%%%%%%%%%%%%


%%%%%%%%%%%%%%%%%%%%%%%%%%%%%%%%%%%%%%%%%%%%%%%%%%%%%%%%%%%%%%%%%%%%%%%%%%%%%%%
% COMMENT OUT THIS CHAPTER.
% FIXME: is this material moot now that we include the compliance guide?
% Either way, it should be merged into compliance guide.
%\chapter{Good Practices for Compliance}

Generally, from the experience of GPL enforcement, we glean the following
general practices that can help in GPL compliance for organizations that
distribute products based on GPL'd software:

\begin{itemize}

\item Talk to your software engineers and ask them where they got the
  components they use in the products they build. Find out if GPL'd
  components are present.

\item Teach your engineering staff to pay attention to license documents.
  Give them easy-to-follow policies to get approval for using a Free
  Software component.

\item Build a ``Free Software Licensing'' committee that handles requests
  and questions about the GPL and other Free Software licenses.

\item Add ``What parts of your products are under the GPL or other Free
  Software licenses?'' to your checklist of questions to ask when you
  consider mergers, acquisitions, or joint ventures.

\item Encourage your engineers to participate collaboratively with GPL'd
  software development. The more knowledge about the Free Software world
  your organization has, the better equipped it is to deal with this
  rapidly changing field.

\item When someone points out a potential GPL violation in one of your
  products, do not assume the product line is doomed. The GPL is not a virus;
  merely having GPL'd code in one part of a product does not necessarily
  mean that every related product must also be GPL'd. And, even if some
  software needs to be released that was not before, the product will
  surely survive. In FSF's enforcement efforts, we have not yet
  seen a product line die because source was released to customers in
  compliance with the GPL.

\end{itemize}

%%%%%%%%%%%%%%%%%%%%%%%%%%%%%%%%%%%%%%%%%%%%%%%%%%%%%%%%%%%%%%%%%%%%%%%%%%%%%%%
% LocalWords:  proprietarize redistributors sublicense yyyy Gnomovision EULAs
% LocalWords:  Yoyodyne FrontPage improvers Berne copyrightable Stallman's GPLs
% LocalWords:  Lessig Lessig's UCITA pre PDAs CDs reshifts GPL's Gentoo glibc
% LocalWords:  TrollTech administrivia LGPL's MontaVista OpenTV Mitek Arce DVD
% LocalWords:  unprotectable protectable Unfreedonia chipset CodeSourcery Iqtel
% LocalWords:  impermissibly Bateman faire minimis Borland uncopyrightable Mgmt
% LocalWords:  franca downloadable Bortez Bortez's
% LocalWords:  Slashdot sublicensed Vigorien Vigorien's Haxil Polgara
% LocalWords:  Thesulac Polgara's Haxil's Thesulac's SDK CD's
