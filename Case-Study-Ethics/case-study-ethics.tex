% case-study-ethics.tex                                           -*- LaTeX -*-

%      Tutorial Text for GPL Compliance Case Studies
%                    and Legal Ethics in Free Software Licensing
%
% Copyright (C) 2004 Free Software Foundation, Inc.

% Verbatim copying and distribution of this entire document is permitted in
% any medium, provided this notice is preserved.

\documentclass[12pt]{report}
% FILTER_PS:  \usepackage[dvips]{graphicx}

\newcommand\href[2]{#2}

%\newcommand\textcolor[2]{#2}

\input{/usr/share/texmf/tex/plain/dvips/colordvi.tex}

% FILTER_PDF: \usepackage[pdftex]{graphicx}
\usepackage[
pdftex=true,
latex2html=false,
pdftitle={The GNU General Public License for Businesspeople and Developers},
pdfauthor={Bradley M. Kuhn},
pdfsubject={GNU General Public License},
pdfkeywords={computer, science, free, software, freedom, licensing, licenses, GPL, GNU, general, public, license}
]{hyperref}

% I could not get this to work!
%\usepackage[pdftex,usenames]{color}
% So I used this:

\input{/usr/share/texmf/pdftex/plain/misc/pdfcolor.tex}

%\usepackage[ref]{backref}

% FILTER_HTML: 
%\usepackage{html}

%\usepackage[dvips]{graphicx}

%\usepackage[latex2html]{hyperref}
\newcommand\href[2]{\htmladdnormallink{#1}{#2}}
%\renewcommand\cite[1]{\hypercite{#1}}
%\renewcommand\ref[1]{\hyperref{#1}}

\usepackage[usenames]{color}

%\input{/usr/share/texmf/tex/plain/dvips/colordvi.tex}

% one-inch-margins.tex                                            -*- LaTeX -*-
%      code to create one inch margins in LaTeX
%
% COPYRIGHT (C) 1994 Bradley M. Kuhn
%
% Written   :   Bradley M. Kuhn         Loyola College
%   By


%% This code creates one inch margins for a LaTeX document

\oddsidemargin 0in
\evensidemargin 0in
\textwidth 6.5in

\topmargin 0in
\textheight 8.0in


%\setlength\parskip{0.7em}
%\setlength\parindent{0pt}

\newcommand{\defn}[1]{\emph{#1}}

%\pagestyle{empty}

\begin{document}

\begin{titlepage}


\begin{center}

\vspace{.5in}

{\Large {\sc GPL Compliance Case Studies and Legal Ethics in Free Software
    Licensing } \\

\vspace{.7in}

Sponsored by the Free Software Foundation \\


\vspace{.3in}

Columbia Law School, New York, NY, USA \\
\vspace{.1in}
Wednesday 21 January 2003 
}

\vspace{.7in}

{\large
Bradley M. Kuhn

Executive Director

Free Software Foundation
}

\vspace{.3in}


{\large
Daniel Ravicher

Senior Counsel 

Free Software Foundation
}

\end{center}

\vfill

{\parindent 0in
Copyright \copyright{} 2004 \hspace{.2in} Free Software Foundation, Inc.

\vspace{.3in}

Verbatim copying and distribution of this entire document is permitted in
any medium, provided this notice is preserved.
}

\end{titlepage}

\pagestyle{plain}
\pagenumbering{roman}

\begin{abstract}


This one-day course presents the details of five different GPL compliance
cases handled by FSF's GPL Compliance Laboratory.  Each case offers unique
insights into problems that can arise when the terms of GPL are not
properly followed, and how diplomatic negotiation between the violator and
the copyright holder can yield positive results for both parties.

This course also includes a unit on the ethical considerations for
attorneys who want to represent clients that make use of or sell Free
Software products.

Attendees should have successfully completely the course, a ``Detailed
Study and Analysis of GPL and LGPL'', as the material from that course
forms the building blocks for this material.

The course is of most interest to lawyers who have clients or employers
that deal with Free Software on a regular basis.  However, technical
managers and executives whose businesses use or distribute Free Software
will also find the course very helpful.

\end{abstract}

\tableofcontents

\pagebreak

\pagenumbering{arabic}

%%%%%%%%%%%%%%%%%%%%%%%%%%%%%%%%%%%%%%%%%%%%%%%%%%%%%%%%%%%%%%%%%%%%%%%%%%%%%%%
\chapter{Overview of FSF's GPL Compliance Lab}

%%%%%%%%%%%%%%%%%%%%%%%%%%%%%%%%%%%%%%%%%%%%%%%%%%%%%%%%%%%%%%%%%%%%%%%%%%%%%%%
\chapter{Case Study A}

%%%%%%%%%%%%%%%%%%%%%%%%%%%%%%%%%%%%%%%%%%%%%%%%%%%%%%%%%%%%%%%%%%%%%%%%%%%%%%%
\chapter{Case Study B}

%%%%%%%%%%%%%%%%%%%%%%%%%%%%%%%%%%%%%%%%%%%%%%%%%%%%%%%%%%%%%%%%%%%%%%%%%%%%%%%
\chapter{Case Study C}


%%%%%%%%%%%%%%%%%%%%%%%%%%%%%%%%%%%%%%%%%%%%%%%%%%%%%%%%%%%%%%%%%%%%%%%%%%%%%%%
\chapter{Case Study D}

%%%%%%%%%%%%%%%%%%%%%%%%%%%%%%%%%%%%%%%%%%%%%%%%%%%%%%%%%%%%%%%%%%%%%%%%%%%%%%%
\chapter{Good Practices for Compliance}
%%%%%%%%%%%%%%%%%%%%%%%%%%%%%%%%%%%%%%%%%%%%%%%%%%%%%%%%%%%%%%%%%%%%%%%%%%%%%%%
\chapter{Ethical Considerations for the Attorney Practicing Free Software}

\end{document}

% LocalWords:  proprietarize redistributors sublicense yyyy Gnomovision EULAs
% LocalWords:  Yoyodyne FrontPage improvers Berne copyrightable Stallman's GPLs
% LocalWords:  Lessig Lessig's UCITA pre PDAs CDs reshifts GPL's Gentoo glibc
% LocalWords:  TrollTech administrivia LGPL's MontaVista OpenTV
